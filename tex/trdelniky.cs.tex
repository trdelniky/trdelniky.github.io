\documentclass[a5paper,10pt]{book}
\usepackage[english,hungarian,czech]{babel}
\usepackage{luavlna}
\usepackage{trdelnik-book}
\usepackage{lipsum}
\title{Trdelník}
\begin{document}

\frontmatter
\maketitle
\ifdefined\HCode\else
\tableofcontents
\fi

\mainmatter
\chapter*{Úvod}
Tohle je úvodní text k dokumentu. Je to první kapitola, která se



\lipsum[1-12]

Trdelník je český a slovenský označení pro koláč původně pečenej na rožni.
Existujou tři základní typy: pruh těsta namotanej na válec, pak rozválený těsto
namotaný na válec jako jeden velkej plát a nakonec řídký těsto, který se na
válec lije v několika vrstvách.

Historicky nejstarší je první typ (taky to je to, co je známý jako trdelník
nebo kurtoskalács dneska). Podobný pečivo bylo známý už antice, ale reálně z
kuchařek je od 15. století známý jako spiesskuchen, česky vaječník. Z našeho
území je známej z roku 1554 v kázání Johana Mathesia z Jáchymova. Pak se o něm
zmiňuje Komenský a nachází se v několika slovnících.

Druhej typ je o dost vzácnější, našel jsem o něm jen několik zmínek. Ale v
kuchařce J. C. Thiema z roku 1694 je na něj recept, označenej jako Bohmische
Küchlein.

Třetí typ je nejnovější, je to v podstatě dort a je známej z Německa jako
baumkuchen, z Rakouska prügelkrapfen, je rozšířenej i v Polsku, Švédsku a
Litvě. Na Moravě byl rozšířenej na Znojemsku. V Čechách byl populární koncem
19. století a v první půlce století 20. jako dort trdlovec. Nedělal se ale
doma, nabízely ho různý cukrárny a taky se prodával na poutích. Vtipný je, už
tenkrát se prodával jako staročeský trdlovec a lidi z toho měli legraci. Dneska
je u nás už úplně zapomenutej.

Termín trdelník se objevuje začátkem 19. století. Je tak pojmenovanej recept na
trdelník z litýho těsta v překladu kuchařky od Sibilly Dorizio. Není úplně
jistý datum vydání, ale nejspíš je to rok 1816. Každopádně to není recept na
typ trdelníku, jak ho známe dneska.

V roce 1813 vyšla sbírka básní Muza Morawská od J. H. A. Gallaše. Trdelník se
vyskytuje ve třech básních, třeba v Nestřídmost v jídle jedné hanácké osoby.
Gallaš žil v Hranicích a popisoval prostředí Valašska a Haný.

Pak je celkem velká mezera, ale od 80. let 19. století je o trdelnících velký
množství zmínek, protože byly populární různý vlastenecký spolky, který
zkoumaly lidovou kulturu a trdelníky a náčiní pro jejich pečení zkoumali jako
historickou kuriozitu. V roce 1895 byla v Praze velká národopisná výstava a
před ní byly menší výstavy i v regionech. Trdelníky jako tradiční pečivo
představovali na Slovácku, Brněnsku, Haný nebo Třebíčsku.

Už v týhle době byl tradiční trdelník pečenej na ohni na ústupu, protože v
domácnostech se přecházelo z černý kuchyně (to byl v podstatě otevřenej oheň v
kuchyni) na sporáky s pecí a klasický trdelníky se v ní dělat nemohly. Vznikly
proto jiný variaty, který se daly píct v troubě, nebo se smažily v sádle.
Protože byly mnohem menší, daly se plnit krémem a vzniklo něco podobnýho
kremrolím. Respektive kremrole samotný možná vznikly původně z trdelníku.

Během první půlky 20. století je o trdelníku ještě spousta zmínek a vyskytuje
se ve spoustě kuchařek (jak litá, tak navíjená varianta), ale po 2. světový
válce už celkem mizí a vyskytuje se spíš v etnografických publikacích, ale i
kuchařkách. Třeba Lidové pečivo v Čechách a na Moravě z roku 1988 se trdelníkům
věnuje dost podrobně a popisuje recepty na všechny možný varianty. V našem
století se pak někde začaly obnovovat tradice a trdelníky se zase začaly
objevovat v některých lokalitách, kde bejval oblíbenej.

Na Slovensku pak v 80. letech začala skalická pobočka Západoslovenských pekární
vyrábět trdelníky a ty se staly celkem popuární, dokonce získal cenu Zlatý
kosák na výstave Agrokomplex. Pak se prodával po různých jarmarcích a podobně.

Pokud jde o historii trdelníku v Praze, tak podle tohohle článku se objevil
poprvý v roce 2000. Jeden chlapík si ho všiml ve Štúrově na jarmarku a zaujal
ho natolik, že ho začal vyrábět. Nejdřív taky na různejch jarmarcích a pak na
vánočních trzích na Staromáku. Stal se z toho hit, začali ho kopírovat další
prodejci, vznikaly různý varianty a tím jsme se dostali do dnešního stavu.
Takže pražskej trdelník původně vycházel ze Skalickýho trdelníku, ale používají
jinou recepturu na těsto, aby se líp připravovalo (a taky je levnější) a různý
náplně a posypky jsou veskrze místní inovace.

V Maďarsku je ten průběh šíření dost podobnej. Původně byl kürtőskalács
rozšířenej docela hojně, ale v průběhu 20. století mizel, až se zachoval jen v
Sedmihradsku v Rumunsku. Někdy v 70. nebo 80. letech, když se začalo povoloval
drobný podnikání, začal se šířit v různejch stáncích i jinam, postupně se
dostal i do Maďarska, kde se stal taky populárním.

Pokud jde o legendu o původu trdelníku ve Skalici, nejstarší zmínka o ní,
kterou jsem našel, je z roku 1998. I v dokumentu pro registraci chráněnýho
geografickýho označení se píše, že se to místně traduje. Vzhledem k tomu, že
byl u nás spiesskuchen známej stovky let před tím, než se měl údajně dostat do
Skalice spolu s kuchařem ze Sedmihradska, a že jen pár let potom ho popisuje
Galaš ze střední Moravy, není moc pravděpodobný, že by se u nás skutečně
rozšířil z Rumunska.

\chapter{Typy trdelníku}
Trdelník se v průběhu historie vyvinul do několika odlišných typů:

\section{Podle způsobu přípravy}
\begin{itemize}
\item \textbf{Pečené na ohni} - Tradiční způsob přípravy na dřevěném válci (trdlu) nad otevřeným ohněm. Typické pro jižní Moravu a Slovensko (Skalický trdelník).
\item \textbf{Smažené} - Varianta rozšířená na Moravě, kde se těsto smažilo v sádle na plechových formách.
\item \textbf{Pečené v troubě} - Pozdější varianta, která se objevila s rozšířením sporáků.

\subsection{Pečené v troubě}

\lipsum[1-3]

\subsection{Smažené}

\lipsum[4-6]

\end{itemize}

\section{Podle typu těsta}
\begin{itemize}
\item \textbf{Z nekynutého těsta} - Nejstarší varianta, doložená již v 16. století.
\item \textbf{Z kynutého těsta} - Rozšířená od 18. století, často s přidáním másla a vajec.
\item \textbf{Litý trdelník (trdlovec)} - Baumkuchen připravovaný litím těsta na rožeň.
\end{itemize}

\section{Regionální varianty}
\begin{itemize}
\item \textbf{Skalický trdelník} - Chráněné zeměpisné označení EU, pečený na dřevěném válci.
\item \textbf{Moravské trdelníky} - Často smažené varianty z kynutého těsta.
\item \textbf{Valašský trdelník} - Údajně přinesen Valachy z Rumunska v 17. století.
\item \textbf{Kürtőskalács} - Maďarská/transylvánská varianta s cukrovou polevou.
\end{itemize}

\chapter{Historický vývoj přípravy}
\section{Rané zmínky (14.-18. století)}
\begin{itemize}
\item První zmínky v českých pramenech: Klaret (1360), Mathesiova kázání (1554), Thiema kuchařka (1694)
\item Německé názvy: Spießkuchen, Prügelkrapfen, Ringelkrapfen
\subsection{Zkouším subsection}
\item Latinský název: obolum
\subsection{Další subsection}
\item Příprava na rožni v měšťanských kuchyních
\end{itemize}

\section{19. století}
\begin{itemize}
\item Rozšíření na venkově, zejména na Moravě
\item Pečení v černých kuchyních na otevřeném ohništi
\item Použití při svátcích (svatby, křtiny, masopust)
\item Objevují se první tištěné recepty (Dorizio 1816, Rettigová 1826)
\end{itemize}

\section{20. století}
\begin{itemize}
\item Úpadek tradiční přípravy s mizením černých kuchyní
\item Průmyslová výroba na Slovensku (Piešťany 1984)
\item Obnova tradice ve Skalici
\end{itemize}

\section{Současnost}
\begin{itemize}
\item Turistická atrakce od 2000 let
\item Spory o původ (český vs. maďarský)
\item Moderní variace s náplněmi
\end{itemize}

\chapter{Zdroje receptů}
Historické recepty lze nalézt v těchto pramenech:

\section{Tištěné kuchařky}
\begin{itemize}
\item Sibylla Dorizio: "Moje skrz čtyřicetileté vykonávání známá Kuchařská kniha" (1816)
\item Magdalena Dobromila Rettigová: "Domácí kuchařka" (1826)
\item Marie Úlehlová-Tilschová: "Česká strava lidová" (1945)
\end{itemize}

\section{Národopisné prameny}
\begin{itemize}
\item Augusta Šebestová: "Lidské dokumenty" (1900)
\item František Bartoš: Dialektický slovník moravský (1906)
\item Časopis Malovaný kraj (od 1949)
\end{itemize}

\section{Archivní materiály}
\begin{itemize}
\item Moravský zemský archiv (svatební menu, účty)
\item Národopisné sbírky muzeí (Muzeum Hodonínska, Slovácké muzeum)
\item EU dokumentace k chráněnému označení Skalický trdelník
\end{itemize}

\chapter{Závěr}

Trdelník prošel složitým vývojem od středověkého měšťanského pečiva přes venkovskou specialitu až po moderní turistickou atrakci. Jeho historie odráží změny ve způsobech bydlení (černé kuchyně → sporáky) i společenské proměny (od obřadního pečiva k komerčnímu produktu). Přes spory o původ zůstává nedílnou součástí kulinárního dědictví střední Evropy.

  
\end{document}
