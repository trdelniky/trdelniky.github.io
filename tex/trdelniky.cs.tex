\documentclass[a5paper,10pt]{book}
\usepackage[english,hungarian,czech]{babel}
\usepackage{luavlna}
\usepackage{trdelnik-book}
\title{Trdelník}
\begin{document}

\frontmatter
\maketitle
\ifdefined\HCode\else
\tableofcontents
\fi

\mainmatter
\chapter*{Úvod}
Tohle je úvodní text k dokumentu. Je to první kapitola, která se



\chapter{Typy trdelníku}
Trdelník se v průběhu historie vyvinul do několika odlišných typů:

\section{Podle způsobu přípravy}
\begin{itemize}
\item \textbf{Pečené na ohni} - Tradiční způsob přípravy na dřevěném válci (trdlu) nad otevřeným ohněm. Typické pro jižní Moravu a Slovensko (Skalický trdelník).
\item \textbf{Smažené} - Varianta rozšířená na Moravě, kde se těsto smažilo v sádle na plechových formách.
\item \textbf{Pečené v troubě} - Pozdější varianta, která se objevila s rozšířením sporáků.
\end{itemize}

\section{Podle typu těsta}
\begin{itemize}
\item \textbf{Z nekynutého těsta} - Nejstarší varianta, doložená již v 16. století.
\item \textbf{Z kynutého těsta} - Rozšířená od 18. století, často s přidáním másla a vajec.
\item \textbf{Litý trdelník (trdlovec)} - Baumkuchen připravovaný litím těsta na rožeň.
\end{itemize}

\section{Regionální varianty}
\begin{itemize}
\item \textbf{Skalický trdelník} - Chráněné zeměpisné označení EU, pečený na dřevěném válci.
\item \textbf{Moravské trdelníky} - Často smažené varianty z kynutého těsta.
\item \textbf{Valašský trdelník} - Údajně přinesen Valachy z Rumunska v 17. století.
\item \textbf{Kürtőskalács} - Maďarská/transylvánská varianta s cukrovou polevou.
\end{itemize}

\chapter{Historický vývoj přípravy}
\section{Rané zmínky (14.-18. století)}
\begin{itemize}
\item První zmínky v českých pramenech: Klaret (1360), Mathesiova kázání (1554), Thiema kuchařka (1694)
\item Německé názvy: Spießkuchen, Prügelkrapfen, Ringelkrapfen
\item Latinský název: obolum
\item Příprava na rožni v měšťanských kuchyních
\end{itemize}

\section{19. století}
\begin{itemize}
\item Rozšíření na venkově, zejména na Moravě
\item Pečení v černých kuchyních na otevřeném ohništi
\item Použití při svátcích (svatby, křtiny, masopust)
\item Objevují se první tištěné recepty (Dorizio 1816, Rettigová 1826)
\end{itemize}

\section{20. století}
\begin{itemize}
\item Úpadek tradiční přípravy s mizením černých kuchyní
\item Průmyslová výroba na Slovensku (Piešťany 1984)
\item Obnova tradice ve Skalici
\end{itemize}

\section{Současnost}
\begin{itemize}
\item Turistická atrakce od 2000 let
\item Spory o původ (český vs. maďarský)
\item Moderní variace s náplněmi
\end{itemize}

\chapter{Zdroje receptů}
Historické recepty lze nalézt v těchto pramenech:

\section{Tištěné kuchařky}
\begin{itemize}
\item Sibylla Dorizio: "Moje skrz čtyřicetileté vykonávání známá Kuchařská kniha" (1816)
\item Magdalena Dobromila Rettigová: "Domácí kuchařka" (1826)
\item Marie Úlehlová-Tilschová: "Česká strava lidová" (1945)
\end{itemize}

\section{Národopisné prameny}
\begin{itemize}
\item Augusta Šebestová: "Lidské dokumenty" (1900)
\item František Bartoš: Dialektický slovník moravský (1906)
\item Časopis Malovaný kraj (od 1949)
\end{itemize}

\section{Archivní materiály}
\begin{itemize}
\item Moravský zemský archiv (svatební menu, účty)
\item Národopisné sbírky muzeí (Muzeum Hodonínska, Slovácké muzeum)
\item EU dokumentace k chráněnému označení Skalický trdelník
\end{itemize}

\chapter{Závěr}

Trdelník prošel složitým vývojem od středověkého měšťanského pečiva přes venkovskou specialitu až po moderní turistickou atrakci. Jeho historie odráží změny ve způsobech bydlení (černé kuchyně → sporáky) i společenské proměny (od obřadního pečiva k komerčnímu produktu). Přes spory o původ zůstává nedílnou součástí kulinárního dědictví střední Evropy.

  
\end{document}
