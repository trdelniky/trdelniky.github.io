\documentclass[a5paper,10pt]{book}
\usepackage[english,hungarian,czech]{babel}
\usepackage{luavlna}
\usepackage{trdelnik-book}
\usepackage{lipsum}
\title{Trdelník}
\begin{document}

\frontmatter
\maketitle
\ifdefined\HCode\else
\tableofcontents
\fi

\mainmatter
\chapter*{Úvod}

\begin{tcolorbox}[title=Upozornění]
   Stránky jsou ve výstavbě, většina textu je zatím neuspořádaná a neúplná. Zveřejňuji je především kvůli kapitole \nameref{Anotovanuxe1ux20chronologickuxe1ux20bibliografie} na straně \pageref{Anotovanuxe1ux20chronologickuxe1ux20bibliografie}.
\end{tcolorbox}

Trdelník je české a slovenské označení pro pečivo původně pečené na rožni. Takové pečivo existuje
v různých variantách v několika zemích střední Evropy, ale i jinde, nejméně od konce středověku. 

Existují tři základní typy: 

\begin{enumerate}
  \item pruh těsta namotaný na formu
  \item těsto rozválené do širokého plátu, poté opečené na formě
  \item řídké těsto, které se na formu lije v několika vrstvách, poté co se předešlé vrstvy zapečou
\end{enumerate}

Historicky nejstarší je první typ (z něj se vyvinulo, to, co dnes nazýváme jako trdelník
nebo \foreignlanguage{hungarian}{Kürtőskalács} v Rumunsku a Maďarsku). 

Podobné pečivo bylo známé už antice, ale reálně z
kuchařek je od 15. století známý jako Spiesskuchen, česky vaječník. Z našeho
území ho zmiňuje roku 1554 ve svém kázání Johannes Mathesius z Jáchymova. Pak se o něm
zmiňuje Komenský a nachází se v několika slovnících.

Druhý typ je o dost vzácnější, našel jsem o něm jen několik zmínek. Ale v
kuchařce J. C. Thiema z roku 1694 je na něj recept, označenej jako Bohmische
Küchlein.

Třetí typ je nejnovější, je to v podstatě dort a je známej z Německa jako
baumkuchen, z Rakouska prügelkrapfen, je rozšířený i v Polsku, Švédsku a
Litvě. Na Moravě byl rozšířený na Znojemsku. V Čechách byl populární koncem
19. století a v první půlce století 20. jako dort trdlovec. Nedělal se ale
doma, nabízely ho  cukrárny, také se prodával na poutích. Zajímavé je, už
tenkrát se prodával jako staročeský trdlovec a lidi z toho měli legraci. Dneska
je u nás už prakticky zapomenutý.

Termín trdelník se objevuje začátkem 19. století. Je tak pojmenovaný recept na
trdelník z litého těsta v překladu kuchařky od Sibilly Dorizio. Není úplně
jisté datum vydání, ale nejspíš je to rok 1816. Každopádně to není recept na
typ trdelníku, jak ho známe dneska.

V roce 1813 vyšla sbírka básní Muza Morawská od J. H. A. Gallaše. Trdelník se
vyskytuje ve třech básních, třeba v Nestřídmost v jídle jedné hanácké osoby.
Gallaš žil v Hranicích a popisoval prostředí Valašska a Haný.

Pak je celkem velká mezera, ale od 80. let 19. století je o trdelnících velký
množství zmínek, protože byly populární různý vlastenecký spolky, který
zkoumaly lidovou kulturu a trdelníky a náčiní pro jejich pečení zkoumali jako
historickou kuriozitu. V roce 1895 byla v Praze velká národopisná výstava a
před ní byly menší výstavy i v regionech. Trdelníky jako tradiční pečivo
představovali na Slovácku, Brněnsku, Haný nebo Třebíčsku.

Už v týhle době byl tradiční trdelník pečenej na ohni na ústupu, protože v
domácnostech se přecházelo z černý kuchyně (to byl v podstatě otevřenej oheň v
kuchyni) na sporáky s pecí a klasický trdelníky se v ní dělat nemohly. Vznikly
proto jiný variaty, který se daly píct v troubě, nebo se smažily v sádle.
Protože byly mnohem menší, daly se plnit krémem a vzniklo něco podobnýho
kremrolím. Respektive kremrole samotný možná vznikly původně z trdelníku.

Během první půlky 20. století je o trdelníku ještě spousta zmínek a vyskytuje
se ve spoustě kuchařek (jak litá, tak navíjená varianta), ale po 2. světový
válce už celkem mizí a vyskytuje se spíš v etnografických publikacích, ale i
kuchařkách. Třeba Lidové pečivo v Čechách a na Moravě z roku 1988 se trdelníkům
věnuje dost podrobně a popisuje recepty na všechny možný varianty. V našem
století se pak někde začaly obnovovat tradice a trdelníky se zase začaly
objevovat v některých lokalitách, kde bejval oblíbenej.

Na Slovensku pak v 80. letech začala skalická pobočka Západoslovenských pekární
vyrábět trdelníky a ty se staly celkem popuární, dokonce získal cenu Zlatý
kosák na výstave Agrokomplex. Pak se prodával po různých jarmarcích a podobně.

Pokud jde o historii trdelníku v Praze, tak podle tohohle článku se objevil
poprvý v roce 2000. Jeden chlapík si ho všiml ve Štúrově na jarmarku a zaujal
ho natolik, že ho začal vyrábět. Nejdřív taky na různejch jarmarcích a pak na
vánočních trzích na Staromáku. Stal se z toho hit, začali ho kopírovat další
prodejci, vznikaly různý varianty a tím jsme se dostali do dnešního stavu.
Takže pražskej trdelník původně vycházel ze Skalickýho trdelníku, ale používají
jinou recepturu na těsto, aby se líp připravovalo (a taky je levnější) a různý
náplně a posypky jsou veskrze místní inovace.

V Maďarsku je ten průběh šíření dost podobnej. Původně byl kürtőskalács
rozšířenej docela hojně, ale v průběhu 20. století mizel, až se zachoval jen v
Sedmihradsku v Rumunsku. Někdy v 70. nebo 80. letech, když se začalo povoloval
drobný podnikání, začal se šířit v různejch stáncích i jinam, postupně se
dostal i do Maďarska, kde se stal taky populárním.

Pokud jde o legendu o původu trdelníku ve Skalici, nejstarší zmínka o ní,
kterou jsem našel, je z roku 1998. I v dokumentu pro registraci chráněnýho
geografickýho označení se píše, že se to místně traduje. Vzhledem k tomu, že
byl u nás spiesskuchen známej stovky let před tím, než se měl údajně dostat do
Skalice spolu s kuchařem ze Sedmihradska, a že jen pár let potom ho popisuje
Galaš ze střední Moravy, není moc pravděpodobný, že by se u nás skutečně
rozšířil z Rumunska.

% další diskuzní příspěvky:

%%%%%%%%%%%%%%%%%%%%%%%%%%%%%%%%%%%%%%%%%%%%%%%%%%%%%%%%%%
% odpověď na https://www.youtube.com/shorts/7mUGpBYcYp8
%%%%%%%%%%%%%%%%%%%%%%%%%%%%%%%%%%%%%%%%%%%%%%%%%%%%%%%%%%

% You know what’s really sad? That most people who take part in debates about the history of trdelník haven’t bothered to find out whether trdelník actually is, or is not, an Old Czech dish.
% When vendors claim that it is, they do so because they made it up. Trdelník haters claim that it isn’t, but only because they know that the vendors made it up. The problem is that even though the vendors invented their claim, they are, strictly speaking, technically right.

% You can see this partly just by searching for the term trdelník in the Czech Digital Library, where there are hundreds of articles and books from the 19th and 20th centuries that mention it. But when you dig a bit deeper, you find that various forms of trdelník were widespread in Germany and across Central Europe from at least the mid-15th century.

% In the Czech lands, its presence is documented as early as 1554 in Jáchymov, where local women ate it at the market and drank beer with it. Comenius mentions it among baked goods in his works in the 16th century. A recipe for one variant of trdelník appears in German cookbooks from the late 17th and early 18th centuries under the name Böhmische Küchlein.

% So trdelník, in the sense of a cake baked on a spit, was indeed an Old Bohemian dish. How widespread it was up to the 19th century, I don’t know, nor do I know which specific forms of trdelník were common here. According to Josef Jungmann’s Czech-German dictionary from 1838, trdelník is „pečivo z mauky, protože na trdle, t. dřewěném rožni se pekau“. In other words, trdelník referred to any pastry baked on a spit.

% In fact, under the term trdelník I have found at least five different forms, incidentally, one of them is basically a kremrole. By the end of the 19th century, trdelník was already being described as an archaic dish, prepared only for special occasions such as weddings, carnival, or the birth of a child, and it was gradually disappearing.
% This was mainly because open hearths in rural houses were being replaced by stoves with ovens, in which the classic large trdelníky could no longer be made. After the Second World War, they were hardly prepared anywhere at all.

% More information about the spread of trdelník and recipes for its various forms can be found in the book Lidové pečivo v Čechách a na Moravě (1988), or in my bibliography at trdelniky [dot] github [dot] io, which contains excerpts from more than five hundred publications related to trdelník.

%  That said, I certainly don’t want to claim that I’m a fan of the way trdelník is sold in Prague today, nor do I claim that I consider it a “traditional” dish. This actually doesn’t interest me at all. What bothers me is when arguments are made that are not based on facts.

% Trdelník did not originate in Romania. According to Hungarian historians who have studied the history of kürtőskalács, it reached Transylvania from Germany or Austria sometime in the late 17th or early 18th century, so more than hundred years after it was already known in the Czech lands. Also, there is no evidence that it spread from there to Skalica and then on to Moravia and Slovakia. The fact that articles and videos on Czech Television, the BBC, or Deutsche Welle present this as fact is genuinely sad. If they had at least bothered to look at the PGI registration for Skalický trdelník, they would have read that this is merely an orally transmitted tradition. And if they had taken a look at the Czech or Slovak digital libraries, they would have found that no book or article about trdelník and Skalica from the 19th century or from most of the 20th century contains any legend claiming that trdelník came from Transylvania.

% And I think that if people in Skalica had had any idea that Hungarian nationalists would today use this story to claim that Skalický trdelník is merely an imitation of kürtőskalács, they would never have cited that legend in the first place. In any case, treating a legend that is about as credible as stories of White Ladies haunting Czech castles as evidence in articles that claim to be “setting the record straight” is rather absurd.

% If someone doesn’t like trdelník and wants to discourage tourists from buying it, I think it’s perfectly fine to say that it’s overpriced, not very tasty, and not made from high-quality ingredients. There’s really no need to argue using a fabricated history and supposed non-traditinality.

%%%%%%%%%%%%%%%%%%%%%%%%%%%%%%%%%%%%%%%%%%%%%%%%%%%
%  https://www.reddit.com/r/czech/comments/1plo0ao/jak_je_to_teda_s_t%C3%ADm_trdeln%C3%ADkem/
%%%%%%%%%%%%%%%%%%%%%%%%%%%%%%%%%%%%%%%%%%%%%%%%%%%
% Trdelník guy here. Mojí bibliografii o historii trdelníku tu už pár lidí posílalo. Aktuálně jsou tam výpisky z o něco víc, než 500 publikací, tak jen trošku shrnu, co jsem zatím zjistil.

% Pokud jde o to, jak moc je dneska tradiční na Moravě, tak odpověď na to je celkem komplikovaná.

% Na konci 19. století byl rozšířenej na velkým území Moravy. V tý době se totiž šířil zájem o národopis a lidový tradice a dělaly se různý výstavy tradičního života, v Praze pak byla v roce 1895 Národopisná výstava Českoslovanská. Díky popisům těch výstav víme, že trdelníky byly rozšířený od Dačic přes Znojemsko, Brněnsko na Hanou, Valašsko, na jihu pak na Slovácku. Zároveň ale asi byly už v tý době spíš na ústupu, protože třeba v časopise Vlasteneckého spolku musejního v Olomouci, kde popisujou výstavku v Němčicích nad Hanou v roce 1887, zmiňujou trdelníky jako starodávné pečivo, jim neznámé. V tý době se taky trdelníky objevujou v různých povídkách a románech z lidovýho prostředí, byly taky mezi cenama na různých dobročinných večírcích a podobně.

% Proč se trdelníky moc neobjevujou dřív? Hlavně asi proto, že se tolik nepsalo o lidovým prostředí. V pozdější literatuře se objevujou zmínky i z průběhu 19. století, založený na rukopisných pamětech, třeba s Ivančic nebo Černé hory. Nejstarší výskyt slova trdelník jsem našel v Muze Morawské z roku 1813, jejíž autor sbíral různý příběhy z okolí Hranic a na jejich základě psal mravokárný básničky.

% V průběhu 20. století trdelník mizel z kuchyní, ale zase se o něj zajímali etnografové, od 50. let je otázka na něj součástí dotazníků České národopisné společnosti, konkrétně dotazník na Chléb a pečivo. Výsledky nejsou digitalizovaný, ale vycházeli z nich autoři knihy Lidové pečivo v Čechách a na Moravě , kde je o trdelnících spousta informací a recepty na jejich varianty.

% Trdelník totiž není jen ten typ, kterej známe dneska z trhů, je jich minimálně pět - klasickej z hada těsta omotanýho kolem formy, pak plát těsta širokej stejně jako forma, další varianta je z litýho těsta (v Německu známá jako Baumkuchen) a pak existovaly dvě menší varianty, který se rozšířily pozdějc. Jedna varianta je v podstatě kremrole (tu dneska propaguje spolek Tetičky z Kobylí, druhá byla smažená v oleji nebo sádle. Důvod, proč se tyhle dvě varianty rozšířily byl ten, že z venkova mizela černá kuchyně. To bylo v podstatě otevřený ohniště uvnitř chalup, na kterým se peklo na rožni nebo vařilo ve speciálním nádobí. Černou kuchyni v průběhu 19. století nahradily sporáky, který byly jednodušší na ovládání a hlavně líp odváděly splodiny. Součástí sporáků byly trouby, do kterejch se původní formy na trdelníky nevešly, takže se místo toho začaly dělat ty menší varianty a ty původní velký, dělaný na ohni skoro všude zanikly.

% Jinak velký trdelníky se dělaly hlavně na různý slavnostní příležitosti. Svatby, masopust, jako dárek pro matku po narození dítěte a podobně. Ani v tom 19. století to nebylo něco, co by se dělalo nějak pravidelně. Hlavně proto, že byly z docela nákladnýho těsta a příprava byla dost náročná. Určitě to nebylo něco, co by se peklo jen tak, že byla zrovna chuť.

% Z hlediska tvojí otázky je taky zajímavý, že trdelník pronikl i do USA, našel jsem několik dopisů od krajanů, kde popisujou, jak pečou trdelník. Třeba tenhle, kde popisuje jak dělat trdelník naplněnej krémem s křenem a paprikou pro zlobivý manžely. Ale nemyslím si, že by se v krajanských komunitách udržel až do dneška.

% Po pádu komunismu pak na jedný straně vznikl zájem o obnovu lokálních tradic, takže někde se skutečně mohl trdelník znovu objevit na jejich základě (třeba v tom Kobylí). Zároveň ale se trdelník začal šířit na různejch jarmarcích, vánočních trzích a podobně. Tam se to podle mě šířilo primárně nápodobou. Třeba do Prahy to v roce 2000 přinesl jeden týpek, kterej viděl trdelník na jarmarku na Slovensku. Začal objíždět jarmarky a vánoční trhy a od něj to pak okopírovali další prodejci v Praze. Předpokládám, že i na Moravě to primárně probíhalo tímhle způsobem.

% Každopádně dneska je trdelník na různejch slavnostech rozšířenej víceméně všude. Hodně jezdím na vejlety po Čechách a stánky s trdelníkem jsem potkal na takovejch místech jako jsou Velvary nebo Prachovský skály. A rozhodně se nedalo říct, že by neměly úspech, byly tam fronty. Taky sleduju nový zmínky o trdelníku na Googlu a můžu říct, že každej tejden vychází několik článků nebo videí s receptem (aktuálně třeba tohle), takže v podstatě v současný český kuchyni je trdelník živej pokrm. Do jaký míry je to obnova reálný historický tradice je samozřejmě věc pro diskuzi.

% Pro mě osobně je na trdelníku nejzajímavější spíš ten příběh, kterej se kolem něj šíří. To, že ho některý prodejci označujou za tradiční a staročeskej, spustilo reakci a vznikl zvláštní literární útvar, historie trdelníku. A co mě na něm fascinuje je to, že ti co ho šířej se nikdy nenamáhali udělat nějakej reálnej výzkum v historickejch zdrojích. Když prodejci o trdelníku tvrděj, že je tradiční a staročeskej, tak si samozřejmě nezjistili, jestli to je pravda, je to z jejich strany jen marketingovej trik. To, že jsou tyhle tvrzení do jistý míry pravda, je z jejich strany čistě náhodný.

% Víc fascinující je ale druhá strana, která na to reaguje "pravdou" o historii trdelníku. Je spousta článků v češtině i angličtině, který šířej příběh, kterej je místo na faktech založenej na legendách ze zemí, kde varianty trdelníku taky znají a kde se začal komerčně šířit o pár let dřív, než u nás. Čili primárně Slovensko a Maďarsko.

% Ten příspěvek od Slavic Bros vyvolal článek BBC, kterej úplně ignoruje historii trdelníku na Moravě. Pak máme třeba video od DW nebo Český televize. A zvlášť to druhý video je fakt vtipný. Začátek je totiž po faktický stránce v pořádku. Trdelník byl totiž skutečně dost rozšířenej v Německu od 15. století a z 16. století je známý kázání z Jáchymova, kde si místní kněz stěžuje, že ženy rády pijou pivo a jedí trdelník, místo aby se věnovaly domácnosti a manželům. Pak ale tvrdí, že z Českých zemí mizí a až v 18. století se šíří do slovenskýho města Skalice z Transylvánie, odkud ho přinesl kuchař hraběte Gvadániho. A na konci 19. století se pak rozšířil i do přilehlý oblasti Moravy.

% Existuje spousta dalších článků, kde tu starší část historie trdelníku úplně ignorujou a rovnou tvrděj, že vznikl v Transylvánii (citujou třeba legendu o tom, že vznikl, když se místní obyvatelé schovávali v jeskyních před nájezdníky). Což je úplná blbost, v tý mojí bibliografii mám několik receptů z německejch a rakouskejch kuchařek z 16. až 18. století, který popisujou víceméně stejnou variantu jako se pak objevila v nejstarším receptu z Transylvánie z roku 1784. I maďarský autoři věnující se historii Kürtőskalácse uznávají, že se do Transylvánie dostal prostřednictvím Sasů nebo Rakušanů.

% Pro to, že by se trdelník dostal do Skalice prostřednictvím hraběte Gvadániho a odtud se šířil po Slovensku a Maďarsku taky neexistujou žádný důkazy. Třeba v PGI registraci pro Skalický trdelník se píše: "ústnom podaní sa hovorí, že výrobný recept na trdelníky, ktorý sa rozšíril po Skalici a jej okolí, mal kuchár grófa, básnika a spisovateľa Jozefa Gvadániho, ktorý žil v Skalici v období rokov 1783 – 1801". To je celej důkaz, na jehož základě desítky článků a videí tuhle informaci předkládají jako fakt. V tý mojí bibliografii jsou desítky zdrojů, který se věnujou Skalickýmu trdelníku nebo historii trdelníku obecně a žádnej z nich se o tomhle ústním podání nezmiňuje až do roku 1998. Přitom třeba Marie Úlehlová-Tilschová ve Skalici žila několik let a o trdelníku toho napsala hodně. A k historii napsala jen, že je to starodávný pečivo, který do její doby přežilo jen na Slovácku a v okolí Skalice. V Československé vlastivědě se zase píše že trdelníky na Slovensko přišly z Moravy. Legenda o Gvadánim se objevuje až v době, kdy se v Maďarsku šíří Kürtőskalács, takže nejpravděpodobnější je, že si jeho existence všiml někdo ze Skalice a spojil si ho s reálně existujícím Gvadánim, kterej skutečně pocházel z Transylvánie a na základě toho ta legenda vznikla.

% I kdyby ta legenda byla pravdivá, tak rozhodně se nedá říct, že se trdelník rozšířil ze Skalice a že na Moravě a Slovensku se rozšířil až koncem 19. století. Jak už jsem psal, v 16. století byl v Jáchymově a v bibliografii mám i další zmínky z českých zemí do konce18. století, kdy se ještě nejmenoval trdelník, ale vaječník. Na začátku 19. století se objevuje název trdelník na Hané. Jaký měl skutečný rozšíření nevíme, protože obecně není z tý doby moc zdrojů o lidový kuchyni, spíš se řeší šlechtická a v průběhu 19. století i měšťanská. Na konci 19. století už trdelník spíš mizí v souvislosti s tím, že mizí černá kuchyně, kde se mohl připravovat.






\lipsum[1-12]

Trdelník je český a slovenský označení pro koláč původně pečenej na rožni.
Existujou tři základní typy: pruh těsta namotanej na válec, pak rozválený těsto
namotaný na válec jako jeden velkej plát a nakonec řídký těsto, který se na
válec lije v několika vrstvách.

Historicky nejstarší je první typ (taky to je to, co je známý jako trdelník
nebo kurtoskalács dneska). Podobný pečivo bylo známý už antice, ale reálně z
kuchařek je od 15. století známý jako spiesskuchen, česky vaječník. Z našeho
území je známej z roku 1554 v kázání Johana Mathesia z Jáchymova. Pak se o něm
zmiňuje Komenský a nachází se v několika slovnících.

Druhej typ je o dost vzácnější, našel jsem o něm jen několik zmínek. Ale v
kuchařce J. C. Thiema z roku 1694 je na něj recept, označenej jako Bohmische
Küchlein.

Třetí typ je nejnovější, je to v podstatě dort a je známej z Německa jako
baumkuchen, z Rakouska prügelkrapfen, je rozšířenej i v Polsku, Švédsku a
Litvě. Na Moravě byl rozšířenej na Znojemsku. V Čechách byl populární koncem
19. století a v první půlce století 20. jako dort trdlovec. Nedělal se ale
doma, nabízely ho různý cukrárny a taky se prodával na poutích. Vtipný je, už
tenkrát se prodával jako staročeský trdlovec a lidi z toho měli legraci. Dneska
je u nás už úplně zapomenutej.

Termín trdelník se objevuje začátkem 19. století. Je tak pojmenovanej recept na
trdelník z litýho těsta v překladu kuchařky od Sibilly Dorizio. Není úplně
jistý datum vydání, ale nejspíš je to rok 1816. Každopádně to není recept na
typ trdelníku, jak ho známe dneska.

V roce 1813 vyšla sbírka básní Muza Morawská od J. H. A. Gallaše. Trdelník se
vyskytuje ve třech básních, třeba v Nestřídmost v jídle jedné hanácké osoby.
Gallaš žil v Hranicích a popisoval prostředí Valašska a Haný.

Pak je celkem velká mezera, ale od 80. let 19. století je o trdelnících velký
množství zmínek, protože byly populární různý vlastenecký spolky, který
zkoumaly lidovou kulturu a trdelníky a náčiní pro jejich pečení zkoumali jako
historickou kuriozitu. V roce 1895 byla v Praze velká národopisná výstava a
před ní byly menší výstavy i v regionech. Trdelníky jako tradiční pečivo
představovali na Slovácku, Brněnsku, Haný nebo Třebíčsku.

Už v týhle době byl tradiční trdelník pečenej na ohni na ústupu, protože v
domácnostech se přecházelo z černý kuchyně (to byl v podstatě otevřenej oheň v
kuchyni) na sporáky s pecí a klasický trdelníky se v ní dělat nemohly. Vznikly
proto jiný variaty, který se daly píct v troubě, nebo se smažily v sádle.
Protože byly mnohem menší, daly se plnit krémem a vzniklo něco podobnýho
kremrolím. Respektive kremrole samotný možná vznikly původně z trdelníku.

Během první půlky 20. století je o trdelníku ještě spousta zmínek a vyskytuje
se ve spoustě kuchařek (jak litá, tak navíjená varianta), ale po 2. světový
válce už celkem mizí a vyskytuje se spíš v etnografických publikacích, ale i
kuchařkách. Třeba Lidové pečivo v Čechách a na Moravě z roku 1988 se trdelníkům
věnuje dost podrobně a popisuje recepty na všechny možný varianty. V našem
století se pak někde začaly obnovovat tradice a trdelníky se zase začaly
objevovat v některých lokalitách, kde bejval oblíbenej.

Na Slovensku pak v 80. letech začala skalická pobočka Západoslovenských pekární
vyrábět trdelníky a ty se staly celkem popuární, dokonce získal cenu Zlatý
kosák na výstave Agrokomplex. Pak se prodával po různých jarmarcích a podobně.

Pokud jde o historii trdelníku v Praze, tak podle tohohle článku se objevil
poprvý v roce 2000. Jeden chlapík si ho všiml ve Štúrově na jarmarku a zaujal
ho natolik, že ho začal vyrábět. Nejdřív taky na různejch jarmarcích a pak na
vánočních trzích na Staromáku. Stal se z toho hit, začali ho kopírovat další
prodejci, vznikaly různý varianty a tím jsme se dostali do dnešního stavu.
Takže pražskej trdelník původně vycházel ze Skalickýho trdelníku, ale používají
jinou recepturu na těsto, aby se líp připravovalo (a taky je levnější) a různý
náplně a posypky jsou veskrze místní inovace.

V Maďarsku je ten průběh šíření dost podobnej. Původně byl kürtőskalács
rozšířenej docela hojně, ale v průběhu 20. století mizel, až se zachoval jen v
Sedmihradsku v Rumunsku. Někdy v 70. nebo 80. letech, když se začalo povoloval
drobný podnikání, začal se šířit v různejch stáncích i jinam, postupně se
dostal i do Maďarska, kde se stal taky populárním.

Pokud jde o legendu o původu trdelníku ve Skalici, nejstarší zmínka o ní,
kterou jsem našel, je z roku 1998. I v dokumentu pro registraci chráněnýho
geografickýho označení se píše, že se to místně traduje. Vzhledem k tomu, že
byl u nás spiesskuchen známej stovky let před tím, než se měl údajně dostat do
Skalice spolu s kuchařem ze Sedmihradska, a že jen pár let potom ho popisuje
Galaš ze střední Moravy, není moc pravděpodobný, že by se u nás skutečně
rozšířil z Rumunska.

\chapter{Typy trdelníku}
Trdelník se v průběhu historie vyvinul do několika odlišných typů:

\section{Podle způsobu přípravy}
\begin{itemize}
\item \textbf{Pečené na ohni} - Tradiční způsob přípravy na dřevěném válci (trdlu) nad otevřeným ohněm. Typické pro jižní Moravu a Slovensko (Skalický trdelník).
\item \textbf{Smažené} - Varianta rozšířená na Moravě, kde se těsto smažilo v sádle na plechových formách.
\item \textbf{Pečené v troubě} - Pozdější varianta, která se objevila s rozšířením sporáků.

\subsection{Pečené v troubě}

\lipsum[1-3]

\subsection{Smažené}

\lipsum[4-6]

\end{itemize}

\section{Podle typu těsta}
\begin{itemize}
\item \textbf{Z nekynutého těsta} - Nejstarší varianta, doložená již v 16. století.
\item \textbf{Z kynutého těsta} - Rozšířená od 18. století, často s přidáním másla a vajec.
\item \textbf{Litý trdelník (trdlovec)} - Baumkuchen připravovaný litím těsta na rožeň.
\end{itemize}

\section{Regionální varianty}
\begin{itemize}
\item \textbf{Skalický trdelník} - Chráněné zeměpisné označení EU, pečený na dřevěném válci.
\item \textbf{Moravské trdelníky} - Často smažené varianty z kynutého těsta.
\item \textbf{Valašský trdelník} - Údajně přinesen Valachy z Rumunska v 17. století.
\item \textbf{Kürtőskalács} - Maďarská/transylvánská varianta s cukrovou polevou.
\end{itemize}

\chapter{Historický vývoj přípravy}
\section{Rané zmínky (14.-18. století)}
\begin{itemize}
\item První zmínky v českých pramenech: Klaret (1360), Mathesiova kázání (1554), Thiema kuchařka (1694)
\item Německé názvy: Spießkuchen, Prügelkrapfen, Ringelkrapfen
\subsection{Zkouším subsection}
\item Latinský název: obolum
\subsection{Další subsection}
\item Příprava na rožni v měšťanských kuchyních
\end{itemize}

\section{19. století}
\begin{itemize}
\item Rozšíření na venkově, zejména na Moravě
\item Pečení v černých kuchyních na otevřeném ohništi
\item Použití při svátcích (svatby, křtiny, masopust)
\item Objevují se první tištěné recepty (Dorizio 1816, Rettigová 1826)
\end{itemize}

\section{20. století}
\begin{itemize}
\item Úpadek tradiční přípravy s mizením černých kuchyní
\item Průmyslová výroba na Slovensku (Piešťany 1984)
\item Obnova tradice ve Skalici
\end{itemize}

\section{Současnost}
\begin{itemize}
\item Turistická atrakce od 2000 let
\item Spory o původ (český vs. maďarský)
\item Moderní variace s náplněmi
\end{itemize}

\chapter{Zdroje receptů}
Historické recepty lze nalézt v těchto pramenech:

\section{Tištěné kuchařky}
\begin{itemize}
\item Sibylla Dorizio: "Moje skrz čtyřicetileté vykonávání známá Kuchařská kniha" (1816)
\item Magdalena Dobromila Rettigová: "Domácí kuchařka" (1826)
\item Marie Úlehlová-Tilschová: "Česká strava lidová" (1945)
\end{itemize}

\section{Národopisné prameny}
\begin{itemize}
\item Augusta Šebestová: "Lidské dokumenty" (1900)
\item František Bartoš: Dialektický slovník moravský (1906)
\item Časopis Malovaný kraj (od 1949)
\end{itemize}

\section{Archivní materiály}
\begin{itemize}
\item Moravský zemský archiv (svatební menu, účty)
\item Národopisné sbírky muzeí (Muzeum Hodonínska, Slovácké muzeum)
\item EU dokumentace k chráněnému označení Skalický trdelník
\end{itemize}

\chapter{Závěr}

Trdelník prošel složitým vývojem od středověkého měšťanského pečiva přes venkovskou specialitu až po moderní turistickou atrakci. Jeho historie odráží změny ve způsobech bydlení (černé kuchyně → sporáky) i společenské proměny (od obřadního pečiva k komerčnímu produktu). Přes spory o původ zůstává nedílnou součástí kulinárního dědictví střední Evropy.

  
\end{document}
