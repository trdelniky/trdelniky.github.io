\subsection{Recept na Spiesskuchen z knihy Balthasara
Staindla}\label{250922-1759}

Zdroj: Staindl, Balthasar. Ain künstlichs und nutzlichs Kochbuch.
Augspurg : Otmar 1547. Dostupné z:
\url{https://www.digitale-sammlungen.de/view/bsb00023833?page=72\%2C73}

Přepis v němčině:

\subsubsection{Ain höflich essen / haisst der
raiff.}\label{250922-1759}

Gib für ain richt / oder ain bachens / nym ain halb meßlin guͤt semel mel
/ nymis halb herdann / in ain schuͤssel / die tüpfstern oder zinsen sey /
nym mel / das erwerm in ainer stuben / nym ain maßlin lüssen raum / laß
im warm werden / das du kain finger darinn haben magst / nym ain löffel
vol garben / under den lüssen raum / nym zway ayer / schlag auch den den
raum / raum / rües als ab / ses auff der schuͤssel zu der würm / so geer
auffser muß auffgehn / bey ainer viertel stund / das er sein wach wir /
vnd sich glat blatern am rüern / nym dann bey ain halben vierel weinber
/ schön erlaubet / die gar trucki sey / dies rür in den taig / nym auch
ain halb lot muscatblüt / brock es klain / rürs auch in taig / so er
balo abgedöts ist / es sey ain lust / so setz in wider zu der würm / in
der schuͤssel / das er auffgehe / wie vor / bey einer viertel stund / so
nym den spiß / der muß darzu gemachet seyn / salb jn mit schmalcz ain
wenig / doch das er nit naß sey / nym dann den taig / leg jn fein an den
spiß vnd vmb jn gleicher dick ab / das er fein glat werd / vnd halßm /
dann so nymb key aier dotter / die salz ain wenig / der streich den taig
am spiß fleißig vmb vnd vmb / das er fein gel werd / so er nun
besprichen ist / so nym ain groben zacker / ain zwingen / den bind vmb
den taig / in massen wie ain raiff / schaw eben das der faden / so len
sey / auff den taig lig / vnd gar nit

Einbeißen/bertaig gienge fonf mit vom folß/wann er unbwur- ben ist mit
dem spieß / so laß in zu einem bumenden feuer/ nur gar fluchs vmb vmb
dretzen/bis er erwarmbt/dann nym ain schmalz/das laß zergren in aim
pfännlein/das du wol ain finger darinn haben magst / Thue dann ain
kleines würfel/ ainer spannen lang/und zwayer finger breit/ die wind in
ain knöpfel / ebäsin das schmalz / streich den braten/ wie ain span
{[}?{]} / dann so brat es mer flugs vmb/ so wirt er seimen/ so streich
in mer wie vor / vnd brat in fast / bis er sich fein beine/ so thue in
zum dritten mal/ und brat in offt / so lang bis er fein lieblich{[}?{]}
wirt/ So nym in vom feuer/ und wend den spieß umb / diewell zeuch den
spen{[}?{]} / und ejl {[}?{]} bamic{[}?{]} / auff ain sauber weiß tüch/
nym ain messer / löse den braten von baiden seiten ab vom ort/ dann so
nym ain tüch in warmes/ zeuch in sein gemächlich herab / so geet er sein
herab vom spieß/ deck in sein zu/ thue ain tüchlein in warmes öl/ das
die warm nie her- auf gee/ so zeucht es sich dipfig an/ und wirt sein
trucken/ inn- wendig/ und ist also bereit/ den taig mit man solte{[}?{]}
/ wann man ain ersten anmachet / Auch so du den taig hast angemachet/ so
brat ein tüch auff den tüch/ und wälzer den spieß hin und her/ so kompt
der taig gleich an den spieß.

\subsection{Překlad pomocí ChatGPT a
Deepseeku:}\label{250922-1759}

\subsubsection{Urozené jídlo, které se nazývá „der
Raiff``.}\label{250922-1759}

Vezmi na jednu porci (nebo na jedno pečení) půl měřice dobré pšeničné
mouky, polovinu dej do mísy (která má být čistá). Zahřej mouku v teplé
místnosti, vezmi malé množství kvasu, nech ho v teple vzejít, dokud není
tak horký, že do něj sotva můžeš strčit prst. Vezmi lžíci kvasu a dej ji
k ohřáté mouce. Vezmi dvě vejce, rozšlehej je a vlij k mouce. Zamíchej,
dokud se těsto nespojí, dej mísu na teplé místo, aby těsto nakynulo --
asi čtvrthodinu -- dokud se nevytvoří pěkné bubliny. Potom vezmi asi půl
čtvrtky rozinek (dobře opraných a usušených) a vmíchej je do těsta.
Vezmi také půl lotu muškátového květu, roztluč jej nadrobno a zamíchej
do těsta. Když je těsto dobře vymícháno, vrať je znovu na teplé místo,
aby znovu vykynulo -- zase asi čtvrthodinu. Potom vezmi špíz, který je k
tomu určen, potřísni jej trochou sádla, ale tak, aby nebyl mokrý. Vezmi
těsto a hezky je naviň na špíz v rovnoměrné tloušťce, aby bylo pěkně
hladké, a dobře je uhlaď. Potom vezmi vaječný žloutek, osol jej trochu a
těsto na špízu jím pečlivě potírej dokola, aby pěkně zežloutlo. Když je
takto potřeno, vezmi hrubý cukr a obvaž jej kolem těsta (nebo důkladně
posyp tak, aby vytvořil jakýsi prstenec). Dbej na to, aby cukr ležel
pěkně na těstě a nespadl.

Pokračuje překlad deepseekem, nepřijde mi tak kvalitní:

(Když je) obalené v těstě a nasunuté na rožeň, když je napíchnuté na
rožni, tak je dej k planoucímu ohni. Velmi rychle s ním otáčej dokola,
dokud se neohřeje. Pak vem sádlo, které nech rozškvařit v malé pánvičce,
do které můžeš strčit prst.

Pak udělej malou kostičku {[}možná z těsta?{]}, dlouhou na píď a širokou
na dva prsty. Tu zviň do kuličky. Když je sádlo rozpuštěné, potři jím
pečeni. Pak ji peč rychleji dokola, aby se zatáhla. Pak ji potírej více
jako předtím a peč ji prudce, dokud hezky neztvrdne. Pak to udělej
potřetí a peč často, tak dlouho, dokud nebude hezky vonět.

Pak ho {[}bratek{]} vezmi z ohně a otoč rožeň. Mezitím sundej ... a ...
na čisté bílé plátno. Vem nůž a uvolni pečeni z obou stran od konce. Pak
vem plátno do teplého ... a stáhni ji zvolna dolů, takže sjede z rožně
dolů. Přikryj ji. Dej malé plátno do teplého oleje, aby teplo nestoupalo
nahoru. Tak se hezky obalí a uvnitř zůstane suché. A je tak připravena.

Těsto se má... když se zadělává poprvé. Také, když jsi těsto zadělal,
tak... plátno na plátno a válíme rožeň sem a tam, tak se těsto přilepí
na rožeň.

\subsection{Poznámky}\label{250922-1759}

„Raiff`` = prstenec/kruh
