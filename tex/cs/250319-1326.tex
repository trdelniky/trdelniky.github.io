\subsection{Seklerský
kürtőskalács}\label{250319-1326}

Autor: Pozsony Ferenc Pozsony, Ferenc. A székely kürtőskalács. Acta
Siculica, 2014--2015, s. 497-518. ISSN
\url{https://epa.oszk.hu/03300/03308/00007/pdf/EPA03308_acta_siculica_2014-2015_497_518.pdf}

V naší práci stručně nastíníme název, původ, maďarské a erdeljské
rozšíření, šíření, evropské paralely, nově vzniklé funkce a významy
seklerské sladkosti připravované nad žhavým uhlím otevřeného ohniště.
Poté postupně ukážeme, jakou roli hraje tento tradiční druh koláče
připravovaný starobylými technikami v reprezentaci lokální, regionální,
etnické a národní identity a v konstrukci evropského kulturního
dědictví?

\subsubsection{Název}\label{250319-1326}

V maďarštině se koláč připravovaný na otevřeném ohništi nad hořícím
uhlím, navinutý na válec z kynutých těstových proužků a poté upečený do
křupava a s cukrovou polevou, nazývá kürtőskalács. Přestože se podle
historických pramenů dříve objevoval v severním Zadunají, Horní zemi, na
východním okraji Velké uherské nížiny, v Partiu, Sedmihradsku, Gyimesi,
Bukovině a Moldávii, dnes je v povědomí především jako seklerský
produkt, který se připravuje na slavnostní stůl, při křtinách a svatbách
a při přijímání významnějších hostů.

V Sedmihradsku se používají dva základní názvy: podle našich lingvistů
pochází varianta s krátkým ö a dlouhým ő, kürtős- a kürtőskalács, ze
slov kürt a kürtő. Gyula Csefkó odvozuje předponu názvu koláče spíše ze
slova kürt (hudební nástroj), protože ženy navíjely jeho prameny na
válec ve tvaru komolého kužele stejným způsobem, jakým mladí Sekleři
kdysi na jaře vyráběli své trubky z spirálovitě oloupané kůry vrbových
nebo vrbových proužků.

T. Attila Szabó vyjádřil zcela odlišný názor: „...když se sundá z válce,
celý koláč tvoří jeden kus přibližně 25-30 cm dlouhého koláče ve tvaru
trubky nebo trubky. Protože se tento koláč ve tvaru trubky podává členům
rodiny a hostům, a spotřebitelé vidí páskovitě se odlamující těsto v
tomto charakteristickém tvaru, je zřejmé, že název mohl vzniknout pouze
z trubkovitého tvaru těsta. Trubka je na vesnici i ve městě běžná,
všeobecně známá věc, trubka už méně.`` Lingvista z Kluže jednoznačně
doporučil používání tvaru kürtőskalács. Variantou T. Attily Szabóa
posiluje i výraz schornstein-kolatsch používaný sedmihradskými Sasy,
který je prostým zrcadlovým překladem maďarského kürtőskalács.

\subsubsection{Šíření v maďarsky mluvící
oblasti}\label{250319-1326}

Anna Bornemisza, manželka erdeljského knížete Michala Apafiho, nechala v
roce 1680 přeložit Jánosem Keszeiem do maďarštiny kuchařku Marxe
Rumpolta Ein neue Kochbuch, vytištěnou ve Frankfurtu nad Mohanem v roce
1581, ve které se Spiesskuchen, populární v německých oblastech,
doporučovaný jako třetí chod, nazývá botfánk. Jeho erdeljská maďarská
verze se objevila již na konci 17. století v soupisu z Uzdiszentpéteru z
roku 1679: Dřevo na pečení kürtős Fánk... Maďarský název koláče, kürtő
kalács, se objevil až o několik desetiletí později, v roce 1723, v
písemných pramenech Seklerska, konkrétně Háromszéku. Ve většině
sedmihradských sídel se nazýval především kürtőskalács, ale méně často
se vyskytovaly i výrazy kürtőspánkó a kürtősfánk (1679).

Nejstarší písemná zmínka o kürtőskalács pochází z první třetiny 18.
století, kdy hraběnka Ferattiné, rozená Ágnes Kálnoki, napsala ve svém
dopise z 22. prosince 1723 z moldavského knížecího dvora v Jászvásáru
následující slova své tetě z erdeljské Torje, Borbále Kálnoki, manželce
Petera Apora: „Ctihodná kněžna si žádá... abyste vyslali sluhu, kterého
byste laskavě nechali naučit všem druhům pečení, mimo jiné i kürtő
kalács... Ctihodná kněžna si žádá... o kterém sluhovi jste laskavě
slíbili, že ho necháte naučit... Drahá teto... nelitujte laskavosti,
nechte ho naučit pečení chleba a dalším jemným pečením a kürtő kalács a
nějakému paštikovi a jemnému jídlu.`` Tento údaj také naznačuje, že mezi
erdeljskou a moldavskou elitou v té době panovaly úzké vztahy, a díky
nim se módnější prvky maďarské gastronomické kultury pravidelně šířily i
do Moldávie ležící východně od Karpat v 18. století.

Pokud by tento koláč byl v Sedmihradsku a jeho východní části,
Seklersku, populární již dříve, členové moldavské elity by se s ním
mohli seznámit a převzít ho již dříve. Je pozoruhodné, že Peter Apor ve
svém díle Metamorphosis Transylvaniae, dokončeném v roce 1736, nezmiňuje
kürtőskalács mezi starobylými, maďarskými a erdeljskými jídly, ačkoli
výše uvedený úryvek z dopisu dokazuje, že se již připravoval v kuchyni
jeho manželky. Přestože konzervativní šlechtic z Torje mohl kürtő dobře
znát, zjevně ho považoval za novou módu zprostředkovanou Rakušany, a
proto ho ve svých pamětech ani nezaznamenal, přestože byl tento druh
koláče mezi erdeljskou elitou již tehdy poměrně oblíbený. Zdůrazňujeme,
že Peter Apor si stěžoval na zmizení dřívější erdeljské stravovací
kultury: „Jídla, na která byli zvyklí naši otcové, nemůžeme jíst, pokud
nemáme německého kuchaře.``

Zdůrazňujeme, že nejstarší forma kürtőskalács se do maďarských komunit
dostala buď prostřednictvím Rakušanů usazených v Sedmihradsku, nebo
Sasů, kteří měli dobré vztahy s vnitřními německými oblastmi. Je
zajímavé, že vydání kuchařky Szakács mesterségnek könyvecskéje z Kluže z
roku 1771 se o kürtőskalács, tehdy již velmi oblíbeném koláči, vůbec
nezmiňuje. Tato skutečnost může naznačovat, že tento koláč v té době
ještě nebyl organickou součástí sváteční stravy rodin střední a nižší
třídy.

Spojená forma slova kürtőskalács se poprvé objevuje v kuchařce Dániel
Istvánné Gróf Mikes Mária A Gazda Aszszonyi Böltseségnek Tárháza,
napsané v roce 1781 a vydané v roce 1784, a v ní se nachází i první
maďarský recept na tento koláč: „Těsto nenech tvrdé, protože pak nebude
vláknité uvnitř.`` V případě „kürtőskalács podle Porániné`` se
doporučovalo, aby se těsto po rozválení a nakrájení navinulo na máslem
vymazané dřevo a peklo se potírané máslem.

Maďarské názvy koláče (dorongos fánk, dorongfánk, botra tekercs) jsou v
podstatě zrcadlovými překlady německého Baumkuchen. První písemná zmínka
o koláči pečen Kürtőskalács, pečený na rožni, se poprvé objevuje v textu
komedie z roku 1789, jejímž autorem byl rodák z Komárna. Kuchařka
Kristófa Simaie „Způsob přípravy některých jídel`` z roku 1795 z
Kremnice také nazývá „Dorongos fánk`` koláč z kynutého těsta s
rozinkami, potřený třtinovým medem, a zveřejňuje podrobnější recept.
„Vezmi dva verdugy jemné bílé mouky, mléko spařené, ale vlažné, dva nebo
tři žloutky, dvě kaly pivovarských kvasnic, ty rozpusť v mléce a
rozmíchej, a nalij do mouky, vezmi verdug dobrého másla, rozehřej na
vlažno a nalij i to do mouky, a s malými rozinkami rozmíchej a udělej
těsto, aby se dalo rozválet, rožeň pomaž máslem, rozválej tenké dlouhé
těsto a oviň ho kolem, konce přilep k rožni, a jako pečeni otáčej nad
plápolajícím ohněm, když se opeče, stáhni z rožně a pomaž třtinovým
medem.`` Simaiova kuchařka obsahovala také významnou novinku,
doporučovala kuchařům, aby koláč po upečení ochutili tehdy typickým
sladidlem, tekutým cukrem zvaným třtinový med. Poznamenejme, že kynutí
těsta se v té době provádělo především pivovarskými kvasnicemi. Moderní
kvasnice byly vynalezeny až v roce 1848, zatímco průmyslová výroba
krystalového cukru začala až na konci 19. století.

T. Attila Szabó ve svém „Malém slovníku`` Dávida Baróti Szabóa (1792) a
v díle „Květy Maďarska`` (1803) našel pod heslem „kürt`` dobové názvy:
„Kürtős Kaláts, botra tekerts, botkaláts, rudas fánk``. Podle klužského
lingvisty, zatímco na konci 18. století se v Seklersku používala forma
kürtőskalács, výrazy botkalács a rudasfánk byly známé pouze mimo
Sedmihradsko a David Baróti Szabó, rodák ze Seklerska, se s nimi mohl
seznámit až během svého pozdějšího pobytu v Horním Uhersku. Varianta
koláče pečená na uhlí se v Maďarsku nejprve rozšířila v severních
malošlechtických vesnicích Zadunají. Tento typ koláče se později stal
slavnostním pokrmem na rolnických svatbách v severním Zadunají (např. v
Cseszneku). Na konci 19. století, podle kuchařky Ágnes Zilahyové, byl
vršek koláče bohatě posypán cukrovými mandlemi.

V 18. století se ve větších sedmihradských městech připravoval poměrně
často. Například klužské inventáře a soupisy z konce 18. století přesně
dokládají přítomnost dřeva na pečení kürtőskalács v tehdejších rodinných
domácnostech. T. Attila Szabó našel v hospodářských záznamech rodiny
hraběte Mikóa dokument z roku 1772, který zaznamenával vybavení tehdejší
klužské kuchyně a mezi jehož náčiním se nacházel i „Hrnec na pečení
kürtőskalács``. V archivu László Telekiho dokument z roku 1773
zaznamenával název nástroje používaného při pečení koláče jako „Forma na
kürtős kaláts``. Podle jiného dokumentu se v roce 1811 v soupisu majetku
dvora v Mezőőru v župě Kluž vyskytovalo „dřevo na pečení kürtős
kaláts``. Mezi majetkem Mihálye Szijgyártó Trintsiniho z Marosvásárhelye
se v roce 1810 objevilo „dřevo na pečení kürtős koláts``. V roce 1822 se
„dřevo na pečení kürtős kaláts`` nacházelo i v kurii barona Józsefa
Gyulakuti Lázára v Nyárádszentanně. Mezi majetkem Karla Petrichevich
Horvátha z Felsőzsuku v župě Kluž, sepsaným v roce 1827, se nacházelo
„dřevo na pečení kürtős kaláts s hrncem``. Uvedené údaje naznačují, že
na konci 18. a počátku 19. století se kürtőskalács pekl s pomocí
charakteristického dřeva především v bohatších městských kuchyních a
vesnických kuriích, a na mnoha místech se používala i jeho varianta
obložená hrncem.

Podle archivního výzkumu Dénese Cs. Bogátse se dřevo používané k pečení
koláče v Háromszéku objevuje častěji až v soupisech z první třetiny 19.
století: „Dřevo na pečení kürtős kaláče obložené hrncem...`` (1810),
„Dřevo na pečení kürtős`` (1834), „Čtyři malé lopatky a dřevo na pečení
kürtős kaláče...`` (1838). Údaje z Háromszéku naznačují, že v
jihovýchodním Sedmihradsku se tento druh koláče pekl na začátku 19.
století na válcovém dřevě s rukojetí, obloženém hrncem, které se pomalu
otáčelo nad žhavými uhlíky otevřeného ohniště. Tyto písemné prameny také
přesně dokládají, že kürtőskalács se v Seklersku rozšířil ve větším
měřítku ve vesnických komunitách až na začátku 19. století.

Balázs Orbán ve svém prvním svazku „Popisu Seklerska`` (1868) zveřejnil
máréfalvskou legendu o původu, která také odráží, že kürtőskalács byl v
té době již hluboce zakořeněn v životě regionu: „Když Tataři pustošili
krajinu, lidé z Máréfalvy se uchýlili do svých ochranných jeskyní, a
když Tataři vcházeli do údolí, ti v jeskyni na ně stříleli šípy a
skolili několik z nich, včetně jejich vůdce. Tatarské vojsko se
rozzuřilo a začalo útočit na úkryt; ale protože se k silně položenému
úkrytu nemohli dostat, oblehli ho, aby ho vyhladověním donutili ke
kapitulaci. Obléhaným i obléhatelům již došly všechny zásoby, když ti v
jeskyni udělali z slámy velký kürtős kalács, ukázali ho a křičeli dolů:
Hle, jak se máme dobře, zatímco vy hladovíte! Když to Tataři viděli, po
zničení vesnice odešli.``

Kuchařka tety Rézi, vydaná v Szegedu v roce 1876, informovala o další
významné změně ve vývoji koláče. Její autorka doporučuje, aby se povrch
koláče před pečením posypal mandlovým cukrem. Ágnes Zilahyová ve své
„Pravé maďarské kuchařce``, vydané v Budapešti v roce 1892, již
doporučovala válení v cukru bez mandlí.

Krby se v maďarsky mluvící oblasti, v okrajové části Seklerska, udržely
až do konce 19. století. Kürtőskalács se tedy mohl i nadále péct nad
uhlíky na otevřených ohništích a v otevřených předsíních pecí na chleba
za tradičních podmínek. Tato skutečnost do značné míry přispěla k tomu,
že kürtőskalács byl až do druhé poloviny 20. století považován za krále
slavnostních koláčů v sekl

Poznamenejme, že slazení krystalovým cukrem se v maďarských vesnicích
rozšířilo až koncem 19. století. Proto se forma koláče pečeného na uhlí
s cukrovou polevou mohla vyvinout až od té doby a teprve poté se
rozšířila.

Podle údajů Maďarského etnografického atlasu byl kürtőskalács na počátku
20. století již významným slavnostním koláčem v župách Bihar, Hajdú,
Szatmár a Ung, ale byl již široce známý také v Silágysku, historickém
Sedmihradsku, včetně Seklerska, a u Seklerů žijících v Gyimesi, Bukovině
a u moldavských Čangů. Podle mapy Atlasu čangských dialektů byl v
Moldávii obecně rozšířen v seklerských vesnicích. Obvykle se připravoval
při svatbách, novoročních a masopustních oslavách, ale nosil se i ženám
po porodu v košíku radina. V Gyimesi se do košíku radina obvykle
vkládaly 4 kusy kürtőskalács, k nim se přidaly 3-4 preclíky, 4 vrstvené
placky a 1 svatojánský chléb. Menší varianta smažená na tuku byla ve 20.
století rozšířena i v zemědělských městech Kiasalföldu a v Sedmihradsku.

Recepty týkající se kürtőskalács v maďarských kuchařkách ze 17.-19.
století a jejich gastronomicko-historické poznatky nedávno shrnul Balázs
Füreder. Ve své práci zdůraznil, že se v průběhu posledních čtyř století
neustále měnily suroviny, způsob přípravy a ochucování tohoto maďarského
druhu koláče. Recepty ze 17.-18. století dokazují, že se již tehdy hnětl
z kynutého těsta, občas se ochucoval rozinkami a poté se pekl navinutý
na dřevo nad uhlíky. Od počátku 19. století se jeho příprava neustále
měnila: mohl se péct z kynutého nebo křehkého těsta, nad uhlíky nebo na
pánvi, v oleji, mohl se ochucovat nejen mandlemi nebo ořechy, ale mohl
se také válet v cukru ještě před pečením, čímž se na vnějším povrchu
koláče vytvořila křupavá cukrová poleva. Ve 20. století se pekl vždy z
kynutého těsta, zatímco jeho povrch se mohl ochucovat karamelizovanou,
mandlovou karamelizovanou a ořechovou karamelizovanou cukrovou polevou.
Na počátku 21. století se v komerčně vyráběných variantách máslo
nejčastěji nahrazuje margarínem, odvážně se koření a peče se do křupava
v moderních topeništích.

\subsubsection{Původ a evropské
paralely}\label{250319-1326}

O původu koláče nemáme přesnější údaje. Podle výzkumu Eszter Kisbánové
nebyl na konci 16. století ve středoevropském kulinářském umění žádnou
novinkou. V německy mluvících oblastech je jeho zvláštní varianta známá
jako Baumkuchen, jejíž původ někteří odvozují z německého města
Salzwedel. Jiní se domnívají, že se tam dostal z maďarských oblastí
prostřednictvím kuchařky Marxe Rumpolta „Ein Neues Kochbuch``, vydané ve
Frankfurtu nad Mohanem v roce 1581, který dříve pracoval jako mistr
kuchař v Maďarsku. Zároveň se dá předpokládat, že se podobné koláče
vyvíjely nezávisle na sobě v různých oblastech Evropy v rámci různých
gastronomických kultur, protože pečení těsta navinutého na tyč nebylo
neznámé ani ve starověké řecké, římské a dálněvýchodní kuchyni.

Variantami vyskytujícími se v německých oblastech se zabýval slezský
Hahn Fritz (1908-1977), jehož dokumentační materiál se dnes nachází v
archivu Steiermarkischen Landesmuseum na zámku Stainz. Podle Hahnova
výzkumu staří Řekové již pekli chléb Obelias z obilné mouky, nad uhlíky,
z spirálovitých, předem nakynutých těstových pásů navinutých na tyč,
často dlouhý i metr. Starověká vyobrazení dokazují, že v rámci Dionýsova
kultu obvykle dva lidé nosili velký koláč na ramenou pomocí tyče. Tento
způsob výroby a pečení, který v podstatě napodoboval pečení masa, se od
Řeků naučili i Římané, nebo ho možná objevili sami. Domníváme se, že se
z Itálie přes Alpy dostal do dnešních německých oblastí, kde se kolem
roku 1550, v těžkých válečných časech, stále pekl venku na tyči nad
ohněm takzvaný Notbrot, tedy nouzový chléb.

V roce 1450 se v německých oblastech připravoval koláč z těsta
navinutého na dřevěný válec, jehož vnější strana se potírala žloutkem a
poté se otáčela nad žhavými uhlíky ohniště, dokud se nespekla. O více
než sedmdesát let později, v roce 1526, benátská kuchařka zaznamenala,
že jeho jemná textura ve tvaru pásků se již připravovala z mouky, vajec,
smetany a koření a poté se pekla na otevřeném ohni na dřevěné tyči,
přičemž se neustále potírala tukem, máslem a vejci. Podle Fritze Hahna
je na prvním titulním listu italské kuchařky Epulario z roku 1526
vyobrazen koláč pečený na tyči nad otevřeným ohništěm.

V 15.-16. století byl slavnostním koláčem především bohatších
šlechtických a později bohatších měšťanských rodin v německých osadách.
Podle policejních zpráv a písemných pramenů se v Norimberku v roce 1485
konzumoval především na svatbách bohatších patricijských rodin. Protože
tehdejší úřady považovaly pečení koláče za plýtvání, nelibě nesly jeho
konzumaci a všemi prostředky se snažily omezit počet osob pozvaných na
svatební konzumaci „vaječného koláče`` (ayrkuchen). Ve Frankfurtu nad
Mohanem byl v roce 1576 dokonce zakázán jeho příprava. Pravděpodobně v
roce 1539 vznikl podrobný recept, který se dochoval v dominikánské
kuchařce. Rozhodné omezující kroky úřadů však nemohly jeho pečení zrušit
a v německých osadách se pekl dál.

Kuchařka Balthasara Steina (pozn.: ve skutečnosti Steindl), vydaná v
Dillingenu v roce 1547, již odráží, že v životě koláče došlo na konci
16. století k významné změně. Zatímco v maďarských a českých oblastech
se stále připravoval ze spirálovitě navinutých copů na dřevě, v
německých osadách se rozválené pláty kynutého těsta kladly přímo na
pečicí dřevo. Pláty koláče z mouky, vajec, smetany slazené medem a
koření se tedy navíjely na předehřáté dřevo, zplošťovaly a poté se
obalovaly provázkem, na jehož vnějším povrchu se během pečení objevovaly
spirálovité drážky a prsteny. Tyto druhy koláčů se připravovaly
především na knížecích dvorech. Tento postup pečení doporučoval i Marx
Rumpolt ve své renesanční kuchařce, vydané v roce 1581, kterou nechala
přeložit do maďarštiny i manželka knížete Apafiho. Kuchařka Marie
Schellhammerin, vydaná v roce 1697, již obsahovala i kresbu k receptu.
Kuchařka Christopha Thiemena z roku 1682 dokumentuje rozšíření
Baumkuchenu na Moravě.

Poznamenejme, že tento starší druh koláče, který se nepekl ze
spirálovitých copů, ale z předem rozválených plátů, připravovali i
barcasští Sasové, kteří žili v sousedství Seklerska. V saských vesnicích
v okolí Brašova (Höltövény/Heldsdorf, Prázsmár/Tartlau,
Szászhermány/Honigberg a Volkány/Wolkendorf) se koláč připravoval z
tenčího nebo silnějšího těsta, které se nemuselo připevňovat provázkem k
válcovitému rožni nebo válečku na těsto, protože hospodyně pláty těsta
silně přitlačily dlaněmi na válcovité dřevo. Když se plát kynutého těsta
přilepil na dřevo, nejprve se obalil krystalovým cukrem, který se během
pečení nad uhlíky postupně rozpouštěl a nakonec karamelizoval do
křupava. Ačkoli se sedmihradští Sasové v desetiletích po druhé světové
válce hromadně vystěhovali do Německa, v rámci svých letničních setkání
v Dinkelsbühlu každoročně hrdě pečou, prodávají a konzumují Baumstriezel
jako starý, typický saský slavnostní koláč, který si přivezli ze svého
sedmihradského rodiště.

Koláč připravovaný na rožni nad žhavými uhlíky otevřeného ohniště zůstal
populární i na konci 16. století a dokonce se připravovala jeho nová
forma z tekutého materiálu, který se kapáním nanášel na pečicí dřevo,
podle kuchařky Susany Gewandtschneiderin z roku 1585. V následujícím
století, zejména po třicetileté válce (1618-1648), se jeho popularita
znovu zvýšila. Na konci 17. století se v německých osadách připravoval z
tekutého vaječného základu, který se vrstveně naléval na dřevo otáčené
nad ohništěm, čímž se na něm vytvořily charakteristické hrbolky z
kapajícího základu z jemné mouky, másla, vajec a smetany. V 18. století
se v německy obývaných oblastech pekl koláč nejen z tvárného, ale i z
tužšího těsta. Poznamenejme, že podobnou technikou a z tekutého základu
se kapáním připravuje litevský a polský ragoulis-sakotis-sekacz, který
později převzali i Francouzi a nazývají ho gateu-a-la-broche, a švédský
populární spettekaka. Poznamenejme, že Rakušané dodnes připravují
podobnou technikou a ze stejných surovin prügeltorte, který se reliktně
zachoval především v vesnicích v údolí potoka Brandberg v Tyrolsku.
Tekuté těsto se nalévá na předehřátý kotouč obalený pečicím papírem,
pomalu otáčený nad ohništěm, a když se vlivem tepla ztuhne, nalije se
další dávka na jeho stále hrbolatější povrch.

Podle kuchařky Markuse Loofta z roku 1769 začali němečtí městští cukráři
od konce 18. století obvykle šlehat základ z vajec, smetany, másla a
cukru do pěny a později jej ochucovat čokoládou. Péter Hantz upozornil,
že nejprve se do těsta s větším množstvím másla vmíchaly pouze žloutky,
zatímco bílky se zvlášť smíchaly se špetkou soli a poté se vmíchaly do
těsta na koláč. Dodnes se zachovala praxe rytmického prořezávání
hladkého povrchu koláče, tedy jeho zvlnění, a následného polévání
cukrovou nebo čokoládovou polevou.

Koláč se postupně začal připravovat i v kuchyních měšťanských rodin. V
19. století se vrstvený koláč s cukrovou polevou nebo čokoládovou
polevou pekl především pro přijímání významných hostů. Když z bytů
zmizela otevřená ohniště a objevila se uzavřená kovová kamna, pečení
tohoto charakteristického koláče ve velké části německých oblastí
zaniklo, ale cukrářské knihy vydané v mnoha městech (např. v Berlíně,
Magdeburgu, Salzwedelu) jej dále popularizovaly. Díky tomu a také díky
poptávce po něm se v německy mluvících oblastech dodnes připravuje
sladký koláč z vrstev stejné tloušťky, často potažený čokoládou. Ve 20.
století se nahřátá kovová tyč nejprve namáčela do tekutého těsta nebo se
na ni základ postupně naléval. Když zhnědl, nanášela se na něj vrstva za
vrstvou a jeho povrch se často tvaroval do vln. Historie německé
varianty také dokazuje, že se koláč neustále měnil až do současnosti a
postupně integroval nové a nové vlivy.

Sladký slavnostní koláč pečený na rožni nad otevřeným ohništěm se dodnes
připravuje v mnoha evropských zemích. V Německu se nazývá Baumkuchen, v
Rakousku Prügelkrapfen, na Slovensku a v Česku Trdelník, v Polsku
Sekacz, v Litvě Ragoulis-Sakotis, ve Švédsku Spettakaka, v Lucembursku
Baamkuch a ve Francii Gateau a la broche.

Domníváme se, že německá varianta Baumkuchenu z tekutého základu se
nejprve rozšířila do polských a litevských oblastí, kde se její
technologie a základ specificky dále vyvíjely. Litevský, francouzský a
polský koláč se obvykle připravoval z řidšího základu, takže se na
povrchu těsta vlivem odstředivé síly vytvořily stalaktitové hrbolky. To
potvrzuje i litevský název tohoto druhu koláče, který znamená špičatý,
vícevětvový, zatímco polský název znamená hrbolatý. Domníváme se, že se
tato specifická forma přípravy tohoto typu koláče v této oblasti mohla
vyvinout kolem 18. století. Protože Litevci považují Ragoulis-sakotis za
své typické jídlo, prodávají jej na mnoha evropských akcích jako svůj
národní symbol. Je pozoruhodné, že Francouzi připravují podobný koláč s
názvem gateau-a-la-broche pouze v Pyrenejích. Podle jejich ústní tradice
si koláč oblíbili vojáci Napoleona během svých východních tažení na
počátku 19. století, poté se s jeho receptem vrátili domů a jejich
potomci jej připravují dodnes.

V jižním Švédsku se tradičně pekl z bramborové mouky od první poloviny
18. století. Jeho nejstarší recept se nachází ve švédském vydání
kuchařky Susany Egerin z Německa z roku 1733, což dokazuje německý původ
koláče. Základ z másla, cukru, vajec a bramborové mouky se obvykle plní
do sáčku, z něhož se postupně vytlačují proužky na kuželovitý tvar
otáčený nad ohništěm, dokud jeho povrch není zcela pokrytý pásky. Povrch
suchého koláče s pórovitou strukturou se často zdobil barevnou cukrovou
polevou. Spettakaka byl zástupci provincie Scania zaregistrován u
Evropské unie jako typický švédský gastronomický produkt.

Protože trdlo nebo tredlenice, které se vyskytují u Čechů a Moravanů,
jsou velmi blízké starému německému spiesskuchen-ayrkuchenu, domníváme
se, že se v jejich komunitách rozšířily vlivem Německa. Jejich základním
rysem je, že se kynuté těsto navinuté na rožeň před pečením neobaluje v
cukru, ale pouze se posype mletými ořechy, přičemž se během pečení
potírá máslem a bílky, takže se na jeho drsnějším povrchu nevytvoří
cukrová poleva. V Skalici na Slovensku se po sejmutí z pečicího dřeva
posype jemněji mletým vanilkovým moučkovým cukrem. První písemná zmínka
o skalickém trdelníku se nachází v textu maďarského básníka Gyuly
Juhásze z roku 1911. Protože se koláč v meziválečných letech již
nepřipravoval, byl znovu objeven až v prvních letech 21. století a
zaregistrován jako typický slovenský koláč. Zdůrazňujeme, že v českých a
slovenských městech se stal skutečně populárním především v posledních
dvou desetiletích, kdy si Švédsko a Slovensko nechaly u Evropské unie
zaregistrovat své varianty jako typické švédské, respektive slovenské
produkty.

Koláč nápadně podobný seklerskému kürtőskalácsi, připravovaný stejnou
technikou a formou, se dnes objevuje i v Turecku a Japonsku.
Poznamenejme, že vynalézaví seklerské, maďarské a jiné národnostní
podniky v posledních dvou desetiletích připravují a prodávají tradičně
připravovaný seklerský kürtőskalács ve všech koutech světa.

Sedmihradský seklerský kürtőskalács se v posledních deseti letech dostal
do specifického evropského kontextu. Ve Skalici na Slovensku vzniklo v
roce 2004 občanské sdružení, jehož cílem bylo nechat zaregistrovat koláč
pečený na uhlí u evropských institucí. V důsledku toho získal skalický
trdelník (Skalicky trdelnik) 21. dubna 2007 ochrannou známku EU. V
materiálu a odůvodnění slovenského místního zájmového sdružení se uvádí,
že podle skalické ústní tradice recept na kürtőskalács připravovaný ve
městě a okolí zavedl sedmihradský kuchař maďarského spisovatele a
básníka Józsefa Gvadányiho, který zde žil v letech 1783-1801.

Slovenská registrace kürtőskalácse vyvolala v Seklersku specifickou
konkurenci. Od té doby se mnoho seklerských komunit snaží připravit
nejdelší kürtőskalács na světě. Například obyvatelé Uzonu v Háromszéku,
kteří v posledních deseti letech prodávají koláč i na ulici turistům a
cestujícím projíždějícím obcí, nejprve upekli 2,5 m dlouhý kürtőskalács
v roce 2007 v rámci místních vesnických slavností a poté v roce 2009
připravili již 10 m dlouhý koláč, který ve stejném roce překonali
obyvatelé sousední obce Szentivánlaborfalva 14metrovým koláčem.
Skaličané na jaře 2011 připravili „největší na světě`` 1,93 m (!) dlouhý
kürt

\subsubsection{Shrnutí}\label{250319-1326}

Koláč navinutý na tyč a pečený otáčením nad žhavými uhlíky ohniště má v
Evropě v zásadě tři formy: a) první varianta se připravuje ze
spirálovitých copů, b) druhá z rozválených plátů těsta, c) třetí se na
tyč nanáší postupně kapáním a během pečení se pomalu a rovnoměrně otáčí
nad ohněm.

V maďarsky mluvící oblasti, včetně Seklerska, se zachoval
nejarchaističtější způsob přípravy tohoto evropského druhu koláče: z
kynutého těsta se nejprve připraví spirálovitý cop, který se navine na
válcovitou tyč, dlaněmi se pevně přitlačí a poté se peče otáčením nad
uhlíky do křupava. Výzkumníci evropské kultury stravování našli
historické předchůdce této techniky pečení nejen u starých Řeků a
Římanů, ale i u středověkých Němců.

Od 16. století se v německých osadách vyvinula i forma tohoto koláče,
kdy se hladce rozválené pláty těsta navíjely ve více vrstvách na předem
nahřáté válcovité dřevo a povrch se spirálovitě obvazoval provázkem,
čímž se na povrchu koláče vytvořily spirálovité drážky a prsteny.
Zdůrazňujeme, že podobné varianty z tužšího základu se vyskytovaly v
různých německých osadách ještě v 18. století.

Třetí varianta tohoto jídla (například u Poláků, Litevců a Francouzů,
Švédů) se připravuje tak, že se z řidšího základu postupně kape na dřevo
pomalu otáčené nad otevřeným ohništěm, čímž povrch koláče nakonec získá
rozmanité tvary. Zdůrazňujeme, že v německých osadách se již od 17.
století připravoval koláč z řidšího základu, nalévaný na nahřátou tyč a
otáčený nad ohništěm. Jeho vrstvené varianty se od konce 18. století
stávaly stále náročnějšími a na konci 19. století se jejich povrch
dokonce poléval čokoládou.

Evropské varianty koláče příbuzného seklerskému kürtőskalácsi jsou tedy
dodnes populární v různých německých, moravských, českých, slovenských,
polských, litevských, švédských a francouzských komunitách. Na základě
písemných pramenů a dobových vyobrazení, jakož i zkoumání oblasti
rozšíření, se domníváme, že se seklerský kürtőskalács rozšířil v
maďarsky mluvící oblasti především z německých komunit, případně
prostřednictvím Sasů ze západu.

Podle našeho názoru se Sekleři seznámili s nejstarší variantou německého
druhu koláče, připravovanou z tužšího základu, spirálovitě navinutou na
tyč a pečenou nad uhlíky, kterou převzali, dále rozvíjeli a zachovali
dodnes. Charakteristická cukrová poleva tohoto slavnostního koláče,
často posypaná mletými ořechy a ochucená, se v Seklersku vyvinula až od
konce 19. století, po širší výrobě a rozšíření krystalového cukru.

Poznamenejme, že skalický „slovenský`` kürtőskalács je kopií
seklersko-maďarské varianty. Koláč z proužků kynutého těsta spirálovitě
navinutých na tyč a pečených otáčením nad uhlíky je na Slovensku v
podstatě neměnný, zatímco v maďarsky mluvící oblasti se tento koláč
vyskytuje široce a jeho přítomnost lze průběžně dokumentovat písemnými
prameny již od konce 17. století.

Dochované písemné a tištěné prameny naznačují, že se tento jemnější
koláč z kynutého těsta v Sedmihradsku usadil až v 17. století, kde byl
zpočátku slavnostním koláčem pouze elity, šlechtické vrstvy, a od konce
18. století se postupně široce rozšířil a usadil i v jiných
společenských vrstvách, začal se připravovat i v kuchyních městských
rodin a venkovských zemědělců. Protože se otevřená topeniště v
seklerských vesnických domech zachovala až do konce 19. století, bylo
možné kürtőskalács nadále připravovat za tradičních podmínek nad žhavými
uhlíky krbu při různých rodinných a kalendářních svátcích. Tento
prestižní koláč tak mohl zůstat významným druhem koláče na seklerských
maďarských rodinných a společenských svátcích i na konci 19. století.
Protože v seklerských vesnických zemědělských komunitách se cukr začal
ve větší míře používat právě v té době, cukrová forma kürtőskaláče se
mohla vyvinout až po širším rozšíření cukru vyráběného v továrnách.

Významný symbol maďarské vesnické zemědělské kultury se připravoval
především na významné události vesnických rodin (křtiny, svatby,
přijímání významných hostů). Je zajímavé, že v ostatních částech
maďarsky mluvící oblasti, vlivem rychlejší a dřívější měšťanizace a
následné modernizace, se již nepekl. V maďarských osadách mimo Seklersko
jeho místo zaujaly především dorty městského měšťanského původu.

V životě seklerského kürtőskaláče nastala nová situace s rumunským
ideologickým uvolněním v roce 1968. Bukurešťská moc, bezprostředně po
invazi do Československa, dočasně zmírnila svou dřívější protimenšinovou
politiku a z taktických důvodů krátce tolerovala prodej maďarských
lidových uměleckých předmětů v různých turistických centrech. Mnoho
agilních, podnikavých seklerských rodin brzy využilo této situace a
začalo péct a prodávat kürtőskalács nejen v sedmihradských, ale i v
valašských a moldavských rumunských letoviscích. Ve druhé polovině 20.
století se v důsledku ideologického uvolnění v roce 1968 poměrně rychle
rozšířil i do rumunských regionů mimo Seklersko, především v pobřežních
a horských turistických centrech.

Rumunská změna režimu v roce 1989 přinesla ještě zajímavější fenomén
obrození. Zpočátku se jím nabízeli pouze maďarští turisté navštěvující
seklerské vesnice a postupně se stal nejvýznamnějším, symbolickým
koláčem stále populárnějších místních slavností. Koláč, který se dříve
připravoval pouze v Seklersku a rumunských turistických centrech, se
tedy stále více šířil jako typická seklerská sladkost. Populární
kürtőskalács se po roce 1990 nejprve postupně šířil na západ, tedy stal
se oblíbenou sladkostí v severozápadním pohraničí země, v Partiu, a poté
se postupně usadil i v různých maďarských městech a letoviscích.

Kürtőskalács se v důsledku prezentovaného procesu, podobně jako guláš,
dnes stal nejen seklerským, ale celomaďarským symbolem, organickou
součástí seklerské a maďarské identity, přičemž i nadále zůstává
významným prvkem turistiky směřující do Seklerska. Kürtőskalács se na
přelomu 20. a 21. století jednoznačně stal seklerským maďarským
symbolem. Zatímco se stal nepostradatelným prvkem obrazu Seklerska a
Sedmihradska propagovaného

Díky pracovní migraci a v neposlední řadě turismu se dnes připravuje v
mnoha zemích, kde se stal nejen seklerským nebo maďarským, ale i
typickým evropským koláčem, specifickým gastronomickým symbolem.

Životní dráha kürtőskaláče dobře ilustruje, jak se gastronomický prvek
rozšířil z elitních kruhů, postupně se začlenil do populární kultury,
jak se během sta let stal organickou tradicí ve venkovských komunitách a
jak se nakonec v rámci folklorismu a turismu vrátil k vyšším vrstvám
společnosti.

Když Rumunsko vstoupilo do Evropské unie, začalo se postupně právně
chránit zdejší typické, často zdůrazňované tradice s etnickým významem.
Když si Slovensko nechalo ochránit skalický trdelník, nastala nová
situace i v postmoderním životě kürtőskaláče, který byl dříve považován
pouze za seklerskou tradici.

Především v Seklersku se rozvinula zajímavá rivalita a
hyperreprezentace: v rámci zde pořádaných místních slavností se místo
tradičního 40-50 centimetrového koláče pekly obrovské, 10-20 metrů
dlouhé, nadrozměrné seklerské koláče. Vzhledem k tomu, že
východoevropské komunity s různými jazyky žily po staletí mozaikovitě
vedle sebe, rozvinula se prudká debata a soutěž o ochranu mnoha dalších
tradičních kulinářských prvků (např. maďarská pálenka a rumunská cujka,
rumunský brânză a slovenská bryndza, maďarské a slovenské tokajské víno
atd.). Seklerský kürtőskalács se v posledních desetiletích stal vybraným
prvkem a součástí této symbolické mezi-etnické soutěže.
