\subsection{Kochen im Bild: Herstellung von
Prügelkrapfen}\label{250615-0047}

(Vaření v obraze: Příprava Prügelkrapfenů)

\url{https://web.archive.org/web/20201230125915/http://kurtos.eu/dl/11664.pdf}

Pro tento v české kuchyni velmi oblíbený sladký pokrm zvaný „Trdelnice``
je zapotřebí železný pečicí rošt, který se vejde do trouby, jak je vidět
na obrázku (uprostřed nahoře). Dále jsou potřeba dvě kónická kulatá
dřeva s vyčnívajícími železnými osami. --- K samotnému těstu, které je
jemné kynuté těsto, se přidá 3 dekagramy rozdrceného droždí (30 gramů
droždí) posypaného kávovou lžičkou cukru, poté se rozpustí v ⅛ litru
vlažného mléka, aby se nyní přidalo tolik 1 kilogramu hrubé nebo
polohrubé pšeničné mouky, aby vzniklo poloměkké těsto, tzv. „Dampfel``,
které se krátce prohněte, poté hustě posype moukou a přikryté utěrkou se
nechá na teplém místě dobře vykynout. Zbytek mouky se smíchá v misce s 1
kávovou lžičkou soli, 5 dekagramy jemného cukru a jemně nastrouhanou
citronovou kůrou. Dále se nechá 30 dekagramů másla zcela rozpustit, poté
vychladnout, aby se toto přepuštěné máslo spolu s ¼ litrem vlažného
mléka, 4 žloutky a jedním celým vejcem dobře rozšlehalo a to vše se
nalije do mouky. Nyní se přidá i již vykynuté „Dampfel`` a za přidání
vlažného mléka podle potřeby se z celku vypracuje pevné, přesto vláčné
těsto, které se v míse hněte, dokud se neodlepuje od nádobí a rukou.
Vytvarované do bochníku se těsto nechá v míse půl hodiny přikryté
odpočívat, poté se vyklopí na lehce pomoučenou desku, aby se ještě
jednou dobře prohnětlo a rozdělilo na kousky zhruba o velikosti
dvojnásobné pěsti, které se vyválí do silných hadů o tloušťce palce. ---
Mezitím se pečicí rošt s oběma dřevy trochu předehřeje v troubě, poté se
dřeva dobře potřou vepřovým sádlem, ale pak se papírem velmi důkladně
otřou (aby se zabránilo tomu, že těsto spadne, pokud by dřevo bylo
příliš mastné). Na připravená dřeva se nyní spirálovitě nanese vyválené
těsto ve tvaru hada, jak je vidět na obrázku, přičemž jedna ruka otáčí
dřevem a druhá vede těstový pramen. Začátek a konec těstových pramenů se
pevně přitlačí na dřevo, aby se role nerozpadla. Nakonec se těstová
spirála zevnitř potře navlhčeným vejcem a hustě posype nahrubo
nasekanými vlašskými ořechy. Nyní je čas zasunout rošt s Prügelkrapfeny
do dobře předehřáté trouby na střední teplotu a krapfeny se v zavřené
troubě pečou poměrně rychle. Jakmile vnější části získají barvu, otáčejí
se dřeva v jejich ose; v případě potřeby se otočí i celý rošt, aby
Prügelkrapfeny získaly krásnou barvu po celém obvodu a dobře se
propekly. Nakonec musí mít zlato-hnědou barvu a doba pečení by měla být
asi 30 minut, poté se „trubky`` opatrně a za horka snímají z dřev, a to
tak, že se dřeva na užší straně rozklepou. --- Velmi dobré je, po
vyjmutí z trubek „vyfouknout`` pečicí páru, aby krapfeny nebyly uvnitř
lepkavé. Prügelkrapfeny se ještě teplé bohatě posypou vanilkovým
moučkovým cukrem a volně, avšak zcela zabalené do papíru se nechají
vychladnout. --- Pro podávání se potřebné množství Prügelkrapfenů
nakrájí na kroužky široké asi 3 centimetry, které se, jak je patrné z
obrázku, naskládají na hromadu a pocukrují. Prügelkrapfeny, které nejsou
určeny k okamžité spotřebě, by se neměly krájet, nýbrž zůstat zabalené v
papíře a uchovávat na chladném místě, kde vydrží čerstvé asi týden.
