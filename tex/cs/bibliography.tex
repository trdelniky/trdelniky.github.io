\section{Wikipedie}\label{Wikipedie}

\begin{itemize}
\tightlist
\item
  jak
  \href{https://cs.wikipedia.org/wiki/N\%C3\%A1pov\%C4\%9Bda:Reference_ve_wikik\%C3\%B3du}{psát
  citace ve wikipedii}

  \begin{itemize}
  \tightlist
  \item
    \href{https://cs.wikipedia.org/wiki/Wikipedista:Michal_h21/P\%C3\%ADskovi\%C5\%A1t\%C4\%9B}{moje
    stránka na cvičení}
  \end{itemize}
\item
  \href{https://bara.ujc.cas.cz/psjc/search.php?hledej=Hledej&heslo=trdeln\%C3\%ADk&where=hesla&zobraz_ps=ps&zobraz_cards=cards&pocet_karet=100&ps_heslo=trdlovec&ps_startfrom=0&ps_numcards=7&numcchange=no&not_initial=1}{Heslo
  na kartotéce UČJ}
\end{itemize}

\section{Historie}\label{Historie}

\begin{itemize}
\tightlist
\item
  Skalice

  \begin{itemize}
  \tightlist
  \item
    legenda o kuchařovi maďarskýho grófa a spisovatele Jozefa Gvadániho,
    kterej si recept přinesl z Transylvánie
  \item
    nikde nemůžu najít relevantní důkazy
  \item
    i v \href{https://www.tmdn.org/giview/gi/EUGI00000013744}{dokumentu}
    (\href{https://eur-lex.europa.eu/legal-content/EN/ALL/?uri=CELEX\%3A52007XC0421\%2802\%29}{EU
    zákon}), kde se skalickej trdelník registruje v EU, se píše, že to
    je jen ústně tradovaná historie
  \end{itemize}
\end{itemize}

\section{Alternativní názvy}\label{Alternativnuxedux20nuxe1zvy}

\begin{itemize}
\tightlist
\item
  trdlovec, trdláč, kotúč, vaječník, náčepka (litej trdelník)
\item
  německy Spiß Kuchen, Spiesskuchen, Prügelkrapfen, Ringelkrapfen (na
  Znojemsku), možná Eyerkuchen a Ayerkuch,
\item
  latinsky obolum, Obelias
\item
  trdlo může znamenat taky vajíčka (třeba rybí potěr nebo žabí), takže
  to je taky spojení s vaječníkem

  \begin{itemize}
  \tightlist
  \item
    asi to pochází od tření, trdlem se třou různý potraviny
  \end{itemize}
\end{itemize}

\section{Další poznatky}\label{Dalux161uxedux20poznatky}

\begin{itemize}
\tightlist
\item
  Marie Úlehlová-Tilschová žila po svatbě ve 20. letech ve Skalici, asi
  znala Trdelník odtud. Její tchán Josef Úlehla o něm ale psal ve svých
  pamětech (nevydaných).
\end{itemize}

\begin{itemize}
\tightlist
\item
  Většinou obřadní, na svatbách, křtinách, masopustech a taky se dával
  matkám po narození dítěte
\end{itemize}

\section{Anotovaná chronologická
bibliografie}\label{Anotovanuxe1ux20chronologickuxe1ux20bibliografie}

\begin{itemize}
\tightlist
\item
  136? viz níž, Klaret a jeho družina
\item
  145? \href{https://gams.uni-graz.at/o:corema.h2a.46}{Von essenn eins
  kuchenn an einem spisz}

  \begin{itemize}
  \tightlist
  \item
    recept na Spiesskuchen, Německo. překlad pomocí ChatGPT:

    \begin{itemize}
    \tightlist
    \item
      „Jíst koláč na špejli Chceš-li udělat koláč na špejli z pohanského
      (tedy nekvašeného nebo jednoduchého) těsta, tak jej připrav dobře
      z dvoubarevného nebo tříbarevného těsta, a polož je vedle sebe
      podél tak, aby se jeden druhého nedotýkal. Pak jeden díl na jednom
      místě nožem nařízni a omotej to kolem dřevěné špejle. Potři to v
      těch místech žloutkem, tak ti to drží pohromadě. A nepeč to příliš
      horko (tedy nepeč to při příliš vysoké teplotě).``
    \end{itemize}
  \end{itemize}
\item
  1464
  \href{https://www.digitale-sammlungen.de/view/bsb10798133?page=354\%2C355}{Fontes
  rerum austriacarum. II. oddíl: Diplomataria et acta, svazek 20.} Vídeň
  1860. Palacký Fr., Listinné příspěvky k dějinám Čech a jejich
  sousedních zemí v době Jiřího z Poděbrad (1450 až 1471), str. 329, č.
  317.

  \begin{itemize}
  \tightlist
  \item
    1464, Jun. 2.
  \item
    víc informací níže, Zíbrt, 1927
  \end{itemize}
\item
  1513 Dictionarius trium linguarum, latinae, teutonicae, boemicae
  potiora vocabula continens, peregrinantibus apprime utilis, Viennae,
  1513, 4°; ve Varšavě, 1513, 4°; vydání 1532, 12°.

  \begin{itemize}
  \tightlist
  \item
    tady je
    \href{https://dbc.wroc.pl/dlibra/publication/33643/edition/30385/content?ref=L3B1YmxpY2F0aW9uLzM0Njg5L2VkaXRpb24vMzE0MTQ}{digitalizovaná
    verze}, s. 19
  \item
    Ouarium - wagečznik - eyerkuch
  \end{itemize}
\item
  1532
  \href{https://bara.ujc.cas.cz/slovniky/slovknm/slovknm31.html\#CTX1129}{Velmi
  užitečná kníška mládencóm}:

  \begin{itemize}
  \tightlist
  \item
    slovníček De ferculis et pertinentiis eorum
  \item
    Ouarium \textbar{} vaječník \textbar{} wagečnik \textbar{} ayerkuch
  \item
    De ferculis et pertinentiis eorum
  \end{itemize}
\item
  1581 \href{https://cakes.institute/rumpolt.html}{Ein new Kochbuch}:

  \begin{itemize}
  \tightlist
  \item
    Marx Rumpolt, Německo
  \item
    slavnej recept na Spiesskuchen, ale je to plát těsta s rozinkama,
    dost podobnej Böhmische Kuchlein o století pozdějc u Thiema
  \end{itemize}
\item
  1582 \href{https://cakes.institute/harsdorfer.html}{Was zu 180 Zuckern
  Leckkuchleinn von Wirtz und Zucker genomen wirt}:

  \begin{itemize}
  \tightlist
  \item
    Susanna Harsdörfer, Německo
  \item
    recept na litej spiss kuchenn, asi nejstarší
  \end{itemize}
\item
  1586
  \href{https://books.google.cz/books?id=wN9YAAAAcAAJ&vq=ayerkuch&hl=cs&pg=PA101-IA1\#v=onepage&q&f=false}{Sÿrach
  Mathesij, Das ist, Christliche, Lehrhaffte, Trostreiche vvd lustige
  Erklerung und Außlegung des schönen Haußbuchs, so der weyse Mann
  Syrach zusammen gebracht vnd geschrieben}:

  \begin{itemize}
  \tightlist
  \item
    Johannes Mathesius, kázání z Jáchymova 1554
  \item
    s. 101:

    \begin{itemize}
    \tightlist
    \item
      vnd gehet der Mann fornen naus / so gehet sie hinden naus / zum
      Krahles / für die Wochen / zum Spießkuchen 3 / fauffen einander
      ein Plessch Bier zu / wil eis ne nicht bescheid thun / so bringet
      eine der andern einen Kayentrunck
    \item
      a když muž přijede, ona odejde, a když muž odejde předními dveřmi,
      ona odejde zadními dveřmi, do krámu, na týdenní nákupy, na špízový
      koláč, pijí spolu džbánek piva, když se jedna nechce podělit,
      druhá jí přinese kajenský nápoj.
    \end{itemize}
  \end{itemize}
\item
  1586
  \href{https://bara.ujc.cas.cz/slovniky/velnom/velnom7.html\#CTX1229}{Daniel
  Adam z Veleslavína: Nomenclator omnium rerum propria nomina tribus
  linguis -- Latina, Boiemica et Germanica}:

  \begin{itemize}
  \tightlist
  \item
    slovník
  \item
    {[}01/07/01/091{]} Panis crocatus: croco et ovi b luteo infectus
    atque imbutus \textbar{} vaječník \textbar{} Wagečnijk \textbar{}
    Eyerkuchle
  \item
    Šafránový chléb: obarvený a nasáklý šafránem a žloutkem vejce.
  \end{itemize}
\item
  1591
  \href{https://ndk.cz/uuid/uuid:27bccb70-88d6-11e3-997d-005056827e52}{Kuchařství
  to jest Knižka o rozličných krmích, kterak se užitečně s chutí}:

  \begin{itemize}
  \tightlist
  \item
    Bavor Rodovský z Hustiřan
  \item
    VARMUŽE Z VAJEČNÍKA: Vezmi vaječník a skrájej jej na čtyry uhly,
    jako kostka jest, a to vlož v mléko. Potom vezmi jablek a v krájej
    je tam a přičiň vaječných žloutkův. Vař to dobře. A daj na mísu.
  \item
    recept na vaječník samotnej tam není, ale omeleta to nebude, tu
    jinde popisuje jako vaječný svítek, navíc by ten recept nedával
    smysl
  \end{itemize}
\item
  1593
  \href{https://www.deutschestextarchiv.de/book/view/coler_kochbuch_1593/?hl=Spi\%C3\%9Fkuchen&p=85}{Coler,
  Johannes: Buch III: Vom Kochen. Aus: Oeconomia. Oder Haußbuchs}

  \begin{itemize}
  \tightlist
  \item
    jakási německá knížka, ale je tam recept \hyperref[250605-2334]{Einen
    Eyerkuchen oder Spißkuchen zu machen.}
  \end{itemize}
\item
  \href{https://bara.ujc.cas.cz/slovniky/rosa/rosafst1480.html}{Václav
  Jan Rosa: Thesaurus linguae Bohemicae} - nedokončený dílo

  \begin{itemize}
  \tightlist
  \item
    žil 1630-1689
  \item
    Wagečnjk, (wagečný kołáč) placenta ex ouis.
  \end{itemize}
\item
  1659
  \href{https://ceskadigitalniknihovna.cz/uuid/uuid:d380c040-c80a-11ea-b7a2-005056827e51}{Ambrosii
  di Praga Liber medicinalis}:

  \begin{itemize}
  \tightlist
  \item
    je tam latinskej popis toho, co je vaječník:

    \begin{itemize}
    \tightlist
    \item
      21v tortam de farina cum ovis et papauere factam, id est wagecznik
    \item
      překlad: koláč z mouky s vejci a mákem udělaný, to jest vaječník
    \end{itemize}
  \item
    ale není teda vůbec jasný, jak se ten koláč vlastně dělá
  \end{itemize}
\item
  1669
  \href{https://vokabular.ujc.cas.cz/moduly/mluvnice/digitalni-kopie-detail/KomJanua1669/strana-76}{Komenský,
  Zlaté dveře jazykův otevřené...}:

  \begin{itemize}
  \tightlist
  \item
    koláčů rozdílowě gsau
  \item
    obeliae = Vaječník = Spiesskuchen
  \item
    je to položka
    \href{https://vokabular.ujc.cas.cz/moduly/mluvnice/digitalni-kopie-detail/KomJanua1669/strana-76}{408}
  \item
    \href{https://vokabular.ujc.cas.cz/moduly/mluvnice/digitalni-kopie-info/KomJanua1669}{Tady
    je celá knížka}
  \item
    zajímavý je, že v
    \href{https://www.digitalniknihovna.cz/mzk/uuid/uuid:c53f79b0-b747-11e4-a7a2-005056827e51}{Brána
    jazyků otevřená} je založená na jiný verzi a jsou tam jiný pečiva,
    například boží milosti
  \end{itemize}
\item
  1689
  \href{https://www.digitale-sammlungen.de/view/bsb11105115?page=42\%2C43}{Vocabularium
  trilingue}

  \begin{itemize}
  \tightlist
  \item
    Vocabularium trilingue: pro usu scholarum = Vocabulár̆ latinský,
    cz̆eský a nĕmecký = Vocabularium lateinisch, teutsch und böhmisch
  \item
    Olomouc
  \item
    obelum = wagečnjk = spisskuchen
  \end{itemize}
\item
  1694
  \href{https://books.google.cz/books?id=PytAAAAAcAAJ&hl=cs&pg=PA876\#v=onepage&q&f=false}{Kuchařka
  Johanna Christopha Thiema}:

  \begin{itemize}
  \tightlist
  \item
    Haus- Feld- Arzney- Koch- Kunst und Wunderbuch
  \item
    s. 876 Bohmische Küchlein - je to plát těsta na válci:

    \begin{itemize}
    \tightlist
    \item
      Vezměte mléko / mouku a vejce / a udělejte těsto / přidejte trochu
      kvásku / a nechte ho trochu vykynout / pak ho zpracujte / aby bylo
      pěkně pevné / a rozválejte ho do šířky / posypte rozinkami /
      skořicí a muškátovým květem / pak vezměte rožeň / který je k tomu
      určený / a opečte ho na ohni / potřete ho sádlem / a naviňte na
      něj těsto / a upečte ho / můžete také nejprve obalit papírem / aby
      nespadlo z rožně / dokud je ještě měkké.
    \end{itemize}
  \item
    na s. 872 je recept na Spiesskuchen, je to klasickej pruh těsta na
    válci:

    \begin{itemize}
    \tightlist
    \item
      Vezměte teplé mléko a vmíchejte do něj vejce / udělejte těsto z
      jemné bílé mouky / přidejte trochu pivního kvásku a másla / nechte
      ho chvíli stát za kamny / aby vykynulo / pak ho znovu promíchejte
      a trochu prohněťte / poté ho čistě rozválejte / posypte ho malými
      černými rozinkami. Vezměte váleček / který je teplý a pomazaný
      máslem / a položte ho na těsto / těsto na něj naviňte / a svažte
      ho provázkem / aby nespadlo / položte ho k ohni / a pomalu
      otáčejte / tak se upeče dozlatova. A když zhnědne / vezměte štětec
      / namočte ho do horkého másla / a potřete tím koláč / tak bude
      krásně hnědý. A když je upečený / sundejte ho z válečku / a oba
      otvory ucpěte čistými hadříky / aby se udrželo teplo / nechte ho
      tak vychladnout / podávejte studené na stůl / tak bude jemný a
      dobrý.
    \end{itemize}
  \end{itemize}
\item
  český koláč byl z kvásku, spiesskuchen používá pivovaský kvasnice
\item
  1704
  \href{https://bara.ujc.cas.cz/slovniky/vokabular/vokabular12.html}{Vokabulář
  latinský a český, nyní vnově spravený a rozšiřený}:

  \begin{itemize}
  \tightlist
  \item
    Vojtěch Jiří Koniáš
  \item
    O Potřebách priſluſſegjcých k Stolu.
  \item
    Obelum, n. 2. = Wagečnjk
  \end{itemize}
\item
  1727
  \href{https://ndk.cz/uuid/uuid:b59a1ff0-9f23-11ea-b6e0-005056827e51}{das
  kleine nürnberger kochbuch oder die curiöse köchin}:

  \begin{itemize}
  \tightlist
  \item
    nemůžu to najít online, ale cituje to Zíbrt na stranách 58-59
  \item
    každopádně další verze Bohmische Kuchlein, zase je to plát těsta:

    \begin{itemize}
    \tightlist
    \item
      Str. 374 citované knihy (s. 59 u Zíbrta):
    \item
      České koláčky (Böhmische Küchlein) Vezme se mléko, mouka a vejce a
      zadělá se těsto. Přidá se trochu droždí a nechá se trochu
      vykynout, pak se zpracuje, aby byl pěkně pevný, a vyválí se do
      šířky. Posype se rozinkami, skořicí, muškátovým květem a
      kardamomem. Poté se vezme rožeň, který je k tomu určen, a opéká se
      nad ohněm. Potře se sádlem, těsto se na něj přehodí a opéká. Na
      začátku se může těsto převázat papírem, aby -- dokud je měkké --
      nespadlo z rožně.
    \item
      zní to jako kopie Thiemena
    \end{itemize}
  \end{itemize}
\item
  17??
  \href{https://bara.ujc.cas.cz/slovniky/VusSlovBLG1729/vusin22.html\#CTX14944}{Kašpar
  Zachariáš Vusín: Dictionarium bohemo-latino-germanicum}:

  \begin{itemize}
  \tightlist
  \item
    slovník
  \item
    Vaječník \textbar{} Wagečnjk/ m. Obeliæ, f. p. Spieß-kuchen. m.
  \end{itemize}
\item
  1811
  \href{https://ceskadigitalniknihovna.cz/view/uuid:e11769d0-4c42-11e7-aac4-005056827e51?page=uuid\%3A628a69f0-6565-11e7-94b3-005056825209&source=mzk}{Auplné
  vměnj kuchařské, aneb, Pochopitedlný poukaz, kterak}:

  \begin{itemize}
  \tightlist
  \item
    je tam recept na plněný trubičky, č. 208. jsou z máslovýho těsta,
    navíjený na dřívka asi 20 cm dlouhý, pečený a pak plněný zadělávaným
    rybízem nebo meruňkama.
  \item
    Nadívané štucličky. Připraví se máslové těsto, ne ale příliš mastné,
    sbalí se do bochníku a pak se znovu, ne příliš tence, rozválí;
    nakrájí se z něj kousky široké na dva prsty a dlouhé na dvě pídě;
    připraví se na prst tlustá a na píď dlouhá dřívka, namažou se máslem
    a to těsto, které se předtím pomaže vejci, se natáčí na ta dřívka
    tak, aby se polovina překrývala a aby z každého dřívka oba konce
    vyčnívaly. Povrch se pomaže vejci, posype cukrem, upeče se pěkně do
    žluta a nakonec se z každého toho štuclíku dřívko vytáhne a naplní
    se buď zadělávaným rybízem nebo meruňkami.
  \end{itemize}
\item
  1813
  \href{https://ndk.cz/view/uuid:c0247bb0-d3b5-11dc-b9b7-000d606f5dc6?page=uuid:f986d11e-aa9e-41c6-b183-52fcf3fc3f80}{Muza
  Morawská}:

  \begin{itemize}
  \tightlist
  \item
    Gallaš, Josef Heřman Agapit,
  \item
    s.
    \href{https://ndk.cz/view/uuid:c0247bb0-d3b5-11dc-b9b7-000d606f5dc6?page=uuid:c4c850d1-41c7-4c42-8940-f42f45196b76}{225}
    tam jsou verše o trdelníku:

    \begin{itemize}
    \tightlist
    \item
      Jiná mlatců při mlácení Pšenice
    \item
      3. „Také se z ní koláče pečou na svátky makovníky, pagáče, buchty
      na sňatky --- bez ní nejsou žádné hody, svatby a ženské úvody.

      \begin{enumerate}
      \tightlist
      \item
        A Bůh ví jakých jídel panští kuchaři podle kunštu pravidel z
        pšenice vaří --- lukší a mastných lívanců pašték, oplatků,
        drobanců ---
      \item
        tašek, Božích milostí a trdelníků, divné způsobivosti lehkých
        dortíků --- bělek, piškotů, buchteček a cukrových lahůdeček.``
      \end{enumerate}
    \end{itemize}
  \item
    s.
    \href{https://ndk.cz/view/uuid:c0247bb0-d3b5-11dc-b9b7-000d606f5dc6?page=uuid:7f1464d8-a54f-4f38-9c1e-70fa4c0285e1}{347}:

    \begin{itemize}
    \tightlist
    \item
      Uršula Mlsánková: Flašky sladké rozolie kafe, jejž vím ráda pije
      každá z nás ctné paničky --- k tomu dorty, trdelníky koláčky a
      makovníky kořalku i rohlíčky.
    \end{itemize}
  \item
    s.
    \href{https://ndk.cz/view/uuid:c0247bb0-d3b5-11dc-b9b7-000d606f5dc6?page=uuid\%3Ada5b6cb7-cfb0-4ebd-b6be-84d59f5e6f72}{417}:

    \begin{itemize}
    \tightlist
    \item
      báseň "Nemjrnost w gidle ženské hanácké osoby" (Nestřídmost v
      jídle ženské hanácké osoby.)
    \item
      někomu je špatně a popisuje co všechno snědla
    \item
      Potom na wečero wagec randliček Belo gich gen dwanact, a
      trdelniček
    \end{itemize}
  \item
    psal to už kolem roku 1805, hlavně hanácký a valašský prostředí
  \end{itemize}
\item
  1814
  \href{https://ndk.cz/uuid/uuid:4672a240-c28c-11dc-9931-000d606f5dc6}{Karl
  Ignaz Thams neuestes ausführliches und vollständiges
  deutsch-böhmisches synonymisch-phraseologisches Nationallexikon oder
  Wörterbuch nebst einem besonders nothwendingen Suplement oder
  Ergänzungsanhang von allen eigenen Namen der Länder ... Männer, Weiber
  und dergl.}:

  \begin{itemize}
  \tightlist
  \item
    slovník německo-český,
  \item
    s.
    \href{https://ndk.cz/uuid/uuid:4672a240-c28c-11dc-9931-000d606f5dc6}{152}

    \begin{itemize}
    \tightlist
    \item
      Prügelkarpfen = trdelnjk
    \end{itemize}
  \item
    s.
    \href{https://ceskadigitalniknihovna.cz/uuid/uuid:9b4f95a7-eb8b-4008-8023-15cc5563f0ed}{293}:

    \begin{itemize}
    \tightlist
    \item
      heslo Spießkuchen = wagečnjk
    \end{itemize}
  \end{itemize}
\item
  1816 (nebo 1805?)
  \href{https://books.google.cz/books?id=7SZjAAAAcAAJ&hl=cs&pg=PA227\#v=onepage&q&f=false}{Moje
  skrz čtyřicetileté vykonávání známá Kuchařská kniha pro velké i menší
  tabule}, Sibylla (nebo Sibilla) Dorizio:

  \begin{itemize}
  \tightlist
  \item
    s. 227 - 228: recept přečtenej pomocí chatgpt a ručně opravenej,
    ChatGPT hrozně halucinovalo

    \begin{itemize}
    \tightlist
    \item
      Trdelníky (Prügelkrapfen)
    \item
      Vezmi 24 vaječných žloutků do hrnce a otluč je trochu, dej funt
      potlučeného cukru do toho, tluč to ještě alespoň půl hodiny dobře,
      udělej z 12 vaječných bílků tuhý sníh, dej ho zponenáhla do toho,
      ale jen tolik najednou, co se ze dvou vaječných bílků vymíchá, a
      míchej to ještě jednu celou hodinu.
    \item
      Naposledy dej ještě trochu utlučené vanilie do toho, jak mnoho
      vaječncý žloutků vemeš, tolik loftů přesívané škrobové moky musíš
      vzíti. Když tedy ta mouka v tom jest, tak to zamíchej jen tak
      mnoho, aby to všecko dohromady přišlo.
    \item
      Omotej špagát silně na to trdlo a toč ho při ohni dokola, až jak
      náleží teplé jest. Vlej potom těsto na něj, napřed ho nech dobře
      spustit, aby dolů nespadlo. Když jsi ho již dvakrát polil, tak ho
      vem od ohně, ale ten spodní díl polívej třikrát. Když ten spodní
      díl po třetí políváš, tak postav něco před oheň, aby se ten vrchní
      díl nespálil, a okrášli ho potom s ledem.
    \item
      Můžeš taky Erdáple(???) z toho udělat, anebo také Sfruc
      (Struc???), ten musí ale něco tenčí dělaný býti.
    \item
      takže je to v podstatě baumkuchen
    \end{itemize}
  \item
    recept přečtenej pomocí Gemini, je to mnohem lepší než ChatGPT:

    \begin{itemize}
    \tightlist
    \item
      Trdelníky (Prügelkrapfen) Vezmi 24 žloutků z vajec do hrnce,
      trochu je rozšlehej a přidej k nim rozemletý cukr. Šlehej to ještě
      půl hodiny. Poté přidej tuhý sníh z dvanácti bílků, přidávej ho
      postupně, ale vždy jen tolik, kolik se vytvoří ze dvou bílků.
      Šlehej to ještě celou hodinu. Nakonec přidej ještě trochu mleté
      vanilky a tolik mouky, kolik váží žloutky. Mouku musíš prosít.
      Když je mouka v těstě, zamíchej to dohromady, aby se všechny
      ingredience spojily. Silně omotej provázek kolem trdla a na ohni
      ho otáčej, dokud nebude teplé. Poté na něj navrstvi těsto a nech
      ho dobře oschnout, aby nespadlo. Když jsi ho dvakrát polil, sundej
      ho z ohně, ale spodní část ještě chvíli nech na ohni.
    \item
      Spodní díl polij třikrát. Když ten spodní díl poliješ potřetí, tak
      postav něco před oheň, aby se ten vrchní díl nespalil, a ozdob ho
      potom polevou. Můžeš z toho také udělat bramborové trdelníky, nebo
      také {[}něco jako{]} štolu, ta ale musí být udělaná o něco tenčí.
    \end{itemize}
  \item
    Víc informací o knížce je v
    \href{https://kulturni-dejiny.slu.cz/data/uploads/067/upvysledky/pamet_chuti_odborna-kniha_uplatnny_2020.pdf}{Paměti
    chuti}:

    \begin{itemize}
    \tightlist
    \item
      je to asi překlad z němčiny, češtinu to má dost špatnou, jako by
      to nepřekládal rodilej mluvčí
    \item
      používá moravský slova a fonetickej přepis cizích slov
    \item
      autorka byla asi z Rakouska, z okolí Grazu
    \item
      jejich přepis:

      \begin{itemize}
      \tightlist
      \item
        Vraz 24 vaječných žloutků do hrnce, a otluč je trochu dej funt
        potlučeného cukru do toho, tluč to ještě ale půlhodiny dobře,
        udělej ze dvanácti vaječných bílků tuhý sníh, dej ho zponenáhla
        do toho ale jenomroli na jedenkráte, co ze dvauch vajec bílek
        vynáší, a míchej to ještě jednu celau hodinu. Naposledy dej
        ještě trochu potlučenej vynýlie do toho, jak mnoho vaječných
        žlautků vemeš, tolik lotů musíš přeosívanej škrobovej mauky
        vzíti.
      \item
        zbytek receptů nepřeložili. ale zní to pochopitelnějc, než to,
        co vytvořil chatgpt
      \end{itemize}
    \end{itemize}
  \end{itemize}
\item
  1821
  \href{https://bara.ujc.cas.cz/slovniky/dobrovsky/dobrovsky587.html}{Josef
  Dobrovský: Deutsch-böhmisches Wörterbuch, Praha 1821}:

  \begin{itemize}
  \tightlist
  \item
    jen slovníkový heslo: Spießkuchen, m. wagečnjk, sl. trdelnjk,
    obelia.
  \item
    další
    \href{https://bara.ujc.cas.cz/slovniky/dobrovsky/dobrovsky195.html}{heslo}:

    \begin{itemize}
    \tightlist
    \item
      Eyerkuchen, m. Eyerfladen, Eyerplatz, ſwjtek z wagec.
    \end{itemize}
  \end{itemize}
\item
  1821
  \href{https://ndk.cz/view/uuid:46b33740-ac9f-11dc-bcf0-000d606f5dc6?page=uuid\%3A52eb1c50-0ecb-11e8-bdb0-005056827e51&fulltext=trdelnjk}{Böhmisch-deutsch-lateinisches
  Wörterbuch}:

  \begin{itemize}
  \tightlist
  \item
    vyšlo v Bratislavě, autor Palkovič, Juraj
  \item
    heslo odkazuje na Spiesskuchen a prügelkrapfen
  \item
    Trdlo, a, n. Stoßel, m. pistillum; 2) {[}. nemeblo, D. 3) (sl.)
    holzerner Bratspieß, zum Backen der Spießkuchen, veru ligneum
    cylindriforme.
  \item
    Trdlo, a, n. Palička/hmoždíř, m. pestík; 2) {[}. nemeblo, D. 3)
    (sl.) dřevěný pečený špíz, k pečení špízových koláčů, pravý,
    dřevěný, válcovitého tvaru.
  \end{itemize}
\item
  1825
  \href{https://www.google.cz/books/edition/Slow\%C3\%A1r_Slowensk\%C3\%AD_\%C4\%8Cesko_La\%C5\%A5insko_\%C5\%87em/Sd1bAAAAcAAJ?hl=cs&gbpv=1&dq=trdeln\%C3\%ADk&pg=PA1051&printsec=frontcover}{Slowár
  Slowenskí Česko-Laťinsko-Ňemecko-Uherskí}:

  \begin{itemize}
  \tightlist
  \item
    Anton Bernolák
  \item
    s. 1051
  \item
    † Kotúč, é, m. v. Karika. 2) v. Kolečko (Taličři) 7 Nro. 3) v.
    Trdelňík. 4) v. Spricla, Sprickrafel, Spříckrafla
  \item
    Kotúlka, i. f. v. Karika 2) v. Trdelňík
  \end{itemize}
\item
  1838:
  \href{https://ndk.cz/uuid/uuid:b6d79670-8232-11dc-abad-000d606f5dc6}{Slownjk
  česko-německý Josefa Jungmanna}:

  \begin{itemize}
  \tightlist
  \item
    na straně 624 docela přesnej popis, zmiňuje alternativní názvy
    vagečnjk (vaječník?) a kotúč
  \item
    trdelnjk, u, m, (r. trdlo, protože na trdle, t. dřewěném rožni se
    pekau), pečivo z mauky, ginak: wagečnjk, slc. též kotúč, Ringel-
    ober Spieß- ober Prügelkrapfen, Ringkuchen, obelia, D. torta
    annularia, Brn. placenta in veru ligneo cocta.
  \item
    pak následujou ukázky z Múz morawských
  \end{itemize}
\item
  1839:
  \href{https://vokabular.ujc.cas.cz/moduly/slovniky/digitalni-kopie-detail/JgSlov05/strana-9}{Slovník
  česko-německý, Díl V}:

  \begin{itemize}
  \tightlist
  \item
    strana 9, heslo wagečnjk: koláč vaječný (Ros.) jest pečité jídlo,
    těsto na koláče se rozválí, při plamenu na způsob kuřete se otáčí a
    peče, pak cukrem posypané se jedí, Us.

    \begin{itemize}
    \tightlist
    \item
      značka Ros. znamená
      \href{https://bara.ujc.cas.cz/slovniky/rosa/rosafst1480.html}{Václav
      Jan Rosa: Thesaurus linguae Bohemicae} - nedokončený dílo

      \begin{itemize}
      \tightlist
      \item
        žil 1630-1689
      \item
        Wagečnjk, (wagečný kołáč) placenta ex ouis.
      \end{itemize}
    \item
      zkratka Us. znamená, že to spojení pochází z běžný řeči a ne z
      písemných pramenů
    \end{itemize}
  \item
    mor: trdelnjky, že se na trdle pekau
  \item
    slc: mrwaň, obeliae
  \item
    Tr. o tlustých dětech: to djtě má wagečnjky, t. záhyby, neboli faldy
    na rukau, na stehnech; protože to podobu má s oněmi wagečnjky
  \item
    ale taky koláč na pánvi nebo rendlíku z vajec a mouky pečený, svitek
  \item
    vysoký koláč v prostřed děravý, po stranách obyčejně vroubkovaný -
    německy Scherbenkuchen, Kugelhupf:

    \begin{itemize}
    \tightlist
    \item
      to nebude trdelník, ale bábovka
    \end{itemize}
  \end{itemize}
\item
  1845
  \href{https://www.google.cz/books/edition/Ouplny_Kapesni_slownik_ceshoslowanskeho/i8dUAAAAcAAJ?hl=cs&gbpv=1&dq=trdeln\%C3\%ADk&pg=PA513&printsec=frontcover}{Ouplny\_Kapesni\_slownik\_ceshoslowanskeho}:

  \begin{itemize}
  \tightlist
  \item
    Johann Nepomuk Konecny
  \item
    s. 513 Trdelník - Ringel-, prugelkrapfen
  \item
    s. 539 Waječník = Eierstock, Eierkuchen
  \end{itemize}
\item
  1845
  \href{https://ceskadigitalniknihovna.cz/view/uuid:f683dda0-81be-11de-9a7b-000d606f5dc6?page=uuid:5a5cbb90-7ff7-11de-8619-000d606f5dc6&fulltext=trdeln\%C3\%ADk\%20OR\%20trdeln\%C3\%ADky\%20OR\%20trdeln\%C3\%ADku\%20&source=cbvk}{Časopis
  českého Museum}:

  \begin{itemize}
  \tightlist
  \item
    slovníček slovenskejch slov
  \item
    trdlo = lipový válec, na němž se při končinách trdelníky pečou.
    Těsto se roztáhne v dlouhý provaz a namotá se na trdlo, to je
    trdelník, Spiesskuchen. Skal.
  \end{itemize}
\item
  1853
  \href{https://www.google.cz/books/edition/Taschen_W\%C3\%B6rterbuch_der_b\%C3\%B6hmischen_und/tDlFAAAAYAAJ?hl=cs&gbpv=1&dq=trdeln\%C3\%ADk&pg=PA462&printsec=frontcover}{Taschen-Wörterbuch
  der böhmischen und deutschen Sprache}

  \begin{itemize}
  \tightlist
  \item
    s. 462
  \item
    Josef Franta Šumavský, Josef Rank
  \item
    Spies-krapfen, -kuchen = vaječník, trdelník
  \item
    existuje spousta dalších vydání, třeba z roku 1846, kde je to samý
    heslo
  \end{itemize}
\item
  1855
  \href{https://ceskadigitalniknihovna.cz/uuid/uuid:2f58bcf0-f6ae-11dd-8224-000d606f5dc6}{Babička}:

  \begin{itemize}
  \tightlist
  \item
    Božena Němcová
  \item
    pečený vaječník, nejsou tuze kalé, ale není jasný, co to vlastně je
  \end{itemize}
\item
  1860
  \href{https://ceskadigitalniknihovna.cz/uuid/uuid:f5239995-7f7b-4b26-8ceb-d90eebcd974c}{Kapesní
  slovník jazyka českého a německého}:

  \begin{itemize}
  \tightlist
  \item
    Rank, Josef, Franta Šumavský, Josef
  \item
    Spiesskrapfen, Spiesskuchen = vaječník, trdelník
  \item
    s. 193: Eierkuchen = svítek, slitek, litá bába, vaječník
  \item
    Prügelkrapfen ani Baumkuchen nemají hesla
  \end{itemize}
\item
  1864
  \href{https://dikda.snk.sk/uuid/uuid:43f40487-25fd-438a-b8b6-e0c25985b2c7}{Zornička}:

  \begin{itemize}
  \tightlist
  \item
    Slovensko
  \item
    hra Korheľ a jeho nebožiatka, Josef Podhradský
  \item
    švagři popíjejí a jeden řiká, že jejich prapředkyně spolu chodily k
    Trdelníkom na priadky, takže se musejí napít
  \item
    to vypadá, že Trdelníkové bude něčí jméno
  \end{itemize}
\item
  1865
  \href{https://www.google.cz/books/edition/Moravske_pohadkya_povesti_M\%C3\%A4hrische_M\%C3\%A4/42qLW6-LrScC?hl=cs&gbpv=1&pg=PA139&printsec=frontcover}{Moravské
  pohádky a pověsti}:

  \begin{itemize}
  \tightlist
  \item
    Nemírnost v jídle ženské hanácké osoby
  \item
    přepis básničky od Gallaše z Muz Morawských
  \end{itemize}
\item
  1865
  \href{https://ceskadigitalniknihovna.cz/uuid/uuid:a738d13c-8062-4352-9707-b5ae338b4873}{Nový
  slovník kapesní jazyka českého i německého}:

  \begin{itemize}
  \tightlist
  \item
    Rank, Josef
  \item
    náčepka = der Prügelkrapfen, (e. Speise)
  \item
    s. 834: trdelník = Ringelkarpfen, Spiesskr.
  \item
    s. 874: vaječník = Eierstoff; Spießkuchen, Spießkrapfen; Eierkuchen,
    Pfannkuchen, der Scherbenkuchen
  \end{itemize}
\item
  1875
  \href{https://www.google.cz/books/edition/Slavia/WrJiAAAAcAAJ?hl=cs&gbpv=1&dq=trdeln\%C3\%ADk&pg=PA312&printsec=frontcover}{Slavia
  - románové listy}:

  \begin{itemize}
  \tightlist
  \item
    Schovanka, původní román A. H. Sokola
  \item
    doktor vyšetřuje holčičku Amálku, její teta ho zve na trdelník,
    zvláštní jídlo z mouky, jen jedna z tisíce kuchařek ho umí dělati
  \end{itemize}
\item
  1878
  \href{https://www.google.cz/books/edition/\%C4\%8Cesko_n\%C4\%9Bmeck\%C3\%BD_slovn\%C3\%ADk_zvl\%C3\%A1\%C5\%A1t\%C4\%9B_gra/4r00AQAAMAAJ?hl=cs&gbpv=1&dq=trdeln\%C3\%ADk&pg=PA586&printsec=frontcover}{Česko-německý
  slovník zvláště grammaticko-fraseologický}:

  \begin{itemize}
  \tightlist
  \item
    František Štěpán Kott
  \item
    heslo Pletenec:

    \begin{itemize}
    \tightlist
    \item
      Pletenec, nce, pletének, nku, m., pletený koláč, věnec, pečivo, na
      Mor. trdelník, (Bž.), šestinedělkám do kouta posílané, calta,
      kolenč, Kranzsemmel, Flechte, f., geflochtener Kuchen, Strizel, m.
      -\/- P., koš. Plk. -\/- P. bavlněný, Baumwollzopf, m. Dch. -\/- P.
      fíků, Feigenbündel, n. Mřk.
    \item
      v jinejch zdrojích je pletenec spíš vánočka, takže to vypadá na
      omyl
    \end{itemize}
  \item
    heslo Nalévanec:

    \begin{itemize}
    \tightlist
    \item
      Nalévanec, nce, m. vaječník, eine Mehlspeise. Na Mor.
    \item
      Mehlspeise znamená moučník
    \item
      zní to jako prugelkrapfen? Ale taky to může bejt lívanec.
    \end{itemize}
  \end{itemize}
\item
  1882
  \href{https://ceskadigitalniknihovna.cz/uuid/uuid:ee6245d1-435d-11dd-b505-00145e5790ea}{Světozor}

  \begin{itemize}
  \tightlist
  \item
    článek Návštěvy: Ze života lidu na jižní Moravě, Jan Herben:
  \item
    historka, jak si půjčou kolíky na pečení trdelníků
  \item
    Stará Ves - fiktivní obec v srdci Slovácka, založená na Brumovicích
  \item
    stejná historka, jako v Moravských obrázcích z roku 1889
  \end{itemize}
\item
  1883
  \href{https://www.google.cz/books/edition/Hand_W\%C3\%B6rterbuch_der_b\%C3\%B6hmischen_und_deu/c6KWlveere4C?hl=cs&gbpv=1&dq=trdeln\%C3\%ADk&pg=PA452&printsec=frontcover}{Hand-Wörterbuch
  der böhmischen und deutschen Sprache }

  \begin{itemize}
  \tightlist
  \item
    s. 452
  \item
    Spiesskrapfen = m. vaječník, trdelník
  \end{itemize}
\item
  1884
  \href{https://ndk.cz/uuid/uuid:78814bf1-f124-4c26-99b4-3ce6ef0458ea}{Česko-německý
  slovník zvláště grammaticko-fraseologický}:

  \begin{itemize}
  \tightlist
  \item
    Kott, František
  \item
    vaječník, na sl. též kotúč, eine Eierspeise, der Ringel-,
    Spiesskrapfen, Ringkuchen
  \end{itemize}
\item
  1886
  \href{https://www.google.cz/books/edition/Dialektologie_moravsk\%C3\%A1/IWowAQAAMAAJ?hl=cs&gbpv=1&dq=trdeln\%C3\%ADk&pg=RA1-PA484&printsec=frontcover}{Dialektologie
  moravská}:

  \begin{itemize}
  \tightlist
  \item
    František Bartoš
  \item
    s. 484
  \item
    trdeľník: těsto se roztáhne v dlouhý provaz, který se otočí o trdlo
    a upeče (val.)
  \item
    takže Valašsko?
  \end{itemize}
\item
  1887
  \href{https://ndk.cz/view/uuid:58907250-abd8-11dd-9aac-000d606f5dc6?page=uuid\%3Aadeec330-246a-11e9-bc55-5ef3fc9bb22f&fulltext=trdeln\%C3\%ADky}{Časopis
  Vlasteneckého spolku musejního v Olomouci}, roč. 4

  \begin{itemize}
  \tightlist
  \item
    starohanácká kuchyně v Němčicích nad Hanou - trdelníky na rožni a v
    peci pečené, vysmažované
  \item
    byla to exkurze vlasteneckýho spolku, jak se žilo za starých časů
  \end{itemize}
\item
  1887
  \href{https://ceskadigitalniknihovna.cz/uuid/uuid:cac319e0-dea6-11e6-9964-005056825209}{Nový
  slovník kapesní jazyka českého i německého dle}:

  \begin{itemize}
  \tightlist
  \item
    Rank, Karel
  \item
    Náčepka = Prügelkrapfen
  \item
    Trdelník = Ringelkarpfen, Spiesskarpfen
  \item
    Vaječník = Eierkuchen, Spiesskuchen, Spiesskrapfen, der Pfannkuchen,
    Scherbenkuchen
  \end{itemize}
\item
  1888
  \href{https://ndk.cz/view/uuid:8d89f8c0-abdb-11dd-ae2a-000d606f5dc6?page=uuid\%3Afbdf17f0-247b-11e9-90cf-5ef3fc9bb22f}{Časopis
  Vlasteneckého spolku musejního v Olomouci}, roč. 5,

  \begin{itemize}
  \tightlist
  \item
    popis svatby v Ořechově u Brna
  \end{itemize}
\item
  1889
  \href{https://ndk.cz/view/uuid:df0f4040-895f-11dd-a7d7-000d606f5dc6?page=uuid\%3A1bb49280-82c0-11e7-94b3-005056825209&fulltext=trdeln\%C3\%ADky}{Moravské
  obráky}:

  \begin{itemize}
  \tightlist
  \item
    Jan Herben - jen zmínka o navinutých trdelnících
  \end{itemize}
\item
  1889
  \href{https://www.digitalniknihovna.cz/vkol/uuid/uuid:3a12df2c-c1ab-42a4-a423-5746952a0111}{Hlas}:

  \begin{itemize}
  \tightlist
  \item
    pomlázka Vesny v Besedním domě (v Brně?)
  \item
    hanácké buchty, trdelníky a další dobroty
  \end{itemize}
\item
  1889
  \href{https://dikda.snk.sk/view/uuid:1ecf9029-7613-401d-843c-e2938a484970?page=uuid:43b26cf7-af4a-4231-8901-ba41c009c9e7&fulltext=trdeln\%C3\%ADk\%20OR\%20trdeln\%C3\%ADka\%20OR\%20trdeln\%C3\%ADku}{Slovenská
  pohľady}:

  \begin{itemize}
  \tightlist
  \item
    Slovensko, popis archeologický lokality, že nějaký bronzový "pušky"
    se dají roztáhnout jako trdelník (drúkový koláč)
  \end{itemize}
\item
  1890
  \href{https://ndk.cz/view/uuid:18b40310-07ad-11dd-8ccf-000d606f5dc6?page=uuid\%3Aae6c9544-ef09-495a-bf20-9e525fdb15b2&fulltext=Pr\%C3\%BCgelkrapfen}{Česko-německý
  slovník zvláště grammaticko-fraseologický}:

  \begin{itemize}
  \tightlist
  \item
    Kott, František Štěpán
  \item
    Prügelkrapfen se překládá jako náčepka, další alternativní název? V
    zásadě pod tim termínem nejde nic vyhledat.
  \end{itemize}
\item
  1891
  \href{https://ceskadigitalniknihovna.cz/uuid/uuid:04e43cad-32f0-11de-992b-00145e5790ea}{Moravská
  orlice}:

  \begin{itemize}
  \tightlist
  \item
    s. 2, pomlázka vesnina, z pečiva budou bábovky, mazance, koláče,
    trdelníky, koblihy aj.
  \end{itemize}
\item
  1891
  \href{https://ceskadigitalniknihovna.cz/view/uuid:04e48ad6-32f0-11de-992b-00145e5790ea?page=uuid:27df031c-32f0-11de-992b-00145e5790ea&fulltext=vaje\%C4\%8D*\%20trdel*&source=mzk}{Moravská
  orlice}:

  \begin{itemize}
  \tightlist
  \item
    Kosauerová darovala mnoho trdelníkův a vajec na pomlázku Vesninu
  \end{itemize}
\item
  1891
  \href{https://www.google.cz/books/edition/Liter\%C3\%A1rne_listy/iTmko65VuoQC?hl=cs&gbpv=1&dq=trdeln\%C3\%ADk&pg=RA5-PA64&printsec=frontcover}{Literárne
  listy}:

  \begin{itemize}
  \tightlist
  \item
    seznam slov z Bystrickej doliny, které se v jiných místech Slovenska
    často nevyskytují
  \item
    trdelník = druh koláčov
  \end{itemize}
\item
  1892
  \href{https://ceskadigitalniknihovna.cz/uuid/uuid:852faf14-821a-11e0-b92b-0050569d679d}{Našinec}:

  \begin{itemize}
  \tightlist
  \item
    vzpomínka na národopisnou výstavku v Němčicích na Hané z 21. srpna
    1887
  \item
    popisujou akci co už popisoval Časopis vlasteneckého spolku
    musejního v Olomouci
  \item
    trdelníky na rožni a v peci pečené, vysmažované a spousta dalších
    druhů pečiva
  \end{itemize}
\item
  1892
  \href{https://ceskadigitalniknihovna.cz/uuid/uuid:0869a320-dc1a-11e6-9e7e-001018b5eb5c}{Neues
  Taschenwörterbuch der böhmischen und deutschen Sprache}:

  \begin{itemize}
  \tightlist
  \item
    Rank, Josef
  \item
    Spiesskrapfen, Spiesskuchen = vaječník, trdelník
  \item
    Prügelkrapfen nemá heslo
  \end{itemize}
\item
  1892
  \href{https://www.google.cz/books/edition/Komensk\%C3\%BD/twVGAQAAMAAJ?hl=cs&gbpv=1&dq=trdeln\%C3\%ADk&pg=RA2-PA181&printsec=frontcover}{Komenský}:

  \begin{itemize}
  \tightlist
  \item
    příspěvky na pomník Komenského v Uh. Brodě, v Nivnici se vydražil
    trdelník za 1. zl 20 kr.
  \end{itemize}
\item
  1892 -
  \href{https://ndk.cz/view/uuid:289bb2c0-9f1e-11dc-bf2d-000d606f5dc6?page=uuid\%3Aca676860-e56d-11e7-8cdd-5ef3fc9bb22f&fulltext=trdeln\%C3\%ADky}{Moravská
  svatba}

  \begin{itemize}
  \tightlist
  \item
    František Bartoš
  \item
    popisuje městskou svatbu v Bzenci
  \item
    trvá celý týden
  \item
    ve čtvrtek se mládenci sejdou u družby a s družkami a hudbou jdou k
    prvnímu svatebnímu hostovi po cestě, kde dostanou koblihy,
    trdelníky, uzeniny a víno
  \end{itemize}
\item
  1892
  \href{https://ceskadigitalniknihovna.cz/uuid/uuid:04f469ce-32f0-11de-992b-00145e5790ea}{Moravská
  orlice}:

  \begin{itemize}
  \tightlist
  \item
    Pomlázka vesnina VII
  \item
    trdelníky darovala rodina Bartůňkova z Letovic
  \end{itemize}
\item
  1892
  \href{https://www.digitalniknihovna.cz/vkol/uuid/uuid:db92b18a-5f60-46e9-a666-022fa9c0f4f6}{Pozor}:

  \begin{itemize}
  \tightlist
  \item
    národopisná výstava v Tovačově, mezi kuchyňským vybavením trdelník
    (očividně rožeň)
  \end{itemize}
\item
  1893
  \href{https://ceskadigitalniknihovna.cz/uuid/uuid:765160d0-b6d7-11ea-b68c-005056827e52}{Národopisná
  výstava českoslovanská v Praze}:

  \begin{itemize}
  \tightlist
  \item
    národopisná výstavka v Ořechově, měli tam svatební koláč, trdelníky,
    makovňáky, trnáčky, kmiňáky, frgály a boží milosti
  \end{itemize}
\item
  1893
  \href{https://ceskadigitalniknihovna.cz/view/uuid:04fd1c48-32f0-11de-992b-00145e5790ea?page=uuid:298ab776-32f0-11de-992b-00145e5790ea&fulltext=vaje\%C4\%8D*\%20trdel*&source=mzk}{Moravská
  orlice}:

  \begin{itemize}
  \tightlist
  \item
    na pomlázku vesninu darovala paní Hálová mísu trdelníků
  \end{itemize}
\item
  1893
  \href{https://ceskadigitalniknihovna.cz/view/uuid:04ff8d70-32f0-11de-992b-00145e5790ea?page=uuid:299cdfa7-32f0-11de-992b-00145e5790ea&fulltext=trdeln\%C3\%AD*&source=mzk}{Moravská
  orlice}:

  \begin{itemize}
  \tightlist
  \item
    sokolská slavnost v Ivančicích, ivančická vzdělávací jednota Vesna
    přispěla jídlem, třeba pravými slovanskými trdelníky
  \end{itemize}
\item
  1893
  \href{https://www.digitalniknihovna.cz/vkol/uuid/uuid:7584cb6d-4978-4820-9af6-7d206917db08}{Moravské
  noviny}:

  \begin{itemize}
  \tightlist
  \item
    Vesnina pomlázka v Brně, národní trdelníky od pí. Páralové z
    Ořechova
  \end{itemize}
\item
  1894 -
  \href{https://ndk.cz/view/uuid:04ec7ad2-32f0-11de-992b-00145e5790ea?page=uuid\%3A81b0ec10-3115-11e8-b257-005056825209&fulltext=trdeln\%C3\%ADk}{Moravská
  orlice}:

  \begin{itemize}
  \tightlist
  \item
    Národopisná výstavka ve Šlapanicích
  \item
    v sále s nádobím a nářadím domácností v okolí Šlapanic je trdelník i
    s hotovým výrobkem
  \end{itemize}
\item
  1894
  \href{https://ceskadigitalniknihovna.cz/view/uuid:04ecc8f4-32f0-11de-992b-00145e5790ea?page=uuid:2a2ac745-32f0-11de-992b-00145e5790ea&fulltext=trdeln\%C3\%AD*&source=mzk}{Moravská
  orlice}:

  \begin{itemize}
  \tightlist
  \item
    další článek o národopisný výstavce ve Šlapanicích, hody, jsou tam
    pečený koláče, buchty, trdelníky a smažený závitky hoblovanky
  \end{itemize}
\item
  1894
  \href{https://ceskadigitalniknihovna.cz/uuid/uuid:b8222e44-435d-11dd-b505-00145e5790ea}{Lidové
  noviny}:

  \begin{itemize}
  \tightlist
  \item
    s. 3, další článek o výstavě ve Šlapanicích
  \end{itemize}
\item
  1894
  \href{https://ndk.cz/uuid/uuid:ab9cd010-b908-11dc-83b0-000d606f5dc6}{Všeobecný
  výstavní trh hospodářský}:

  \begin{itemize}
  \tightlist
  \item
    katalog vystavovatelů
  \item
    Votruba, Karel, cukrář v Praze, Ferdinandova třída: cukrářské
    výrobky a trdlovce
  \end{itemize}
\item
  1894
  \href{https://ceskadigitalniknihovna.cz/uuid/uuid:c041b7c0-a37f-11de-95bf-000d606f5dc6}{Velehrad}:

  \begin{itemize}
  \tightlist
  \item
    Národopisná výstava v Kroměříži, mezi darovaným jídlem byla místa
    trdelníků od paní Hejné
  \end{itemize}
\item
  1894
  \href{https://ceskadigitalniknihovna.cz/uuid/uuid:a0df34ac-8228-11e0-b92b-0050569d679d}{Našinec}:

  \begin{itemize}
  \tightlist
  \item
    ženský vzdělávací spolek Vlasta v Přerově měl za úkol pořádání
    loterie věcné a bufetu na velké národní slavnosti "Radhošťské" v
    Michalově 19. srpna
  \item
    paní Leciánová věnovala trdelníky a koláče
  \end{itemize}
\item
  1895 -
  \href{https://ceskadigitalniknihovna.cz/view/uuid:a69c34d0-f3fe-11dc-a2df-000d606f5dc6?page=uuid\%3Aa5170ee0-7b0a-11e6-ad48-005056825209&fulltext=trdeln\%C3\%ADk\%20OR\%20trdeln\%C3\%ADky\%20OR\%20trdeln\%C3\%ADk\%C5\%AF&source=nkp}{
  Národopisná výstava českoslovanská v Praze 1895}:

  \begin{itemize}
  \tightlist
  \item
    s. 109. trdelníky mezi svátečním pečivem, spolu s božími milostmi,
    koblihama
  \item
    popisuje horácký grunt, po vzoru Sazomína u Velkýho Meziříčí
  \end{itemize}
\item
  1895
  \href{https://ceskadigitalniknihovna.cz/uuid/uuid:a7ab5cf0-1dae-11e7-a38c-005056827e51}{Německočeský
  slovník}:

  \begin{itemize}
  \tightlist
  \item
    Prügelkrapfen = víd. náčepky
  \item
    Ringkuchen = trdelník, koláč
  \item
    Ringelkrapfen = trdelník, vaječník
  \item
    Spiesskrapfen, Spiesskuchen = m. vaječník, trdelník
  \end{itemize}
\item
  1895
  \href{https://ceskadigitalniknihovna.cz/uuid/uuid:67436920-adcf-11ed-9763-5ef3fc9bb22f}{Národopisná
  výstava československá: výstavní reperoir Českých novin}:

  \begin{itemize}
  \tightlist
  \item
    trdelníky, koule a houby z Klobouk u Brna
  \end{itemize}
\item
  1895
  \href{https://ceskadigitalniknihovna.cz/view/uuid:0519cc62-32f0-11de-992b-00145e5790ea?page=uuid:2a76016a-32f0-11de-992b-00145e5790ea&fulltext=trdel*&source=mzk}{Moravská
  orlice}:

  \begin{itemize}
  \tightlist
  \item
    pomlázka Vesny v Brně, sl. Páralová z Ořechova darovala trdelníky
  \end{itemize}
\item
  1895
  \href{https://ceskadigitalniknihovna.cz/view/uuid:051a1a8d-32f0-11de-992b-00145e5790ea?page=uuid:2a7c1c1f-32f0-11de-992b-00145e5790ea&fulltext=trdeln\%C3\%AD*&source=mzk}{Moravská
  orlice}:

  \begin{itemize}
  \tightlist
  \item
    Národopisná výstava českoslovanská
  \item
    brněnsko na výstavě, světnička z podhorácka, mají tam cukrové kule,
    trdelíky a huby z Klobouk a ze Šlapanic
  \end{itemize}
\item
  1895
  \href{https://ceskadigitalniknihovna.cz/uuid/uuid:f6731d10-a2e1-11e7-a093-005056825209}{Věstník
  Ústřední Matice Školské}:

  \begin{itemize}
  \tightlist
  \item
    1. září, Klobouky u Brna, výlet a zábava spojená s bufetem
  \item
    paní Horáková darovala chléb a trdelník
  \end{itemize}
\item
  1895
  \href{https://www.digitalniknihovna.cz/vkol/uuid/uuid:a7f4aea1-a2cd-4d9f-a1fa-b43aa3c25dff}{Dačické
  listy}:

  \begin{itemize}
  \tightlist
  \item
    národopisná výstava v Dačicích, trdelník mezi vybavením kuchyně v
    selský světnici
  \end{itemize}
\item
  1895 -
  \href{https://ndk.cz/view/uuid:77d41380-a2e4-11e7-a093-005056825209?page=uuid\%3A9cb54070-a343-11e7-a093-005056825209&fulltext=trdeln\%C3\%ADk}{Věstník
  Ústřední Matice Školské}:

  \begin{itemize}
  \tightlist
  \item
    farnost z Těšetic odevzdala Matici školské výtěžek ze silvesrovkýho
    večírku 1895
  \item
    Ústin u Těšetic, pár kilometrů na západ od Olomouce
  \item
    \href{https://ndk.cz/view/uuid:eb8e3d99-435d-11dd-b505-00145e5790ea?page=uuid\%3A04295899-610e-49f5-b5b3-35f73d41b05d&fulltext=trdeln\%C3\%ADk}{tady
    je ta samá akce popsaná detailnějc}
  \end{itemize}
\item
  1895
  \href{https://ceskadigitalniknihovna.cz/uuid/uuid:6ef09ed4-435f-11dd-b505-00145e5790ea}{Národní
  listy}:

  \begin{itemize}
  \tightlist
  \item
    reklama cukráře Karla Votruby na Ferdinandově třídě
  \item
    trdlovec od 3 zlatých
  \end{itemize}
\item
  1895
  \href{https://dikda.snk.sk/uuid/uuid:dde694e6-6191-4983-bed1-e6328f0c9bf4}{Literárne
  listy}

  \begin{itemize}
  \tightlist
  \item
    Slovensko, slovníkový heslo, trdelník druh koláčov
  \item
    slovník speciálních slov od Bánskej Bystrice
  \end{itemize}
\item
  1895
  \href{https://ceskadigitalniknihovna.cz/uuid/uuid:22933190-6ef8-11ed-a5ba-005056827e51}{Lidové
  noviny}:

  \begin{itemize}
  \tightlist
  \item
    Brněnsko na výstavě Národopisné v Praze
  \item
    popisují, co se vystvovalo na výstavě v Praze z Brněnska
  \item
    s. 2 na stole stojí pečivo - trdelníky, kule a huby, kterými měla
    uctívati babička vážené kmotry
  \end{itemize}
\item
  1895
  \href{https://ceskadigitalniknihovna.cz/uuid/uuid:f392b3ef-435d-11dd-b505-00145e5790ea}{Národní
  listy}:

  \begin{itemize}
  \tightlist
  \item
    povídka Dárek k Ježíšku, Josef Kořínek
  \item
    kluk na Slovácku se dvoří dívce, padají takové hubičky sladké, že by
    mohli ocukrovat košatinu trdelníků
  \item
    jsem si jistej, že stejnou povídku už jsem viděl jinde
  \end{itemize}
\item
  1896
  \href{https://ceskadigitalniknihovna.cz/uuid/uuid:b805f3a6-435d-11dd-b505-00145e5790ea}{Lidové
  noviny}:

  \begin{itemize}
  \tightlist
  \item
    Výroční výstava dívčích škol Vesny v Brně
  \item
    už 10. výroční výstava
  \item
    s. 2: měli tam kuchařskou výstavu s oddělením lidového pečiva, s
    chvalně známými trdelníky
  \end{itemize}
\item
  1897
  \href{https://www.digitalniknihovna.cz/vkol/uuid/uuid:1a1f3964-02ac-463e-8a5b-8d7577a0ad33}{Moravské
  noviny}:

  \begin{itemize}
  \tightlist
  \item
    školní zkoušky v Křižanovicích u Slavkova, za odměnu byly slavnosti
    se spoustou jídla, i trdelníky
  \end{itemize}
\item
  1897
  \href{https://ceskadigitalniknihovna.cz/uuid/uuid:c3a426e0-2b81-11e7-a77b-001018b5eb5c}{Chrudim}:

  \begin{itemize}
  \tightlist
  \item
    reklama na cukráře V. Uricha v Pardubicích, dělá různý dorty,
    mimojiný i Baumkuchen
  \end{itemize}
\item
  1898
  \href{https://ceskadigitalniknihovna.cz/view/uuid:38bef4ba-03a2-433c-a4ed-a9914d68b1a5?page=uuid\%3A5c5ae01f-9a43-4111-b72b-1cfe6710f70a&fulltext=trdeln\%C3\%ADk\%20OR\%20trdeln\%C3\%ADky\%20OR\%20trdeln\%C3\%ADk\%C5\%AF&source=knav}{Český
  lid}:

  \begin{itemize}
  \tightlist
  \item
    článek Národopisné drobty ze Slovenska moravského
  \item
    Karmín (v chorvatských ostadách u Mikulova to znamená pohřební
    hostina, ale tady zabijačka) u sedláka
  \item
    někde si ženský smažej koblihy nebo trdelníky, ale málokde (z
    Velkých Pavlovic)
  \end{itemize}
\item
  1900:
  \href{https://ndk.cz/view/uuid:7d517d10-0583-11dd-85d4-000d606f5dc6?page=uuid\%3A445fd636-253c-4b04-b533-3b0a96e7e0b4}{
  Lidské dokumenty a jiné národopisné poznámky z několika jihomoravských
  dědin}:

  \begin{itemize}
  \tightlist
  \item
    Augusta Šebestová, národopisný materiál z Kobylího, Vrbice,
    Pavlovic, Bořetic, Němčiček, Rakvic, Žížkova, Čejkovic, Štarviček
  \item
    na straně 110 recept na trdelník nad ohněm, na další straně je pak
    recept na smaženej v sádle
  \item
    s. 12 popis pečiva posílanýho do kouta, trdelník ve třetí zásilce
  \item
    s. 36 popis her, hra na mlsnú kozu
  \item
    s. 113 trdelník se jí na zabijačce
  \end{itemize}
\item
  1900
  \href{https://dikda.snk.sk/uuid/uuid:7f57fb99-b746-49c2-9b31-a09b7a646c02}{Hlas
  : mesačník pre literatúru, politiku a otázku sociálnu}:

  \begin{itemize}
  \tightlist
  \item
    s. 389 Slovensko, ve Skalici a okolí se posílají trdelníky jako
    dárky
  \end{itemize}
\item
  1900
  \href{https://ceskadigitalniknihovna.cz/uuid/uuid:24bda940-8ab0-11ed-8f7f-5ef3fc9bb22f}{Květy}:

  \begin{itemize}
  \tightlist
  \item
    na pokračování vychází Rok na vsi od Aloise Mrštíka, v tomhle čísle
    únor
  \item
    článek Škaredá středa
  \item
    Habrůvka - fiktivní vesnice, založená na obci Diváky u Hustopečí
  \item
    popisují průvod opilých vesničanů po vesnici, kde vyvádí a chodí po
    domech, kde hrajou a tancujou a jedí jídlo

    \begin{itemize}
    \tightlist
    \item
      Všecko se do něho vešlo: i koblihy i trdelníky, jitrnice a celé
      šrútky --- vším chasa brala za vděk
    \end{itemize}
  \end{itemize}
\item
  1897
  \href{https://ceskadigitalniknihovna.cz/uuid/uuid:225c96bf-3ac7-4d81-addb-3ab9562be388}{Archiv
  pro lexikografii a dialektologii}:

  \begin{itemize}
  \tightlist
  \item
    je to digitalizovaný pod Podrobný seznam slov Rukopisu
    Kralodvorského z roku 1897, takže byly svázaný dvě knihy do jednoho
    svazku
  \item
    heslo trdelník:

    \begin{itemize}
    \tightlist
    \item
      těsto se vyválí na provázky, natočí na horký dřevěný trdlo,
      pomastí vejčetem, posype tlučenýma ořechama. Šeb, 110
    \item
      cituje Lidské dokumenty od Šebestové z roku 1900, takže to
      očividně nevyšlo v roce 1897
    \end{itemize}
  \end{itemize}
\item
  1901
  \href{https://ndk.cz/view/uuid:c714e5a0-81eb-11e3-a70e-005056822549?page=uuid\%3A884b7690-8641-11e3-b315-001018b5eb5c&fulltext=trdeln\%C3\%ADky}{Brněnské
  noviny}

  \begin{itemize}
  \tightlist
  \item
    svatební hody v okolí Brna - trdelníky smažený
  \end{itemize}
\item
  1902
  \href{https://ceskadigitalniknihovna.cz/view/uuid:51ede477-8259-11e0-b92b-0050569d679d?page=uuid\%3A540811e9-8259-11e0-b92b-0050569d679d&fulltext=trdeln\%C3\%ADk\%20OR\%20trdeln\%C3\%ADky\%20OR\%20trdeln\%C3\%ADk\%C5\%AF&source=mzk}{Lidové
  drobotiny z Polešovic, časopis Našinec}:

  \begin{itemize}
  \tightlist
  \item
    vyprávění kluka o tom, jak byl populární ve škole, když přinesl
    trdelník, co dostala jeho máma po narození bratra
  \end{itemize}
\item
  1902
  \href{https://ceskadigitalniknihovna.cz/view/uuid:b0836ae0-7a03-11e3-ae4b-001018b5eb5c?page=uuid\%3A5bf570c0-7bd1-11e3-ae4b-001018b5eb5c&fulltext=trdeln\%C3\%ADk\%20OR\%20trdeln\%C3\%ADky\%20OR\%20trdeln\%C3\%ADk\%C5\%AF&source=mzk}{Brněnské
  noviny}:

  \begin{itemize}
  \tightlist
  \item
    Z Juliánova (poděkování) - čajový večírek ve prospěch chudých žáků,
    mezi dary byly trdelníky
  \end{itemize}
\item
  1903
  \href{https://www.digitalniknihovna.cz/vkol/uuid/uuid:b799eafc-435d-11dd-b505-00145e5790ea}{Lidové
  noviny}:

  \begin{itemize}
  \tightlist
  \item
    Ze Židenic, Hody Moravanky pro chudou školní mládež, mezi dary
    trdelníky
  \end{itemize}
\item
  1904
  \href{https://ceskadigitalniknihovna.cz/view/uuid:62df2c30-8367-11e3-a606-005056827e51?page=uuid\%3A3bd12fb0-86e3-11e3-b6b2-005056822549&fulltext=trdeln\%C3\%ADk\%20OR\%20trdeln\%C3\%ADky\%20OR\%20trdeln\%C3\%ADk\%C5\%AF&source=mzk}{Brněnské
  noviny}:

  \begin{itemize}
  \tightlist
  \item
    článek Z útulny ženské v Brně
  \item
    výlet dětí útulny do Tuřan, 80 lidí, dostali koláče, koblihy,
    smažiny, trdelníky atd.
  \end{itemize}
\item
  1904
  \href{https://ceskadigitalniknihovna.cz/uuid/uuid:e6cbfaf4-435d-11dd-b505-00145e5790ea}{Humoristické
  listy}:

  \begin{itemize}
  \tightlist
  \item
    dopis Antonina Jaborskyho
  \item
    zve pana Randu k nim do Plešovca, olizoval by se, kdyby Manda
    napekla křeháče a trdelníky
  \end{itemize}
\item
  1904
  \href{https://ceskadigitalniknihovna.cz/view/uuid:245714d0-e323-11ec-a26a-5ef3fc9bb22f?page=uuid\%3Acacf7f15-f1a5-4cca-a344-05b1ee66b972&fulltext=trdlovec&source=nkp}{Slovník
  umění kuchařského}:

  \begin{itemize}
  \tightlist
  \item
    s. 679 - recept na trdlovec - stromovej dort, litej trdelník
  \end{itemize}
\item
  1905 \href{https://www.jstor.org/stable/42693111}{Lidová jídla na
  Podřipsku} - František Homolka,

  \begin{itemize}
  \tightlist
  \item
    časopis
    \href{https://ceskadigitalniknihovna.cz/uuid/uuid:27984073-3c0d-48de-8c3a-9b56c8b34d7a}{Český
    lid}
  \item
    píšou, že se vaří a řadí je mezi knedlíky. Neni tam žádnej detail,
    co že to vlastně je, takže je možný, že to je jen shoda jména.
  \end{itemize}
\item
  1905
  \href{https://ceskadigitalniknihovna.cz/view/uuid:89c00f20-ac41-11dd-b6fd-000d606f5dc6?page=uuid\%3A2b67a058-6dd3-4a26-8b98-58bc2dd95f6a&fulltext=trdeln\%C3\%ADk\%20OR\%20trdeln\%C3\%ADky\%20OR\%20trdeln\%C3\%ADk\%C5\%AF&source=nkp}{Obzor}
  - časopis z Brna:

  \begin{itemize}
  \tightlist
  \item
    etiketa přikazuje matkám, sestrám, kmotrám, tetám, ale můžou i
    kamarádky, nosit šestinedělkám do kouta: 1. zásilka bývá vařená
    slepice, lukšová polívka, koláče, 2. lukšová polívka, hovězí s
    omáčkou, koblihy, víno, 3. smažený kuřata, trdelník nebo boží
    milosti a víno, 4. živá drůbež, syrová káva, bábovka i syrový maso
  \end{itemize}
\item
  1905
  \href{https://ceskadigitalniknihovna.cz/uuid/uuid:b78503e3-435d-11dd-b505-00145e5790ea}{Lidové
  noviny}:

  \begin{itemize}
  \tightlist
  \item
    s. 4, Hanácký večer
  \item
    národopisná výstava Hanácká v Úprkově dvorance ve Vesně
  \item
    páni dávali pozornost beleškám, trdelníkům, prstkám a plecovníku
    (zauzené plecko)
  \end{itemize}
\item
  1905
  \href{https://ceskadigitalniknihovna.cz/uuid/uuid:e6cc2207-435d-11dd-b505-00145e5790ea}{Humoristické
  listy}:

  \begin{itemize}
  \tightlist
  \item
    nějakej hanák Antonin Jaborské děkuje Mandě, za trdelníky, co si od
    ní z Plešovca přivezl, paní Randové moc chutnali a musí jí přivízt
    trdlo na pečení
  \end{itemize}
\item
  1906
  \href{https://ceskadigitalniknihovna.cz/uuid/uuid:50b3ac32-1b3f-4639-9f61-c01f3883690d}{Deset
  rozprav lidopisných}

  \begin{itemize}
  \tightlist
  \item
    František Bartoš
  \item
    cituje zvyky nošení šestinedělce do kouta
  \end{itemize}
\item
  1906
  \href{https://ceskadigitalniknihovna.cz/view/uuid:d04052b0-a08e-11ed-bf61-5ef3fc9bb22f?page=uuid:50e42324-813b-4538-aaf5-2b1efbf63513&fulltext=trd*ln*k*&source=nkp}{Dialektický
  slovník moravský}:

  \begin{itemize}
  \tightlist
  \item
    František Bartoš, cituje Šebestovou, Lidové dokumenty, 1900
  \end{itemize}
\item
  1906
  \href{https://ndk.cz/uuid/uuid:6a5d3470-e943-11e2-9923-005056827e52}{Třetí
  příspěvek k česko-německému slovníku}:

  \begin{itemize}
  \tightlist
  \item
    trdelník = knedlík z hrubé pšeničné mouky na Podřipsku
  \item
    trdlovec = do výšky lité cukroví, Baumkuchen. prodává se na kusy
    sekané
  \end{itemize}
\item
  1906
  \href{https://dikda.snk.sk/view/uuid:a5d481cd-45e0-4830-8eb3-f6a70599c799?page=uuid:400792a6-389c-46bb-9b66-3bd44ddecca3&fulltext=trdeln\%C3\%ADk\%20OR\%20trdeln\%C3\%ADka\%20OR\%20trdeln\%C3\%ADku}{Slovník
  Maďarský A Slovenský} - Slovensko, heslo Sodrófánk = trdelník
\item
  1906
  \href{https://ceskadigitalniknihovna.cz/view/uuid:50b3ac32-1b3f-4639-9f61-c01f3883690d?page=uuid:ad4dfae7-7a29-11ed-b508-001b63bd97ba&fulltext=trdeln\%C3\%ADky&source=kfbz}{Deset
  rozprav lidopisných}:

  \begin{itemize}
  \tightlist
  \item
    František Bartoš, národopisný texty o Moravě
  \item
    popis co se dává šestinedělkám, trdelník ve třetí zásilce
  \end{itemize}
\item
  1907
  \href{https://ceskadigitalniknihovna.cz/uuid/uuid:9a11a910-6d75-11e8-be68-5ef3fc9bb22f}{Besedy
  času}:

  \begin{itemize}
  \tightlist
  \item
    povídka od Marie Dolečková-Krhovská: Kubíkových
  \item
    ze slováckýho prostředí, popisuje dětství, kdy tetička Kubíková
    dávala dětem trdláče
  \end{itemize}
\item
  1908
  \href{https://www.digitalniknihovna.cz/vkol/uuid/uuid:c11919dc-435d-11dd-b505-00145e5790ea}{Lidové
  noviny}:

  \begin{itemize}
  \tightlist
  \item
    pozvánka na svatbu v Luhačovicích, budou trdelníky a další dobroty
  \end{itemize}
\item
  1908
  \href{https://ceskadigitalniknihovna.cz/uuid/uuid:d373d613-8285-11e0-b92b-0050569d679d}{Našinec}:

  \begin{itemize}
  \tightlist
  \item
    J. Vyhlídal: Obrázky z mého hanáckého alba - o hanáckých vdolkách a
    bochtách a roztodivnym žetnym
  \item
    popis hanáckýho pečiva, trdelníky - troubky smažené
  \end{itemize}
\item
  1909
  \href{https://www.digitalniknihovna.cz/vkol/uuid/uuid:8de01b83-a076-483b-b755-bd2e9de6e830}{Severní
  Morava}:

  \begin{itemize}
  \tightlist
  \item
    Kuchařská výstava splolku Zora v Zábřehu, paní Vránová z Čejčí
    prezentovala slovácký trdelníky
  \end{itemize}
\item
  1909
  \href{https://ceskadigitalniknihovna.cz/uuid/uuid:a6ed5320-aa9f-11e3-bb86-005056825209}{Českoněmecký
  slovník}:

  \begin{itemize}
  \tightlist
  \item
    Herzer, Jan
  \item
    náčepka = Prügelkrapfen
  \end{itemize}
\item
  1909
  \href{https://ceskadigitalniknihovna.cz/uuid/uuid:c455e0f0-cd7f-11e3-b110-005056827e51}{Brněnské
  noviny}:

  \begin{itemize}
  \tightlist
  \item
    Martinské hody ve Vídni
  \item
    pořádal slovácký krajanský spolek Slovák
  \item
    trnčené buchty, makovníky, vdouků a trdelníků napečú hospodyně
    přespolní
  \end{itemize}
\item
  1909
  \href{https://ceskadigitalniknihovna.cz/uuid/uuid:649fe960-411f-47df-9f66-f407e20f9b16}{Plzeňské
  besedy}:

  \begin{itemize}
  \tightlist
  \item
    O lázních, které jsou naše
  \item
    Luhačovice, v búdě jsou pověstné trdelníky
  \end{itemize}
\item
  1910
  \href{https://ceskadigitalniknihovna.cz/view/uuid:b06a8860-7d9d-11e9-85ec-005056825209?page=uuid\%3Ab3e88f80-7e35-11e9-b171-5ef3fc9ae867&fulltext=trdeln\%C3\%ADk\%20OR\%20trdeln\%C3\%ADky\%20OR\%20trdeln\%C3\%ADk\%C5\%AF&source=nkp}{Čas}:

  \begin{itemize}
  \tightlist
  \item
    Luhačovice (protest, toužení a prosba)
  \item
    děti by měly chodit v lázních na výlety, třeba na autorovu stráň do
    búdy na sladké trdelníky
  \end{itemize}
\item
  1910
  \href{https://ceskadigitalniknihovna.cz/view/uuid:9619abc6-e10c-479a-a5c6-82a48f7789f4?page=uuid\%3Adefe860e-1878-4938-89bc-3a04bc29b94b&source=kkp}{Neodvislé
  listy}:

  \begin{itemize}
  \tightlist
  \item
    Slovenský večer v Havlíčkově Brodě, podávali slovenský pokrmy
    trdelíky a zázvorníky
  \end{itemize}
\item
  1910
  \href{https://ceskadigitalniknihovna.cz/uuid/uuid:f19aaf80-126a-11e8-8cd8-5ef3fc9bb22f}{Časopis
  Moravského musea zemského}:

  \begin{itemize}
  \tightlist
  \item
    článek Z Gallašovy literární pozůstalosti
  \item
    pokračování článku z minulýho čísla
  \item
    popisuje masopust, autor článku připomíná, že zapomněl na trdelníky,
    protože v rukopisech popisuje, že koblihy a trdelníky jsou pokrm pro
    řemeslníky
  \item
    s. 219 v panských kuchyních i trdelníky
  \end{itemize}
\item
  1911
  \href{https://www.digitalniknihovna.cz/vkol/uuid/uuid:be493458-5dbc-42df-9ca1-fa5bef6ae4c6}{Pozor}:

  \begin{itemize}
  \tightlist
  \item
    článek Besídka, položili základní kámen Uměleckého domu v Hodoníně,
    pak jeli do Bojanovic
  \item
    jedí na rožni pečený trdelníky
  \end{itemize}
\item
  1911
  \href{https://ceskadigitalniknihovna.cz/view/uuid:5a976b57-57a4-491f-b403-535f71c739a6?page=uuid\%3A2ec0de06-85d4-11e4-8faf-00155d010f03&fulltext=trdlovec&source=svkhk}{Podkrkonošské
  rozhledy}: reklama na cukrářství v Hronově, dělá trdlovec v pěti
  různých druzích
\item
  1912
  \href{https://ndk.cz/view/uuid:e8f12a30-1b67-11e7-96ce-005056827e51?page=uuid\%3Af8cffa60-2933-11e7-a38c-005056827e51}{Velká
  kuchařka}:

  \begin{itemize}
  \tightlist
  \item
    novější vydání od M. D. Rettigové, doplněný novýma receptama od Idy
    Caligris a Elišky Neubauerový
  \item
    s. 608-609 dva recepty
  \item
    1. je litej trdelník z nekynutýho vaječnýho těsta
  \end{itemize}
\item
  1911
  \href{https://www.digitalniknihovna.cz/vkol/uuid/uuid:d4380ff6-8a62-49b7-8196-d84c1ad69a53}{Hlasy
  z Hané}:

  \begin{itemize}
  \tightlist
  \item
    Majales ženského odboru Národní Jednoty v Prostějově
  \item
    slovácké trdelníky šly na dračku
  \end{itemize}
\item
  1912
  \href{https://ceskadigitalniknihovna.cz/view/uuid:2c2f1570-b8b8-11de-b399-000d606f5dc6?page=uuid:625e3e20-b7e9-11de-8b7d-000d606f5dc6&source=mzk}{Moravská
  orlice}:

  \begin{itemize}
  \tightlist
  \item
    Poštorná a venkov na matiční slavnosti v Břeclavi
  \item
    dámské výbory připravují bufety, například pečivo slovácké,
    trdelníky, břeclavské koláče
  \end{itemize}
\item
  1913
  \href{https://ceskadigitalniknihovna.cz/view/uuid:3bc6373e-804a-44b4-b42f-50d5bf37e90d?page=uuid\%3Af7d19e61-5740-11e3-852c-0050569d679d&fulltext=trdelniky&source=mzk}{Vlastivěda
  moravská}:

  \begin{itemize}
  \tightlist
  \item
    č. 34 Znojemský kraj
  \item
    strava lidu, smažené trdelníky nosí hospodyně šestinedělkám do kouta
  \end{itemize}
\item
  1913
  \href{https://ceskadigitalniknihovna.cz/view/uuid:01729a50-78ad-11e3-a388-5ef3fc9ae867?page=uuid\%3A3eefbb60-79ca-11e3-b0d1-005056827e51&fulltext=trdeln\%C3\%ADk\%20OR\%20trdeln\%C3\%ADky\%20OR\%20trdeln\%C3\%ADk\%C5\%AF&source=mzk}{Brněnské
  noviny} - tombola na výlet mateřský školy, trdelník jedna z cen
\item
  1913
  \href{https://www.digitalniknihovna.cz/vkol/uuid/uuid:2997b9af-6b24-4bda-b3cb-6dc7e7e97004}{Hanácký
  kraj}:

  \begin{itemize}
  \tightlist
  \item
    Z Otnic (národopisná slavnost)
  \item
    choď předsedy hasičů darovala místu trdelníků a smažený boží milosti
  \end{itemize}
\item
  1913
  \href{https://ceskadigitalniknihovna.cz/view/uuid:6c6a5e90-917d-11e8-9588-5ef3fc9bb22f?page=uuid\%3A034ac5a0-9a3b-11e8-b814-5ef3fc9bb22f&fulltext=trdlovec&source=nkp}{Mládenec}:

  \begin{itemize}
  \tightlist
  \item
    reklama na cukrárnu na Vinohradech u Riegrových sadů, specialita je
    trdlovec
  \end{itemize}
\item
  1914
  \href{https://www.digitalniknihovna.cz/uzei/uuid/uuid:dcea1ada-9d5a-4be8-a92a-fdab1e9e93e0}{Velká
  vzorná česká kuchařka}:

  \begin{itemize}
  \tightlist
  \item
    s. 732-733 recept na litej trdelník, stejnej jako v jinejch
    kuchařkách z tý doby
  \end{itemize}
\item
  1915
  \href{https://ceskadigitalniknihovna.cz/view/uuid:f679b550-74ec-11ef-b80e-5ef3fc9bb22f?page=uuid:091aa6cd-7fcc-4ef0-99cd-a2a5f9d72724&fulltext=trd*ln*k*&source=nkp}{Zpráva
  dívčí průmyslové školy spolku Vesna v Hořicích}:

  \begin{itemize}
  \tightlist
  \item
    časopis, rok vydání 1915, není dostupnej fulltext, ale je vidět
    náhled:

    \begin{itemize}
    \tightlist
    \item
      císařská kýta, plněné placičky, trdelník
    \item
      podkrkonoší, ale nevím, čeho se ten článek vlastně týká
    \end{itemize}
  \end{itemize}
\item
  1916
  \href{https://ceskadigitalniknihovna.cz/uuid/uuid:57d4bfd0-7b1f-11e3-a80c-005056825209}{Českoněmecký
  slovník}, díl S-Ž:

  \begin{itemize}
  \tightlist
  \item
    Herzer, Jan
  \item
    trdelník = Spies-krapfen\textbar kuchen,
    Ring(el)-kuchen\textbar krapfen
  \item
    vaječník = a) Eier-speise f, -fladen m, Eier-, Pfann-, Scherben,
    Spieß-kuchen m, Ringel= Spieß-krapfen
  \end{itemize}
\item
  1917
  \href{https://ceskadigitalniknihovna.cz/view/uuid:bc7a2eb0-944f-11e6-baa5-005056827e51?page=uuid\%3A1fdf0630-a5f1-11e6-8bf1-001018b5eb5c&fulltext=trdeln\%C3\%ADk\%20OR\%20trdeln\%C3\%ADky\%20OR\%20trdeln\%C3\%ADk\%C5\%AF&source=mzk}{Trýzeň
  duše a jiné povídky}:

  \begin{itemize}
  \tightlist
  \item
    Viktor Kamil Jeřábek
  \item
    povídka z Bránic, krmili pana radu jako šeestinedělku, koláči,
    trdelníky
  \end{itemize}
\item
  1919
  \href{https://ceskadigitalniknihovna.cz/view/uuid:14e7b8a0-0fe2-11ee-8c4d-5ef3fc9bb22f?page=uuid\%3A9b6bf140-0ff9-11ee-b46c-5ef3fcdaa9a7&fulltext=trdeln\%C3\%ADk\%20OR\%20trdeln\%C3\%ADky\%20OR\%20trdeln\%C3\%ADk\%C5\%AF&source=nkp}{Praha:
  obrazový deník}:

  \begin{itemize}
  \tightlist
  \item
    výstavka jídel na Střeleckym ostrově, byly tam moravský trdelníky
  \end{itemize}
\item
  1920
  \href{https://ndk.cz/uuid/uuid:d08ced10-7b32-11eb-9f97-005056827e51}{Nejnovější
  illustrovaná kuchařská kniha, obsahující na 3.500 }

  \begin{itemize}
  \tightlist
  \item
    s. 260 recept pečenej v troubě, ale je to normálně kynutej váleček,
    na předcházejících stranách i vyobrazení přípravy
  \end{itemize}
\item
  1920
  \href{https://dikda.snk.sk/uuid/uuid:a65dbdb4-70d4-436e-92af-86192f998fc5}{Slovenská
  kuchárka}:

  \begin{itemize}
  \tightlist
  \item
    recept, pečenej v troubě na plechovejch formách, sypanej mandlema
  \item
    je tam uvedenej název Priegel-Krapfen
  \end{itemize}
\item
  1920
  \href{https://ceskadigitalniknihovna.cz/view/uuid:7a29ece0-be2c-11de-a09c-000d606f5dc6?page=uuid:461eb510-b739-11de-8d2b-000d606f5dc6&fulltext=trdeln\%C3\%AD*&source=mzk}{Moravská
  orlice}:

  \begin{itemize}
  \tightlist
  \item
    fejeton Jho nejtěžší, Viktor Kamil Jeřábek
  \item
    Matys si bere Maryčlu, je to dobrá partie, protože zaplatila dluhy
    na půllánu
  \item
    Maryčla smaží trdelníky a koblihy
  \end{itemize}
\item
  1921
  \href{https://ceskadigitalniknihovna.cz/uuid/uuid:e28419ef-cadc-449a-834c-164adadcc55b}{Tribuna}:

  \begin{itemize}
  \tightlist
  \item
    jakejsi fejeton, popisuje trhy na Kapucínském náměstí v Brně
  \item
    na konci zmiňuje trdelníky jako vydatný hlty
  \end{itemize}
\item
  1922
  \href{https://ndk.cz/view/uuid:d875fc80-2c59-11e4-8f64-005056827e52?page=uuid\%3A4ba3cc00-5b4e-11e4-a6f0-5ef3fc9ae867&fulltext=trdeln\%C3\%ADky}{Moravské
  Slovensko}:

  \begin{itemize}
  \tightlist
  \item
    s. 522 - 523 - slavnostní pokrmy - trdelníky z nekvašenýho těsta
    zadělanýho smetanou, pak se namotaly na plechový válečky s rukojetí,
    kterejma se otáčelo nad ohněm
  \item
    s. 524 - na Brodsku pečený na dřevěných trdlách
  \item
    s. 532 - recept na smaženej trdelník z těsta ze smetany, mouky a
    vajec
  \end{itemize}
\item
  1923
  \href{https://ceskadigitalniknihovna.cz/uuid/uuid:6ef4c460-b7b1-11e9-8f08-5ef3fc9ae867}{Večer:
  lidový deník}:

  \begin{itemize}
  \tightlist
  \item
    článek Brasilské dřevo v českém zpracování
  \item
    poznámka, že nějakej dřevěnej suk vypadá jako obrovský náš koláč
    trdelník (němci mu říkají Baumkuchen)
  \end{itemize}
\item
  1924
  \href{https://ceskadigitalniknihovna.cz/uuid/uuid:2dbf7a20-4967-11e4-aded-005056827e51}{Jak
  se naši škádlívají}:

  \begin{itemize}
  \tightlist
  \item
    Otakar Bystřina
  \item
    lidový humor z Moravy
  \item
    s. 24: O chropiňských moresích, synek z Bilan studoval v Chropini,
    po skončení školního roku ho vítali jako na svatbu, zabili prase,
    napekli vdolky, boží milosti i trdelníky
  \item
    s. 228: štramberské uši se dělají z trdelnicového těsta
  \end{itemize}
\item
  1924
  \href{https://ceskadigitalniknihovna.cz/view/uuid:f0cbc280-f31d-11ed-8fba-005056827e51?page=uuid\%3A7d79e78f-ae5e-497a-8be8-419105a6b0f9&fulltext=trdeln\%C3\%ADky&source=mzk}{Kalendář
  Palacký}

  \begin{itemize}
  \tightlist
  \item
    Slovensko
  \item
    článek Dr. Joz. Ludev. Holuby: Drobnosti z mojho dlheho života
  \item
    narozenej 26. marca 1836 v Lubině pod Javorinou, Nitranská župa
  \item
    trdelníky posypaný ořechama s cukrem a pečený na trdle
  \item
    Maďaři pojmenovávají dorongkalács, Zvolenčani drúkový koláč,
    Rakušáci Prügelkrapfen, což je nesmysl, protože to znamená smažený
    šišky
  \end{itemize}
\item
  1924
  \href{https://ceskadigitalniknihovna.cz/uuid/uuid:a382bc3a-7825-4663-992a-f6e93b38a656}{České
  slovo}:

  \begin{itemize}
  \tightlist
  \item
    s: 5, popis svatby, hosti jedli spoustu jídel, například trdelníky
  \end{itemize}
\item
  1925
  \href{https://www.digitalniknihovna.cz/nulk/uuid/uuid:b2d75327-ce5f-4248-8ec5-52b0f9b2f9af}{Podunajská
  dedina v Československu}:

  \begin{itemize}
  \tightlist
  \item
    popisuje západní Slovensko
  \item
    s. 52: popis vaření na ohni, na železný koně se nasazovaly dřevěné
    nebo železní rožně na masité pečeně a trdelníky. obrázek koně na
    pečení masa a trdelníků
  \item
    s. 92: trdelníky smažené v sádle, dodnes jsou oblíbený na Uh.
    Brodsku na Moravě
  \item
    s. 93: od stravování na Slovácku se dneska popisovaná slovenská
    strava liší, i když některý názvy jídel, např. trdelníky ukazujou na
    někdejší příbuznost
  \item
    s. 224: jídla pro šestinedělku, tredlníky se dodneska dělají na
    Slovácku
  \end{itemize}
\item
  1925
  \href{https://ceskadigitalniknihovna.cz/view/uuid:fc3b3470-0163-11ea-af21-005056827e52?page=uuid\%3Aeb966593-da2e-4197-8137-77bb9b40d4c7&fulltext=trdlovec&source=nkp}{Kniha
  kuchařských předpisů}:

  \begin{itemize}
  \tightlist
  \item
    recept na trdlovec, je to litej Baumkuchen
  \item
    zdá se, že za 1. republiky byl trdlovec známej termín, používá se v
    beletrii - pohádky, (louskáček, o perníkový chaloupec)
  \end{itemize}
\item
  1925
  \href{https://ceskadigitalniknihovna.cz/uuid/uuid:f9d7dc50-c0ec-4cbb-ace5-c60d24598488}{Kvítko
  z čertovy zahrádky}:

  \begin{itemize}
  \tightlist
  \item
    Jak to bude vypadat, až v Československé republice nastoupí vláda
    sovětů
  \item
    popisujou hypotetickej Centrosklad, kam se budou dovážet trakaře,
    syrečky, klobásy, hrábě, kvedlačky, slivovice, trdelníky a jiné
  \end{itemize}
\item
  1926
  \href{https://ndk.cz/view/uuid:be635380-b1b9-11ed-826c-005056827e52?page=uuid\%3A9ae6ef00-82cb-438b-8727-ef820bed7a49}{Paměti
  městyse Černé Hory}

  \begin{itemize}
  \tightlist
  \item
    městečko u Blanska, popisujou místní kuchyni minulých staletí, ale u
    trdelníku neni časový určení
  \item
    klasickej trdelník sypanej ořechama před pečením, pečenej nad ohněm
  \item
    jemný koláčový těsto, válec v průžezu 15 cm, po opečení rozsekanej
    na 20 cm dlouhý části
  \end{itemize}
\item
  1926
  \href{https://ceskadigitalniknihovna.cz/view/uuid:3ffb22b9-dde8-4a93-b4f9-7eb3f9242f73?page=uuid\%3A864331ed-50a3-11e5-8200-0050569d679d&fulltext=trdeln\%C3\%ADk\%20OR\%20trdeln\%C3\%ADky\%20OR\%20trdeln\%C3\%ADk\%C5\%AF&source=mzk}{Vlastivěda
  moravská}:

  \begin{itemize}
  \tightlist
  \item
    II. Místopis Moravy. Dil I místopisu, Brněnský kraj. {[}Čís. 45a{]},
    Hodonský okres - Hodonín
  \item
    trdelník = pečivo co se navinuje při smažení na kolíčky
  \end{itemize}
\item
  1926
  \href{https://ceskadigitalniknihovna.cz/view/uuid:42a39d90-3138-11ea-a83e-005056827e51?page=uuid\%3A0cfdfefd-f5fd-45f2-9d3e-cb5a4563b2c2&fulltext=trdlovec&source=mzk}{Domácí
  vševěd}:

  \begin{itemize}
  \tightlist
  \item
    slovník vědomstí. trdlovec je stromový dort, litej na rožeň, kupuje
    se zpravidla hotový
  \end{itemize}
\item
  1926
  \href{https://ceskadigitalniknihovna.cz/view/uuid:b91df880-8a84-11ef-bcc2-005056825209?page=uuid\%3A3da0dc0d-49de-4574-b04d-eeda66a7b25a&fulltext=trdeln\%C3\%ADk\%20OR\%20trdeln\%C3\%ADky\%20OR\%20trdeln\%C3\%ADk\%C5\%AF&source=nkp}{Právo
  lidu}:

  \begin{itemize}
  \tightlist
  \item
    nějaká povídka, kluci se dohadujou, jaký jídlo je nejlepší
  \item
    podle jednoho jsou trdelníky (krémrole) lepší, než čokoláda
  \item
    další tvrdí, že nejlepší neni ani čokoláda, ani trdelník, ale párky
    s houskou
  \end{itemize}
\item
  1926
  \href{https://ceskadigitalniknihovna.cz/uuid/uuid:58b8aad0-3d9c-11e8-b52f-5ef3fc9ae867}{Česká
  revue}:

  \begin{itemize}
  \tightlist
  \item
    fejeton: buchty
  \item
    vypisuje seznam svátečních pečiv, trdelník mezi nima
  \end{itemize}
\item
  1927
  \href{https://ndk.cz/uuid/uuid:b59a1ff0-9f23-11ea-b6e0-005056827e51}{Staročeské
  umění kuchařské}, Čeněk Zíbrt:

  \begin{itemize}
  \tightlist
  \item
    na několika místech se objevuje vaječník, což je alternativní název
    pro trdelník uváděnej Jungmannem. Jaká je to ale forma není jistý,
    jsou to jen citace ze starších knížek a většinou jen seznamy pokrmů.
    Ale jsou ve skupině se sladkym pečivem.
  \item
    v knížkách z 19. století ale vaječník často značí omeletu
    (\href{https://ndk.cz/uuid/uuid:f7e03900-d300-11dc-9815-000d606f5dc6}{stručný
    všeobecný slovník naučný}, omeleta - francouzský vaječník)
  \item
    s. 90:

    \begin{itemize}
    \tightlist
    \item
      poznámka s německym textem, kde se vyskytuje termín "eyerkuchen",
      překlad pomocí chatgpt
    \item
      Fontes rerum austriacarum. II. oddíl: Diplomataria et acta, svazek
      20. Vídeň 1860. Palacký Fr., Listinné příspěvky k dějinám Čech a
      jejich sousedních zemí v době Jiřího z Poděbrad (1450 až 1471),
      str. 329, č. 317.
    \item
      1464, Jun. 2.
    \item
      Přepis původního textu (s ponechanými archaismy): „Item graue
      Ludwig von Glichen hat mich Rudolff Schenken in sunderheit
      berichet, wie der könig von Beheim myn ald fraw von Sachsen an
      allen enden von Brux angehinde bisz gein Prage und furd hat durch
      sein konigreich und furstentumb sie mit aller usrichtung ufs
      folligst versorgen lassen und zum jungsten bestalt das vor allen
      dorfen zwischin Brux und Prage die bawrn mit mosantzen, eigern,
      kesen und eyerkuchen irergnaden entgegen gelauffen sein und das in
      iren wagen geben.``
    \item
      Překlad do moderní němčiny: „Des Weiteren hat mich der ehrwürdige
      Ludwig von Glichen durch Rudolf Schenken im Besonderen berichtet,
      wie der König von Böhmen meine alte Frau von Sachsen an allen
      Orten von Brüx an bis nach Prag und weiter durch sein Königreich
      und sein Fürstentum sie mit aller Ausrichtung aufs Vollständigste
      versorgen ließ und jüngst veranlasste, dass vor allen Dörfern
      zwischen Brüx und Prag die Bauern mit Mostbrötchen, Eiern, Käse
      und Eierkuchen Ihrer Gnaden entgegen gelaufen sind und dies in
      ihren Wagen gegeben haben.``
    \item
      Dopis z Čech vévodovi Vilémovi Saskému: o pobytu staré vévodkyně
      Saské v Praze: „Také ctihodný Ludvík z Glichen mi sdělil
      podrobnosti skrze Rudolfa Schenka o tom, jak český král přijal mou
      starou paní vévodkyni Saskou, když přijížděla z Mostu do Prahy, a
      jak ji nechal skrze své království a knížectví pečlivě připravit a
      zcela zabezpečit. Nejvíce bylo dbáno na vesnice mezi Mostem a
      Prahou, kde rolníci běželi vstříc s moštovými nápoji (špatnej
      překlad, v originále jsou mazance), vejci, sýry a vaječníky a vše
      to nakládali do jejích vozů.``
    \item
      král byl Jiří z Poděbrad
    \item
      \href{https://www.digitale-sammlungen.de/view/bsb10798133?page=354\%2C355}{tady}
      je digitalizovanej originál
    \end{itemize}
  \item
    s. 98 - cituje Klaretovu družinu, vaječník je uvedenej mezi pečivem
    (podrobnějc viz r. 1928, Klaret a jeho družina)
  \item
    s. 151 - trojjazyčnej slovník vydanej ve Varšavě 1513, vydání z roku
    1532. Vaječník (Ayerkuch) uvedenej mezi mazancem a perníkem

    \begin{itemize}
    \tightlist
    \item
      Dictionarius trium linguarum, latinae, teutonicae, boemicae
      potiora vocabula continens, peregrinantibus apprime utilis,
      Viennae, 1513, 4°; ve Varšavě, 1513, 4°; vydání 1532, 12°.
    \item
      tady je
      \href{https://dbc.wroc.pl/dlibra/publication/33643/edition/30385/content?ref=L3B1YmxpY2F0aW9uLzM0Njg5L2VkaXRpb24vMzE0MTQ}{digitalizovaná
      verze}, s. 19
    \item
      Ouarium - wagečznik - eyerkuch
    \end{itemize}
  \item
    s. 207:

    \begin{itemize}
    \tightlist
    \item
      obrázek titulního listu Kuchařství Pavla Severina z Kapí hory, r.
      1535
    \item
      je tam na rožni nějaká drůbež a vedle ní válec podobnej trdelníku
    \end{itemize}
  \item
    s. 249:

    \begin{itemize}
    \tightlist
    \item
      obrázek titulního listu Kantorova přetisku Kuchařství Severina
      Mladšího, zhruba polovina 16. století
    \item
      je tam černá kuchyně, na roštu nějaká drůbež a vedle ní válec, co
      vypadá jako trdelník
    \item
      je to jinej obrázek, než u Severina, i když je mu dost podobnej
    \end{itemize}
  \item
    s. 318:

    \begin{itemize}
    \tightlist
    \item
      varmuže (kaše) z vaječníku, ale to si vůbec nejsem jistej, co to
      má bejt.
    \item
      Je to součástí Kuchařství Bavora mladšího z Hustiřan z r. 1591
    \item
      na tenhle recept se Zíbrt neodkazuje v rejstříku pod položkou na
      vaječník
    \item
      ale v rejstříku
      \href{https://ndk.cz/uuid/uuid:27bccb70-88d6-11e3-997d-005056827e52}{vydání
      z roku 1975} je vaječník uvedenej jako vaječnej koláč, svítek
    \item
      nicméně vaječný svítky tam popisuje samostatně, ne jako vaječníky
    \end{itemize}
  \item
    s. 376:

    \begin{itemize}
    \tightlist
    \item
      Komenský, Zlaté dveře jazykův otevřené..., 1669
    \item
      Vaječník (Spiesskuchen) je koláč
    \item
      je to položka
      \href{https://vokabular.ujc.cas.cz/moduly/mluvnice/digitalni-kopie-detail/KomJanua1669/strana-76}{408}
    \item
      \href{https://vokabular.ujc.cas.cz/moduly/mluvnice/digitalni-kopie-info/KomJanua1669}{Tady
      je celá knížka}
    \item
      zajímavý je, že v
      \href{https://www.digitalniknihovna.cz/mzk/view/uuid:c53f79b0-b747-11e4-a7a2-005056827e51?page=uuid:c976bf00-bdcf-11e4-ba2b-5ef3fc9bb22f&fulltext=kol\%C3\%A1\%C4\%8D\%C5\%AF}{Brána
      jazyků otevřená} je založená na jiný verzi a jsou tam jiný pečiva,
      například boží milosti
    \end{itemize}
  \item
    s. 378:

    \begin{itemize}
    \tightlist
    \item
      cituje trojjazyčný slovník Hadriani Junii Nomenclator omnium rerum
      propria nomina tribus linguis explicata continens, Pragae, 1686,
    \item
      ale podle Zíbrta je to špatnej název, původně se měl jmenovat
      Vocabolarium trilingue.
    \item
      každopádně je tam seznam jídel s překlady a mezi nima vaječník
      (Spisskuchen) (zajímavý, že ne Spiesskuchen)
    \item
      \href{https://www.digitale-sammlungen.de/view/bsb11105115?page=42\%2C43}{digitalizovanej
      originál}
    \item
      obelum = wagečnjk = spisskuchen
    \end{itemize}
  \item
    s 613:

    \begin{itemize}
    \tightlist
    \item
      Fr. Kropf průvodce po kuhyni české v polovici 18. století
    \item
      koláče na rošti pécti
    \end{itemize}
  \end{itemize}
\item
  1927
  \href{https://ndk.cz/view/uuid:a6e51320-373c-11ee-973a-005056827e51?page=uuid\%3A7905381e-5d69-48f7-97ca-9e0d2c8b5023&fulltext=trdeln\%C3\%ADk}{Weekend
  Na výletě}:

  \begin{itemize}
  \tightlist
  \item
    s. 38, recept na trdelník, ale jsou to šišky z vařenejch brambor s
    moukou, smažený na pánvi
  \item
    není možný, že to je ten trdelník z Podřipska, což je knedlík?
  \end{itemize}
\item
  1928
  \href{https://sources.cms.flu.cas.cz/src/index.php?s=v&bookid=833&page=3}{Klaret
  a jeho družina}:

  \begin{itemize}
  \tightlist
  \item
    sbírka staročeských slovníků od mistra Klareta, asi 60. léta 14.
    století
  \item
    Bohemář, s
    \href{https://sources.cms.flu.cas.cz/src/index.php?s=v&bookid=832&page=84}{50},
    verš 298:

    \begin{itemize}
    \tightlist
    \item
      artopiper = pernik, ovarius = wagecznyk
    \item
      je to v podstatě česko-latinskej slovník
    \end{itemize}
  \item
    s.
    \href{https://nlp.fi.muni.cz/projekty/ahisto/portal/book/832?lpage=171&search=vage}{171},
    verš 1807:

    \begin{itemize}
    \tightlist
    \item
      ovarium = vagecznik
    \end{itemize}
  \item
    Rejstříky v družině II:

    \begin{itemize}
    \tightlist
    \item
      s
      \href{https://sources.cms.flu.cas.cz/src/index.php?s=v&zoom=y&bookid=833&page=411&ft=vaje\%C4\%8Dn\%C3\%ADk}{411}:
      preclík (rým vaječník)
    \item
      s
      \href{https://sources.cms.flu.cas.cz/src/index.php?s=v&zoom=y&bookid=833&page=413&ft=vaje\%C4\%8Dn\%C3\%ADk}{410}:
      seznam pečiva s vaječníkem
    \item
      s
      \href{https://sources.cms.flu.cas.cz/src/index.php?s=v&bookid=833&page=495}{492}:
      heslo vaječník, kontext: koblih, pecen, vaječník, húsce, preclík
    \end{itemize}
  \end{itemize}
\item
  1928
  \href{https://ceskadigitalniknihovna.cz/view/uuid:f80a63e0-611a-11e1-acfb-0013d398622b?page=uuid\%3Ada4abcb0-9dd7-11e7-a093-005056825209&fulltext=trdlovec&source=nkp}{Národní
  osvobození}:

  \begin{itemize}
  \tightlist
  \item
    staročeský trdlovec
  \item
    autor popisuje Baumkuchen s čokoládovou polevou, všehochuť chutí,
    citron, vanilka, ananas, malina
  \end{itemize}
\item
  1928
  \href{https://ceskadigitalniknihovna.cz/view/uuid:d9b0f410-125b-11ed-8635-005056827e52?page=uuid\%3A82be2751-0e47-4147-a2f9-395b5608b5ce&fulltext=trdeln\%C3\%ADk\%20OR\%20trdeln\%C3\%ADky\%20OR\%20trdeln\%C3\%ADk\%C5\%AF&source=mzk}{Bosonohy}:

  \begin{itemize}
  \tightlist
  \item
    ráz života osady brněnského kraje před třicíti lety - Bosonohy jsou
    dneska část Brna
  \item
    dračky, draní peří, vzpomíná se na hody, masopusty, atd.
  \item
    po skončení draček se dělá důděrná, na kterou se napekly koblihy a
    trdelníky
  \end{itemize}
\item
  1928
  \href{https://ceskadigitalniknihovna.cz/uuid/uuid:25f83680-128e-11e8-8ee4-005056825209}{Venkov}:

  \begin{itemize}
  \tightlist
  \item
    Josef Mikulič: Odvolání (O slováckých dětech)
  \item
    kluci jsou venku a sněží, řeší, že by byli radši, kdyby padal cukr
    nebo mouka, aby maminka mohla dělat koláčů a trdelníků, kolik by
    chtěli
  \end{itemize}
\item
  1929
  \href{https://ceskadigitalniknihovna.cz/view/uuid:8eac0a2b-3229-4774-bc7b-59842402b9e4?page=uuid\%3Abd127f65-068f-11ee-8565-00155d01210b&source=svkhk}{Společenské
  klasobraní}:

  \begin{itemize}
  \tightlist
  \item
    Jiří Guth-Jarkovský
  \item
    Baumkuchen, česky snad vaječník nebo trdelník
  \item
    popisuje, jak ho jíst z talířku. ideálně by ho měla nakrátej osoba
    služebná
  \end{itemize}
\item
  1929
  \href{https://ceskadigitalniknihovna.cz/view/uuid:bf47407e-435d-11dd-b505-00145e5790ea?page=uuid\%3A2a4bcb83-435e-11dd-b505-00145e5790ea&fulltext=trdeln\%C3\%ADk\%20OR\%20trdeln\%C3\%ADky\%20OR\%20trdeln\%C3\%ADk\%C5\%AF&source=mzk}{Národní
  listy}:

  \begin{itemize}
  \tightlist
  \item
    Štramberk na Moravě a jeho dobrý genius: k pětašedesátinám dr. A.
    Hrstky
  \item
    poznámka, že trdelníky jsou valašský pečivo na způsob kremrolí
  \end{itemize}
\item
  1930
  \href{https://ceskadigitalniknihovna.cz/view/uuid:e8387412-1c76-452f-9ecf-f705f1af03b3?page=uuid\%3Acd84eba0-525b-11e5-a788-0050569d679d&fulltext=trdeln\%C3\%ADk\%20OR\%20trdeln\%C3\%ADky\%20OR\%20trdeln\%C3\%ADk\%C5\%AF&source=mzk}{Ždáňsko}:

  \begin{itemize}
  \tightlist
  \item
    popis výživy dětí ve Ždánicích, na ostatky se smažívají koblihy,
    trdelníky a boží milosti
  \end{itemize}
\item
  1930
  \href{https://ndk.cz/view/uuid:2d9fa8a0-5063-11e4-8344-005056827e52?page=uuid:1b35e0b0-77b1-11e4-9605-005056825209&fulltext=trdeln\%C3\%ADk\%C5\%AF}{Luhačovské
  Zálesí příspěvky k národopisné hranici Valašska, Slovenska a Hané}:

  \begin{itemize}
  \tightlist
  \item
    s. 138:

    \begin{itemize}
    \tightlist
    \item
      popisujou sváteční jídla kolem roku 1890:

      \begin{itemize}
      \tightlist
      \item
        sem tam pamatují na smažení trdelníků, tolik proslavených na
        mor. Slovensku
      \item
        smažilo se na konopném, bukvicovém nebo lněném oleji
      \end{itemize}
    \item
      popis přípravy pečenýho trdelníku v poznámce pod čarou na straně
      138:

      \begin{itemize}
      \tightlist
      \item
        pečenej na ohni, při tom se maže mlíkem a sype perníkem a
        cukrem. po opečení se potíral medem
      \item
        na Slavičínsku se nosí trdláče z řidšího těsta sypaný mákem
      \end{itemize}
    \end{itemize}
  \item
    s. 318:

    \begin{itemize}
    \tightlist
    \item
      co se dává šestinedělkám
    \item
      v Biskupicích trdláče
    \end{itemize}
  \end{itemize}
\item
  1930
  \href{https://ceskadigitalniknihovna.cz/view/uuid:962edf50-9004-11e3-83a0-005056825209?page=uuid\%3Ac58d1770-9613-11e3-8b69-005056825209&fulltext=trdeln\%C3\%ADk\%20OR\%20trdeln\%C3\%ADky\%20OR\%20trdeln\%C3\%ADk\%C5\%AF&source=mzk}{Zvon}:

  \begin{itemize}
  \tightlist
  \item
    kritizuje že na mezinárodní kuchařský výstavě nebyly víc zastoupený
    tradiční český jídla, jako závin, makovec nebo trdelík a boží
    milosti
  \end{itemize}
\item
  1930
  \href{https://ceskadigitalniknihovna.cz/uuid/uuid:0ef229ae-ef9e-4c80-b6ca-8a7a99a85b67}{Turistický
  průvodce po Uherském Brodě a okolí}:

  \begin{itemize}
  \tightlist
  \item
    povídka Honění krále
  \item
    Vlčnov, selský dvůr Šimona Pavelíka
  \item
    honění krále byla jakási slavnost, kde celá vesnice chytala vybraný
    chlapce
  \item
    tetička Králíčková pekla na ohništi trdláče nadívané trnkami
  \end{itemize}
\item
  1930
  \href{https://ceskadigitalniknihovna.cz/uuid/uuid:c691a682-5723-11ee-b168-005056841fbb}{Sudetendeutsche
  Zeitschrift für Volkskunde}

  \begin{itemize}
  \tightlist
  \item
    článek Der Prügelkrapfen, č. 5, s. 221
  \item
    Ignaz Göth
  \item
    Překlad: Trdelník (nebo doslova „Hůlkový koblih``) To je zvláštní
    jihomoravské pečivo, které se peče na otevřeném ohništi. Přes
    konstrukci podobnou studni leží dřevěná hřídel, do které jsou
    zasazeny boční tyče. Zasazená hřídel se otáčí a těsto se pomalu
    podél hřídele se na ni nalévá (těsto), takže trdelník (koblih) je
    stále tlustší. Po upečení se tyčky vytáhnou, hřídel se také
    odstraní. Když je trdelník vychladlý, pocukruje se a ozdobí. Jihlava
    - Znojmo.
  \item
    obrázek zdobenýho trdelníku
  \end{itemize}
\item
  1930
  \href{https://ceskadigitalniknihovna.cz/uuid/uuid:af245650-9670-11dc-8c7e-000d606f5dc6}{Lidové
  noviny}:

  \begin{itemize}
  \tightlist
  \item
    Neděle ve Skalici, Bohuslav Halusický
  \item
    z Grazu převezli ostatky historika Františka Vítězslava Sasinka
  \item
    trdelníků bylo, co hrdlo ráčilo
  \end{itemize}
\item
  1931
  \href{https://www.digitalniknihovna.cz/vkol/uuid/uuid:c2c8f360-7588-407d-9bc7-d0c59adde597}{Vlastivědný
  sborník střední a severní Moravy}:

  \begin{itemize}
  \tightlist
  \item
    vzpomínky na Xaveru Běhálkovou z Tovačova
  \item
    v roce 1905 výstavka hanácká ve Vesně, hanácké pečivo: trdelníky,
    belešky, atd.
  \end{itemize}
\item
  1931
  \href{https://ndk.cz/view/uuid:dc35f5d0-c266-11ed-ac82-5ef3fc9bb22f?page=uuid\%3Acb35b596-b227-47df-bc08-358154471f54}{Populárne
  spisy}:

  \begin{itemize}
  \tightlist
  \item
    taky Dr. Holuby, ale popisuje novější historku, asi 1873, kdy ho
    navštívil botanik z Brém
  \item
    ze Slovenska, Zemianské Podhradie
  \item
    zmínka, že maďarsky je trdelník dorongkalács
  \end{itemize}
\item
  1931
  \href{https://ceskadigitalniknihovna.cz/uuid/uuid:918bbf20-d4ea-11e7-a047-005056825209}{Encyklopedický
  německo-český slovník}:

  \begin{itemize}
  \tightlist
  \item
    Prügelkrapfen (Spiesskuchen, Brigelkrapfen) = náčepky
  \item
    Ringkuchen - trdelník, koláč
  \item
    Ringelkrapfen - trdelník, vaječník
  \item
    Ringelkuchen - kruhovitý koláč
  \item
    další díl
    \href{https://ceskadigitalniknihovna.cz/uuid/uuid:83ae629b-8a1b-4b33-a2cb-0141786babd8}{1935}:

    \begin{itemize}
    \tightlist
    \item
      Spiesskuchen = koláč pečený na rožni

      \begin{itemize}
      \tightlist
      \item
        takže neznali slovo, který použít
      \end{itemize}
    \end{itemize}
  \end{itemize}
\item
  1932
  \href{https://ceskadigitalniknihovna.cz/view/uuid:60ab056c-5ee1-43ab-9014-e91189880750?page=uuid\%3A31d79ba8-8100-11ed-8b6a-001b63bd97ba&fulltext=trd*&source=kfbz}{Hranice
  mezi zemí Moravskoslezskou a Slovenskem}:

  \begin{itemize}
  \tightlist
  \item
    Břeclav, Sudoměřice u Strážnice - trdelníky se smetanou, i Slovensko
  \end{itemize}
\item
  1932
  \href{https://www.digitalniknihovna.cz/mzk/view/uuid:ea5d2d20-0652-11e8-b1a1-005056827e52?page=uuid\%3A58510a10-3f36-11e8-a7aa-005056825209&fulltext=trdeln\%C3\%ADk}{Jeřabiny:
  obrázky z horního Pojizeří}:

  \begin{itemize}
  \tightlist
  \item
    povídky od Marie Markvartové
  \item
    povídka Jiný svět, o rodině učitele v Bahensku (asi fiktivní
    vesnice), jeho dcera peče trdelníky
  \item
    každopádně je to jeden ze zdrojů pro slovník podkrkonošského nářečí
  \end{itemize}
\item
  1933
  \href{https://ceskadigitalniknihovna.cz/view/uuid:1026bc40-ea8b-11dc-9dcb-000d606f5dc6?page=uuid\%3Ac347cf40-31ff-4912-b767-a8cb9e4e8683&fulltext=trdeln*&source=nkp}{Lidové
  noviny}:

  \begin{itemize}
  \tightlist
  \item
    článek Znáte trdelníky? autorka M. Ú (čili M. Úlehlová-Tilschová)
  \item
    snad ještě pamatujete na vysoké trdlovce nebo Baumkuchen, které
    kdysi byly v cukrárnách
  \item
    pak popisuje rápadoslovenské trdelníky
  \item
    recept (mouka, máslo, sádlo, 5 žloutků, droždí, cukr, mlíko), sypaný
    madhlema, potíraný máslem a bílkama
  \item
    krájí se na kolečka (to je rozdíl oproti tomu, co popisuje v
    pozdějších knížkách, kde se odmotávají)
  \end{itemize}
\item
  1933
  \href{https://ceskadigitalniknihovna.cz/view/uuid:60eab9a0-0be7-11ed-8635-005056827e52?page=uuid\%3A8ee6bb86-db3e-4211-a60a-d927a201dc69&fulltext=trdeln*&source=mzk}{Rektor
  Kvěch a jiná próza}:

  \begin{itemize}
  \tightlist
  \item
    Viktor Kamil Jeřábek
  \item
    příběhy o venskovskym učiteli v Třešňůvkách na Moravě, nejdřív se mu
    dvoří místní dívky a podstrkujou trdelníky, pak jsou tdelníky i na
    svatbě
  \item
    autor byl ve skutečnosti učitelem v
    \href{https://kramerius.lib.cas.cz/view/uuid:48220bc0-4294-11e2-b246-005056827e52?page=uuid\%3A04efbdca-01ec-4111-8df2-32a5f52f4c16}{Silůvkách
    v letech 1882-87}, ty mu sloužily jako vzor pro Třešňůvky
  \end{itemize}
\item
  1933
  \href{https://www.digitalniknihovna.cz/vkol/uuid/uuid:349c38c2-fadc-4992-a172-9aacc818e434}{Slovácké
  noviny}:

  \begin{itemize}
  \tightlist
  \item
    do muzea Uh. Brodu věnovala Mar. Hladká sekáč, kopál a trdelník
  \end{itemize}
\item
  1934
  \href{https://ceskadigitalniknihovna.cz/view/uuid:e1259815-3caf-4b6b-af1d-8ada80900484?page=uuid\%3A8bb1ba0a-525b-11e5-a788-0050569d679d&fulltext=trdeln\%C3\%ADk\%20OR\%20trdeln\%C3\%ADky\%20OR\%20trdeln\%C3\%ADk\%C5\%AF&source=mzk}{Nářečí
  na Kyjovsku a Ždánsku}:

  \begin{itemize}
  \tightlist
  \item
    popisuje změny venkoskýho života, ubývá dodraných se smaženýma
    koblihama a trdleníkama
  \end{itemize}
\item
  1934
  \href{https://ceskadigitalniknihovna.cz/view/uuid:c1ecf480-9c2d-11e3-8b69-005056825209?page=uuid\%3Ac53ff231-d798-11e3-94ef-5ef3fc9ae867&fulltext=trdlovec&source=nkp}{Polední
  list}:

  \begin{itemize}
  \tightlist
  \item
    1934, č. 57, s. 3
  \item
    článek o matějský pouti, mimojiný nabízeli staročeský trdlovec za
    kačku
  \end{itemize}
\item
  1935
  \href{https://ceskadigitalniknihovna.cz/view/uuid:9e3e00c0-1eb9-11ed-bb16-005056827e52?page=uuid\%3Ac70b483f-d1ae-48f7-9b87-ec788605bd6c&fulltext=trdeln*&source=mzk}{Bouře
  a jiné dědinské kroniky}:

  \begin{itemize}
  \tightlist
  \item
    Viktor Kamil Jeřábek
  \item
    hospodyně smaží trdelníky pro manžela, ten ale vyráží na zálety, tak
    ty trdelníky dá pacholkovi
  \end{itemize}
\item
  1936
  \href{https://ceskadigitalniknihovna.cz/view/uuid:21c28337-816c-4103-a6ae-143ab29b452a?page=uuid\%3A138254c0-7d8d-11e9-b171-5ef3fc9ae867&fulltext=trdeln\%C3\%ADk\%20OR\%20trdeln\%C3\%ADky\%20OR\%20trdeln\%C3\%ADk\%C5\%AF&source=nkp}{Moravský
  deník}:

  \begin{itemize}
  \tightlist
  \item
    prezident Beneš navštívil Slováckou búdu v Luhačovicích a přivítali
    ho trdelníkem
  \end{itemize}
\item
  1936
  \href{https://ceskadigitalniknihovna.cz/view/uuid:5e938270-0675-11df-93e5-000d606f5dc6?page=uuid:37544fe0-fb63-11de-ba13-000d606f5dc6&fulltext=trdeln\%C3\%AD*&source=mzk}{Moravská
  orlice}:

  \begin{itemize}
  \tightlist
  \item
    další článek o tom, jak prezident Bendeš dostal v Luhačovicích
    trdelníky
  \end{itemize}
\item
  1936
  \href{https://ceskadigitalniknihovna.cz/view/uuid:f2e890b0-435d-11dd-b505-00145e5790ea?page=uuid\%3A6d16d5b9-435f-11dd-b505-00145e5790ea&fulltext=trdeln\%C3\%ADk\%20OR\%20trdeln\%C3\%ADky\%20OR\%20trdeln\%C3\%ADk\%C5\%AF&source=svkhk}{Národní
  listy}:

  \begin{itemize}
  \tightlist
  \item
    článek Posvícenské a sváteční pečivo: sláva české kuchyně
  \item
    popisuje různý pečiva
  \item
    o masopustu se dělaj smaženky: kobližky, šišky, makovníky,
    trdelníky, boží milosti, smažený na másle nebo sádle
  \end{itemize}
\item
  1936
  \href{https://ceskadigitalniknihovna.cz/view/uuid:d4dd6060-bfc4-11e3-aec3-005056827e52?page=uuid\%3A23cc3e33-7f1f-49cd-8c09-67351d084c07&fulltext=trdeln*&source=nkp}{Příruční
  slovník německo-český}:

  \begin{itemize}
  \tightlist
  \item
    baumkuchen = trdelník, trdlovec
  \end{itemize}
\item
  1937
  \href{http://digitalna.kniznica.info/zoom/90599/view?search=trdeln\%C3\%ADk\%20OR\%20trdeln\%C3\%ADka\%20OR\%20trdeln\%C3\%ADku&page=6&p=separate&tool=search&view=1502,2130,4778,2339}{Slovenská
  obrana}:

  \begin{itemize}
  \tightlist
  \item
    Slovensko, dopis z Ameriky, popisuje život místní slovenský
    komunity, byly na návštěvě u známejch na farmě a dostali trdelník,
    kterej ale nešel sundat z formy, takže ho museli upilovat
  \end{itemize}
\item
  1937
  \href{http://digitalna.kniznica.info/zoom/90688/view?search=trdeln\%C3\%ADk\%20OR\%20trdeln\%C3\%ADka\%20OR\%20trdeln\%C3\%ADku&page=6&p=separate&tool=search&view=2631,1599,4628,2266}{Slovenská
  obrana}:

  \begin{itemize}
  \tightlist
  \item
    Slovensko, další dopis, autorka dává recept na náplň trdelníku ze
    sněhu s paprikou a křenem, pro zlobivý manžely
  \end{itemize}
\item
  1937
  \href{https://ceskadigitalniknihovna.cz/uuid/uuid:b42af96b-ef00-4cc8-ac42-7413f52b7595}{České
  slovo}:

  \begin{itemize}
  \tightlist
  \item
    není dostupný online, ale v náhledu je:

    \begin{itemize}
    \tightlist
    \item
      zato mají o hodech na Maršově koláče, buchty, makovníky a trdláče,
      již z první letošní mouky,
    \end{itemize}
  \end{itemize}
\item
  1937
  \href{https://ceskadigitalniknihovna.cz/uuid/uuid:6f308ed0-4eaf-11ed-8756-005056827e51}{Nejnovější
  ilustrovaná kuchařská kniha obsahující 3705 vyzkoušenýc}:

  \begin{itemize}
  \tightlist
  \item
    Ondráčková, Luisa
  \item
    s. 293: obrázek trdelníků
  \item
    s, 294: recept na trdelník v troubě
  \end{itemize}
\item
  1937
  \href{https://ceskadigitalniknihovna.cz/uuid/uuid:0e56bea0-5587-11e5-81eb-001018b5eb5c}{Měsíc}:

  \begin{itemize}
  \tightlist
  \item
    Josef Sekora: Čtyči lovci kolem ohně
  \item
    povídka, dohadujou se, co jíst, jeden dělá skalický trdelníky, peče
    na ohni, sype mandlema
  \end{itemize}
\item
  1938
  \href{https://ndk.cz/view/uuid:001e82b0-302a-11e9-b81e-005056827e52?page=uuid\%3A3123a720-50df-11e9-8854-005056827e51&fulltext=trdeln\%C3\%ADk}{Rajský
  ostrov}:

  \begin{itemize}
  \tightlist
  \item
    Jaromír John
  \item
    knížka pro děti, fiktivní příběh o požáru Národního divadla
  \item
    vypráví, jak se tatínek po roce 1850 ženil ve Štěpánově u Olomouce a
    v hospodě měli nejlepší moravský lahůdky, mezi nimi i ohromný
    trdelník
  \item
    (ale jeho skutečnej otec se narodil 1847, takže kdo ví, jak moc je
    to založený na realitě)
  \end{itemize}
\item
  1938
  \href{https://web.archive.org/web/20201230125915/http://kurtos.eu/dl/11664.pdf}{Wiener
  Kuche}:

  \begin{itemize}
  \tightlist
  \item
    Kochen im Bild: Herstellung von Prügelkrapfen = Vaření v obraze:
    Příprava Prügelkrapfenů
  \item
    rakouskej článek, recept na trdelník pečenej v troubě
  \item
    obrázky roštu na držení forem, schéma navíjení, obrázky hotovejch
    trdelníků
  \item
    pojmenovávají trdelník trdelnice, což je normálně polívka
  \item
    \hyperref[250615-0047]{Překlad receptu z Wienech Kuche}
  \end{itemize}
\item
  1938
  \href{https://ceskadigitalniknihovna.cz/uuid/uuid:e6636d2b-f486-4e2f-b5ae-fb4e89b772e4}{Jihočeské
  listy}:

  \begin{itemize}
  \tightlist
  \item
    Návštěvou na Moravském Slovácku - tentokrát bez metafory
  \item
    rozhovor dvou osob
  \item
    jeden vzpomíná na rok 1911, kdy byl v Luhačovicích, chodil k dr.
    Blahovi na Slováckou búdu, kde měl trdelníky, což jsou sladké
    preclíčky na tyči sušené
  \end{itemize}
\item
  1940
  \href{https://ndk.cz/view/uuid:911977c0-052e-11e8-816d-5ef3fc9bb22f?page=uuid\%3Aa0797b50-0cb5-11e8-8ee4-005056825209}{Vzpomínky,
  Božena Mrštíková}:

  \begin{itemize}
  \tightlist
  \item
    vzpomínky na Viléma Mrštíka z Diváků u Klobouků
  \item
    kynutý těsto s rozinkama
  \end{itemize}
\item
  1940
  \href{https://ceskadigitalniknihovna.cz/view/uuid:d065e200-90ad-11e8-87bd-005056827e52?page=uuid:47609990-a79c-11e8-ba56-5ef3fc9bb22f&fulltext=trdeln\%C3\%ADky&source=nkp}{Křečkovice}
  - část Vyškova, trdelníky dostávali dělníci po žních
\item
  1940
  \href{https://ceskadigitalniknihovna.cz/uuid/uuid:c6220b36-bdd1-4eec-94b2-2bed755a042a}{Jihočeské
  listy}:

  \begin{itemize}
  \tightlist
  \item
    s. 2, Moravští učitelé zblízka
  \item
    učitelé z Hané a Slovácka jsou na návštěvě v Písku a Českých
    Budějovicích
  \item
    přivezli moravské klobásy, trdelníky a bzenecké víno
  \end{itemize}
\item
  1940
  \href{https://ceskadigitalniknihovna.cz/uuid/uuid:9582fdd0-6e03-11dd-a59e-000d606f5dc6}{Lidové
  noviny}:

  \begin{itemize}
  \tightlist
  \item
    anketa čtenářek Lidových novin
  \item
    Na svatbě a do kouta
  \item
    Anna Marková z Lysic
  \item
    popisuje, jak žili lidé za časů její babičky u Letovic, na rozhraní
    Malé Hané a Horácka
  \item
    do kouta se posílala velká buchta, smažené trubičky, trdelníky, nebo
    jiné jemnější pečivo
  \end{itemize}
\item
  1941
  \href{https://ceskadigitalniknihovna.cz/view/uuid:5247f390-0408-11ed-bd12-005056827e51?page=uuid\%3Ac9bca5eb-64b6-4861-a251-c29522db0865&fulltext=trdlovce&source=mzk}{Pod
  staropražským nebem}:

  \begin{itemize}
  \tightlist
  \item
    vyprávění babičky autorky o starých časech
  \item
    maminka jí brala jako malou do Nuslí, na jarmark na Fidlovačce a
    kupovali si marcipán a kus trdlovce
  \end{itemize}
\item
  1941
  \href{https://ceskadigitalniknihovna.cz/view/uuid:5a644b90-81df-11e4-9d8c-005056827e51?page=uuid\%3A645f2680-95c9-11e4-a2db-005056825209&source=nkp}{Hvězdy
  nad Roupkovem}:

  \begin{itemize}
  \tightlist
  \item
    humoristickej román, na trhu prodávají národní staročeské jídlo
    trdlovec, jinej prodejce prodává pravej tureckej trdlovec
  \end{itemize}
\item
  1941
  \href{https://ceskadigitalniknihovna.cz/view/uuid:3d76f2a0-2cf6-11e4-8e0d-005056827e51?page=uuid\%3A4ffb7cf0-4b61-11e4-aded-005056827e51&fulltext=trdeln\%C3\%ADk\%20OR\%20trdeln\%C3\%ADky\%20OR\%20trdeln\%C3\%ADk\%C5\%AF&source=nkp}{Lidové
  umění na Hané}:

  \begin{itemize}
  \tightlist
  \item
    kuchyňské výrobky, trdelníky v peci na rožni pečené
  \end{itemize}
\item
  1941
  \href{https://ceskadigitalniknihovna.cz/view/uuid:196e9f80-fea6-11de-bd64-000d606f5dc6?page=uuid:30a645e0-fac8-11de-b0ad-000d606f5dc6&fulltext=trdeln\%C3\%AD*&source=mzk}{Moravská
  orlice}:

  \begin{itemize}
  \tightlist
  \item
    Jak štodyroval první Hanák na Chropyňské škole
  \item
    historka, jak syn sedláka z Bilan u Kroměříže se vrátil po školním
    roce domů a vítali ho jak na svatbě, maso, trdelníky, boží milosti
    atd.
  \item
    stejná historka je už v Jak se naši škádlívají z roku 1924
  \end{itemize}
\item
  1942
  \href{https://ceskadigitalniknihovna.cz/view/uuid:a493cef0-fea6-11de-81a7-000d606f5dc6?page=uuid:30eaf000-fac8-11de-9bea-000d606f5dc6&fulltext=trdeln\%C3\%AD*&source=mzk}{Moravská
  orlice}:

  \begin{itemize}
  \tightlist
  \item
    Rudolf Kynčl, Hore Velkú, Uječek Matěj Škrobák
  \item
    vzpomínají na zesnulýho dědečka, i babičku a její trdelníky
  \end{itemize}
\item
  1942
  \href{https://ceskadigitalniknihovna.cz/view/uuid:b2ec6826-435d-11dd-b505-00145e5790ea?page=uuid\%3A5f028ec9-435f-11dd-b505-00145e5790ea&fulltext=trdeln\%C3\%ADk\%20OR\%20trdeln\%C3\%ADky\%20OR\%20trdeln\%C3\%ADk\%C5\%AF&source=mzk}{Pestrý
  týden}:

  \begin{itemize}
  \tightlist
  \item
    Slovácká jídla, smaženej trdelník mezi nima
  \end{itemize}
\item
  1942:

  \begin{itemize}
  \tightlist
  \item
    MATĚJÍČEK, Rudolf: Trdelníky. Lidová tvorba 3, 1942, č. 5, s. 7--8
  \item
    musím získat, najít
  \end{itemize}
\item
  1942
  \href{https://ceskadigitalniknihovna.cz/uuid/uuid:086b8270-e420-11e8-a5a4-005056827e52}{Velký
  česko-německý slovník Unikum s mluvnicí, pravopisem, frazeologií}:

  \begin{itemize}
  \tightlist
  \item
    trdlovec = Baumkuchen
  \end{itemize}
\item
  1942
  \href{https://ceskadigitalniknihovna.cz/uuid/uuid:2c588580-119c-11e9-a03f-5ef3fc9bb22f}{Mrštíkové}:

  \begin{itemize}
  \tightlist
  \item
    Božena Mrštíková
  \item
    vzpomínky na paní Hrdličkovou, která občas dělala trdelníky, který
    měl rád především Vilém
  \end{itemize}
\item
  1943
  \href{https://ceskadigitalniknihovna.cz/uuid/uuid:dc1db170-6fe9-11ed-a5ba-005056827e51}{Přišlá}:

  \begin{itemize}
  \tightlist
  \item
    román, Stanislav Kovanda
  \item
    po svatbě posílají výslužku příbuzným, koš plný koláčů, koblih,
    božích milostí a trdelníků
  \end{itemize}
\item
  1944
  \href{https://ceskadigitalniknihovna.cz/uuid/uuid:bca2a111-2a25-42f8-b998-2567d7c7e193}{Hanácké
  povídky}:

  \begin{itemize}
  \tightlist
  \item
    Nevšímalová, Krista (1854-1935), takže vyšlo posmrtně
  \item
    Naši stařeček

    \begin{itemize}
    \tightlist
    \item
      svatba, podávají se buchty, makovníky, trdelníky a beleše
    \end{itemize}
  \end{itemize}
\item
  1944
  \href{https://ceskadigitalniknihovna.cz/uuid/uuid:78576090-e45a-11e5-8d5f-005056827e51}{Povídky
  slunného kraje}:

  \begin{itemize}
  \tightlist
  \item
    Bohumila Dubňanská
  \item
    s. 37: Strúhalka na starým ohništi peče tradiční slováckou lahůdku,
    trdelníky
  \item
    s. 38: chystá se do Budapešti asi za synem, kterýho dlouho neviděla,
    trdelníky veze jako dárek
  \item
    s. 47: setkala se synem, není nadšenej. trdelníky jsou pro manželku
  \item
    s. 50: maďarská manželka se směje trdelníkům, nechápe co to je
  \item
    s. 53: pijou víno, už i trdelníky jedí, ale manželka se jim pořád
    směje
  \item
    s. 63: Strúhalka se chystá ne cestu zpátky, služebná jí dá zbylý
    trdelníky, nechtějí je jíst
  \end{itemize}
\item
  1945
  \href{https://ceskadigitalniknihovna.cz/view/uuid:89562bf0-7962-11e5-9690-005056827e51?page=uuid\%3A521d38f0-92de-11e5-bf6c-005056825209&fulltext=trdeln\%C3\%ADk\%20OR\%20trdeln\%C3\%ADky\%20OR\%20trdeln\%C3\%ADk\%C5\%AF&source=mzk}{Česká
  strava lidová}:

  \begin{itemize}
  \tightlist
  \item
    Marie Úlehlová-Tilschová
  \item
    s. 118 - stařenka z Podřipska si stěžuje, že uherská mouka nemá
    žádnou sílu, že za mladejch let navařili trdelníky a měly je celej
    den - ale to budou asi knedlíky
  \item
    s. 367 - na Mikuláše přišla stařenka s různými dobrotami, včetně
    trdelníčků
  \item
    s. 376 - na podluží hostí návštěvy obarovicí, jitrnicema se zelím a
    trdelníkem nebo koblihama
  \item
    s. 385 - na Ostrožsku se trdelníky smažily,
  \item
    s. 396 - cituje dárky pro šestinedělku z Lidských dokumentů,
  \item
    s. 415 - trdelníky jako svatební pečivo na Slovácku, pečený nad
    uhlíkama

    \begin{itemize}
    \tightlist
    \item
      podle pamětí Josefa Úlehly se mají odvíjet pramínky, jen cizinci
      je krájejí
    \item
      to je zajímavý, protože v jinejch zdrojích se právě krájí na
      kroužky
    \item
      \href{https://encyklopedie.brna.cz/home-mmb/?acc=profil-osobnosti&load=3473}{Josef
      Úlehla} (1852 - 1933), tchán Tilschový, byl učitel, působil na
      hodně místech na Moravě
    \end{itemize}
  \item
    s. 571 - jedinou památkou na pečení na rožni v naší lidové kuchyni
    není maso, ale trdelníky, zvláštní slavnostní pečivo, co se navinuje
    na válec zvaný trdlo a peče při stálém otáčení nad ohněm. Dělá se
    ještě na Slovácku, hlavně o křtinách, svatbách nebo v masopustě
  \end{itemize}
\item
  1945
  \href{https://ceskadigitalniknihovna.cz/view/uuid:a0628fa0-c266-11ed-9eb6-005056827e52?page=uuid\%3Ac795fb09-57f5-4221-8dc6-28537310e770&fulltext=trdeln\%C3\%ADk\%20OR\%20trdeln\%C3\%ADky\%20OR\%20trdeln\%C3\%ADk\%C5\%AF&source=mzk}{Paměti
  uječka Matěja Škrobáka}:

  \begin{itemize}
  \tightlist
  \item
    smažený trdelníky na kostovym oleji, velký jako Tomšova trumpétka
  \item
    odehrává se v roce 1896
  \end{itemize}
\item
  1945
  \href{https://ceskadigitalniknihovna.cz/view/uuid:418433d0-7277-11dc-b60a-000d606f5dc6?page=uuid\%3Ad7f9cd6a-a3fe-4479-9f34-63429f6ac6d1&fulltext=trdeln\%C3\%ADk\%20OR\%20trdeln\%C3\%ADky\%20OR\%20trdeln\%C3\%ADk\%C5\%AF&source=nkp}{Náš
  kraj} - článek o masopustu na Slovácku, trdelníky smažený
\item
  1946
  \href{https://ceskadigitalniknihovna.cz/view/uuid:e0e42f00-3284-11e4-8f64-005056827e52?page=uuid\%3A6ed8c580-46e6-11e4-a450-5ef3fc9bb22f&fulltext=trdeln\%C3\%ADk\%20OR\%20trdeln\%C3\%ADky\%20OR\%20trdeln\%C3\%ADk\%C5\%AF&source=nkp}{Slovník
  nářečí mistřického}:

  \begin{itemize}
  \tightlist
  \item
    trdelník = dřevěný váleček na způsob rumpálu
  \item
    trdláč = těsto namotaný na trdelníku opečený nad ohněm a pomašťovaný
  \end{itemize}
\item
  1946
  \href{https://ceskadigitalniknihovna.cz/uuid/uuid:c22c5a5d-8140-11ed-b210-001b63bd97ba}{Projekt
  Bartošova národopisného musea v přírodě}:

  \begin{itemize}
  \tightlist
  \item
    popis roku v národopisné dědině na Hradisku, v říjnu se slaví
    slovácký vinobraní a podávají se rázovitý trdelníky
  \end{itemize}
\item
  1946
  \href{https://ceskadigitalniknihovna.cz/uuid/uuid:4b7ffab0-68c3-11e8-a583-005056827e51}{Národní
  obroda}:

  \begin{itemize}
  \tightlist
  \item
    článek trdelníky, maminka peče trdelníky v troubě pro děti jen tak.
    Nivnice
  \item
    \href{https://ceskadigitalniknihovna.cz/uuid/uuid:747b7700-68c3-11e8-943b-5ef3fc9ae867}{tady
    je pokračování}
  \end{itemize}
\item
  1948
  \href{https://dikda.snk.sk/uuid/uuid:92a95dfe-83fe-44e4-9c0a-5c7cec70c80c}{Zlatá
  kniha kuchárska}:

  \begin{itemize}
  \tightlist
  \item
    Slovensko, recept na Skalický trdelník
  \end{itemize}
\item
  1948
  \href{https://ceskadigitalniknihovna.cz/uuid/uuid:1536d800-488a-11e4-aded-005056827e51}{Památník
  musea J.A. Komenského v Uh. Brodě}:

  \begin{itemize}
  \tightlist
  \item
    popis národopisné výstavy v Uh. Brodě 1895
  \item
    byl tam slovácký šenk, kde se podávaly trdláče a boží milosti
  \end{itemize}
\item
  1948
  \href{https://dikda.snk.sk/uuid/uuid:fe08d509-63e7-4e4f-bf4f-0d9cef56b5a8}{Anička
  Jurkovičová}:

  \begin{itemize}
  \tightlist
  \item
    slovenská národní buditelka
  \item
    v roce 1848 se skrývá u svojí sestry v Rusavě na Moravě a narodí se
    jí syn
  \item
    místní babky je zásobovaly trdelníkama, radostníkama, božíma
    milostma a šiškama
  \end{itemize}
\item
  1949
  \href{https://ceskadigitalniknihovna.cz/uuid/uuid:66097490-f209-11e3-a012-005056825209}{Malovaný
  kraj}:

  \begin{itemize}
  \tightlist
  \item
    ples na dědině, trdelníky mezi dalšíma dobrotama v rohu sálu
  \end{itemize}
\item
  1949
  \href{https://ceskadigitalniknihovna.cz/uuid/uuid:3c65e0dc-eea0-4315-9819-a34d4fe39616}{Rok
  v naší kuchyni}:

  \begin{itemize}
  \tightlist
  \item
    M. Úlehlová-Tilschová
  \item
    s. 128: na svatbě na Moravsko-Slovenském pomezí trdelníky, pečené
    ještě nad otevřeným ohněm
  \item
    s. 326 recept, pečený na ohni, posypaný mandlema
  \end{itemize}
\item
  1951
  \href{https://ceskadigitalniknihovna.cz/view/uuid:20d0de00-02ba-11e5-9506-005056827e51?page=uuid\%3Ad5741950-0ec5-11e5-b562-005056827e51&fulltext=trdlovec*\%20OR\%20trdlovcem&source=mzk}{Tak
  jsme žili}:

  \begin{itemize}
  \tightlist
  \item
    popis přípravy trdlovce, ale je to překlad z němčiny
  \end{itemize}
\item
  1953
  \href{https://ceskadigitalniknihovna.cz/uuid/uuid:b65792bb-dbc4-4d41-b1da-e0aec484459d}{Příruční
  slovník jazyka českého}:

  \begin{itemize}
  \tightlist
  \item
    s. 212 slovníkový Heslo
  \item
    s. 213 heslo pro trdlovec
  \end{itemize}
\item
  1954
  \href{https://dikda.snk.sk/uuid/uuid:38df4d2e-5eea-4484-9e22-e5a1080ba4e3}{Varíme
  zdravo, chutno a hospodárne}:

  \begin{itemize}
  \tightlist
  \item
    slovenská kuchařka, recept na skalickej trdelník
  \end{itemize}
\item
  1956
  \href{https://dikda.snk.sk/uuid/uuid:906d894f-daa0-4fa8-9756-804e856a0d44}{Ľudová
  tvorivosť}:

  \begin{itemize}
  \tightlist
  \item
    List z babkárskej Skalice
  \item
    jen zmínka o trdelníku, zázvorníku a skalickym dortu, jako tradičním
    pečivu
  \end{itemize}
\item
  1958
  \href{https://ceskadigitalniknihovna.cz/uuid/uuid:4d82e3c0-db26-11e2-9923-005056827e52}{Proč
  mlčí živí}:

  \begin{itemize}
  \tightlist
  \item
    Pecháčková, Františka, beletrie
  \item
    stařeček cítí, že umře, tak prosí stařenku, jestli by mu napekla
    starodávného masopustního pečiva, trdelníků, jaké dělávala, když
    byli mladí
  \item
    stařenka vyhrabala v polici železnou tyč, péct ale musí u Josefy
    Pazderkové, která má ještě ohniště
  \item
    pak popisuje, jak připravily ohniště a rožeň, trdelníky mastily
    lojem
  \item
    stařeček mezitím zemřel
  \end{itemize}
\item
  1959
  \href{https://ceskadigitalniknihovna.cz/uuid/uuid:d7bb5685-7a29-11ed-b508-001b63bd97ba}{Výroční
  obyčeje a lidové umění}:

  \begin{itemize}
  \tightlist
  \item
    jen drobný poznámky, poz. 10, s. 486, že se trdelníky dělaj z
    nekvašenýho těsta, s. 487, že se na Slovácku a v okolí Skalice smaží
    a dřív dělaly na rožni
  \end{itemize}
\item
  1960
  \href{https://ceskadigitalniknihovna.cz/view/uuid:747980d0-a9c9-11e7-920d-005056827e51?page=uuid\%3A6d279530-a5e3-11e8-8b41-005056822549&fulltext=trdlovec&source=nkp}{Ethnographica}:

  \begin{itemize}
  \tightlist
  \item
    Soňa Švecová, Vývoj ohnišť na jižní Moravě a v přilehlých oblastech
    západního Slovenska.
  \item
    článek v němčině, tohle jsou překlady

    \begin{itemize}
    \tightlist
    \item
      Ohniště jihomoravské oblasti prošla, pokud je to dnes ještě v
      terénu zjistitelné, v průběhu posledních 70 až 80 let významnými
      změnami. Všechny tyto změny se soustředí téměř výhradně na
      nejdůležitější pravěké ohniště, pec. Pokud si představíme tuto
      jihomoravskou oblast, která tvoří klín mezi hranicemi Rakouska a
      Slovenska, rozšířenou o zhruba 20 až 30 kilometrů hluboký pruh
      východním směrem ke Slovensku, pak můžeme hovořit o třech
      výrazných vývojových fázích ohnišť takto vymezené oblasti, které
      nejsou bez významu ani z hlediska středoevropského členění. Kolem
      přelomu století se zde ještě setkáváme s pecemi vytápěnými podle
      předovkového principu, vedle toho se vyskytuje i princip zadovkový
      a konečně je v tomto období zaznamenán i rozpad pecí, který se
      projevuje v různých formách; někdy se pec v posledně jmenovaném
      období stahuje do chodby, v jiných případech zcela mizí. Tyto tři
      základní etapy vývoje, které v západněji položených částech naší
      vlasti v terénu již nenacházíme, nejsou ve východním Rakousku a na
      Slovensku ani dnes vzácností. Popsaná oblast je však zajímavá tím,
      že představuje nejjižnější výběžek oblasti rozšíření předovkového
      principu na své západní hranici z doby kolem přelomu století.
      Nejstarší forma těchto ohnišť tak zároveň tvoří most k dalšímu
      výběžku dolnorakouské oblasti rozšíření. Mnoho badatelů také
      nastolilo otázku, zda kdysi existovala kontinuita oblasti dýmnic
      východních Alp s našimi dýmnicemi. Pro zkoumání takové otázky by
      přicházelo v úvahu především srovnání materiálu, který byl nalezen
      v bezprostředně sousedících oblastech.1)
    \end{itemize}
  \item
    popis nádobí v němčině. přibližnej překlad:

    \begin{itemize}
    \tightlist
    \item
      K tomuto časově mladšímu inventáři (trojnožka, rendlíky atd.),
      který je obecně charakteristický pro západní střední Evropu, na
      rozdíl od výše zmíněného způsobu vaření, který je typický pro
      východní oblasti, přibyly další předměty, které sloužily k
      přípravě jídel na otevřeném ohništi. Bylo to především nádobí na
      pečení moučníků, tzv. belešníky, hliněné pekáče s nožičkami, a
      trdelníky (zhruba „koláčový strom`` -- trdlo se nazývá Baumkuchen,
      pozn. překl.). Posledně jmenované byly válce, na které se navinulo
      těsto a které se pak otáčely nad ohněm.18) Tato dvě zařízení měla
      rozměry, které by se na poměrně úzký sporák už nevešly. Jak ale
      brzy uslyšíme, tvar ohnišť se od minulého století již natolik
      změnil, že už nemůžeme mluvit o úzkém sporáku, který by se
      používal k vaření, který by se tvarem ještě shodoval s předovkami.
    \end{itemize}
  \end{itemize}
\item
  1961
  \href{https://ceskadigitalniknihovna.cz/view/uuid:aab22200-ca09-11ec-af1b-005056827e51?page=uuid:879b9f5b-9287-4f6f-9958-eca21704f1b6&fulltext=trdeln\%C3\%ADk\%20OR\%20trdeln\%C3\%ADky\%20OR\%20trdeln\%C3\%ADku\%20&source=mzk}{Lidová
  strava na Brněnsku}, Ludvíková, Miroslava:

  \begin{itemize}
  \tightlist
  \item
    s. 108 konec masopustu, válec nazývají dřevko
  \item
    s. 110 důdirek, slavený v masopustě, hospodyně připravila pro
    pomocníky čaj nebo kávu, koblihy, trdelníky a koláče
  \item
    s. 111:

    \begin{itemize}
    \tightlist
    \item
      pro šestinedělku do kúta, nosily to kmotry, proto se často za
      kmotry vybíraly bohaté selky, i když to vytvářelo podřízené
      postavení
    \item
      na křtiny u bohatých se podávaly trdelníky, koláče, koblihy,
      slepice, řízky
    \end{itemize}
  \item
    s. 114:

    \begin{itemize}
    \tightlist
    \item
      topeniště:

      \begin{itemize}
      \tightlist
      \item
        asi do konce 80, let se vařilo na ohništi (vohnisko) tvarovaném
        z hlíny, bylo v samostatný místnosti, nebo jen ve výklenku
      \item
        v zimě se topilo v kamnech (peci)
      \item
        nad ohništěm byl otevřený komín
      \item
        později se komín zabednil a přistavil sporák
      \item
        kolem roku 1900 už mají skoro všechny domácnosti kuchyni jako
        místnost
      \end{itemize}
    \item
      nádobí - kozička a dřevka na trdelníky
    \end{itemize}
  \item
    s. 115 po zmizení otevřených ohnišť zmizely dřévka
  \item
  \end{itemize}
\item
  1963 LUDVÍKOVÁ, Miroslava, Lidová strava v jižní části Drahanské
  vrchoviny, Časopis Moravského musea -- Acta Musei Moraviae, roč. 48,
  1963, s. 161--198.

  \begin{itemize}
  \tightlist
  \item
    není online, mají v
    \href{https://www.knihovny.cz/Record/nkp.SLK01-000511327?sid=30934569}{NK}
  \item
    podívat
  \end{itemize}
\item
  1964 \href{https://cakes.institute/hahn64.html}{Fritz Hahn: The
  Baumkuchen Family}:

  \begin{itemize}
  \tightlist
  \item
    Die Familie der Baumkuchen, originally published simultaneously in
    Die Konditorei and Der Konditormeister, July 1964.
  \item
    překlad německýho článku, historie koláčů na rožni
  \item
    popisuje několik generací stromovýho koláče:

    \begin{itemize}
    \tightlist
    \item
      1. generace - antický Obelias
    \item
      2. generace - Spiesskuchen, had z těsta
    \item
      3. generace - Baumstriezel, Böhmische Kuchlein, recept od Rumpolta
      - plát těsta
    \item
      4. generace - populární v 17. století, těsto jako na lívance, leje
      se na špíz - P
    \item
      5. generace - od poloviny 18. století, Baumkuchen, Prügelkrapfen,
      atd.
    \end{itemize}
  \item
    V rukopisu Codex Palatinus Germanicus z doby kolem roku 1450 se
    kromě jiných vynikajících kuchařských receptů píše toto: „Jíst koláč
    na špízu. Chceš-li udělat koláč na špízu z pohanského těsta, připrav
    ho dobře ze dvou nebo tří barev. A polož je vedle sebe podélně, aby
    nebyly kratší než ostatní. A pak jeden díl na jednom místě nožem
    seřízni a oviň kolem dřevěného špízu. A potři ho žloutkem na
    místech, aby se ti celý nezlepil. A nepeč ho příliš horký.``

    \begin{itemize}
    \tightlist
    \item
      Co se rozumělo pod „pohanským nebo českým`` těstem, resp. koláčem,
      je patrné z o 100 let staršího würzburského pergamenového
      rukopisu. Podle něj se do těsta přidával med, vejce, pepř,
      „koření`` a mandle. Barevná těsta byla tehdy velmi oblíbená a
      uvádějí se odpovídající pokyny pro barvení. Šafrán pro žlutou,
      špenátová šťáva pro zelenou, chrpa pro modrou a další.
    \end{itemize}
  \item
    v Jáchymově 1554: In Joachimsthal (Bohemia), spit cakes even raised
    the ire of the church. The great preacher Mathesius, a friend of
    Martin Luther, wrote the following in his posthumous work Syrach
    (1554): ``As the men go out the front, the women go out back to the
    Krahles, for weeks, eating spit cakes and guzzling beer.''
  \item
    Němci v Transylvánii používají název Baumstritzel nebo
    Schornstein-Kolatschen (komínovej koláč)
  \item
    v knížce Haus-, Feld-, Artzney-, Koch-, Kunst- und Wunder-Buch od
    Johann Christoph Thieme, 1682 je recept na trdelník pojmenovanej
    Bohemian Sweetcakes:

    \begin{itemize}
    \tightlist
    \item
      \href{https://books.google.cz/books?id=PytAAAAAcAAJ&pg=PA1604&hl=cs&source=gbs_selected_pages&cad=1\#v=onepage&q=bohmisch&f=false}{s.
      876}
    \item
      je to obdélníkovej kus těsta navinutej na rožeň a omotanej drátem
    \item
      Vezměte mléko, mouku a vejce a udělejte těsto, dejte do něj trochu
      kvasnic a nechte je trochu vykynout, poté těsto zpracujte, aby
      bylo pevné, rozválejte ho do šířky, posypte rozinkami, skořicí a
      muškátovým květem, poté vezměte rožeň, který je k tomu určený, a
      opečte to na ohni, potřete šmalcem a těsto na něj navlékněte a
      opečte ho, můžete také nejprve obalit papírem, aby nespadlo z
      rožně, dokud je ještě měkké.
    \item
      na s. 871 je recept na klasickej spiesskuchen:

      \begin{itemize}
      \tightlist
      \item
        Příprava dortu na rožni
      \item
        Vezměte teplé mléko a rozklepněte do něj vejce, udělejte těsto z
        krásné bílé mouky, přidejte trochu pivního kvásku a másla,
        nechte ho chvíli stát za kamny, aby vykynulo, pak ho znovu
        zpracujte a trochu promíchejte, poté ho pěkně rozválejte,
        posypte malými černými rozinkami. Vezměte váleček, který je
        teplý a potřený máslem, položte ho na těsto, těsto na něj
        přeložte a svažte ho dohromady nití, aby nespadlo, dejte ho k
        ohni a pomalu otáčejte, aby se pěkně opeklo. A když zhnědne,
        vezměte štětec, namočte ho do horkého másla a potřete jím dort,
        aby byl pěkně nahnědlý. A když je upečený, sundejte ho z
        válečku-rožně a oba otvory ucpěte čistými hadříky, aby v něm
        zůstalo teplo, nechte ho tak, dokud nevychladne, podávejte ho
        studený na stůl, bude pěkně křehký a dobrý.
      \end{itemize}
    \end{itemize}
  \end{itemize}
\end{itemize}

\begin{itemize}
\tightlist
\item
  1965
  \href{https://ndk.cz/uuid/uuid:e1a51a10-09d1-11ea-a20e-005056827e51}{Horácká
  chasa}:

  \begin{itemize}
  \tightlist
  \item
    vysočina v okolí Třebíčska, Znojemska atd.
  \item
    na tučný čtvrtek se začaly připravovat tučný jídla, koblihy,
    křupance, trdelníky
  \end{itemize}
\item
  1966
  \href{https://dikda.snk.sk/uuid/uuid:656263cf-a9ec-466d-81b8-5a1ffef20756}{Výživa
  a zdravie}

  \begin{itemize}
  \tightlist
  \item
    Slovensko, propagace grilování, zmínka, že trdelník pekly babičky v
    otevřený straně komína a že to bylo zpestření pro pospolitej lid.
    Trdelník byl pomazanej bílkem a posypanej sekanýma ořechama.
  \end{itemize}
\item
  1966
  \href{https://ndk.cz/uuid/uuid:6fc6e800-b623-11dd-aa80-000d606f5dc6}{rudé
  právo}:

  \begin{itemize}
  \tightlist
  \item
    článek o Skalici, trdelníky z máslového těsta
  \end{itemize}
\item
  1968
  \href{https://ndk.cz/view/uuid:86449e60-654c-11e2-9d9f-005056827e52?page=uuid\%3A7c5078bd258a6084e96ff7134ee24101}{Československá
  vlastivěda}, díl 3:

  \begin{itemize}
  \tightlist
  \item
    s. 579: odstavec o koláčích na Slovensku, trdelník je na západním
    Slovensku, pochazí z Moravy
  \end{itemize}
\item
  1968
  \href{https://ceskadigitalniknihovna.cz/uuid/uuid:9e8373c6-0e68-11e4-a3ad-0050569d679d}{Lidová
  strava na Kloboucku a Ždánicku}, Ludvíková, Miroslava:

  \begin{itemize}
  \tightlist
  \item
    s. 17:

    \begin{itemize}
    \tightlist
    \item
      starší typ dělanej na ohni, sypanej ořechama a maštěnej,
    \item
      mladší pečenej na turčisku z buchtovýho těsta, nebyly tak
      slavnostní
    \item
      v rodině Mrštíkových se dělaly z kynutého těsta s rozinkama,
      sypaný ořechy
    \item
      tatínkovi připomínaly dětství v Dambořicích
    \item
      z poslední doby jsou známé trdelníky smažené, z nového typu těsta
    \end{itemize}
  \item
    s. 34: trdelníky do kouta
  \item
    s. 35: svatební hostina, jedly se koláče, bábovky, koblihy, boží
    milosti, trdelníky, pégny
  \item
    s. 44: slovníček:

    \begin{itemize}
    \tightlist
    \item
      1. obřadní pečivo, pečené na dřevěném válci nad ohništěm
    \item
      2. později trubičky z buchtového těsta na dřevěných válečcích
    \item
      3. máslový trubičky smažený
    \end{itemize}
  \end{itemize}
\item
  1969
  \href{https://ceskadigitalniknihovna.cz/uuid/uuid:be0ae270-94a1-11e8-9690-005056827e51}{Nesmrtelný
  poutník}:

  \begin{itemize}
  \tightlist
  \item
    Díl 1. Mladá léta Jana Ámose
  \item
    Mašínová, Leontina
  \item
    popisuje mládí J. A. Komenského, rok 1592
  \item
    když se narodil, přinášely sousedky vařený kohouti a trdláče sypané
    mákem
  \item
    ale je to beletrie, o tom dětství neměla autorka moc zpráv
  \end{itemize}
\item
  1970
  \href{https://ceskadigitalniknihovna.cz/uuid/uuid:874ae8f0-f920-11e3-b4ad-005056827e51}{Malovaný
  kraj}:

  \begin{itemize}
  \tightlist
  \item
    fašank na Kopčanoch (Slovensko, jižně od Skalice, levej břeh
    Moravy), kluci chodí koledovat a dostávaj trdelníky
  \end{itemize}
\item
  1970
  \href{https://ceskadigitalniknihovna.cz/view/uuid:733fd410-3b55-11e5-8b04-5ef3fc9bb22f?page=uuid\%3A9992ddf0-3b5b-11e5-a525-5ef3fc9ae867&fulltext=trdeln\%C3\%ADk\%20OR\%20trdeln\%C3\%ADky\%20OR\%20trdeln\%C3\%ADk\%C5\%AF&source=mzk}{Národopisné
  aktuality}:

  \begin{itemize}
  \tightlist
  \item
    vzpomínky na Slovácký den ve Strážnici 1926, jen poznámka, že se
    jedl trdleník
  \end{itemize}
\item
  1971
  \href{https://dikda.snk.sk/uuid/uuid:286ab1e8-9e6d-4d9e-acc5-c7a4ece3b1d4}{Pečieme
  na sviatky}:

  \begin{itemize}
  \tightlist
  \item
    Slovensko, recept na trdelníčky v troubě
  \end{itemize}
\item
  1971
  \href{https://dikda.snk.sk/uuid/uuid:7daf6367-3c66-45b1-95be-b9216087f65a}{Zora
  východu}:

  \begin{itemize}
  \tightlist
  \item
    Slovensko, další recept na skalickej trdelník v troubě
  \end{itemize}
\item
  1972
  \href{https://ceskadigitalniknihovna.cz/view/uuid:7f125360-7a61-11e2-b212-005056827e52?page=uuid:c55826ef-5b0a-c263-256a-4b1ed7dcf198&fulltext=trdeln\%C3\%ADk*\%20OR\%20trdelnjk*&source=mzk}{Svatební
  cesta do Jiljí}:

  \begin{itemize}
  \tightlist
  \item
    beletrie, tetina specialita valašský trdelníky
  \end{itemize}
\item
  1972
  \href{https://ceskadigitalniknihovna.cz/view/uuid:8cf964a0-baf0-11e5-8c9e-001018b5eb5c?page=uuid\%3A13cfe450-bafa-11e5-82dc-5ef3fc9bb22f&fulltext=trdeln\%C3\%ADk\%20OR\%20trdeln\%C3\%ADky\%20OR\%20trdeln\%C3\%ADk\%C5\%AF&source=nkp}{Naše
  rodina}:

  \begin{itemize}
  \tightlist
  \item
    Svatodušní "kosmatice" a trdelníky
  \item
    Marie Úlehlová-Tilschová, je tam recept, ale jinak opakuje co psala
    jinde
  \end{itemize}
\item
  1972
  \href{https://dikda.snk.sk/uuid/uuid:0c6787b6-5b99-4d97-9642-caa75ca6d9be}{Trnavský
  hlas}

  \begin{itemize}
  \tightlist
  \item
    Slovensko, v Krakovanech (kousek od Piešťan) byla delegace z NDR,
    pohostili je trdelníkem a dali jim recept
  \end{itemize}
\item
  1972
  \href{https://ceskadigitalniknihovna.cz/view/uuid:aaca4510-b6b2-11ed-9ffb-5ef3fc9bb22f?page=uuid\%3Aea81fc20-c1fc-11ed-979d-5ef3fcdaa9a7&fulltext=trdeln\%C3\%ADk\%20OR\%20trdeln\%C3\%ADky\%20OR\%20trdeln\%C3\%ADk\%C5\%AF&source=mzk}{Rovnost:
  list sociálních demokratů českých}:

  \begin{itemize}
  \tightlist
  \item
    článek Zajímavosti z kuchyně před půl stoletím
  \item
    recept na moravský smažený trdelníky
  \end{itemize}
\item
  1972
  \href{https://ceskadigitalniknihovna.cz/uuid/uuid:f7dfd885-ee4d-4815-9bb8-26f566d50485}{Muzejní
  a vlastivědná práce}:

  \begin{itemize}
  \tightlist
  \item
    přehled druhů a funkce lidového pečiva,
  \item
    spousta druhů pečiva spojená s narozením dítěte, někde koblihy,
    šišky, boží milosti, trdelníky, preclíky
  \end{itemize}
\item
  1973
  \href{https://ceskadigitalniknihovna.cz/view/uuid:16e70800-3350-11e8-9dd8-005056827e51?page=uuid\%3Ab94f2080-53a5-11e8-946a-005056825209&fulltext=trdeln\%C3\%ADk\%20OR\%20trdeln\%C3\%ADky\%20OR\%20trdeln\%C3\%ADk\%C5\%AF&source=mzk}{Slovácká
  obec Rohatec}:

  \begin{itemize}
  \tightlist
  \item
    svatba, jen vzácných byly zázvorníky a trdelníky
  \end{itemize}
\item
  1974\href{https://ceskadigitalniknihovna.cz/view/uuid:01efbca0-9ed8-11e3-8e84-005056827e51?page=uuid\%3A605e51f0-a73b-11e3-87a3-001018b5eb5c&fulltext=trdeln\%C3\%ADk\%20OR\%20trdeln\%C3\%ADky\%20OR\%20trdeln\%C3\%ADk\%C5\%AF&source=mzk}{Vedení
  domácnosti a výchova v rodině}:

  \begin{itemize}
  \tightlist
  \item
    s. 242: recept na skalický trdelníky

    \begin{itemize}
    \tightlist
    \item
      na západním Slovensku se cukrují, na Moravě se plní jablkovou
      pěnou
    \item
      poměr: 60 Dg hrubé mouky, 1 vejce, 1 žloutek, 5 Dg másla, 4 dl
      kysaný smetany, 3 Dg droždí, sůl, citronová kůra
    \end{itemize}
  \item
    s. 243, 244: fotky přípravy trdelník
  \end{itemize}
\item
  1974
  \href{https://dikda.snk.sk/uuid/uuid:d94332c2-6255-4120-ae75-24a5112dd9b9}{Senica}:

  \begin{itemize}
  \tightlist
  \item
    Slovensko, slovníček nářečí kolem Senice
  \end{itemize}
\item
  1974
  \href{https://ceskadigitalniknihovna.cz/uuid/uuid:7c79b650-3b72-11e5-b57a-005056825209}{Národopisné
  aktuality}:

  \begin{itemize}
  \tightlist
  \item
    LUDVÍKOVÁ, Miroslava, Obřadní pokrmy na Znojemsku, Národopisné
    aktuality, roč. 11, 1974, s. 185--198.
  \item
    s. 188 Bohutice, hostina na popeleční středu
  \item
    s. 191 obrázek stojanů na pečení trdelníků
  \item
    s. 194:

    \begin{itemize}
    \tightlist
    \item
      rozšíření: Slovácko až Třebíčsko, i v Rakousku
    \item
      pekly se nad uhlíky z březovýho dřeva, pozdějc jen v troubě
    \item
      v současnosti nejčastějc smažený na plechový formě
    \item
      vždycky neplněné
    \end{itemize}
  \end{itemize}
\item
  1974
  \href{https://ndk.cz/uuid/uuid:630185b0-f30f-11e3-a012-005056825209}{Malovaný
  kraj}:

  \begin{itemize}
  \tightlist
  \item
    recept z Vlčnova, taky zmínka, že se někdy trdelník říká trubičkám
    plněným šlehačkou
  \item
    v Hradčovicích se omotával krajkou, bylo to asi hlavně na parádu
  \end{itemize}
\item
  1974
  \href{https://ceskadigitalniknihovna.cz/uuid/uuid:9a981180-f289-11ea-9c2e-005056827e51}{Český
  slovník věcný a synonymický}:

  \begin{itemize}
  \tightlist
  \item
    s. 528 trdlovec = věžovité pečivo z těsta litého na otáčející se
    rožeň
  \item
    trdelník, trdelníček = z jemného mastného těsta, pečený na trdle
    nebo na kolíčkách na způsob závitků
  \end{itemize}
\item
  1975
  \href{https://ceskadigitalniknihovna.cz/view/uuid:b23f9410-20b3-11ee-a8bd-005056827e51?page=uuid\%3A00d297d0-20cd-11ee-9137-005056822549&fulltext=trdeln\%C3\%ADk\%20OR\%20trdeln\%C3\%ADky\%20OR\%20trdeln\%C3\%ADk\%C5\%AF&source=mzk}{Rovnost}:

  \begin{itemize}
  \tightlist
  \item
    děj se odehrává v roce 1900, autorka Růžena Bergerová
  \item
    tetička Hlavsová z Ivančic prodává sněhem pleněný trdelníky
  \item
    malá Fanynka se těší na hody, který se v Zakřanech konají třetí
    říjnovou neděli, ale místo toho musí jít sbírat uhlí na haldu
  \end{itemize}
\item
  1976
  \href{https://ceskadigitalniknihovna.cz/view/uuid:ed1e8590-5937-11e2-b816-001018b5eb5c?page=uuid:42702f2d-2ffb-ed27-fb9d-4071dce2a1d8&fulltext=trdeln\%C3\%ADk\%20OR\%20trdeln\%C3\%ADky\%20OR\%20trdeln\%C3\%ADku\%20&source=mzk}{Ivančický
  zpravodaj}:

  \begin{itemize}
  \tightlist
  \item
    číslo z února, s. 6, článek A vypilo se devět věder
  \item
    popis svatby z roku 1848 v Ivančicích, pekly se trdelníky
  \end{itemize}
\item
  1976
  \href{https://ceskadigitalniknihovna.cz/view/uuid:c28d9bc0-d57e-11ee-8ea1-5ef3fc9bb22f?page=uuid\%3A97237c77-34d2-4422-98b3-6d767baeb2bd&fulltext=trdeln\%C3\%ADk\%20OR\%20trdeln\%C3\%ADky\%20OR\%20trdeln\%C3\%ADk\%C5\%AF&source=mzk}{Rovnost}:

  \begin{itemize}
  \tightlist
  \item
    o brněnských hodech, chtěli by tam nabízet smažený trdelníky
  \end{itemize}
\item
  1976
  \href{https://ceskadigitalniknihovna.cz/uuid/uuid:16272350-bc77-11e2-b6da-005056827e52}{Po
  zarostlém chodníčku: Sblížení s Leošem Janáčkem}:

  \begin{itemize}
  \tightlist
  \item
    v roce 1906 vybudoval po vzorů ze skalických vinohradů búdu, jeho
    sestry tam pekly trdelníky a nalévaly skalický rubín
  \end{itemize}
\item
  1977
  \href{https://ndk.cz/uuid/uuid:84ffac90-ec3f-11e3-97c9-001018b5eb5c}{Malovaný
  kraj}:

  \begin{itemize}
  \tightlist
  \item
    dřív se pekly v celym pomoraví, tenhle článek je o Skalici
  \item
    obrázek zařízení k pečení trdelníku a recept
  \end{itemize}
\item
  1977
  \href{https://ceskadigitalniknihovna.cz/uuid/uuid:287f33fd-b113-44e3-b64c-9c82f3afa022}{Rovnost}:

  \begin{itemize}
  \tightlist
  \item
    článek Je to jako pohádka
  \item
    autorka vzpomíná, jak jezdila k babičce a voněly u ní jablkovým
    sněhem trdelníky
  \end{itemize}
\item
  1979
  \href{https://ceskadigitalniknihovna.cz/view/uuid:be7361f0-4ec5-11ed-8291-5ef3fc9bb22f?page=uuid\%3A7ae36698-432f-4b36-b735-c85727b4d482&source=nkp}{Miscellanea
  Brunensia}:

  \begin{itemize}
  \tightlist
  \item
    slovník nářečí Žarošic, trdélňik = jakési motance na dřevěných
    tyčkách
  \end{itemize}
\item
  1979
  \href{https://ceskadigitalniknihovna.cz/uuid/uuid:15aadd4e-cf35-4c33-afac-de13c4594761}{Rovnost:
  list sociálních demokratů českých}:

  \begin{itemize}
  \tightlist
  \item
    článek Vlasáková, Olga. Vařte s námi: masopustní pečivo
  \item
    pěknej popis trdelníků
  \item
    recept na Skalickej, smaženej z Brna a pečený Vlčňovský
  \end{itemize}
\item
  1980
  \href{https://ceskadigitalniknihovna.cz/uuid/uuid:c9416770-a264-41ee-9b5f-6d74a6a748aa}{Muzeum
  Hodonínska}:

  \begin{itemize}
  \tightlist
  \item
    průvodce expozicemi, mají vystavený trdlo
  \end{itemize}
\item
  1980
  \href{https://ceskadigitalniknihovna.cz/uuid/uuid:79871740-f2ac-11e3-a2c6-005056827e51}{Malovaný
  kraj}:

  \begin{itemize}
  \tightlist
  \item
    národopisná výstava 1895 a Slovácko. lokálně se organizovaly sbírky
    předmětů na výstavu, poptávali trdla na trdelníky
  \item
    další článek "péče o dítě", je tam pečivo do kúta a trdláče
  \end{itemize}
\item
  1980
  \href{https://ceskadigitalniknihovna.cz/uuid/uuid:48f04200-3b6c-11e5-8b04-5ef3fc9bb22f}{Národopisné
  aktuality}:

  \begin{itemize}
  \tightlist
  \item
    křtiny v Tuřanech u Brna, z vyprávění pamětnice v roce 1950
  \end{itemize}
\item
  1981
  \href{https://ndk.cz/uuid/uuid:6927e180-d0b8-11e6-8032-005056827e52}{Agricultura
  carpatica}:

  \begin{itemize}
  \tightlist
  \item
    řeší básníka Josefa Heřmana Gallaše, kterej psal i o lidovým jídle
    začátkem 19. století
  \item
    s. 129 - trdelníky v panském prostředí, rok 1805
  \item
    s. 130 - báseň, popisuje křtiny a v seznamu sladkostí trdelníky.
    možná městský prostředí
  \end{itemize}
\item
  1981
  \href{https://ceskadigitalniknihovna.cz/uuid/uuid:b1f3ed0d-b70a-4847-a4fe-baa804e675fc}{Rovnost}:

  \begin{itemize}
  \tightlist
  \item
    v brněnský části Komín Olga Vlasáková připomíná tradiční věci, jako
    třeba trdelníky
  \end{itemize}
\item
  1982
  \href{https://ceskadigitalniknihovna.cz/uuid/uuid:605643a0-4d40-11e5-a525-5ef3fc9ae867}{Národopisné
  aktuality}:

  \begin{itemize}
  \tightlist
  \item
    Lidová strava ve Strážnici
  \item
    starodávné trdelníky jsou téměř zapomenuty
  \item
    dřív v pecích, pozdějc i ve sporákových troubách
  \end{itemize}
\item
  1982
  \href{https://ceskadigitalniknihovna.cz/uuid/uuid:8e4f05a0-1a98-11f0-8eb6-5ef3fc9bb22f}{Rovnost}:

  \begin{itemize}
  \tightlist
  \item
    článek Horácký Masopust, Olga Vlasáková
  \item
    smažily se a pekly šišky, křupance, křehotiny, koblihy, někde taky
    trdelníky
  \end{itemize}
\item
  1982
  \href{https://ceskadigitalniknihovna.cz/uuid/uuid:55d93f60-3319-11e8-8cf8-005056827e52}{Lidové
  zvyky a slavnosti na Podluží}:

  \begin{itemize}
  \tightlist
  \item
    Kružík, Jan
  \item
    s. 29: popis co se nosilo novorodičce do kúta, smažené koblihy,
    šišky, trdelníky, vepřová pečeně
  \item
    s. 65: strava o žních, na "domlácenou" pekl hospodář na volném
    prostranství nad uhlíky velké (nenaplněné) trdelníky, na trdle nad
    uhlíky, peklo se do zlatova
  \end{itemize}
\item
  1983
  \href{https://ceskadigitalniknihovna.cz/uuid/uuid:630563b0-2c45-11e3-b62e-005056825209}{Moravské
  národní pohádky a pověsti}:

  \begin{itemize}
  \tightlist
  \item
    pohádka chudý švec
  \item
    přespává v hospodě, maj tam trdelníky
  \end{itemize}
\item
  1983
  \href{https://ndk.cz/uuid/uuid:2dc5f8a0-fd98-11ef-b1df-5ef3fc9bb22f}{Rovnost}:

  \begin{itemize}
  \tightlist
  \item
    Listopad a lidová tradice, Olga Vlasáková
  \item
    A. Kachlík z Brna-Komína vzpomíná (1907), že po skončení draní peří
    se slavil důděrek, kdy se jedli trdelníky a koblihy a zapíjelo se to
    čajem
  \end{itemize}
\item
  1983
  \href{https://ndk.cz/uuid/uuid:afab0700-f75c-11e3-8232-5ef3fc9ae867}{Malovaný
  kraj}:

  \begin{itemize}
  \tightlist
  \item
    článek Lidové pečivo 1,
  \item
    zmínka, že trdelníky víceméně zanikly proto, že se dělaly na ohni
  \item
    v některejch obcích se ale trdelníky začalo řikat máslovejm
    trubičkám
  \end{itemize}
\item
  1983
  \href{https://ceskadigitalniknihovna.cz/view/uuid:a3661330-e99a-11e6-9964-005056825209?page=uuid\%3A83054300-e9a3-11e6-8010-005056827e51&fulltext=trdeln\%C3\%ADk\%20OR\%20trdeln\%C3\%ADky\%20OR\%20trdeln\%C3\%ADk\%C5\%AF&source=mzk}{Vlasta}:

  \begin{itemize}
  \tightlist
  \item
    představení 80letý malířky Ludmily Procházkový, popisujou obrazy,
    třeba jak se pečou trdelníky
  \end{itemize}
\item
  1983
  \href{https://dikda.snk.sk/uuid/uuid:b775b7f6-6bff-41ea-81f8-291194dc217c}{Trnavský
  hlas}

  \begin{itemize}
  \tightlist
  \item
    Slovensko, Krakovany, mládežnické klubové večery, podávají se místní
    speciality - langoše, lokše, trdelníky
  \end{itemize}
\item
  1984
  \href{https://dikda.snk.sk/uuid/uuid:f4d70313-b883-466c-972e-3cfd5e07a41e}{Záhorák}:

  \begin{itemize}
  \tightlist
  \item
    Slovensko
  \item
    článek o pekárně, kde trdelník vyrábí. Je součástí Západoslovenské
    pekárne, n. p., závod Piešťany, má 12 pracovníků
  \item
    těsto je kynuté, pečený v pecích s infrazářiči za 18-20 minut
  \item
    kapacita 600 kusů, plánujou zvýšit výrobu na 800
  \item
    i přes průmyslovou výrobu kvalita neutrpěla, na výstavě Agrokomplex
    84 získali cenu Zlatý kosák
  \end{itemize}
\item
  1984
  \href{https://dikda.snk.sk/uuid/uuid:b873200a-106a-429f-b02a-29e45a750c35}{Výživa
  a zdravie}

  \begin{itemize}
  \tightlist
  \item
    Slovensko, Západoslovenské pekárne, n. p. Piešťany získaly za
    trdelník VHJ (netuším co to je) na výstavě Agrokomplex 84 za mlýnský
    a pekárenský průmysl
  \end{itemize}
\item
  1984
  \href{https://dikda.snk.sk/uuid/uuid:a8b9dc68-bf2f-4a1c-93e9-81f3cff1ab98}{Večerník}

  \begin{itemize}
  \tightlist
  \item
    Slovensko, trdelník z n.p. Mlýny a pekárne zpestří nabídku
    bratislavských prodejen
  \end{itemize}
\item
  1984
  \href{https://dikda.snk.sk/uuid/uuid:de963c61-5ac8-4073-8c7c-12d5370aa4f0}{Piešťany}

  \begin{itemize}
  \tightlist
  \item
    Západoslovenské pekárne, n. p., závod Piešťany otřevřely průzkumnou
    prodejnu. Největší zájem je o trdelník, kterýho by i přes vysokou
    cenu prodali víc, než můžou nabídnout.
  \end{itemize}
\item
  1984
  \href{https://ceskadigitalniknihovna.cz/uuid/uuid:065a3d70-f844-11e5-92c7-5ef3fc9ae867}{Socializace
  vesnice a proměny lidové kultury}:

  \begin{itemize}
  \tightlist
  \item
    článek o přežívání tradičního nádobí a náčiní při domácích prací v
    obci u UH. Hradiště
  \item
    do roku 1915 se běžně vařilo na otevřeným ohni ve většině zkoumaný
    domácnostech ve Starým Městě, Mařatice, Kunovice
  \item
    pak se rozšířily kamna s plotnou
  \item
    obrázek babičky, jak peče trdelník ve Vlčnově v roce 1948. vypadá to
    jako plát těsta.
  \end{itemize}
\item
  1984
  \href{https://ndk.cz/uuid/uuid:831adb20-0cb3-11f0-9e41-5ef3fc9bb22f}{Rovnost}:

  \begin{itemize}
  \tightlist
  \item
    Masopuste, masopuste... - Olga Vlasáková
  \item
    popisuje různý smažený masopustní pečivo z Moravy, jen zmiňuje, že
    méně známé jsou vrkoče a trdelníky různých velikostí a úprav
  \end{itemize}
\item
  1984
  \href{https://ceskadigitalniknihovna.cz/uuid/uuid:0571ff20-0cbb-11f0-9e41-5ef3fc9bb22f}{Rovnost}:

  \begin{itemize}
  \tightlist
  \item
    O skalickém trdlování, Lenka Vlasáková,
  \item
    vyjímečností národopisné tradice masopustního období z počátku
    minulého století je tvorba lidového pečiva, trdelníků velkých
    rozměrů, dodnes rožněných na otevřeném ohništi
  \item
    článek o Skalici, ale jsou tam i zmínky o Moravě:

    \begin{itemize}
    \tightlist
    \item
      trdelník byl slavnostní pečivo v Ořechově u Brna, Velkém Meziříčí,
      Oslavansku, Hustopečsku i místně z Hané
    \item
      dodnes se smaží v Brně-Komíně
    \item
      v Napajedlích byl součástí jarních obřadů trdelník navíjenej na
      vrbovym proutí
    \item
      na proutku smažený voničky v Mostišti na Horácku
    \item
      Sobotecký ratolesti z Čech mají podobnej základ
    \end{itemize}
  \item
    ve Skalici dělá trdelník posledních pět žen
  \item
    popisuje, jak 64 letá tetička Mazůrová připravuje trdelníky:

    \begin{itemize}
    \tightlist
    \item
      dřevo jasanové, bukové, nebo javorové
    \item
      většinou peče v zimě, když není práce na vinohradě
    \item
      popis pečení
    \item
      sypou se cukrem a vanilkou
    \item
      formu na trdelník vyrání místní stolař a kování mechanik z JRD
    \end{itemize}
  \end{itemize}
\item
  1984
  \href{https://ceskadigitalniknihovna.cz/uuid/uuid:606979c0-efae-11e3-adbd-5ef3fc9bb22f}{Malovaný
  kraj}:

  \begin{itemize}
  \tightlist
  \item
    Slovenská búda v Luhačovicích, vybudovaná 1906 Jurkovičem, nabízeli
    trdelníky jako Skalický speciality
  \end{itemize}
\item
  1985
  \href{https://ceskadigitalniknihovna.cz/uuid/uuid:a10e2690-0153-11e4-99ee-001018b5eb5c}{Kuchařská
  technologie}:

  \begin{itemize}
  \tightlist
  \item
    recept na svatební trdelník, je to klasickej kynutej trdelník,
    posypanej mandlema, potíranej bílkem
  \end{itemize}
\item
  1986
  \href{https://ceskadigitalniknihovna.cz/uuid/uuid:6c7effb0-14fa-11e5-b9a6-5ef3fc9ae867}{Amatérský
  film}:

  \begin{itemize}
  \tightlist
  \item
    scénář a popis dokumentárního filmu, kde stará rodina ze Skalice
    připravuje trdelník
  \end{itemize}
\item
  1986
  \href{https://ceskadigitalniknihovna.cz/uuid/uuid:efea92f0-525a-11e5-a788-0050569d679d}{Žarošice}:

  \begin{itemize}
  \tightlist
  \item
    s. 170 popisujou slavnostní jídla, trdelníky se připravovaly nad
    ohněm, v domácnostech ještě meli černý kuchyně
  \end{itemize}
\item
  1986
  \href{https://ndk.cz/uuid/uuid:c91c2e60-1d1a-11e4-a8ab-001018b5eb5c}{Malovaný
  kraj}:

  \begin{itemize}
  \tightlist
  \item
    článek: Vlasáková, Olga. O masopustě a masopustním pečivu.
  \item
    o masopustě a masopustním pečivu
  \item
    zmínky o různejch vesnicích, kde se historicky dělal (u Brna,
    Vyškova, Velký Meziříčí)
  \item
    na většině míst se dělají na menších formách, jen ve Skalici na
    původních trdlech na ohništi
  \item
    je tam několik jmen žen, co ve Skalici trdelníky dělají. každá má
    svůj recept i techniku
  \item
    popis přípravy u jedný z nich, už používá motorek na otáčení
  \end{itemize}
\item
  1986
  \href{https://ndk.cz/uuid/uuid:882d50d0-42f4-11f0-9463-005056825209}{Rovnost}:

  \begin{itemize}
  \tightlist
  \item
    Plní poslání, ovlivňuje výrobu
  \item
    článek o pekárně ve Žďáru nad Sázavou
  \item
    chtějí vyrábět žďárník, žďárské trdekníky
  \end{itemize}
\item
  1986
  \href{https://dikda.snk.sk/uuid/uuid:bcfbf53a-ce88-41a9-9e57-676c018f79a7}{Slovenské
  pohľady}

  \begin{itemize}
  \tightlist
  \item
    Slovensko, jen zmínka o existenci vynikajícího skalickýho trdelníku
  \end{itemize}
\item
  1986
  \href{https://ceskadigitalniknihovna.cz/uuid/uuid:9e7ab0f0-3a6b-11e4-8e0d-005056827e51}{Vánoční
  pečivo}:

  \begin{itemize}
  \tightlist
  \item
    Davidová, Olga
  \item
    s. 23: recept na trdelníčky, těsto s rumen, sypaný perníkem, fíkama,
    čokoládou
  \end{itemize}
\item
  1987
  \href{https://ceskadigitalniknihovna.cz/uuid/uuid:fb80fd90-a376-11e3-aa54-5ef3fc9bb22f}{Květy}:

  \begin{itemize}
  \tightlist
  \item
    fejeton Je pan choť pruďas?
  \item
    o tom, proč je pestrej jídelníček nezbytnej
  \item
    když se chlap přejí vepřovýho, vezme za vděk i škubánky nebo
    trdlovcem
  \end{itemize}
\item
  1988
  \href{https://dikda.snk.sk/uuid/uuid:f1160e6b-b724-4a15-ba1e-ad92b54d2423}{Večerník}:

  \begin{itemize}
  \tightlist
  \item
    Slovensko
  \item
    městské muzeum v Bratislavě připravilo na nádvoří Staré radnice akci
    Tradicia Vánoc v Bratislavě
  \item
    autor kritizuje, že pečivo až na praclíky, který byly kruhový a ne
    typický, bylo obyčejný a dá se snadno koupit v obchodě a na každý
    pouti - s. trdelník, tyčinky, perníky
  \end{itemize}
\item
  1988
  \href{https://ceskadigitalniknihovna.cz/uuid/uuid:d410e1e0-4611-11e1-1331-001143e3f55c}{Český
  lid}:

  \begin{itemize}
  \tightlist
  \item
    článek: Tendence vývoje sváteční stravy v Brně od počátku 20.
    století, Helena Bočková
  \item
    masopustní pečivo: trdelníky a boží milosti během 20. a hlavně 30.
    let zanikly, koblihy se smažily ve většině rodin
  \end{itemize}
\item
  1988
  \href{https://ceskadigitalniknihovna.cz/uuid/uuid:0a416360-305d-11e9-844c-005056827e51}{Lidové
  pečivo v Čechách a na Moravě}:

  \begin{itemize}
  \tightlist
  \item
    s. 15:

    \begin{itemize}
    \tightlist
    \item
      vaječný těsto, užší teritoriální rozšíření, Brněnsko, jižní Morava
    \item
      trdelníky měly několik vývojových fází, v závislosti na typu
      topeniště:

      \begin{itemize}
      \tightlist
      \item
        na ohni - pramínky těsta pečený na trdle, to se dalo na cihly
        nebo stojan nad oheň a otáčelo. někdy se trdelník pomakoval,
        později posypal nebo obalil cukrem
      \item
        smažený s cukrem a skořicí
      \item
        v troubě
      \end{itemize}
    \end{itemize}
  \item
    s. 16:

    \begin{itemize}
    \tightlist
    \item
      popis pečení na ohništi v Boršicích u Blatnice
    \item
      odkazy na literaturu
    \end{itemize}
  \item
    s. 17:

    \begin{itemize}
    \tightlist
    \item
      koláče, buchty, koblihy, boží milosti, trdelníky, bábovky a jiné
      druhy pečiva na křtinách
    \end{itemize}
  \item
    s. 20: pečivo na svatbách, máme zprávy až od 19. století, hlavně
    poslední třetinu
  \item
    s. 56: někde byly trdelníky i vánoční pečivo
  \item
    s. 66:

    \begin{itemize}
    \tightlist
    \item
      nevíme, jak stará je tradice trdelníků, nejstarší recept je od
      Sibilly Dorizio, 1798, s 15 (Moje skrz čtyřicetileté vykonávání
      známá Kuchařská kniha pro velké i menší tabule) - ve skutečnosti
      asi vyšla pozdějc, kolem 1801 -
      \href{https://books.google.cz/books?vid=NKP:1002400169&printsec=frontcover\#v=onepage&q&f=false}{digitalizovaná
      verze}
    \item
      \href{https://books.google.cz/books?vid=NKP:1002400169&printsec=frontcover\#v=onepage&q=trdel&f=false}{recept
      na trdelník}
    \end{itemize}
  \item
    s. 78-79: třísky - pečivo z litýho těsta, který konzistencí
    připomíná kruhanice a provedením trdelníky
  \item
    s. 84:

    \begin{itemize}
    \tightlist
    \item
      popis rozšíření - Slovácko, Znojemsko, Třebíčsko, Brněnsko,
      Drahanská vrchovina, Haná
    \item
      původně z nekvašenýho přesnýho těsta (přesný těsto znamená, že se
      všecky ingredience musej přesně odměřovat)
    \item
      trdlo mělo průměr 5 až 10 cm, dýlka 30 až 50, vyjímečně 70 cm
      (Švirga - Břeclav-Celnice)
    \item
      dřevo březový nebo bukový, nad živým ohněm
    \item
      informace získali z průzkumů České národopisné společnosti, která
      měla
      \href{https://www.narodopisnaspolecnost.cz/index.php/virtualni-badatelna/itemlist/category/31-dotaznikydotazníky\%20na\%20různá\%20témata}{dotazníky
      na různá témata}:

      \begin{itemize}
      \tightlist
      \item
        dotazník Chléb a pečivo se explicitně ptal na trdelník
      \end{itemize}
    \end{itemize}
  \item
    s. 85:

    \begin{itemize}
    \tightlist
    \item
      pekly se v černý kuchyni (Ševčík - Březová u UH), nebo na kraji
      pece (Mrázek, Ponětovice u Brna; Gorlová, Beroun!), nebo na volnym
      ohništi
    \item
      trdlo umístěný na cihlách nebo železnejch stojanech, trdlo mělo
      kliku, aby šlo točit
    \item
      pozdějc trouby, forma jen tak velká, aby se tam vešla
    \item
      formy dřevo, pozdějc plechový, někde taky kukuřičný lodyhy
      (turčisko)
    \item
      v troubě se pak peklo ve formách položenejch na plechu
    \item
      před pečením se potíraly rozšlehanym vejcem, nebo jen bílkem či
      žloutkem, sypaly mákem. Jindy se obalovaly tukem a cukrem s mákem
      nebo ořechama, nebo jen cukrem.
    \item
      po upečení se cukrovaly
    \item
      vývoj těsta: z přesnýho nekynutýho se přešlo ke kypřenýmu, pak
      kynutýmu a nakonec překládanýmu máslovýmu
    \item
      z máslovýho těsta se pak dělaly i trubičky plněný sněhem, krémem,
      nebo šlehačkou
    \end{itemize}
  \item
    s. 86 až 88:

    \begin{itemize}
    \tightlist
    \item
      recept na trdláče na živym ohni
    \end{itemize}
  \item
    s. 88:

    \begin{itemize}
    \tightlist
    \item
      ve Vlčnově je tradice pečení trdelníků v několika rodinách živá do
      současnosti,
    \item
      v Březové u UH se trdelníky udržely jen do druhé světové války
    \item
      další recepty z Uherského Hradiště, Blanska, Brna-Tuřan
    \item
      pokud se sype mákem, tak nemletým
    \end{itemize}
  \item
    s. 89:

    \begin{itemize}
    \tightlist
    \item
      taky zmínka, že menší trdelníky se v Brně dělaly z buchtového
      těsta, namotávaly na kolíky dlouhé 15cm, průměr 2cm
    \item
      další recept Bzenec a Brmovice: menší trdelníky v troubě se někde
      stále udržujou, někde se dělají taky z koláčového těsta
    \end{itemize}
  \item
    s. 89 až 90:

    \begin{itemize}
    \tightlist
    \item
      recepty na smažený trdelníky
    \item
      nejdřív se smažilo na bukovym nebo lněnym oleji, pozdějc sádlo
    \item
      zajímavý je, že se kladly na formu jako čtverečky, místo válečků
    \item
      pekly se hlavně na kovových kónických formách, smažený trdelníky
      se rozšířily, když se tyhle formy objevily
    \item
      recept Bilíkovice UH, Polešovice UH
    \end{itemize}
  \item
    s. 149:

    \begin{itemize}
    \tightlist
    \item
      další popis smaženejch trdelníků, z koblihovýho těsta, válečky,
      celý ponořený v sádle (Archlebov - Hodonín)
    \item
      jsou známý po celý jižní Moravě, na Brněnsku a Drahanské vrchovině
    \end{itemize}
  \end{itemize}
\item
  1988
  \href{https://ndk.cz/uuid/uuid:21bdf290-4ae8-11f0-ace4-005056825209}{Rovnost}:

  \begin{itemize}
  \tightlist
  \item
    Masopustní pohoštění, Olga Vlasáková
  \item
    jen pár zmínek, co už byly ve starších článcích autorky
  \item
    recepty na masopustní jídla, smažený komínský trdelníky
  \end{itemize}
\item
  1988
  \href{https://ndk.cz/uuid/uuid:21bdf290-4ae8-11f0-ace4-005056825209}{Rovnost}:

  \begin{itemize}
  \tightlist
  \item
    Strom života z Archlebova (u Ždánic), Olga Vlasáková
  \item
    Františka Poláčková, nar. 1926
  \item
    popis tradičního svatebního koláče - strom života. Poláčková ho
    vytvořila na základě povídání starších sousedů
  \item
    při koláčovém povídání zavoněl trdelník stařenky předešlé generace
    pečený na loukoti (loukoť je součást dřevěnýho kola)
  \end{itemize}
\item
  1988
  \href{https://ndk.cz/uuid/uuid:bd7a25f0-4acf-11f0-b498-5ef3fc9bb22f}{Rovnost}:

  \begin{itemize}
  \tightlist
  \item
    Otevřená kronika, Olga Vlasáková
  \item
    Rosicko-oslavanská uhelná pánev
  \item
    o Růženě Bergerové (nar. 1926), která vzpomínala na smažený
    trdelníky z dětství
  \end{itemize}
\item
  1989
  \href{https://ceskadigitalniknihovna.cz/uuid/uuid:139c9470-ad04-11e3-9d7d-005056827e51}{Ukrojte
  si u nás: kapitoly z dějin chleba}:

  \begin{itemize}
  \tightlist
  \item
    s. 187 - jen drobná zmínka, zámožnější rodiny pekly štolu, trdelník
    i trdlovec a štrůdl
  \end{itemize}
\item
  1989
  \href{https://ceskadigitalniknihovna.cz/uuid/uuid:5da1a820-f9d2-11e2-9923-005056827e52}{Dranciáš}:

  \begin{itemize}
  \tightlist
  \item
    Pavel Verner, román
  \item
    chlapík po úraze v nemocnici má sen z dětství, tatínek je asi pekař,
    je s nim v pekárně, kde mají spousty pečiva, včetně trdelníků
  \end{itemize}
\item
  1990
  \href{https://ndk.cz/uuid/uuid:78486a30-1a13-11e4-8e0d-005056827e51}{Malovaný
  kraj}:

  \begin{itemize}
  \tightlist
  \item
    s. 29 - Vlasáková, Olga. Masopust a ostatkové pečivo.
  \item
    v původní podobě nad ohněm se zachoval jen ve Skalici
  \item
    recept na smaženej trdelník z Hustopečska, dělanej na kukuřičných
    kotrlačkách (lodyhy) nebo trubičkovejch formách
  \end{itemize}
\item
  1990
  \href{https://dikda.snk.sk/uuid/uuid:d4c3bafc-0e1e-4cb1-8e37-24b0294ec64b}{Záhorák}:

  \begin{itemize}
  \tightlist
  \item
    Slovensko, článek o skalický pekárně
  \item
    recept, obsahuje vejce, máslo, mlíko, citropastu, rum, vanilku
  \item
    týdně vyroběj 4000 kusů
  \end{itemize}
\item
  1990
  \href{https://dikda.snk.sk/uuid/uuid:51e3241b-d1cc-4430-a318-822c58fc4dae}{Slovenská
  kuchárka}

  \begin{itemize}
  \tightlist
  \item
    Slovensko, recept na skalický trdelník,
  \item
    s. 288 obrázek nakrájenýho trdelníku
  \end{itemize}
\item
  1991
  \href{https://dikda.snk.sk/uuid/uuid:7e3fed6a-4442-4a06-9976-aac07df998ce}{Výživa
  a zdravie}

  \begin{itemize}
  \tightlist
  \item
    Slovensko, obrázek trdelníku nakrájenýho na kružky, popisek Skalický
    trdelník patří mezi krajové speciality slovenský kuchyně
  \end{itemize}
\item
  1991
  \href{https://dikda.snk.sk/uuid/uuid:fa8b23b1-0cee-454b-898b-73643d919956}{Nap}:

  \begin{itemize}
  \tightlist
  \item
    článek v maďarštině, popis přípravy trdelníku, spousta fotek
  \end{itemize}
\item
  1991
  \href{https://ceskadigitalniknihovna.cz/uuid/uuid:c0d88b90-df55-11e6-b333-5ef3fc9ae867}{Církevní
  rok a lidové obyčeje aneb kalendárium světců a světic}:

  \begin{itemize}
  \tightlist
  \item
    recept na velkej trdelník
  \item
    těsto se vyválelo do válečků dlouhých metr a širokých 5 cm
  \item
    po navinutí na válec a omaštění se převálel v posýpce (máslo s
    cukrem, žloutek, mouka) a pak v máku s cukrem
  \item
    po opečení se posypal cukrem a vykrajovaly se prstence (krůželky)
  \end{itemize}
\item
  1991
  \href{https://ceskadigitalniknihovna.cz/uuid/uuid:b0598360-1be3-11e4-a8ab-001018b5eb5c}{Malovaný
  kraj}:

  \begin{itemize}
  \tightlist
  \item
    s. 19: Za starých časů: Jak se v Pašovicích bydlívalo. Vypráví Jakub
    Zajíc.
  \item
    popis pece s možností pečení trdelníku v Pašovicích (severně od
    Uherskýho Brodu)
  \item
    trdelník na pečení trdláčů
  \end{itemize}
\item
  1991
  \href{https://ceskadigitalniknihovna.cz/uuid/uuid:2e23e200-3d6d-11e3-9c86-005056827e51}{Technologie
  přípravy pokrmů II}:

  \begin{itemize}
  \tightlist
  \item
    Bulková, Věra
  \item
    tabulka se seznamem pokrmů připravovaných z překládaného kynutého
    těsta
  \item
    sladké pečivo: trubičky /trdelníky/ s pěnou, koláče
  \end{itemize}
\item
  1991
  \href{https://ceskadigitalniknihovna.cz/uuid/uuid:f87ff390-1ab3-11e3-9319-005056827e51}{Slovácko
  1905}:

  \begin{itemize}
  \tightlist
  \item
    sestavené na základě pamětí fotografa Karla Dvořáka (1859-1946)
  \item
    propagoval turistiku po českých zemích
  \item
    popisuje střetnutí s katechetou na reálce v Kyjově, pozval ho na
    návštěvu, podávali se slovácký trdelníky
  \end{itemize}
\item
  1992
  \href{https://ceskadigitalniknihovna.cz/uuid/uuid:3cd82b40-62fb-11e3-9ea2-5ef3fc9ae867}{Receptář
  pro každý den}:

  \begin{itemize}
  \tightlist
  \item
    od Přemka Podlahy, je tam recept, zajímavý, že od někoho z Tábora.
    kynutý těsto s máslem a sádlem, potírá se bílkama s mandlema, při
    pečení máslem. na konec vanilkovej cukr.
  \end{itemize}
\item
  1992
  \href{https://ndk.cz/uuid/uuid:2a7057e0-8b35-11e3-8031-001018b5eb5c}{Týdeník
  Květy}:

  \begin{itemize}
  \tightlist
  \item
    článek o darech pro prezidenty, Beneš dostal trdelník (byl to ten z
    Luhačovic?) - podle popisu ale spíš myslí formu, než jen pečivo
  \end{itemize}
\item
  1993
  \href{https://ceskadigitalniknihovna.cz/uuid/uuid:788f9b30-023f-11e4-97de-5ef3fc9ae867}{Malovaný
  kraj}:

  \begin{itemize}
  \tightlist
  \item
    Josef Kukulka, První krůčky malých Vlčnovjanů
  \item
    popis jídel do kúta, makovníčky, vdolečky, trdláče, boží milosti
  \end{itemize}
\item
  1993
  \href{https://ceskadigitalniknihovna.cz/uuid/uuid:c6a2f3b0-0381-11e4-97de-5ef3fc9ae867}{Malovaný
  kraj}:

  \begin{itemize}
  \tightlist
  \item
    s. 15: Vlasáková, Olga. O masopusním pečivu.
  \item
    původně na trdle, pak v troubě, dneska pouze smažený
  \item
    poslední rožněnej zaznamenala v roce 1984 u paní Mazůrové (1920) ve
    Skalici
  \item
    recepty, obrázky pečení z Ořechova u Brna, masopust
  \end{itemize}
\item
  1993
  \href{https://ceskadigitalniknihovna.cz/view/uuid:e294a2f0-8260-11e3-a6e0-005056827e52?page=uuid:32da4ff0-96d7-11e3-ad99-001018b5eb5c&fulltext=trdlovec&source=mzk}{Nová
  velká škola}:

  \begin{itemize}
  \tightlist
  \item
    recept na trdlovec v dortový formě, protože je to kuchařka na doma a
    tam nemáme válec s topným tělesem
  \end{itemize}
\item
  1994
  \href{https://ceskadigitalniknihovna.cz/uuid/uuid:7e0705a0-528b-11e6-beb0-001018b5eb5c}{Naše
  rodina}:

  \begin{itemize}
  \tightlist
  \item
    klub lidové tvořivosti na Praze 5, člnky připravily na setkání
    tradiční pečivo, mimojiný trdelníky
  \end{itemize}
\item
  1993
  \href{https://ceskadigitalniknihovna.cz/uuid/uuid:f9f63a60-98ab-11e3-8e84-005056827e51}{Město
  pod Špilberkem}:

  \begin{itemize}
  \tightlist
  \item
    na vesnicích na Brněnsku se při výročích se připravovaly jídla,
    která se jinak během roku nedělala
  \item
    na ostatke to bylo smažené masopustní pečivo trdelníky, boží milosti
    a koblihy
  \end{itemize}
\item
  1994
  \href{https://ceskadigitalniknihovna.cz/uuid/uuid:1e8fe7c0-21a7-11e4-a8ab-001018b5eb5c}{Malovaný
  kraj}:

  \begin{itemize}
  \tightlist
  \item
    s. 26, Pašovické končiny
  \item
    jak vypadaly fašanky na počátku 20. století z vyprávění babičky Anny
    Hefkové
  \item
    v neděli se na trdelníku nadělaly trdláče
  \end{itemize}
\item
  1994
  \href{https://ceskadigitalniknihovna.cz/uuid/uuid:464cbdb0-4060-11e4-8f33-5ef3fc9ae867}{Malovaný
  kraj}:

  \begin{itemize}
  \tightlist
  \item
    s. 22, Pavelčík, Jiří. Rostliny na našem stole IV. Ořechy.
  \item
    článek o ořeších, poznámka, že na podlužácku (což je kolem soutoku
    Moravy a Dyje) nejsou pravý trdelníky, ale jen trubičky (šmetrole)
  \end{itemize}
\item
  1995
  \href{https://ceskadigitalniknihovna.cz/uuid/uuid:0f104c60-3fbe-11e4-b6b9-001018b5eb5c}{Malovaný
  kraj}:

  \begin{itemize}
  \tightlist
  \item
    Beneš, Josef. Slovácko na národopisné výstavě v Praze 1895
  \item
    uherskobrodská výstava, k ochutnání byly nabízeny trdláče a boží
    milosti
  \end{itemize}
\item
  1995
  \href{https://ceskadigitalniknihovna.cz/uuid/uuid:3cd53df2-ab22-431c-b169-618cc63f3ce6}{Lidová
  tvorba}:

  \begin{itemize}
  \tightlist
  \item
    popis jak se dělá ve Skalici
  \item
    dřív byl rozšířenej po skoro celý Moravě a západním Slovensku, před
    pečením se většinou potíral žloutekm a sypal mákem nebo hrubším
    cukrem, jinde sekanými ořechy
  \item
    podával se krájený na kroužky
  \end{itemize}
\item
  1996
  \href{https://ceskadigitalniknihovna.cz/view/uuid:f7075980-dd45-11ee-bb18-0050568d319f?page=uuid\%3A38f6bbbd-dd95-47e5-b29c-ef351dec4c15&fulltext=trdeln\%C3\%ADk\%20OR\%20trdeln\%C3\%ADky\%20OR\%20trdeln\%C3\%ADk\%C5\%AF&source=mzk}{Co
  se vaří v našich krajích}:

  \begin{itemize}
  \tightlist
  \item
    recept na trdelníky z jižní Moravy
  \end{itemize}
\item
  1996
  \href{https://ceskadigitalniknihovna.cz/uuid/uuid:df7cfd60-8b4d-11e3-aa9f-5ef3fc9ae867}{Týdeník
  Květy}:

  \begin{itemize}
  \tightlist
  \item
    článek o velikonočních zvycích
  \item
    trdláče se pekly v trdelníku na živém ohni
  \item
    dodnes se prý pečou ve Vlčnově
  \end{itemize}
\item
  1996
  \href{https://ceskadigitalniknihovna.cz/view/uuid:955967f0-58cb-11e8-983f-005056827e51?page=uuid\%3A7daef930-7ef0-11e8-9588-5ef3fc9bb22f&fulltext=trdeln\%C3\%ADk\%20OR\%20trdeln\%C3\%ADky\%20OR\%20trdeln\%C3\%ADk\%C5\%AF&source=nkp}{Chuťový
  místopis}: recept podle Úlehlový-Tilschový
\item
  1997
  \href{https://ceskadigitalniknihovna.cz/view/uuid:df5be160-ef7c-11e2-9923-005056827e52?page=uuid:bba8b2b0-0f12-11e3-9439-005056825209&fulltext=trdlovec&source=nkp}{Etymologický
  slovník jazyka českého}:

  \begin{itemize}
  \tightlist
  \item
    heslo trdlo, trdelník a trdlovec druhy pečiva, těsto se ovine kolem
    trdla a pak peče
  \end{itemize}
\item
  1998
  \href{https://ceskadigitalniknihovna.cz/view/uuid:955967f0-58cb-11e8-983f-005056827e51?page=uuid\%3A7daef930-7ef0-11e8-9588-5ef3fc9bb22f&fulltext=trdeln\%C3\%ADk\%20OR\%20trdeln\%C3\%ADky\%20OR\%20trdeln\%C3\%ADk\%C5\%AF&source=nkp}{Střední
  Morava}:

  \begin{itemize}
  \tightlist
  \item
    článek ze života na hanáckém statku na přelomu 19. a 20. století
  \item
    na základě zápisků Aloise Vavroucha z let 1889 až 1891 z Dubu nad
    Moravou
  \item
    smažený trdelníky
  \end{itemize}
\item
  1998
  \href{https://www.digitalniknihovna.cz/mzk/uuid/uuid:dd805301-5c91-11eb-94b4-5ef3fc9ae867}{Zvuk}:

  \begin{itemize}
  \tightlist
  \item
    článek trdelníky od generála
  \item
    první zmínka o pověsti o sedmihradských kuchařích ve Skalici
  \item
    recept na skalický trdelník
  \end{itemize}
\item
  1998
  \href{https://ceskadigitalniknihovna.cz/view/uuid:90c1d400-282b-11ed-9c80-005056827e51?page=uuid\%3Af40f9ec2-58d4-4e4a-b664-846a40f83308&source=nkp}{Klobouky
  u Brna}:

  \begin{itemize}
  \tightlist
  \item
    popis tradičních pečiv, smažený i pečený
  \item
    trdelníky z kynutýho těsta, pruh těsta se ovine kolem až 30 cm
    dlouhýho kolíku a smaží nebo pečou
  \item
    pekly se výhradně pro šestinedělku do kúta
  \end{itemize}
\item
  1999
  \href{https://ceskadigitalniknihovna.cz/view/uuid:57abf773-997c-406c-8ba1-063f2a52d0c4?page=uuid\%3A0ef9ff97-bf5a-11ed-9fa2-001b63bd97ba&fulltext=trdeln\%C3\%ADk\%20OR\%20trdeln\%C3\%ADky\%20OR\%20trdeln\%C3\%ADk\%C5\%AF&source=kfbz}{Slovácko}:
  popis sbírek Slováckýho muzea v Uherským Hradišti

  \begin{itemize}
  \tightlist
  \item
    zvláštní nářadí pro lité trdelníky - v tomhle případě se myslí asi
    forma
  \end{itemize}
\item
  1999
  \href{https://ceskadigitalniknihovna.cz/view/uuid:288aee50-92f1-11ec-a3e4-5ef3fc9bb22f?page=uuid\%3A82f08da1-51aa-455c-8d5d-cb64c45aa594&fulltext=trdeln\%C3\%ADk\%20OR\%20trdeln\%C3\%ADky\%20OR\%20trdeln\%C3\%ADk\%C5\%AF&source=nkp}{Pohledy
  do minulosti Klobouk}:

  \begin{itemize}
  \tightlist
  \item
    Jakub Vrbas (1858 - 1952), sepsal knížku na základě starých pramenů,
    co už nejsou dostupný
  \item
    s. 57 - trdelníky na ohni, sypaný mákem a ořechy
  \item
    s. 76 - hody, smažený trdelníky, zmiňuje rok 1868
  \item
    s. 111. smažený trdélníky
  \end{itemize}
\item
  1999
  \href{https://ceskadigitalniknihovna.cz/view/uuid:1205dd30-22f9-11ef-be46-0050568d319f?page=uuid\%3A744a56c4-b1cb-4846-840b-498bb5fc53a8&fulltext=trdeln\%C3\%ADk\%20OR\%20trdeln\%C3\%ADky\%20OR\%20trdeln\%C3\%ADk\%C5\%AF&source=mzk}{Čtení
  o Strážnici}:

  \begin{itemize}
  \tightlist
  \item
    na svatby se připravovaly trdelníky nad ohněm, maštěný a sypaný
    ořechy
  \end{itemize}
\item
  1999
  \href{https://ceskadigitalniknihovna.cz/view/uuid:9ee83d54-e4ea-4999-9b2d-7fd7bc1b19a1?page=uuid\%3A57ffd5a3-50a8-11e5-8200-0050569d679d&fulltext=trdeln\%C3\%ADk\%20OR\%20trdeln\%C3\%ADky\%20OR\%20trdeln\%C3\%ADk\%C5\%AF&source=mzk}{Veselsko}:

  \begin{itemize}
  \tightlist
  \item
    trdelníky z kynutýho i nekynutýho těsta, pečený nad uhlíky.
    nejmladší informátoři už si ani nepamatovali jejich název
  \end{itemize}
\item
  1999
  \href{https://ceskadigitalniknihovna.cz/view/uuid:42ab79d0-0672-11e4-83c7-005056827e51?page=uuid\%3A60685a60-33e0-11e4-8413-5ef3fc9ae867&fulltext=trdel*&source=nkp}{Podřezávání
  větve}:

  \begin{itemize}
  \tightlist
  \item
    recenze na Medvědí román od Jiřího Kratochvíla, která se odehrává v
    70. letech na Moravě, vypravěč popisuje arcimoravský krmě, po
    kterých by se utloukl, včetně trdelníku
  \end{itemize}
\item
  1999
  \href{https://ceskadigitalniknihovna.cz/uuid/uuid:688f803c-f890-4e66-986a-47d85c06d611}{Krkonoše}:

  \begin{itemize}
  \tightlist
  \item
    časopis Správy Krkonošského národního parku, 1999, č. 9, s. 35
  \item
    Co se vařívalo na poutích a posvíceních
  \item
    na posvícení došlo na husy, kachny, kuřata, krocany, ale i na pašíka
    leckde, obarovici, trdelníky, dandule, presbuřty, pekly koláče
    skládance, roháče a hnětýnky
  \end{itemize}
\item
  1999
  \href{https://ceskadigitalniknihovna.cz/uuid/uuid:b931b0f0-a760-11e7-ae0a-005056827e52}{Urmedvěd}:

  \begin{itemize}
  \tightlist
  \item
    Kratochvíl, Jiří
  \item
    experimentální román odehrávající se během normalizace
  \item
    Helenka propadla svodům slovácké kuchyně na základě kalendáře
    Restaurací a jídelen, kde oprášili různý polozapomenutý jídelníčky a
    jídla, jako trdelníky
  \end{itemize}
\item
  2000
  \href{https://ceskadigitalniknihovna.cz/view/uuid:a52ce0c6-2ad8-45b6-a768-ec516dd103f3?page=uuid\%3Af2321a13-5740-11e3-852c-0050569d679d&fulltext=trdeln\%C3\%ADk\%20OR\%20trdeln\%C3\%ADky\%20OR\%20trdeln\%C3\%ADk\%C5\%AF&source=mzk}{Tišnovské
  noviny}:

  \begin{itemize}
  \tightlist
  \item
    report z výstavy v podhoráckym muzeu Od koblížku k mazanci
  \item
    vystavený obrovský formy na trubičky zvaný trdlovce či trdelníky
  \end{itemize}
\item
  2001
  \href{https://ceskadigitalniknihovna.cz/view/uuid:a9321c14-1021-43dc-8103-3032373856fa?page=uuid\%3A85bc7b46-bf59-11ed-9fa2-001b63bd97ba&source=kfbz}{Slovácko}:

  \begin{itemize}
  \tightlist
  \item
    Lidová strava na Uherskohradišťsku jako součást kulturního dědictví,
    Ludmila Tarcalová
  \item
    v obcích mezi Uh. Hradištěm a Uh. Brodem bylo známý pečení trdelníku
    nad uhlíky, podávaly se rozkrájený na kolečka, v Jakubí i o svatbách
    a jinejch rodinných oslavách
  \end{itemize}
\item
  2001
  \href{https://ceskadigitalniknihovna.cz/view/uuid:322e701c-2381-435e-a7c8-e76568ab8a64?page=uuid:edf52cea-b1b8-11ed-8d8c-001b63bd97ba&fulltext=trdeln\%C3\%ADk\%20OR\%20trdeln\%C3\%ADky\%20OR\%20trdeln\%C3\%ADku\%20&source=kfbz}{Zlínské
  noviny}:

  \begin{itemize}
  \tightlist
  \item
    článek "naše trdelníky vs americký big mac"
  \item
    o festivalu slow food ve Vrbici na Břeclavsku, kde znovuobnovovali
    zapomenutý dovednosti předků
  \item
    obrázky a popis pečení
  \item
    s. 6: Projděme s fašankem Moravu, Olga Vlasáková, masopustní zvyky
    na Moravě, smažený pečivo, boží milosti, křehotiny, šišky, trdelníky
    nebo koblihy
  \end{itemize}
\item
  2001
  \href{https://ceskadigitalniknihovna.cz/uuid/uuid:34c2b160-28d1-11e7-9efd-005056827e52}{Velikonoce}:

  \begin{itemize}
  \tightlist
  \item
    Pavel Toufar
  \item
    s. 107: už i na moravském Slovácku pouze vzpomínají na upečený
    trdelník
  \item
    popis pečení ve vzpomínkách Antonína Václavíka:

    \begin{itemize}
    \tightlist
    \item
      oheň z bukového dřeva, muselo vzniknout hodně řezavých uhlíků
    \item
      dřevěný válec, ke stranám zúžený, zapravený do dvou cihel, s
      kličkou
    \item
      když se dostatečně nehřál, natáčelo se na něj těsto
    \item
      při pečení se otáčelo a máčelo mlékem
    \item
      když se trochu zapeklo, sypalo se strouhaným perníkem a cukrem a
      dál peklo do zlatočervena
    \item
      po upečení se ovinul kožichem, aby se snadněji vytáhl a trdelník
      zůstal neporušený
    \item
      navrch se potíral medem
    \end{itemize}
  \end{itemize}
\item
  \href{https://ceskadigitalniknihovna.cz/uuid/uuid:06cd95f1-915c-4408-8ab5-5e8abdacd4ea}{Acta
  musealia}:

  \begin{itemize}
  \tightlist
  \item
    článek o Slovácké búdě v Luhačovicích
  \item
    s. 107: ceník, trdelník za 30 halířů
  \item
    s. 111: Emilie Jurkovičová měla na starosti kuchyni, nabízeli
    skalický zázvorníky, trdelníky a pagáčky
  \end{itemize}
\item
  2002
  \href{https://www.digitalniknihovna.cz/nulk/uuid/uuid:f582cd95-65fe-4d23-b7a5-29f879aeeab3}{Péče
  o tradiční lidovou kulturu v České republice}:

  \begin{itemize}
  \tightlist
  \item
    18. strážnické sympozium 24.-25. září 2002 : sborník příspěvků
  \item
    v mnoha obcích mezi Uh. Brodem a Hradištěm nesměl o masopustu chybět
    trdelník z nekynutýho taženýho těsta pečenýho na ohni
  \item
    podávaly se rozkrájený na kolečka
  \item
    někde byly součástí svatebního jídelníčku, pohřbech a jinejch
    význačnejch rodinnejch oslavách
  \item
    dneska se pečou jen zcela sporadicky
  \end{itemize}
\item
  2002
  \href{https://www.nosislav.cz/assets/File.ashx?id_org=10486&id_dokumenty=1709}{Nosislavický
  zpravodaj}, únor, s. 8:

  \begin{itemize}
  \tightlist
  \item
    článek tradiční pochoutka nejen z Nosislavi - trdelníky
  \item
    až do první poloviny 20. století se připravovali při slavnostních
    příležitostech v letní černé kuchyni
  \item
    na topení používali speciální otýpky z jasanových nebo vrbových
    větví - chábí
  \item
    trdlo bylo dřevěný, kónický, asi 55 cm dlouhý, 8-10 cm v průměru, s
    klikou a usazený na třínožkách nad ohněm
  \item
    těsto se namotalo na trdlo a rozplácalo, potřelo vejci a posypalo
    ořechy s mákem
  \end{itemize}
\item
  2002
  \href{https://www.nosislav.cz/assets/File.ashx?id_org=10486&id_dokumenty=1709}{Nosislavický
  zpravodaj}, červen, s. 6:

  \begin{itemize}
  \tightlist
  \item
    recept na pečení v troubě, sypanej mandlema
  \item
    fotka pomůcek na pečení
  \end{itemize}
\item
  2002
  \href{https://ceskadigitalniknihovna.cz/uuid/uuid:0e216ec0-0395-11e4-a680-5ef3fc9bb22f}{Malovaný
  kraj}:

  \begin{itemize}
  \tightlist
  \item
    s. 5, Urbančík, Vojtěch. Domy č. 119B, č. 65 a Národopisné sbírky ve
    Vlčnově
  \item
    ve Vlčnově mají památkově chráněný domy, v muzeu v bejvalý škole je
    i trdlo na trdelník
  \end{itemize}
\item
  2002
  \href{https://ceskadigitalniknihovna.cz/uuid/uuid:cd4bfec0-29e6-11ed-af34-5ef3fc9bb22f}{Když
  zazpívají křídlovky}:

  \begin{itemize}
  \tightlist
  \item
    s. 296, Muzikanti okny neutíkají
  \item
    za války hráli muzikati v Krhovicích (u Znojma) a v bufetu byly
    trdelníky
  \end{itemize}
\item
  2002
  \href{https://ceskadigitalniknihovna.cz/uuid/uuid:b70d1490-9c83-11ed-b51a-005056827e52}{Knížka
  o Slatině}:

  \begin{itemize}
  \tightlist
  \item
    není online, v náhledu vidím: na čerstvém sádle se dělaly koblihy
    nebo trdelníky. To se těsto v proužkách namotávalo na dřevěné
    válečky...
  \end{itemize}
\item
  2002
  \href{https://ceskadigitalniknihovna.cz/uuid/uuid:2f7b11a0-3d86-11ed-b06c-005056827e52}{Město
  Velké Pavlovice}:

  \begin{itemize}
  \tightlist
  \item
    s. 227: pečivo do kúta, před 2. světovou válkou, nosily se i
    trdelníky
  \end{itemize}
\item
  2003
  \href{https://ceskadigitalniknihovna.cz/uuid/uuid:99baad60-eb86-11e3-a2c6-005056827e51}{Kouzlo
  rodinného stolu}:

  \begin{itemize}
  \tightlist
  \item
    Marie Kubátová
  \item
    recepty po tetě Valči z valašské obce Pitína
  \item
    Hrozenkovské trdelníky
  \item
    recept na trdelníky pečený v troubě
  \end{itemize}
\item
  2003
  \href{https://ceskadigitalniknihovna.cz/view/uuid:cc3e0ec0-084a-11e6-a611-005056827e51?page=uuid\%3Afaea6360-21de-11e6-8803-005056827e51&fulltext=trdeln\%C3\%ADk\%20OR\%20trdeln\%C3\%ADky\%20OR\%20trdeln\%C3\%ADk\%C5\%AF&source=mzk}{Na
  dobré hodince v Otnicích}:

  \begin{itemize}
  \tightlist
  \item
    s. 80 trdelníky do kouta
  \item
    s. 182 babička dělala trdelníky
  \item
    s. 225 místní nářečí
  \item
    s. 233 slovníček - trdelník = trubičky pečený na kukuřičným stvolu
  \end{itemize}
\item
  2003
  \href{https://ceskadigitalniknihovna.cz/uuid/uuid:d9f114f0-527a-11e3-ac69-005056827e51}{Kniha
  o Těšově (1298-1976)}:

  \begin{itemize}
  \tightlist
  \item
    s. 253 - svatba, na námluvách u rodiny nevěsty na přivítanou dávali
    vdolečky nebo trdláče
  \item
    s. 263 - snídaně u ženicha, dostávají vdolečky nebo trdláče
  \item
    s. 332 - k obědu bývaly trdláče pečené na trdelníku na ohništi nad
    ohněm

    \begin{itemize}
    \tightlist
    \item
      trdláče téměř vymizely v první polovině 20. století kvůli vymizení
      ohnišť z chalup
    \end{itemize}
  \end{itemize}
\item
  2003
  \href{https://ceskadigitalniknihovna.cz/uuid/uuid:72938430-ef5f-11e5-bdc9-005056827e52}{Rok
  ve Vlčnově 1945-46}:

  \begin{itemize}
  \tightlist
  \item
    s. 130: o hodech se dělaly trdláče, těsto se opéká na rožni nad
    žhavým uhlím, natočené na dřevěném válečku,
  \item
    s. 185: v mnoha staveních se zachovalo ohniště, kde se pekly trdláče
  \end{itemize}
\item
  2003
  \href{https://ceskadigitalniknihovna.cz/view/uuid:b004cc10-6352-11e4-b42a-005056827e52?page=uuid\%3A70d8a160-6799-11e4-8fe2-5ef3fc9bb22f&fulltext=trdeln\%C3\%ADk\%20OR\%20trdeln\%C3\%ADky\%20OR\%20trdeln\%C3\%ADk\%C5\%AF&source=mzk}{Slovenská
  kuchařka}:

  \begin{itemize}
  \tightlist
  \item
    s. 53 recept na Skalický trdelník
  \item
    s. 160 recept na máslový trdelnky
  \end{itemize}
\item
  2003
  \href{https://ceskadigitalniknihovna.cz/view/uuid:cdea10f0-f64a-11e5-8d5f-005056827e51?page=uuid\%3A866b8a90-2194-11e6-8145-5ef3fc9bb22f&fulltext=trdeln\%C3\%ADk\%20OR\%20trdeln\%C3\%ADky\%20OR\%20trdeln\%C3\%ADk\%C5\%AF&source=mzk}{Sladkosti
  od moravské maminky}:

  \begin{itemize}
  \tightlist
  \item
    recept na slovácké trdelníky, pečený nad uhlíky
  \end{itemize}
\item
  2003
  \href{https://ceskadigitalniknihovna.cz/uuid/uuid:d6ed8ce0-451c-11e3-9c86-005056827e51}{Věstník
  Historicko-vlastivědného kroužku v Žarošicích}:

  \begin{itemize}
  \tightlist
  \item
    článek o Daně Partykové-Wurstové, tvůrkyni uměleckých figurek z
    těsta,
  \item
    vídala babičku, jak připravuje trdelníky, pégny, malé koláčky se
    sypánkovými drdůlky
  \item
    narodila se v roce 1936
  \end{itemize}
\item
  2004
  \href{https://dadala.hyperlinx.cz/hypsladkuch/ost/ostr0025.html}{Dadalova
  kuchařka}:

  \begin{itemize}
  \tightlist
  \item
    2004 je datum první archivace na archive.org
  \item
    nesí chybět na vinařských slavnostech
  \item
    dodnes pečen v Nosislavicích a Němčičkách u Hustopečí
  \item
    recept na Nosislavický a Skalický trdelník
  \item
    Nosislavickej je pečenej v troubě
  \end{itemize}
\item
  2004 Tradiční pečivo, Milena Habustová, Jiřina Veselá:

  \begin{itemize}
  \tightlist
  \item
    s. 27:

    \begin{itemize}
    \tightlist
    \item
      masopustní pečivo mezi Uh. Hradištěm a Uh. Brodem, ale
      připravovaly se i k jinejm slavnostním příležitostem
    \item
      kynutý těsto na kónickym válci nad žhavými uhlíky
    \end{itemize}
  \item
    s. 40:

    \begin{itemize}
    \tightlist
    \item
      topeniště, popisujou, jak místnosti s černou kuchyní byly špinavý,
      až krby
    \end{itemize}
  \item
    s. 42:

    \begin{itemize}
    \tightlist
    \item
      obrázek s receptem na trdelník od Sibilly Dorizio
    \end{itemize}
  \item
    s. 45:

    \begin{itemize}
    \tightlist
    \item
      popis vaření na ohni ve venkovských kuchyních
    \end{itemize}
  \item
    s. 49:

    \begin{itemize}
    \tightlist
    \item
      na konci 18. století se v měšťanských kuchyních objevujou pláty
      nad ohněm a vznikají sporáky s pečící troubou, ale šířilo se to
      pomalu a na venkově se
    \end{itemize}
  \item
    s. 51:

    \begin{itemize}
    \tightlist
    \item
      někde vznikly samostatný místnosti pro černý kuchyně, takže v
      jizbě nebyl dým
    \item
      někde se ale otevřený ohniště udrželo až do půlky 20. století
      (Těšínsko)
    \end{itemize}
  \item
    s. 54:

    \begin{itemize}
    \tightlist
    \item
      změny topeniště zapříčinily změny pečiva a rozvoj složitějších
      receptů
    \end{itemize}
  \item
    s. 116:

    \begin{itemize}
    \tightlist
    \item
      recept, obrázek válců na pečení trdelníků
    \item
      velký trdelníky se krájí na kroužky, malý se postupně odmotávají
    \item
      původně se dělaly z nekvašenýho přesnýho těsta, pak se přidávaly
      kypřidla, pak těsto koláčový a nakonec i těsto překládaný, ze
      kterýho se dělaly trubičky plněný vařeným sněhem z cukru a bílků
    \end{itemize}
  \end{itemize}
\item
  2004
  \href{https://ceskadigitalniknihovna.cz/uuid/uuid:8ad90ef0-1c91-11ed-bb16-005056827e52}{Jako
  když dýchneš na sklo}:

  \begin{itemize}
  \tightlist
  \item
    Miroslava Ludvíková
  \item
    vzpomíná, jak se v 50. letech poprvé setkala s trdelníky
  \item
    dělala učitelku v jedné vsi u Brna
  \item
    pomáhala při draní peří skupině místních žen, pak jí pohostily
    smaženým trdelníkem
  \end{itemize}
\item
  2004
  \href{https://ceskadigitalniknihovna.cz/uuid/uuid:a088d120-1f29-11e6-8803-005056827e51}{Pečeme
  po celý rok}:

  \begin{itemize}
  \tightlist
  \item
    Štěpánská, Věra
  \item
    není online, najít
  \item
    s. 8: náhled, vyrábí různé tvary. Do tohoto období patří také
    příprava božích milostí, trdelníků, hoblovaček apod
  \item
    s. 26: náhled, VELIKONOČNÍ PEČIVO Trdelníky 560 g hladké mouky 300
    ml čerstvého mléka 40 g krupicového cukru 30 g čerstvého
  \end{itemize}
\item
  2004
  \href{https://ceskadigitalniknihovna.cz/uuid/uuid:6544a590-d474-11e4-8565-005056827e52}{Narození
  a smrt v české lidové kultuře}:

  \begin{itemize}
  \tightlist
  \item
    s. 128: co se posílá do kúta, Tuřany smažený trdelníky, v
    Biskupicích trdláče
  \item
    s. 136: hostiny při úvodě, uplatňovalo se tradiční pečivo,
    trdelníky, pampuchy, babičky, boží milosti
  \end{itemize}
\item
  2004
  \href{https://ceskadigitalniknihovna.cz/uuid/uuid:5823cdf0-1024-11ea-af21-005056827e52}{Babiččiny
  recepty nejen pro dědečky}:

  \begin{itemize}
  \tightlist
  \item
    není online,najít
  \item
    s. 58: se, teplé krájíme na šikmé řezy. Dlouho vydrží. Trdelníčky
    Hradec Králové • 25 dkg polohrubé mouky, 2 dkg hladké mouky
  \end{itemize}
\item
  2005:

  \begin{itemize}
  \tightlist
  \item
    VLASÁKOVÁ, Olga: Jak masopust přišel k trdlům a trdelníkům. Slovácké
    noviny 15, 2005, č. 9 z 2. 2., s. 14.
  \item
    není online, najít
  \end{itemize}
\item
  2005
  \href{https://ceskadigitalniknihovna.cz/uuid/uuid:024b8100-ced7-11e3-b110-005056827e51}{Násedlovice
  v obrazech}:

  \begin{itemize}
  \tightlist
  \item
    jen zmínka o trdelníku, že se nosil do kouta
  \item
    vesnice u Kyjova
  \end{itemize}
\item
  2005
  \href{https://ceskadigitalniknihovna.cz/uuid/uuid:12d02530-5e00-11e5-bf4b-005056827e51}{České
  Vánoce}:

  \begin{itemize}
  \tightlist
  \item
    není online, najít
  \item
    s. 123, vyráběli Jiranovi nejrůznější speciality, například krásné
    fondánové figur- ky, trdelníky, originální bonbony a další dobroty
  \end{itemize}
\item
  2006
  \href{https://ceskadigitalniknihovna.cz/uuid/uuid:d3bddf00-5ca4-11eb-a728-5ef3fc9bb22f}{Zvuk}:

  \begin{itemize}
  \tightlist
  \item
    článek o Luhačovicích, Slovenský trdelníky v Búdě
  \end{itemize}
\item
  2007 Lidová kultura, národopisná encyklopedie, Čech, Moravy a Slezska,
  část O-Ž:

  \begin{itemize}
  \tightlist
  \item
    s. 656, obřadní pečivo - původně z přesnýho těsta, proužky těsta nad
    ohněm, délka 25 až 50 cm, pozdějc kynutý těsto jako na vánočku
  \item
    s. 719 v hesle "pečivo":

    \begin{itemize}
    \tightlist
    \item
      litej trdelník byl nejstarší způsob, pozdějc v troubě a pak
      smažený
    \item
      přejatý z Rakouska, ale i tam je dělali hlavně cukráři
    \item
      tyhle informace jsou možná ze článku Ludvíkové o lidový kultuře na
      Znojemsku z roku 1975. není zatím digitalizovanej, takže jsem ho
      nečetl
    \end{itemize}
  \end{itemize}
\item
  2006
  \href{https://ceskadigitalniknihovna.cz/uuid/uuid:1b16fb51-6afc-11e2-b4a8-001018b5eb5c}{Bílé
  Karpaty}:

  \begin{itemize}
  \tightlist
  \item
    Valašský trdelník Ondřeje Hladkého má chráněnou značku "vyrobeno v
    Beskydech"
  \end{itemize}
\item
  2006
  \href{https://ceskadigitalniknihovna.cz/uuid/uuid:616f5dc0-6adb-11e7-b92d-005056827e51}{3333
  receptů od A do Z}:

  \begin{itemize}
  \tightlist
  \item
    Vašák, Jaroslav
  \item
    není online, najít
  \item
    s. 553: dřevěným uhlím. Při pečení se trdelníky potírají máslem.
    Zlatavé prstence se z válečku sejmou a pocukrují. Podávají se
  \end{itemize}
\item
  2006
  \href{https://ceskadigitalniknihovna.cz/uuid/uuid:a7380140-967b-11e9-9209-005056827e51}{Muzeum
  vesnice jihovýchodní Moravy}:

  \begin{itemize}
  \tightlist
  \item
    průvodce expozicí muzea
  \item
    není online, najít
  \item
    hliněných formách - babůvkách, dále beléše - vdolky pečené na
    plotně, trdelníky - velké trubičky sypané posýpkou s ořechy, které
  \end{itemize}
\item
  2007
  \href{https://dikda.snk.sk/uuid/uuid:36e701c1-1689-4807-876f-6895e3c17a3c}{Spišský
  deník Korzár}:

  \begin{itemize}
  \tightlist
  \item
    report ze Skalice o výrobě trdelníků
  \end{itemize}
\item
  2007
  \href{https://ceskadigitalniknihovna.cz/uuid/uuid:889f61a0-f31b-11e3-97c9-001018b5eb5c}{Malovaný
  kraj}:

  \begin{itemize}
  \tightlist
  \item
    s. 22-23, Pavelčík, Jiří. Každodennost v životě slováckého lidu XXI.
    (Doplňkové pracovní aktivity)
  \item
    těsto podobný jako na vdolečky, jen trochu tužší
  \item
    těsto se válelo do úzkýho pruhu, nebo širokýho plátu
  \item
    zmínka o náplních z tvarohu, máku, ořechů. ty náplně se zaplňovaly
    do plátů (což je další druh trdélníku, podobnej německýmu
    Baumstriezel)
  \item
    taky zmínka, že na Podluží se jako trdelník označujou i šmetrdole
  \item
    představu, že trdelník pochází z Maďarska označuje za mylnou, i když
    nezpochybňuje, že i tam je populární
  \item
    tradici v současnosti obnovily ženy ze Skalice a prodávají se běžně
    na jarmarcích
  \end{itemize}
\item
  2008
  \href{https://ceskadigitalniknihovna.cz/uuid/uuid:e0d7de90-55fc-11e3-bc9f-5ef3fc9bb22f}{Respekt}:

  \begin{itemize}
  \tightlist
  \item
    článek To kyselo je moc kyselý
  \item
    o tradiční kuchyni, popisujou, jak Mária Romančíková ve Skalici
    připravuje trdelník
  \end{itemize}
\item
  2008
  \href{https://ndk.cz/uuid/uuid:d30cd530-5bfb-11e4-97e9-5ef3fc9bb22f}{Folia
  ethnographica}:

  \begin{itemize}
  \tightlist
  \item
    s. 64 - v roce 1906 postavil v Luhačovicích Jurkovič Slovenskou
    búdu, kde byly skalický speciality, pekli trdelníky, obsluhovali
    slovenský číšníci
  \end{itemize}
\item
  \href{https://ceskadigitalniknihovna.cz/uuid/uuid:0986bb00-6623-11e3-ae59-005056827e52}{Bohuslavice
  u Kyjova}:

  \begin{itemize}
  \tightlist
  \item
    není online, najít
  \item
    s. 45: náhled, milosti, snad i perník a trdelníky. Z jarmarku nebo
  \end{itemize}
\item
  2008
  \href{https://ceskadigitalniknihovna.cz/uuid/uuid:b2d480f0-9aeb-11e3-8e84-005056827e51}{Luhačovická
  zastavení starodávná i novější}:

  \begin{itemize}
  \tightlist
  \item
    není online, najít
  \item
    s. 59, jeho přátelé. Holubyho hospodyně (gazdinka) při té
    příležitosti napekla koláče, zázvorníky a trdelníky a dr. Blaho s
  \item
    s. 75, proslavila svými specialitami, trdelníky a zázvomíky.
    Zbojnická a živáňská pečeně se opékala venku před búdou. V hlavní
    sezóně tu
  \item
    s. 106, Slováckou búdu, kde byl vzácný Host přivítán trdelníky a
    tradičním slováckým pečivem. Personál jej obdaroval keramickým
    talířem
  \end{itemize}
\item
  2009
  \href{https://ceskadigitalniknihovna.cz/uuid/uuid:33331cf0-d935-11e5-a3e0-005056827e51}{Recepty
  naší rodiny: 13. ročník. Venkovská kuchařka}:

  \begin{itemize}
  \tightlist
  \item
    Doležal, Vladimír; Doležalová, Alena
  \item
    není online, najít
  \item
    náhledy:

    \begin{itemize}
    \tightlist
    \item
      s. 234: Moučníky \^{} Smažené moučníky TRDELNÍKY 150g polohrubé
      mouky, 60g másla, 60g krupicového cukru, 40g droždí, 4 žloutky, 3
    \item
      s. 237: TRDELNÍČKY 160 ml mléka, 30g cukru, 30g droždí
    \item
      s. 239: „TRDELNÍK`` NA PLECHU 250 ml mléka, 5 lžic cukru, 35 g
      droždí, 3 žloutky, 3 lžíce másla
    \end{itemize}
  \end{itemize}
\item
  2010
  \href{https://ceskadigitalniknihovna.cz/uuid/uuid:553c58c0-69fe-11eb-9f97-005056827e51}{Velikonoční
  lidová zdobnost}:

  \begin{itemize}
  \tightlist
  \item
    Olga Vlasáková
  \item
    není online, najít
  \item
    s. 12: Pečení trdelníků Trdlo, u tetinky Mazurové ve Skalici.
    Zakresleno a zapsáno v roce 1984. -12-
  \item
    s. 13: JAK MASOPUST K „TRDLU" PŘIŠEL Masopust je dávný lidový
    obyčej. Za původem jeho veselých oslav bychom se však
  \item
    s. 14: trdla s patřičným kováním stávala také zbraní. „Před
    rožněním," vysvětluje paní Mazurová, „se trdelník na trdlu obalí
  \item
    s. 15: Tvary masopustnľho pečiva jsou rozmanité. Trdelnľky a koblihy
    musí mft těsto dobré, bohaté, kynuté, boží milosti se
  \item
    s. 16: Pečení trdelníků v Ořechově u Brna
  \end{itemize}
\item
  2010
  \href{https://ceskadigitalniknihovna.cz/uuid/uuid:1b8fd480-8ee4-11ea-ae16-005056827e52}{Chutě
  a vůně slovácké kuchyně}:

  \begin{itemize}
  \tightlist
  \item
    Tarcalová, Ludmila, Kondrová, Marta
  \item
    není online, najít
  \item
    s. 40: Hanácké Slovácko Kysaný trdelník z Kobylí Suroviny: 1 kostka
    kvasnic 200 g cukru 500 ml mléka 250 g tuku (másla) 5
  \item
    s. 57: patřily trdelníky pečené ke svatbě, koblihy na fašank, o
    Velikonocích z formy upečený beránek nebo buchta babovice či figurky
  \item
    s. 71: mákem se jedly příležitostně, stejně jako malé koláčky,
    koblihy, mrváně, trdelníky a perníky, které se pekly к hodům
  \item
    s. 113: dortů, trdelníků, beránků a dalšího obřadního pečiva, která
    je spolu s výrobou medu a vosku dodnes věhlasná
  \end{itemize}
\item
  2010
  \href{https://zlinsky.denik.cz/podnikani/valassky-trdelnik-se-na-trhu-objevil-pred-8a2d.html}{Valašský
  trdelník se na trhu objevil před šesti lety}:

  \begin{itemize}
  \tightlist
  \item
    od roku 2004 týpek dělá trdelník podle rodinnýho receptu
  \item
    na Valašsko se měl dostat někdy v 17. století z Rumunska s Valachama
    - ale to ani necitujou toho chlapíka
  \end{itemize}
\item
  2010
  \href{https://www.idnes.cz/ekonomika/domaci/trdelnik-hrstka-testa-ktera-dobre-vydelava.A101230_125615_ekonomika_spi}{Idnes}:

  \begin{itemize}
  \tightlist
  \item
    článek Trdelník: hrstka těsta, která dobře vydělává
  \item
    citujou prodejce, co dělá trdelníky už 7 let, protože viděl, že měli
    úspěch na nějakym jarmarku
  \item
    náklady na jeden tak 5 korun, je to jen mouka, tuk a droždí
  \end{itemize}
\item
  2010
  \href{https://ceskadigitalniknihovna.cz/uuid/uuid:d23e8f20-80e9-11e7-b92d-005056827e51}{Újezdec
  u Luhačovic}:

  \begin{itemize}
  \tightlist
  \item
    není online, jen úryvek:

    \begin{itemize}
    \tightlist
    \item
      Každá hospodyň napékla deň před tým vdolků lebo trdláčů, všecko
    \end{itemize}
  \end{itemize}
\item
  2010
  \href{https://ndk.cz/uuid/uuid:41ed2580-5e9b-11e4-8b87-001018b5eb5c}{Folia
  ethnographica}:

  \begin{itemize}
  \tightlist
  \item
    s. 50-\/-51:

    \begin{itemize}
    \tightlist
    \item
      topeniště: černá kuchyně se na některejch místech východní Moravy
      dochovala až do začátku 20. století
    \item
      po roce 1915 už se v drtivý většině používaly kamna či sporák s
      litinovejma plátama a troubou ve městech i na venkově
    \item
      z užívání pak vyšlo nádobí určený k přípravě na ohni, jako jsou
      kolíky na trdelníky
    \end{itemize}
  \end{itemize}
\item
  2010
  \href{https://ceskadigitalniknihovna.cz/uuid/uuid:0aef9650-0ddf-11e5-9eb3-005056827e52}{České
  Vánoce v kuchyni}:

  \begin{itemize}
  \tightlist
  \item
    Herynek, Petr
  \item
    tři zmínky, není online, musím sehnat, najít
  \item
    s. 23: pečenými pa- náčky a panenkami, které měly oči a knoflíky z
    hrozinek, se sma- ženými věnečky, preclíčky a trdelníky, pečenými
  \item
    s. 26: kousky. (recept Marie Janků-Sandtnerové) `Trdelníky Tak zní
    slogan vyhlášeného pekaře trdelníků Martina Figury, kte- rý
  \item
    s. 27: Spolu s dalšími dobrými a vydatnými pokrmy se trdelníky pekly
    šestinedělkám „do kouta``, což byla plachtou - koutnicí
  \end{itemize}
\item
  2010
  \href{https://ceskadigitalniknihovna.cz/uuid/uuid:467e18f0-1b9b-11e8-a0cf-005056827e52}{Zdroje
  a cíle jazykové popularizace}:

  \begin{itemize}
  \tightlist
  \item
    Svobodová, Jana
  \item
    není online, najít
  \item
    s. 52: v náhledu: ve slovní zásobě svou renesanci a jako pojmemování
    sladkého pečiva (se slovotvournou variantou trdelník)
  \item
    s. 130: v náhledu: nepostradatelné byly ovšem při pečení takzvaného
    trdlovce, trdelníku, nebo také šmetrdólu, tedy sladkého pečiva,
    které se připravovalo...
  \end{itemize}
\item
  2010
  \href{https://ceskadigitalniknihovna.cz/uuid/uuid:de93a670-fd0d-11e7-b1a1-005056827e52}{Starý
  Poddvorov}:

  \begin{itemize}
  \tightlist
  \item
    není online, najít
  \item
    s. 351: náhled, hospodyně smažily koblihy, Boží milosti nebo z
    litého těsta motýlky a pekly manžety - úzké pásky z trdelníkového
    těsta pečené na
  \item
    s. 365: náhled, milosti a pekly trdelníky, kousky z odpalovaného
    těsta a makové bábovky. Dnes se chystá největší sortiment cukroví na
    hody
  \end{itemize}
\item
  2010
  \href{https://ceskadigitalniknihovna.cz/uuid/uuid:eb53abf0-80d8-11e7-b92d-005056827e51}{Ratíškovice}:

  \begin{itemize}
  \tightlist
  \item
    není online, najít
  \item
    s. 545: náhled, boží milostě a trdelníky. Slepici a polévku
    přinášely zpravidla v hrnci s velkým uchem (nosák), ostatní na
    zádech v nůši či
  \item
    s. 546: náhled, koláče, koláčky, trdelníky, případně maso. Druhy
    jídel a jejich pořadí bývaly poměrně ustálené a dodržovaly se ve
    všech
  \item
    s. 681: náhled, koblihy, podruhé velké tvarohové koláče nakrájené na
    klínky, a potřetí buchty s mákem. Z jiného pečiva pak trdelníky z
  \end{itemize}
\item
  2010
  \href{https://ceskadigitalniknihovna.cz/uuid/uuid:46f5c8f0-d962-11ef-8f57-005056827e51}{Nostalgická
  kuchařka a nostalgický gastronomický slovníček}:

  \begin{itemize}
  \tightlist
  \item
    Vašák, Jaroslav
  \item
    není online, najít
  \item
    s. 191: náhled, bývaly ostatky. Smažily se koblihy z nudlového
    těsta, dále koblovačky, růžice, trdelníky a Boží milosti. Koblihami
    se
  \item
    s. 244: náhled, potíralo a trdelník se pilně potíral rozkverlaným
    vaječným žloutkem. Též palice na rozmělnění potravin nebo koření.
  \end{itemize}
\item
  2010
  \href{https://ceskadigitalniknihovna.cz/uuid/uuid:ed1ed830-6889-11e8-8470-005056827e52}{Vánoce
  na Moravě}:

  \begin{itemize}
  \tightlist
  \item
    není online, najít
  \item
    s. 17: moravsko-slovenském pomezí bývaly trdelníky, které dnes
    můžete ochutnat spíše už jen u stánkařů na poutích. Dobré, jemné
  \item
    s. 18: válečkem stále otáčíme a mažeme máslem, až jsou trdelníky
    dozlatova pečené. Hotový trdelník zbývalo sejmout z válečku posypat
  \item
    asi recept podle M. Úlehlové-Tilschové
  \end{itemize}
\item
  2011
  \href{https://ceskadigitalniknihovna.cz/uuid/uuid:05e8afb0-ef6c-11e5-bdc9-005056827e52}{Od
  věnečku k obálence, aneb, Co kdysi znamenala svatba}:

  \begin{itemize}
  \tightlist
  \item
    není online, najít
  \item
    v náhledu: koblihy, trdláče, v pozdější době napekli i cukroví. V
    Prakšicích byli příchozí přivítáni chlebem
  \end{itemize}
\item
  2011 Jak se dříve žilo a co se jedlo v Hluku, aneb, Opomíjená jídla a
  zvyky

  \begin{itemize}
  \tightlist
  \item
    Marie Lekešová
  \item
    město u Uherského Hradiště
  \item
    najít
  \end{itemize}
\item
  2011
  \href{https://ndk.cz/view/uuid:cc17ef80-fa5e-11e7-9854-5ef3fc9ae867?page=uuid\%3A5f880b20-fa6d-11e7-9854-5ef3fc9ae867}{článek
  v magazínu mf dnes} - o trdelnících v Praze, ale i recept
\item
  2011 VEČERKOVÁ, Eva, Německé obyvatelstvo na Mikulovsku a jeho
  obyčeje, Jižní Morava, roč. 47, 2011, s. 235--248.

  \begin{itemize}
  \tightlist
  \item
    není online, najít
  \end{itemize}
\item
  2012 VEČERKOVÁ, Eva, Lidové obyčeje německého etnika na Znojemsku,
  Jižní Morava, roč. 48, 2012, s. 168--196.

  \begin{itemize}
  \tightlist
  \item
    není online, najít
  \end{itemize}
\item
  2013
  \href{https://ndk.cz/view/uuid:1b545920-4f78-11e4-a830-005056827e51?page=uuid:226078c0-5c34-11e4-a6f0-5ef3fc9ae867}{Folia
  ethnographica}:

  \begin{itemize}
  \tightlist
  \item
    článek Kugluf a povidla - rakouské nebo české? Sladká kuchyně v roli
    kulturního dědictví:

    \begin{itemize}
    \tightlist
    \item
      pojednává o vzájemnym ovlivňování receptů mezi vrstvama
      společnosti
    \item
      je složitá otázka co vůbec považovat za národní kuchyni -
      venkovskou, nebo vyšších společenských vrstev?
    \item
      šlechtická kuchyně byla orientovaná hlavně na italsku a
      francouzskou kuchyni
    \item
      až v 19. století začaly do středostavovský kuchyně pronikat
      podněty z lidových kuchyní a národní kuchyně se vyhranily
    \item
      Vídeň byla tavící kotel kultur a vytvořila svébytnou kuchyni
    \item
      specialitou Vídeňský i Český kuchyně je, že sladký pokrmy fungujou
      i jako hlavní chod, ne jen jako moučník
    \item
      moučný jídla navozujou pocit sytosti
    \item
      s. 170:

      \begin{itemize}
      \tightlist
      \item
        obrázek přípravy trdelníku na pečení, ze Skaštic (severně od
        Kroměříže), kolem r. 1920
      \item
        sladký pečivo bylo v archaických receptech společný pro Čechy i
        Moravu (pečivo z chlebovýho těsta, pečivo na plotně, trdelníky)
      \end{itemize}
    \end{itemize}
  \item
    článek Rukopisné záznamy Miroslavy Ludvíkové k výzkumu lidové
    stravy:

    \begin{itemize}
    \tightlist
    \item
      s. 204 - obrázek kolíků na přípravu trdelníků a hotovejch
      trdelníků z Rebešovic u Brna z roku 1972
    \item
      s. 205 - masopust, ve Vlčnově připravovali trdláče, pecové vdolky
    \item
      s. 216-217 - popis některejch variant trdelníků:

      \begin{itemize}
      \tightlist
      \item
        Dětkovice u Prostějova těsto se žloutkem, zadělaný smetanou,
        tenkej plát se namotal na váleček a smažil v sádle
      \item
        v Kobylí velký smetanový trubičky s průměrem 5-6 cm, dlouhý
        12-14 cm, namotaný na kolíky z turkyniska, upekly a posypaly
        mletejma ořechama s cukrem. Po druhý světový válce se dovnitř
        dávala i šlehačka
      \end{itemize}
    \end{itemize}
  \end{itemize}
\item
  2014 Velké dějiny zemí Koruny české. Tematická řada, Lidová kultura

  \begin{itemize}
  \tightlist
  \item
    s. 240-241
  \item
    řeší, že trdelníky, dřív pečený z přesného těsta k masopustu dnes
    nabízí stánkaři
  \item
    není to nic nového, dělo se to vžycky, třeba s perníkem
  \end{itemize}
\item
  2014
  \href{https://epa.oszk.hu/03300/03308/00007/pdf/EPA03308_acta_siculica_2014-2015_497_518.pdf}{A
  székély kürtőskalács}:

  \begin{itemize}
  \tightlist
  \item
    Pozsony Ferenc
  \item
    překlad: \hyperref[250319-1326]{maďarskej článek o trdelníku}
  \item
    historie v Maďarsku a Transylvánii:

    \begin{itemize}
    \tightlist
    \item
      spiesskuchen v překladu knihy v roke 1680
    \item
      kürtőskalács se začíná vyskytovat kolem 1. třetiny 18. století
    \item
      dostal se do sedmihradska pomocí buď Sasů nebo Rakušanů
    \item
      Kuchařka Kristófa Simaie „Způsob přípravy některých jídel`` z roku
      1795 z Kremnice také nazývá „Dorongos fánk`` koláč z kynutého
      těsta s rozinkami, potřený třtinovým medem:

      \begin{itemize}
      \tightlist
      \item
        používal pivovarský kvasnice, droždí až kolem půlky 19. století
      \end{itemize}
    \item
      během 18. století poměrně častej v sedmihradskejch měšǎnskejch
      domácnostech
    \item
      měli speciální formy na pečení
    \item
      v lidovém prostředí se šíří začátkem 19. století
    \item
      krby se v seklersku udržely až do konce 19. století
    \item
      cukrová poleva se rozšířila taky koncem tohoto století
    \item
      rumunský Sasové pekli koláč z plátů těsta na rožni, ale po 2.
      světový se vystěhovali do Německa, Baumstriezel ale pořád pečou
    \item
      po roce 1968 byla v Rumunsku uvolněnější politická atmosféra a
      kürtoskalács se začal připravovat i mimo Transylvánii, třeba u
      moře ve stáncích
    \item
      po roce 1989 se stal populárnější i u maďarskejch návštěvníků,
      postupně se stal symbolem seklerských Maďarů a populární po celym
      Maďarsku
    \item
      po tom, co se stal skalickej trdelník chráněným označením, nastala
      u Seklerů panika a začali se hyperkompenzovat, takže třeba soutěží
      o co největší trdelník
    \end{itemize}
  \item
    histore v Evropě:

    \begin{itemize}
    \tightlist
    \item
      kolem 1450 v Německu doloženej koláč opejkanej na ohni, potíranej
      bílkem
    \item
      v 15. a 16. století slavnostní koláč bohatších měťanskejch a
      šlechtickejch rodin
    \item
      Kuchařka Balthasara Steina, vydaná v Dillingenu v roce 1547, již
      odráží, že v životě koláče došlo na konci 16. století k významné
      změně. Zatímco v maďarských a českých oblastech se stále
      připravoval ze spirálovitě navinutých copů na dřevě, v německých
      osadách se rozválené pláty kynutého těsta kladly přímo na pečicí
      dřevo
    \item
      Kuchařka Christopha Thiemena z roku 1682 dokumentuje rozšíření
      Baumkuchenu na Moravě (asi Haus- Feld- Arzney- Koch- Kunst und
      Wunderbuch, autor Johann Christoph Thieme, ):

      \begin{itemize}
      \tightlist
      \item
        cituje článek, kde se píše o moravských koláčích, ale v
        originále jsou český
      \item
        s.
        \href{https://books.google.cz/books?id=PytAAAAAcAAJ&pg=PA1604&hl=cs&source=gbs_selected_pages&cad=1\#v=onepage&q=Spie\%C3\%9Fkuchen&f=false}{876}
      \item
        Bohmische Küchlein
      \end{itemize}
    \item
      od 18. století se baumkuchen šířil po Evropě a pořád se vyvíjel,
      rozšířil se do Švédska, Polska, Litvy
    \end{itemize}
  \item
    Česko:

    \begin{itemize}
    \tightlist
    \item
      Protože trdlo nebo tredlenice, které se vyskytují u Čechů a
      Moravanů, jsou velmi blízké starému německému
      spiesskuchen-ayrkuchenu, domníváme se, že se v jejich komunitách
      rozšířily vlivem Německa. Jejich základním rysem je, že se kynuté
      těsto navinuté na rožeň před pečením neobaluje v cukru, ale pouze
      se posype mletými ořechy, přičemž se během pečení potírá máslem a
      bílky, takže se na jeho drsnějším povrchu nevytvoří cukrová
      poleva.
    \end{itemize}
  \end{itemize}
\item
  2014
  \href{https://english.radio.cz/old-bohemian-trdelnik-8286412}{Český
  rozhlas}:

  \begin{itemize}
  \tightlist
  \item
    zajímavej je tam rozhovor s ředitelem gastrnomockýho muzea,
  \item
    redaktor se ptá: Oh, look! An `Old Bohemian' something or other, and
    maybe that is a traditional thing that Czechs are eating.'' The
    history suggests otherwise. It is perhaps Hungarian, or Slovak, or
    Turkic?
  \item
    ředitel: No, it goes way beck to Neolithic times. Trdelník ties very
    directly to the open fire. Prior to bread ovens, and prior to
    kitchen stoves or hot plates, there was no other way to cook dough
    other than to twist it on a stick of wood and rotate it over an open
    fire. In a practical sense, this made it accessible for people who
    were travelling, or people staying with their herds out in the
    countryside. So that is the charm of it. As far as it is known in
    Europe, it starts with ancient Greece and practically every nation
    from Sweden down to the south, east and west, people knew Trdelník.
    The only difference is that it had so many local names. The word
    `Trdelník' is the only thing about this food, which is purely Czech.
    It is a very ancient word, and it essentially denotes the use of a
    wooden stick, mallet or spindle -- when twisting yarn this word was
    used. So this is very nice word, which has many meanings in the
    Czech language. For example `trdlo' is a gentle word for someone who
    is a little bit confused in a given situation.
  \item
    pak je tam rozhovor s prodejcem, co si myslí, že možná pochází z
    Maďarska
  \end{itemize}
\item
  2014
  \href{https://ceskadigitalniknihovna.cz/uuid/uuid:a8cb0db0-fbeb-11ee-a794-5ef3fc9bb22f}{Chutě
  a vůně slovácké kuchyně 2}:

  \begin{itemize}
  \tightlist
  \item
    Kondrová, Tarcalová
  \item
    není online, najít
  \item
    náhledy:

    \begin{itemize}
    \tightlist
    \item
      s. 12: některé druhy pečiva (trdelníky, pecové vdolky, chléb).
      Nové topeniště tvořené kachlovými kamny s troubou a litinovou
      plotnou
    \item
      s. 18: údajně oblíbeného a častého pokrmu se obyvatelům Starého
      Města hanlivě přezdívalo bálešáci. Trdelníky, trdláče „Naše trdlo
      sa
    \item
      s. 19: nebo železné stojánky, na kterých se formou pozvolna
      otáčelo pomocí kliky. Trdelníky se \textbackslash{} původně pekly
      na celé jižní
    \item
      s. 24: ještě v první polovině 20. století udržela příprava starého
      typu obřadního pečiva, kterým jsou trdelníky - trdláče
    \item
      s. 44: vypracujeme těsto, které necháme vykynout. Během doby
      kynutí těsto jedenkrát přemísíme. Formu k pečení trdelník necháme
      nahřát nad
    \item
      s. 45: Staré obřadní pečivo se původně připravovalo na dřevěné
      válečkové formě trdelníku či trdle na otevřeném ohništi
    \end{itemize}
  \end{itemize}
\item
  2014
  \href{https://ceskadigitalniknihovna.cz/uuid/uuid:a07eb793-2212-4764-9edc-25f44e6931a3}{Českoskalický
  zpravodaj}:

  \begin{itemize}
  \tightlist
  \item
    děti ze speciálních tříd v Český Skalici se učili o vánočních
    jídlech, připravili bramborový salát, řízky, vinnou klobásu a
    staročeský trdelníky
  \end{itemize}
\item
  201?
  \href{https://web.archive.org/web/20201230130310/https://www.kurtos.eu/dl/kurtos_konyv.pdf}{KÜRTŐSKALÁCS
  A VILÁG MINDEN TÁJÁN ISMERT SZÉKELY-MAGYAR SÜTEMÉNY}:

  \begin{itemize}
  \tightlist
  \item
    Hantz Péter, Pozsony Ferenc, Füreder Balázs
  \item
    dokument z už neexistujícího maďarskýho webu
  \item
    Nejblíže ke kořenům koláčů pečených na válcích, ke staroněmeckému
    Spiesskuchen-Ayrkuchen, má česko-moravské trdlo a trdelník pečený ve
    slovenské Skalici. Trdlo-trdelník je vyrobeno z relativně silného
    kynutého těsta obaleného kolem válce. Způsob zpracování povrchu
    surového kynutého těsta na válci se však výrazně liší od toho, jak
    dnes postupujeme před pečením kürtőskalács nebo Baumstriezel.
    Trdlo-trdelník se před pečením nepotahuje cukrem a jeho povrch se
    nevyhlazuje, ale posype se mletými ořechy (mandlemi, meruňkovými
    jádry).
  \item
    Kürtősfánk (smažený kürtős) ...se vymyká řadě, jen jeho tvar
    připomíná kürtőskalács. Jeho velikost je mnohem menší, délka obvykle
    nepřesahuje 15--20 centimetrů. Známý je především v maďarských a
    německých oblastech, ale jeho příprava je v lidové kultuře méně
    spjata se slavnostními událostmi. Jeho základem je kynuté těsto
    podobné těstu na kürtőskalács, které se namotává na malý válec, bez
    obalování v cukru a smaží se v horkém oleji za stálého otáčení. Po
    usmažení se podává posypaný moučkovým cukrem, případně mletými
    ořechy, nebo naplněný šlehačkou.
  \item
    Stejně jako trdlo zmizelo z české a sekacz z polské lidové kultury,
    tak i kürtőskalács zmizel z podstatné části maďarského jazykového
    území. V polovině 20. století se pekl téměř výhradně už jen v
    Sedmihradsku (Székelyföld).
  \item
    Důvodem je, že v důsledku měšťanství a urbanizace nahradily tento
    druh koláče ve velké části maďarského jazykového území dorty
    městského a měšťanského původu.
  \item
    Kromě omezeného rozšíření nových koláčů v Sedmihradsku hrálo v
    zachování kürtőskalács roli i to, že na východním okraji maďarského
    jazykového území se krby s otevřenými ohništi udržely až do konce
    19. století. V těchto otevřených ohništích a také v předsálích
    pekařských pecí bylo možné péct kürtőskalács na žhavém uhlí bez
    zvláštních příprav. Kürtőskalács je dodnes považován za krále
    slavnostních koláčů a důležitou součást svatebních pokrmů v
    Sedmihradsku.
  \item
    Ve druhé polovině 20. století, po ideologickém uvolnění v Rumunsku v
    roce 1968, se kürtős postupně rozšířil i do oblastí mimo
    Sedmihradsko, především do pobřežních a horských turistických
    center. Po rumunském systémovém převratu v roce 1989 se jím nabízeli
    především turisté z Maďarska, kteří navštěvovali sedmihradské
    vesnice, a zároveň se stal oblíbeným, symbolickým koláčem stále
    populárnějších místních slavností.
  \item
    Obyvatelé Szentivánlaborfalva vyrobili 14 metrů dlouhý koláč. V roce
    2011 na místní slavnosti v Oroszfalu, která se nachází v sousedství
    Kézdivásárhely, upekli na žhavém uhlí na venkovním ohništi kürtős
    dlouhý už 16,8 metru.
  \item
    The early form of Kürtősh Kalách was imported to Hungarian --
    speaking regions via Austrians and Saxons.
  \end{itemize}
\item
  201?
  \href{https://web.archive.org/web/20201230125626/http://kurtos.eu/history}{History
  of Kürtősh Kalách}:

  \begin{itemize}
  \tightlist
  \item
    HANTZ Péter - POZSONY Ferenc, maďarskej článek o historii špízovejch
    koláčů
  \item
    starověk - chleba Obelias
  \item
    15. století - ayrkuchen, maže se žloutkem
  \item
    Kürtőshfánk - kurtos smaženej v oleji, podobně jako se smažil na
    Moravě
  \item
    domnívají se, že kurtos byl do Maďarska přinesenej Rakušanama nebo
    Sasama
  \end{itemize}
\item
  2015
  \href{http://zivepomezi.cz/wp-content/uploads/2014/12/Publikace-zvyky_vnit.blok_.pdf}{Od
  Hromnic až do Tří Králů : Jiří Mačuda}:

  \begin{itemize}
  \tightlist
  \item
    týká se to Znojemska
  \item
    strana 7, tenhle trdelník byl litej a byl rozšířenej i v Rakousku,
    pak se vyvinul a děalj se menší trubičky i smažený
  \item
    na straně 13 o trdelnícíh při masopustě
  \item
    strana 15 - další zmínky, německej název: prügelkrapfen, česky
    trdláče, pekly se na válečku "pryglík", pak i popis přípravy
  \item
    strana 84 - recept na trdelník v troubě (kynutý těsto, bez másla)

    \begin{itemize}
    \tightlist
    \item
      a taky je tam spousta dalších zajímavejch receptů
    \end{itemize}
  \end{itemize}
\item
  2016 Kondrová, Marta:

  \begin{itemize}
  \tightlist
  \item
    jen citace v Kulinární dědictví Čech, Moravy a Slezska, v seznamu
    literatury ale chybí
  \end{itemize}
\item
  2016
  \href{https://ceskadigitalniknihovna.cz/view/uuid:cc664180-3864-11ea-85b5-005056827e52?page=uuid\%3A2786f459-7045-4ac5-8f33-a52948185820&fulltext=trdeln\%C3\%ADk\%20OR\%20trdeln\%C3\%ADky\%20OR\%20trdeln\%C3\%ADk\%C5\%AF&source=mzk}{Slovník
  podkrkonošského nářečí}:

  \begin{itemize}
  \tightlist
  \item
    trdelňik, tardelňik, pečivo podobný kremroli, pečená na trdlu.
    cituje hlavně různý rukopisy a terénní poznatky
  \end{itemize}
\item
  2017 Zámecké kuchyně : zámecké kuchyně v kontextu evropského vývoje:

  \begin{itemize}
  \tightlist
  \item
    Vítězslav Štajnochr
  \item
    citace v Kulinární dědictví Čech, Moravy a Slezska, s. 80
  \item
    najít -
    \href{https://search.mlp.cz/cz/titul/zamecke-kuchyne/4396772/\#/getPodobneTituly=deskriptory-eq:3212124-amp:key-eq:4396772}{městská
    knihovna}
  \end{itemize}
\item
  2017
  \href{https://www.metro.cz/praha/masarykovi-pekli-dort-cukrarnu-zavrelo-znarodneni.A171019_160631_metro-praha_lam}{metro}:

  \begin{itemize}
  \tightlist
  \item
    za 1. republiky cukrárna Myšák pekla trdlovec, těžké těsto rozpékané
    na rožni
  \end{itemize}
\item
  2018 \href{https://www.ceeol.com/search/viewpdf?id=756237}{Disputed
  Words of Disputed Territories: Whose Is Kürtőskalács?}:

  \begin{itemize}
  \tightlist
  \item
    článek Imoly Katalin Nagy
  \item
    věnuje se historii slova Kürtőskalács a jestli patří Rumunům nebo
    Maďarům
  \item
    s. 75 vypisuje názvy v různejch jazycích - Baumkuchen, staroněmecky
    Ayrkuchen nebo Spiesskuchen, rakousko Prügeltorte, latinsky Obelias
  \end{itemize}
\item
  2018
  \href{https://ceskadigitalniknihovna.cz/uuid/uuid:3f2c7e80-38fe-11e9-b81e-005056827e52}{Věstník
  Historicko-vlastivědného kroužku v Žarošicích}:

  \begin{itemize}
  \tightlist
  \item
    není online, najít
  \item
    s. 44: trnkama a smotaný, v nedělu někdy jako pamlsek sladký
    trdélniky, pečený na dřevěnéch kolíkách. Klukom Adamcovém se
    posměšně
  \end{itemize}
\item
  2019
  \href{https://www.nzm.cz/o-nas/aktuality/teticky-z-kobyli}{Národní
  zemědělské muzeum}:

  \begin{itemize}
  \tightlist
  \item
    klub přátel historie z Kobylí
  \item
    recept na malý trdelníky plněný sněhen
  \item
    \href{https://www.ceskatelevize.cz/porady/13889655314-peceni-na-nedeli/13478-recepty/12089-trdelnicky-z-kobyli/}{Tady
    je i video}
  \end{itemize}
\item
  2019
  \href{https://ceskadigitalniknihovna.cz/uuid/uuid:771c80e0-53ee-11ea-a3ba-005056827e52}{Věstník
  Historicko-vlastivědného kroužku v Žarošicích}:

  \begin{itemize}
  \tightlist
  \item
    není online, najít
  \item
    povidly), pěry (kynuté knedlíky šiškovitého tvaru) rovněž plněné
    trnkami (povidly) a sypané mákem, trdelníky - jakési trubičky
  \end{itemize}
\item
  2020
  \href{https://kulturni-dejiny.slu.cz/data/uploads/067/upvysledky/jidlo-neni-jenom_odborna-kniha_uplatnny_2020.pdf}{Jídlo
  není jenon něco k jídlu}:

  \begin{itemize}
  \tightlist
  \item
    další výstup z projektu Kulinární dědictví českých zemí
  \item
    dělali dotazníkovej průzkum na Hlučínsku, Hanáckym Slovácku a
    Podkrkonoší
  \item
    Slovácko hlavně Hustopeče
  \item
    nejsou si jistý, jestli se znalost trdelníků odvíjí od lokální
    tradice, nebo novodobý komerční úspěšnosti
  \item
    znají ho skoro všichni, ale vaří ho jen dva respondenti
  \end{itemize}
\item
  2021 \href{https://www.cooklikeczechs.com/trdelnik/}{Cook like
  Czechs}:

  \begin{itemize}
  \tightlist
  \item
    recept a trochu historie
  \end{itemize}
\item
  2021
  \href{https://www.nzm.cz/o-nas/veda-a-vyzkum/publikacni-cinnost/odborne-publikace/kulinarni-tradice-moravskych-a-slezskych-regionu}{Kulinární
  tradice moravských a slezských regionů}:

  \begin{itemize}
  \tightlist
  \item
    Lucie Kubásková, Jana Jírovcová
  \item
    potvrzujou, co si myslím i já, a to že se zánikem černý kuchyně
    obliba trdelníků opadla a že se taky změnila jejich forma,
    zmenšovaly se
  \item
    vznikly z nich trubičky plněný sněhem z ušlehanejch bílků (a co
    kremrole?)
  \item
    smažený trdelníky se dělají z koblihovýho těsta
  \end{itemize}
\item
  2022
  \href{https://wirtschaftsmuseum.at/media/downloads/Interreg/Kucharka_Kochbuch_WEB.pdf}{Kulinářské
  dědictví}, autoři Mačuda, Zvonařová, Eckl

  \begin{itemize}
  \tightlist
  \item
    s. 38 recept na litej trdelník z roku 1891, kuchařka Die praktische
    Wiener Köchin, Anna Bauer
  \item
    s 103 trdelníky oblíbený masopustní pečivo u místních Němců a v
    Rakousku zvaný Prügelkrapfen, v moravských obcích trdláče nebo
    trdelníky. hotové se krájely na prstence zvané krůžalky
  \item
    s. 104 - smažený trubičky Rollkrapfen nebo Ringelkrapfen
  \item
    s. 138 trdelníky byly součástí svatebních dortů, jako svatební
    koruny - rozřezaný do prstenců. litej trdelník byl oblíbenej na
    svatbách Němců
  \end{itemize}
\item
  2022
  \href{https://kulturni-dejiny.slu.cz/data/uploads/067/upvysledky/067-2022-4-kulinarni_dedictvi_web_enc.pdf}{Kulinární
  dědictví Čech, Moravy a Slezska}:

  \begin{itemize}
  \tightlist
  \item
    s. 284-285: spousta informací
  \item
    trdelník starodávné pečivo, známé nejmíň od konce středověku
  \item
    tři způsoby výroby trdelníku:

    \begin{itemize}
    \tightlist
    \item
      pruhy navíjený spirálově na válec, Morava a Uhry
    \item
      těsto v pruzích navíjený na válec, pak se zmáčkne, aby se
      nerozpadlo do pruhů, Sedmihradsko, Sasko
    \item
      třetí litý těsto
    \end{itemize}
  \item
    v českých zemích zdomácněly hlavně na jižní Moravě
  \item
    obrázek Prügelkrapfen a trubičky s krémem
  \item
    na Brněnsku a Podhorácku se dělaly z proužků těsta navinutých na
    válec, zvaný dřevko, kozička nebo turčisko
  \item
    na Ždánicku se v 19. století pekly velký trdelníky, dlouhý údajně
    jako ruka, sypaný cukrem a ořechy. pak se krájely na kroužky
  \item
    pozdějc trdelníky menší, z kynutýho těsta
  \item
    počátkem 20. století se prodávaly průmyslově vyráběný otáčecí rožně
    nebo dřevěný trubičky do trouby
  \item
    trdelníky z trouby se už dělaly běžnějc, manželka Viléma Mrštíka je
    měla ráda ke kávě
  \item
    na Slovácku prošly během 19. a 20. století podobným vývojem,
    doložený minimálně tři varianty:

    \begin{itemize}
    \tightlist
    \item
      pruhy pečený na válci, krájený na kroužky
    \item
      na Horňácku se před upečením uhlazoval, aby se nerozpadal do
      spirály, pocukrovanej se krájel na krůžalce
    \item
      modernější varianta se pekla v troubě, ale někde se i smažila, pak
      se plnila bílkovým sněhem nebo šlehačkou, připomínaly kremrole
      nebo šamrole
    \end{itemize}
  \item
    na Zlínsku a Mikulovsku:

    \begin{itemize}
    \tightlist
    \item
      v německým prostředí se nazývaly Prügelkrapfen nebo Ringelkrapfen,
      rožnily se na trdle, označovaným taky jako pryglík
    \item
      podobně jako jinde se postupně zmenšily do podoby trubiček
    \item
      na Znojemsku a možná i Jihlavsku se v německých obcích dělal i
      litej trdelník, jak je známej z Rakouska, po vysídlení Němců byly
      zapomenutý
    \end{itemize}
  \end{itemize}
\item
  2023
  \href{https://www.ceskatelevize.cz/porady/13889655314-peceni-na-nedeli/222544160510009/cast/947232/}{Česká
  televize}:

  \begin{itemize}
  \tightlist
  \item
    pečení na neděli, díl o Pálavě
  \item
    trdelník z Němčiček
  \end{itemize}
\item
  2024
  \href{https://reportermagazin.cz/79410/trdlokomedie-lukrativni-byznys-tahanice-a-tezke-vahy-v-pozadi/}{Trdlokomedie}:

  \begin{itemize}
  \tightlist
  \item
    zajímavej článek o historii trdelníku v Praze
  \item
    píšou tam o prvním výrobci trdelníku v Praze, kterej si otevřel
    stánek v roce 2000, potom, co viděl trdelník na Slovensku
  \end{itemize}
\item
  2024
  \href{https://www.bbc.com/travel/article/20241014-trdelnik-the-czech-food-thats-not-czech}{Trdelník:
  The Czech food that\textquotesingle s not Czech}

  \begin{itemize}
  \tightlist
  \item
    je tam historie prodeje trdelníků u stánků za posledních 20 let
  \end{itemize}
\item
  2024
  \href{https://is.muni.cz/th/livm9/Bakalarska_praca.pdf}{Revitalizácia
  pokrmu trdelník v Českej Republike}:

  \begin{itemize}
  \tightlist
  \item
    bakalářka, zkoumá historii i současnost
  \item
    závěr - na začátku nultejch let přinesli trdelník podnikatelé, co se
    inspirovali v Maďarsku
  \item
    receptura původně skalická, ale pak jí upravili, kvůli hygienckym
    normám (vajíčka) a technologii, aby šlo trdelníky snadnějc vyrábět,
    těsto snadnějc zpracovávat
  \item
    výzkum postojů k trdelníku
  \end{itemize}
\item
  2024
  \href{https://www.czechology.com/trdelnik-history-and-recipe/}{Trdelník
  -- History and Recipe}:

  \begin{itemize}
  \tightlist
  \item
    článek, kde se odkazujou na některý historický knihy, dokonce i na
    recept od Sibilly Dorizio
  \item
    ale opakujou bez důkazů teorii, že přišel z Maďakrska
  \end{itemize}
\item
  Web o koláčích na rožni: \url{https://cakes.institute/cakes.html}

  \begin{itemize}
  \tightlist
  \item
    popisuje různý typy a kategorizuje
  \item
    taky blog:

    \begin{itemize}
    \tightlist
    \item
      \href{https://diybaumkuchen.blogspot.com/2015/06/1581-82-was-boom-year-for-cake.html}{recept
      na Spiesskuchen z roku 1581}
    \item
      \href{https://diybaumkuchen.blogspot.com/2015/08/the-family-tree.html}{Popis
      různejch druhů dortů}
    \end{itemize}
  \end{itemize}
\item
  2025
  \href{https://medium.com/rooted-publication/when-does-a-dish-become-traditional-740fc377bef5}{When
  Does a Dish Become Traditional?}:

  \begin{itemize}
  \tightlist
  \item
    blogovej článek o tom, že trdelník už vlastně je tradiční
  \end{itemize}
\item
  2025 \href{https://www.youtube.com/watch?v=qUezTWRfAMc}{Trdelnik: Why
  tourists love it and locals hate it}:

  \begin{itemize}
  \tightlist
  \item
    video od Deutsche Welle
  \item
    opakujou všecky mýty o trdelníku, fakt špatný
  \end{itemize}
\end{itemize}

\section{Nacionalismus v jídle}\label{Nacionalismusux20vux20juxeddle}

\begin{itemize}
\tightlist
\item
  Itálie

  \begin{itemize}
  \tightlist
  \item
    \href{https://www.youtube.com/watch?v=iZZfwyKa0Lc}{How the U.S. made
    pizza popular (in Italy)}
  \end{itemize}
\end{itemize}
\section{Překlady}
\subsection{Einen Eyerkuchen oder Spißkuchen zu
machen}\label{250605-2334}

Zdroj: Coler, Johannes: Buch III: Vom Kochen. Aus: Oeconomia. Oder
Haußbuch. Bd. 1. Wittenberg, 1593. S. 102-208. Strana: 85

\url{https://www.deutschestextarchiv.de/book/view/coler_kochbuch_1593/?p=85}

Překlad:

Jak připravit vaječný koláč nebo špízový koláč (dort)

Chceš-li upéct vaječný koláč, vezmi dobrou sladkou smetanu a jednu nebo
dvě lžíce droždí. Pokud je špíz velký, musíš vzít více droždí. Nasyp do
toho hodně šafránu, aby to bylo pěkně žluté. Přilij také trochu másla a
všechno dobře prohněť, aby to bylo hezky hladké. Vytvoř z toho těsto,
které je pěkně volné a nelepí se ti na ruce. Přidej do něj malé rozinky.

Špíz namaž máslem, jako na jiný vaječný koláč, ale ne příliš mastně, aby
z něj koláč nespadl. Vezmi těsto a ulom z něj kousky. Vyválej je na
prkénku do dlouhých proužků a namáčej si ruce do mouky, aby se ti těsto
nelepilo. Z těsta udělej pět nebo šest kusů, podle množství těsta.

Když to budeš chtít namotat na špíz, poklepej to trochu rukou, aby se to
roztáhlo. Polož to na začátek špízu a nech někoho, aby ti špíz otáčel,
zatímco ty budeš těsto namotávat kolem dokola, jako bys namotával
provázek. Když jsi jeden kus těsta omotal, trochu ho stlač, aby se
roztáhl a pokryl špíz.

Poté vezmi další kousek těsta, přilož ho na špíz a znovu ho omotej, jako
ten předchozí. Dělej to tak dlouho, dokud se špíz nezaplní. Pak to celé
rovnoměrně uhlaď, aby to bylo všude stejné a nelepilo se to na špíz ani
se netrhalo.

Vezmi nit a volně to s ní svaž. Pak to dej k ohni a nech to péct. Když
je to z poloviny upečené, pokapej to horkým máslem. Těsto předtím osol
správným množstvím a také ho pokrop máslem na špízu.

Když odřezáváš roury (pravděpodobně části koláče), můžeš přidat máslo a
dát ho do malých misek, aby se do nich vaječný koláč namáčel a podával
vedle něj.

\subsection{Kochen im Bild: Herstellung von
Prügelkrapfen}\label{250615-0047}

(Vaření v obraze: Příprava Prügelkrapfenů)

\url{https://web.archive.org/web/20201230125915/http://kurtos.eu/dl/11664.pdf}

Pro tento v české kuchyni velmi oblíbený sladký pokrm zvaný „Trdelnice``
je zapotřebí železný pečicí rošt, který se vejde do trouby, jak je vidět
na obrázku (uprostřed nahoře). Dále jsou potřeba dvě kónická kulatá
dřeva s vyčnívajícími železnými osami. --- K samotnému těstu, které je
jemné kynuté těsto, se přidá 3 dekagramy rozdrceného droždí (30 gramů
droždí) posypaného kávovou lžičkou cukru, poté se rozpustí v ⅛ litru
vlažného mléka, aby se nyní přidalo tolik 1 kilogramu hrubé nebo
polohrubé pšeničné mouky, aby vzniklo poloměkké těsto, tzv. „Dampfel``,
které se krátce prohněte, poté hustě posype moukou a přikryté utěrkou se
nechá na teplém místě dobře vykynout. Zbytek mouky se smíchá v misce s 1
kávovou lžičkou soli, 5 dekagramy jemného cukru a jemně nastrouhanou
citronovou kůrou. Dále se nechá 30 dekagramů másla zcela rozpustit, poté
vychladnout, aby se toto přepuštěné máslo spolu s ¼ litrem vlažného
mléka, 4 žloutky a jedním celým vejcem dobře rozšlehalo a to vše se
nalije do mouky. Nyní se přidá i již vykynuté „Dampfel`` a za přidání
vlažného mléka podle potřeby se z celku vypracuje pevné, přesto vláčné
těsto, které se v míse hněte, dokud se neodlepuje od nádobí a rukou.
Vytvarované do bochníku se těsto nechá v míse půl hodiny přikryté
odpočívat, poté se vyklopí na lehce pomoučenou desku, aby se ještě
jednou dobře prohnětlo a rozdělilo na kousky zhruba o velikosti
dvojnásobné pěsti, které se vyválí do silných hadů o tloušťce palce. ---
Mezitím se pečicí rošt s oběma dřevy trochu předehřeje v troubě, poté se
dřeva dobře potřou vepřovým sádlem, ale pak se papírem velmi důkladně
otřou (aby se zabránilo tomu, že těsto spadne, pokud by dřevo bylo
příliš mastné). Na připravená dřeva se nyní spirálovitě nanese vyválené
těsto ve tvaru hada, jak je vidět na obrázku, přičemž jedna ruka otáčí
dřevem a druhá vede těstový pramen. Začátek a konec těstových pramenů se
pevně přitlačí na dřevo, aby se role nerozpadla. Nakonec se těstová
spirála zevnitř potře navlhčeným vejcem a hustě posype nahrubo
nasekanými vlašskými ořechy. Nyní je čas zasunout rošt s Prügelkrapfeny
do dobře předehřáté trouby na střední teplotu a krapfeny se v zavřené
troubě pečou poměrně rychle. Jakmile vnější části získají barvu, otáčejí
se dřeva v jejich ose; v případě potřeby se otočí i celý rošt, aby
Prügelkrapfeny získaly krásnou barvu po celém obvodu a dobře se
propekly. Nakonec musí mít zlato-hnědou barvu a doba pečení by měla být
asi 30 minut, poté se „trubky`` opatrně a za horka snímají z dřev, a to
tak, že se dřeva na užší straně rozklepou. --- Velmi dobré je, po
vyjmutí z trubek „vyfouknout`` pečicí páru, aby krapfeny nebyly uvnitř
lepkavé. Prügelkrapfeny se ještě teplé bohatě posypou vanilkovým
moučkovým cukrem a volně, avšak zcela zabalené do papíru se nechají
vychladnout. --- Pro podávání se potřebné množství Prügelkrapfenů
nakrájí na kroužky široké asi 3 centimetry, které se, jak je patrné z
obrázku, naskládají na hromadu a pocukrují. Prügelkrapfeny, které nejsou
určeny k okamžité spotřebě, by se neměly krájet, nýbrž zůstat zabalené v
papíře a uchovávat na chladném místě, kde vydrží čerstvé asi týden.

\subsection{Seklerský
kürtőskalács}\label{250319-1326}

Autor: Pozsony Ferenc
\url{https://epa.oszk.hu/03300/03308/00007/pdf/EPA03308_acta_siculica_2014-2015_497_518.pdf}

V naší práci stručně nastíníme název, původ, maďarské a erdeljské
rozšíření, šíření, evropské paralely, nově vzniklé funkce a významy
seklerské sladkosti připravované nad žhavým uhlím otevřeného ohniště.
Poté postupně ukážeme, jakou roli hraje tento tradiční druh koláče
připravovaný starobylými technikami v reprezentaci lokální, regionální,
etnické a národní identity a v konstrukci evropského kulturního
dědictví?

\subsubsection{Název}\label{250319-1326}

V maďarštině se koláč připravovaný na otevřeném ohništi nad hořícím
uhlím, navinutý na válec z kynutých těstových proužků a poté upečený do
křupava a s cukrovou polevou, nazývá kürtőskalács. Přestože se podle
historických pramenů dříve objevoval v severním Zadunají, Horní zemi, na
východním okraji Velké uherské nížiny, v Partiu, Sedmihradsku, Gyimesi,
Bukovině a Moldávii, dnes je v povědomí především jako seklerský
produkt, který se připravuje na slavnostní stůl, při křtinách a svatbách
a při přijímání významnějších hostů.

V Sedmihradsku se používají dva základní názvy: podle našich lingvistů
pochází varianta s krátkým ö a dlouhým ő, kürtős- a kürtőskalács, ze
slov kürt a kürtő. Gyula Csefkó odvozuje předponu názvu koláče spíše ze
slova kürt (hudební nástroj), protože ženy navíjely jeho prameny na
válec ve tvaru komolého kužele stejným způsobem, jakým mladí Sekleři
kdysi na jaře vyráběli své trubky z spirálovitě oloupané kůry vrbových
nebo vrbových proužků.

T. Attila Szabó vyjádřil zcela odlišný názor: „...když se sundá z válce,
celý koláč tvoří jeden kus přibližně 25-30 cm dlouhého koláče ve tvaru
trubky nebo trubky. Protože se tento koláč ve tvaru trubky podává členům
rodiny a hostům, a spotřebitelé vidí páskovitě se odlamující těsto v
tomto charakteristickém tvaru, je zřejmé, že název mohl vzniknout pouze
z trubkovitého tvaru těsta. Trubka je na vesnici i ve městě běžná,
všeobecně známá věc, trubka už méně.`` Lingvista z Kluže jednoznačně
doporučil používání tvaru kürtőskalács. Variantou T. Attily Szabóa
posiluje i výraz schornstein-kolatsch používaný sedmihradskými Sasy,
který je prostým zrcadlovým překladem maďarského kürtőskalács.

\subsubsection{Šíření v maďarsky mluvící
oblasti}\label{250319-1326}

Anna Bornemisza, manželka erdeljského knížete Michala Apafiho, nechala v
roce 1680 přeložit Jánosem Keszeiem do maďarštiny kuchařku Marxe
Rumpolta Ein neue Kochbuch, vytištěnou ve Frankfurtu nad Mohanem v roce
1581, ve které se Spiesskuchen, populární v německých oblastech,
doporučovaný jako třetí chod, nazývá botfánk. Jeho erdeljská maďarská
verze se objevila již na konci 17. století v soupisu z Uzdiszentpéteru z
roku 1679: Dřevo na pečení kürtős Fánk... Maďarský název koláče, kürtő
kalács, se objevil až o několik desetiletí později, v roce 1723, v
písemných pramenech Seklerska, konkrétně Háromszéku. Ve většině
sedmihradských sídel se nazýval především kürtőskalács, ale méně často
se vyskytovaly i výrazy kürtőspánkó a kürtősfánk (1679).

Nejstarší písemná zmínka o kürtőskalács pochází z první třetiny 18.
století, kdy hraběnka Ferattiné, rozená Ágnes Kálnoki, napsala ve svém
dopise z 22. prosince 1723 z moldavského knížecího dvora v Jászvásáru
následující slova své tetě z erdeljské Torje, Borbále Kálnoki, manželce
Petera Apora: „Ctihodná kněžna si žádá... abyste vyslali sluhu, kterého
byste laskavě nechali naučit všem druhům pečení, mimo jiné i kürtő
kalács... Ctihodná kněžna si žádá... o kterém sluhovi jste laskavě
slíbili, že ho necháte naučit... Drahá teto... nelitujte laskavosti,
nechte ho naučit pečení chleba a dalším jemným pečením a kürtő kalács a
nějakému paštikovi a jemnému jídlu.`` Tento údaj také naznačuje, že mezi
erdeljskou a moldavskou elitou v té době panovaly úzké vztahy, a díky
nim se módnější prvky maďarské gastronomické kultury pravidelně šířily i
do Moldávie ležící východně od Karpat v 18. století.

Pokud by tento koláč byl v Sedmihradsku a jeho východní části,
Seklersku, populární již dříve, členové moldavské elity by se s ním
mohli seznámit a převzít ho již dříve. Je pozoruhodné, že Peter Apor ve
svém díle Metamorphosis Transylvaniae, dokončeném v roce 1736, nezmiňuje
kürtőskalács mezi starobylými, maďarskými a erdeljskými jídly, ačkoli
výše uvedený úryvek z dopisu dokazuje, že se již připravoval v kuchyni
jeho manželky. Přestože konzervativní šlechtic z Torje mohl kürtő dobře
znát, zjevně ho považoval za novou módu zprostředkovanou Rakušany, a
proto ho ve svých pamětech ani nezaznamenal, přestože byl tento druh
koláče mezi erdeljskou elitou již tehdy poměrně oblíbený. Zdůrazňujeme,
že Peter Apor si stěžoval na zmizení dřívější erdeljské stravovací
kultury: „Jídla, na která byli zvyklí naši otcové, nemůžeme jíst, pokud
nemáme německého kuchaře.``

Zdůrazňujeme, že nejstarší forma kürtőskalács se do maďarských komunit
dostala buď prostřednictvím Rakušanů usazených v Sedmihradsku, nebo
Sasů, kteří měli dobré vztahy s vnitřními německými oblastmi. Je
zajímavé, že vydání kuchařky Szakács mesterségnek könyvecskéje z Kluže z
roku 1771 se o kürtőskalács, tehdy již velmi oblíbeném klužském zelí,
vůbec nezmiňuje. Tato skutečnost může naznačovat, že tento koláč v té
době ještě nebyl organickou součástí sváteční stravy rodin střední a
nižší třídy.

Spojená forma slova kürtőskalács se poprvé objevuje v kuchařce Dániel
Istvánné Gróf Mikes Mária A Gazda Aszszonyi Böltseségnek Tárháza,
napsané v roce 1781 a vydané v roce 1784, a v ní se nachází i první
maďarský recept na tento koláč: „Těsto nenech tvrdé, protože pak nebude
vláknité uvnitř.`` V případě „kürtőskalács podle Porániné`` se
doporučovalo, aby se těsto po rozválení a nakrájení navinulo na máslem
vymazané dřevo a peklo se potírané máslem.

Maďarské názvy koláče (dorongos fánk, dorongfánk, botra tekercs) jsou v
podstatě zrcadlovými překlady německého Baumkuchen. První písemná zmínka
o koláči pečen Kürtőskalács, pečený na rožni, se poprvé objevuje v textu
komedie z roku 1789, jejímž autorem byl rodák z Komárna. Kuchařka
Kristófa Simaie „Způsob přípravy některých jídel`` z roku 1795 z
Kremnice také nazývá „Dorongos fánk`` koláč z kynutého těsta s
rozinkami, potřený třtinovým medem, a zveřejňuje podrobnější recept.
„Vezmi dva verdugy jemné bílé mouky, mléko spařené, ale vlažné, dva nebo
tři žloutky, dvě kaly pivovarských kvasnic, ty rozpusť v mléce a
rozmíchej, a nalij do mouky, vezmi verdug dobrého másla, rozehřej na
vlažno a nalij i to do mouky, a s malými rozinkami rozmíchej a udělej
těsto, aby se dalo rozválet, rožeň pomaž máslem, rozválej tenké dlouhé
těsto a oviň ho kolem, konce přilep k rožni, a jako pečeni otáčej nad
plápolajícím ohněm, když se opeče, stáhni z rožně a pomaž třtinovým
medem.`` Simaiova kuchařka obsahovala také významnou novinku,
doporučovala kuchařům, aby koláč po upečení ochutili tehdy typickým
sladidlem, tekutým cukrem zvaným třtinový med. Poznamenejme, že kynutí
těsta se v té době provádělo především pivovarskými kvasnicemi. Moderní
kvasnice byly vynalezeny až v roce 1848, zatímco průmyslová výroba
krystalového cukru začala až na konci 19. století.

T. Attila Szabó ve svém „Malém slovníku`` Dávida Baróti Szabóa (1792) a
v díle „Květy Maďarska`` (1803) našel pod heslem „kürt`` dobové názvy:
„Kürtős Kaláts, botra tekerts, botkaláts, rudas fánk``. Podle klužského
lingvisty, zatímco na konci 18. století se v Seklersku používala forma
kürtőskalács, výrazy botkalács a rudasfánk byly známé pouze mimo
Sedmihradsko a David Baróti Szabó, rodák ze Seklerska, se s nimi mohl
seznámit až během svého pozdějšího pobytu v Horním Uhersku. Varianta
koláče pečená na uhlí se v Maďarsku nejprve rozšířila v severních
malošlechtických vesnicích Zadunají. Tento typ koláče se později stal
slavnostním pokrmem na rolnických svatbách v severním Zadunají (např. v
Cseszneku). Na konci 19. století, podle kuchařky Ágnes Zilahyové, byl
vršek koláče bohatě posypán cukrovými mandlemi.

V 18. století se ve větších sedmihradských městech připravoval poměrně
často. Například klužské inventáře a soupisy z konce 18. století přesně
dokládají přítomnost dřeva na pečení kürtőskalács v tehdejších rodinných
domácnostech. T. Attila Szabó našel v hospodářských záznamech rodiny
hraběte Mikóa dokument z roku 1772, který zaznamenával vybavení tehdejší
klužské kuchyně a mezi jehož náčiním se nacházel i „Hrnec na pečení
kürtőskalács``. V archivu László Telekiho dokument z roku 1773
zaznamenával název nástroje používaného při pečení koláče jako „Forma na
kürtős kaláts``. Podle jiného dokumentu se v roce 1811 v soupisu majetku
dvora v Mezőőru v župě Kluž vyskytovalo „dřevo na pečení kürtős
kaláts``. Mezi majetkem Mihálye Szijgyártó Trintsiniho z Marosvásárhelye
se v roce 1810 objevilo „dřevo na pečení kürtős koláts``. V roce 1822 se
„dřevo na pečení kürtős kaláts`` nacházelo i v kurii barona Józsefa
Gyulakuti Lázára v Nyárádszentanně. Mezi majetkem Karla Petrichevich
Horvátha z Felsőzsuku v župě Kluž, sepsaným v roce 1827, se nacházelo
„dřevo na pečení kürtős kaláts s hrncem``. Uvedené údaje naznačují, že
na konci 18. a počátku 19. století se kürtőskalács pekl s pomocí
charakteristického dřeva především v bohatších městských kuchyních a
vesnických kuriích, a na mnoha místech se používala i jeho varianta
obložená hrncem.

Podle archivního výzkumu Dénese Cs. Bogátse se dřevo používané k pečení
koláče v Háromszéku objevuje častěji až v soupisech z první třetiny 19.
století: „Dřevo na pečení kürtős kaláče obložené hrncem...`` (1810),
„Dřevo na pečení kürtős`` (1834), „Čtyři malé lopatky a dřevo na pečení
kürtős kaláče...`` (1838). Údaje z Háromszéku naznačují, že v
jihovýchodním Sedmihradsku se tento druh koláče pekl na začátku 19.
století na válcovém dřevě s rukojetí, obloženém hrncem, které se pomalu
otáčelo nad žhavými uhlíky otevřeného ohniště. Tyto písemné prameny také
přesně dokládají, že kürtőskalács se v Seklersku rozšířil ve větším
měřítku ve vesnických komunitách až na začátku 19. století.

Balázs Orbán ve svém prvním svazku „Popisu Seklerska`` (1868) zveřejnil
máréfalvskou legendu o původu, která také odráží, že kürtőskalács byl v
té době již hluboce zakořeněn v životě regionu: „Když Tataři pustošili
krajinu, lidé z Máréfalvy se uchýlili do svých ochranných jeskyní, a
když Tataři vcházeli do údolí, ti v jeskyni na ně stříleli šípy a
skolili několik z nich, včetně jejich vůdce. Tatarské vojsko se
rozzuřilo a začalo útočit na úkryt; ale protože se k silně položenému
úkrytu nemohli dostat, oblehli ho, aby ho vyhladověním donutili ke
kapitulaci. Obléhaným i obléhatelům již došly všechny zásoby, když ti v
jeskyni udělali z slámy velký kürtős kalács, ukázali ho a křičeli dolů:
Hle, jak se máme dobře, zatímco vy hladovíte! Když to Tataři viděli, po
zničení vesnice odešli.``

Kuchařka tety Rézi, vydaná v Szegedu v roce 1876, informovala o další
významné změně ve vývoji koláče. Její autorka doporučuje, aby se povrch
koláče před pečením posypal mandlovým cukrem. Ágnes Zilahyová ve své
„Pravé maďarské kuchařce``, vydané v Budapešti v roce 1892, již
doporučovala válení v cukru bez mandlí.

Krby se v maďarsky mluvící oblasti, v okrajové části Seklerska, udržely
až do konce 19. století. Kürtőskalács se tedy mohl i nadále péct nad
uhlíky na otevřených ohništích a v otevřených předsíních pecí na chleba
za tradičních podmínek. Tato skutečnost do značné míry přispěla k tomu,
že kürtőskalács byl až do druhé poloviny 20. století považován za krále
slavnostních koláčů v sekl

Poznamenejme, že slazení krystalovým cukrem se v maďarských vesnicích
rozšířilo až koncem 19. století. Proto se forma koláče pečeného na uhlí
s cukrovou polevou mohla vyvinout až od té doby a teprve poté se
rozšířila.

Podle údajů Maďarského etnografického atlasu byl kürtőskalács na počátku
20. století již významným slavnostním koláčem v župách Bihar, Hajdú,
Szatmár a Ung, ale byl již široce známý také v Silágysku, historickém
Sedmihradsku, včetně Seklerska, a u Seklerů žijících v Gyimesi, Bukovině
a u moldavských Čangů. Podle mapy Atlasu čangských dialektů byl v
Moldávii obecně rozšířen v seklerských vesnicích. Obvykle se připravoval
při svatbách, novoročních a masopustních oslavách, ale nosil se i ženám
po porodu v košíku radina. V Gyimesi se do košíku radina obvykle
vkládaly 4 kusy kürtőskalács, k nim se přidaly 3-4 preclíky, 4 vrstvené
placky a 1 svatojánský chléb. Menší varianta smažená na tuku byla ve 20.
století rozšířena i v zemědělských městech Kiasalföldu a v Sedmihradsku.

Recepty týkající se kürtőskalács v maďarských kuchařkách ze 17.-19.
století a jejich gastronomicko-historické poznatky nedávno shrnul Balázs
Füreder. Ve své práci zdůraznil, že se v průběhu posledních čtyř století
neustále měnily suroviny, způsob přípravy a ochucování tohoto maďarského
druhu koláče. Recepty ze 17.-18. století dokazují, že se již tehdy hnětl
z kynutého těsta, občas se ochucoval rozinkami a poté se pekl navinutý
na dřevo nad uhlíky. Od počátku 19. století se jeho příprava neustále
měnila: mohl se péct z kynutého nebo křehkého těsta, nad uhlíky nebo na
pánvi, v oleji, mohl se ochucovat nejen mandlemi nebo ořechy, ale mohl
se také válet v cukru ještě před pečením, čímž se na vnějším povrchu
koláče vytvořila křupavá cukrová poleva. Ve 20. století se pekl vždy z
kynutého těsta, zatímco jeho povrch se mohl ochucovat karamelizovanou,
mandlovou karamelizovanou a ořechovou karamelizovanou cukrovou polevou.
Na počátku 21. století se v komerčně vyráběných variantách máslo
nejčastěji nahrazuje margarínem, odvážně se koření a peče se do křupava
v moderních topeništích.

\subsubsection{Původ a evropské
paralely}\label{250319-1326}

O původu koláče nemáme přesnější údaje. Podle výzkumu Eszter Kisbánové
nebyl na konci 16. století ve středoevropském kulinářském umění žádnou
novinkou. V německy mluvících oblastech je jeho zvláštní varianta známá
jako Baumkuchen, jejíž původ někteří odvozují z německého města
Salzwedel. Jiní se domnívají, že se tam dostal z maďarských oblastí
prostřednictvím kuchařky Marxe Rumpolta „Ein Neues Kochbuch``, vydané ve
Frankfurtu nad Mohanem v roce 1581, který dříve pracoval jako mistr
kuchař v Maďarsku. Zároveň se dá předpokládat, že se podobné koláče
vyvíjely nezávisle na sobě v různých oblastech Evropy v rámci různých
gastronomických kultur, protože pečení těsta navinutého na tyč nebylo
neznámé ani ve starověké řecké, římské a dálněvýchodní kuchyni.

Variantami vyskytujícími se v německých oblastech se zabýval slezský
Hahn Fritz (1908-1977), jehož dokumentační materiál se dnes nachází v
archivu Steiermarkischen Landesmuseum na zámku Stainz. Podle Hahnova
výzkumu staří Řekové již pekli chléb Obelias z obilné mouky, nad uhlíky,
z spirálovitých, předem nakynutých těstových pásů navinutých na tyč,
často dlouhý i metr. Starověká vyobrazení dokazují, že v rámci Dionýsova
kultu obvykle dva lidé nosili velký koláč na ramenou pomocí tyče. Tento
způsob výroby a pečení, který v podstatě napodoboval pečení masa, se od
Řeků naučili i Římané, nebo ho možná objevili sami. Domníváme se, že se
z Itálie přes Alpy dostal do dnešních německých oblastí, kde se kolem
roku 1550, v těžkých válečných časech, stále pekl venku na tyči nad
ohněm takzvaný Notbrot, tedy nouzový chléb.

V roce 1450 se v německých oblastech připravoval koláč z těsta
navinutého na dřevěný válec, jehož vnější strana se potírala žloutkem a
poté se otáčela nad žhavými uhlíky ohniště, dokud se nespekla. O více
než sedmdesát let později, v roce 1526, benátská kuchařka zaznamenala,
že jeho jemná textura ve tvaru pásků se již připravovala z mouky, vajec,
smetany a koření a poté se pekla na otevřeném ohni na dřevěné tyči,
přičemž se neustále potírala tukem, máslem a vejci. Podle Fritze Hahna
je na prvním titulním listu italské kuchařky Epulario z roku 1526
vyobrazen koláč pečený na tyči nad otevřeným ohništěm.

V 15.-16. století byl slavnostním koláčem především bohatších
šlechtických a později bohatších měšťanských rodin v německých osadách.
Podle policejních zpráv a písemných pramenů se v Norimberku v roce 1485
konzumoval především na svatbách bohatších patricijských rodin. Protože
tehdejší úřady považovaly pečení koláče za plýtvání, nelibě nesly jeho
konzumaci a všemi prostředky se snažily omezit počet osob pozvaných na
svatební konzumaci „vaječného koláče`` (ayrkuchen). Ve Frankfurtu nad
Mohanem byl v roce 1576 dokonce zakázán jeho příprava. Pravděpodobně v
roce 1539 vznikl podrobný recept, který se dochoval v dominikánské
kuchařce. Rozhodné omezující kroky úřadů však nemohly jeho pečení zrušit
a v německých osadách se pekl dál.

Kuchařka Balthasara Steina, vydaná v Dillingenu v roce 1547, již odráží,
že v životě koláče došlo na konci 16. století k významné změně. Zatímco
v maďarských a českých oblastech se stále připravoval ze spirálovitě
navinutých copů na dřevě, v německých osadách se rozválené pláty
kynutého těsta kladly přímo na pečicí dřevo. Pláty koláče z mouky,
vajec, smetany slazené medem a koření se tedy navíjely na předehřáté
dřevo, zplošťovaly a poté se obalovaly provázkem, na jehož vnějším
povrchu se během pečení objevovaly spirálovité drážky a prsteny. Tyto
druhy koláčů se připravovaly především na knížecích dvorech. Tento
postup pečení doporučoval i Marx Rumpolt ve své renesanční kuchařce,
vydané v roce 1581, kterou nechala přeložit do maďarštiny i manželka
knížete Apafiho. Kuchařka Marie Schellhammerin, vydaná v roce 1697, již
obsahovala i kresbu k receptu. Kuchařka Christopha Thiemena z roku 1682
dokumentuje rozšíření Baumkuchenu na Moravě.

Poznamenejme, že tento starší druh koláče, který se nepekl ze
spirálovitých copů, ale z předem rozválených plátů, připravovali i
barcasští Sasové, kteří žili v sousedství Seklerska. V saských vesnicích
v okolí Brašova (Höltövény/Heldsdorf, Prázsmár/Tartlau,
Szászhermány/Honigberg a Volkány/Wolkendorf) se koláč připravoval z
tenčího nebo silnějšího těsta, které se nemuselo připevňovat provázkem k
válcovitému rožni nebo válečku na těsto, protože hospodyně pláty těsta
silně přitlačily dlaněmi na válcovité dřevo. Když se plát kynutého těsta
přilepil na dřevo, nejprve se obalil krystalovým cukrem, který se během
pečení nad uhlíky postupně rozpouštěl a nakonec karamelizoval do
křupava. Ačkoli se sedmihradští Sasové v desetiletích po druhé světové
válce hromadně vystěhovali do Německa, v rámci svých letničních setkání
v Dinkelsbühlu každoročně hrdě pečou, prodávají a konzumují Baumstriezel
jako starý, typický saský slavnostní koláč, který si přivezli ze svého
sedmihradského rodiště.

Koláč připravovaný na rožni nad žhavými uhlíky otevřeného ohniště zůstal
populární i na konci 16. století a dokonce se připravovala jeho nová
forma z tekutého materiálu, který se kapáním nanášel na pečicí dřevo,
podle kuchařky Susany Gewandtschneiderin z roku 1585. V následujícím
století, zejména po třicetileté válce (1618-1648), se jeho popularita
znovu zvýšila. Na konci 17. století se v německých osadách připravoval z
tekutého vaječného základu, který se vrstveně naléval na dřevo otáčené
nad ohništěm, čímž se na něm vytvořily charakteristické hrbolky z
kapajícího základu z jemné mouky, másla, vajec a smetany. V 18. století
se v německy obývaných oblastech pekl koláč nejen z tvárného, ale i z
tužšího těsta. Poznamenejme, že podobnou technikou a z tekutého základu
se kapáním připravuje litevský a polský ragoulis-sakotis-sekacz, který
později převzali i Francouzi a nazývají ho gateu-a-la-broche, a švédský
populární spettekaka. Poznamenejme, že Rakušané dodnes připravují
podobnou technikou a ze stejných surovin prügeltorte, který se reliktně
zachoval především v vesnicích v údolí potoka Brandberg v Tyrolsku.
Tekuté těsto se nalévá na předehřátý kotouč obalený pečicím papírem,
pomalu otáčený nad ohništěm, a když se vlivem tepla ztuhne, nalije se
další dávka na jeho stále hrbolatější povrch.

Podle kuchařky Markuse Loofta z roku 1769 začali němečtí městští cukráři
od konce 18. století obvykle šlehat základ z vajec, smetany, másla a
cukru do pěny a později jej ochucovat čokoládou. Péter Hantz upozornil,
že nejprve se do těsta s větším množstvím másla vmíchaly pouze žloutky,
zatímco bílky se zvlášť smíchaly se špetkou soli a poté se vmíchaly do
těsta na koláč. Dodnes se zachovala praxe rytmického prořezávání
hladkého povrchu koláče, tedy jeho zvlnění, a následného polévání
cukrovou nebo čokoládovou polevou.

Koláč se postupně začal připravovat i v kuchyních měšťanských rodin. V
19. století se vrstvený koláč s cukrovou polevou nebo čokoládovou
polevou pekl především pro přijímání významných hostů. Když z bytů
zmizela otevřená ohniště a objevila se uzavřená kovová kamna, pečení
tohoto charakteristického koláče ve velké části německých oblastí
zaniklo, ale cukrářské knihy vydané v mnoha městech (např. v Berlíně,
Magdeburgu, Salzwedelu) jej dále popularizovaly. Díky tomu a také díky
poptávce po něm se v německy mluvících oblastech dodnes připravuje
sladký koláč z vrstev stejné tloušťky, často potažený čokoládou. Ve 20.
století se nahřátá kovová tyč nejprve namáčela do tekutého těsta nebo se
na ni základ postupně naléval. Když zhnědl, nanášela se na něj vrstva za
vrstvou a jeho povrch se často tvaroval do vln. Historie německé
varianty také dokazuje, že se koláč neustále měnil až do současnosti a
postupně integroval nové a nové vlivy.

Sladký slavnostní koláč pečený na rožni nad otevřeným ohništěm se dodnes
připravuje v mnoha evropských zemích. V Německu se nazývá Baumkuchen, v
Rakousku Prügelkrapfen, na Slovensku a v Česku Trdelník, v Polsku
Sekacz, v Litvě Ragoulis-Sakotis, ve Švédsku Spettakaka, v Lucembursku
Baamkuch a ve Francii Gateau a la broche.

Domníváme se, že německá varianta Baumkuchenu z tekutého základu se
nejprve rozšířila do polských a litevských oblastí, kde se její
technologie a základ specificky dále vyvíjely. Litevský, francouzský a
polský koláč se obvykle připravoval z řidšího základu, takže se na
povrchu těsta vlivem odstředivé síly vytvořily stalaktitové hrbolky. To
potvrzuje i litevský název tohoto druhu koláče, který znamená špičatý,
vícevětvový, zatímco polský název znamená hrbolatý. Domníváme se, že se
tato specifická forma přípravy tohoto typu koláče v této oblasti mohla
vyvinout kolem 18. století. Protože Litevci považují Ragoulis-sakotis za
své typické jídlo, prodávají jej na mnoha evropských akcích jako svůj
národní symbol. Je pozoruhodné, že Francouzi připravují podobný koláč s
názvem gateau-a-la-broche pouze v Pyrenejích. Podle jejich ústní tradice
si koláč oblíbili vojáci Napoleona během svých východních tažení na
počátku 19. století, poté se s jeho receptem vrátili domů a jejich
potomci jej připravují dodnes.

V jižním Švédsku se tradičně pekl z bramborové mouky od první poloviny
18. století. Jeho nejstarší recept se nachází ve švédském vydání
kuchařky Susany Egerin z Německa z roku 1733, což dokazuje německý původ
koláče. Základ z másla, cukru, vajec a bramborové mouky se obvykle plní
do sáčku, z něhož se postupně vytlačují proužky na kuželovitý tvar
otáčený nad ohništěm, dokud jeho povrch není zcela pokrytý pásky. Povrch
suchého koláče s pórovitou strukturou se často zdobil barevnou cukrovou
polevou. Spettakaka byl zástupci provincie Scania zaregistrován u
Evropské unie jako typický švédský gastronomický produkt.

Protože trdlo nebo tredlenice, které se vyskytují u Čechů a Moravanů,
jsou velmi blízké starému německému spiesskuchen-ayrkuchenu, domníváme
se, že se v jejich komunitách rozšířily vlivem Německa. Jejich základním
rysem je, že se kynuté těsto navinuté na rožeň před pečením neobaluje v
cukru, ale pouze se posype mletými ořechy, přičemž se během pečení
potírá máslem a bílky, takže se na jeho drsnějším povrchu nevytvoří
cukrová poleva. V Skalici na Slovensku se po sejmutí z pečicího dřeva
posype jemněji mletým vanilkovým moučkovým cukrem. První písemná zmínka
o skalickém trdelníku se nachází v textu maďarského básníka Gyuly
Juhásze z roku 1911. Protože se koláč v meziválečných letech již
nepřipravoval, byl znovu objeven až v prvních letech 21. století a
zaregistrován jako typický slovenský koláč. Zdůrazňujeme, že v českých a
slovenských městech se stal skutečně populárním především v posledních
dvou desetiletích, kdy si Švédsko a Slovensko nechaly u Evropské unie
zaregistrovat své varianty jako typické švédské, respektive slovenské
produkty.

Koláč nápadně podobný seklerskému kürtőskalácsi, připravovaný stejnou
technikou a formou, se dnes objevuje i v Turecku a Japonsku.
Poznamenejme, že vynalézaví seklerské, maďarské a jiné národnostní
podniky v posledních dvou desetiletích připravují a prodávají tradičně
připravovaný seklerský kürtőskalács ve všech koutech světa.

Sedmihradský seklerský kürtőskalács se v posledních deseti letech dostal
do specifického evropského kontextu. Ve Skalici na Slovensku vzniklo v
roce 2004 občanské sdružení, jehož cílem bylo nechat zaregistrovat koláč
pečený na uhlí u evropských institucí. V důsledku toho získal skalický
trdelník (Skalicky trdelnik) 21. dubna 2007 ochrannou známku EU. V
materiálu a odůvodnění slovenského místního zájmového sdružení se uvádí,
že podle skalické ústní tradice recept na kürtőskalács připravovaný ve
městě a okolí zavedl sedmihradský kuchař maďarského spisovatele a
básníka Józsefa Gvadányiho, který zde žil v letech 1783-1801.

Slovenská registrace kürtőskalácse vyvolala v Seklersku specifickou
konkurenci. Od té doby se mnoho seklerských komunit snaží připravit
nejdelší kürtőskalács na světě. Například obyvatelé Uzonu v Háromszéku,
kteří v posledních deseti letech prodávají koláč i na ulici turistům a
cestujícím projíždějícím obcí, nejprve upekli 2,5 m dlouhý kürtőskalács
v roce 2007 v rámci místních vesnických slavností a poté v roce 2009
připravili již 10 m dlouhý koláč, který ve stejném roce překonali
obyvatelé sousední obce Szentivánlaborfalva 14metrovým koláčem.
Skaličané na jaře 2011 připravili „největší na světě`` 1,93 m (!) dlouhý
kürt

\subsubsection{Shrnutí}\label{250319-1326}

Koláč navinutý na tyč a pečený otáčením nad žhavými uhlíky ohniště má v
Evropě v zásadě tři formy: a) první varianta se připravuje ze
spirálovitých copů, b) druhá z rozválených plátů těsta, c) třetí se na
tyč nanáší postupně kapáním a během pečení se pomalu a rovnoměrně otáčí
nad ohněm.

V maďarsky mluvící oblasti, včetně Seklerska, se zachoval
nejarchaističtější způsob přípravy tohoto evropského druhu koláče: z
kynutého těsta se nejprve připraví spirálovitý cop, který se navine na
válcovitou tyč, dlaněmi se pevně přitlačí a poté se peče otáčením nad
uhlíky do křupava. Výzkumníci evropské kultury stravování našli
historické předchůdce této techniky pečení nejen u starých Řeků a
Římanů, ale i u středověkých Němců.

Od 16. století se v německých osadách vyvinula i forma tohoto koláče,
kdy se hladce rozválené pláty těsta navíjely ve více vrstvách na předem
nahřáté válcovité dřevo a povrch se spirálovitě obvazoval provázkem,
čímž se na povrchu koláče vytvořily spirálovité drážky a prsteny.
Zdůrazňujeme, že podobné varianty z tužšího základu se vyskytovaly v
různých německých osadách ještě v 18. století.

Třetí varianta tohoto jídla (například u Poláků, Litevců a Francouzů,
Švédů) se připravuje tak, že se z řidšího základu postupně kape na dřevo
pomalu otáčené nad otevřeným ohništěm, čímž povrch koláče nakonec získá
rozmanité tvary. Zdůrazňujeme, že v německých osadách se již od 17.
století připravoval koláč z řidšího základu, nalévaný na nahřátou tyč a
otáčený nad ohništěm. Jeho vrstvené varianty se od konce 18. století
stávaly stále náročnějšími a na konci 19. století se jejich povrch
dokonce poléval čokoládou.

Evropské varianty koláče příbuzného seklerskému kürtőskalácsi jsou tedy
dodnes populární v různých německých, moravských, českých, slovenských,
polských, litevských, švédských a francouzských komunitách. Na základě
písemných pramenů a dobových vyobrazení, jakož i zkoumání oblasti
rozšíření, se domníváme, že se seklerský kürtőskalács rozšířil v
maďarsky mluvící oblasti především z německých komunit, případně
prostřednictvím Sasů ze západu.

Podle našeho názoru se Sekleři seznámili s nejstarší variantou německého
druhu koláče, připravovanou z tužšího základu, spirálovitě navinutou na
tyč a pečenou nad uhlíky, kterou převzali, dále rozvíjeli a zachovali
dodnes. Charakteristická cukrová poleva tohoto slavnostního koláče,
často posypaná mletými ořechy a ochucená, se v Seklersku vyvinula až od
konce 19. století, po širší výrobě a rozšíření krystalového cukru.

Poznamenejme, že skalický „slovenský`` kürtőskalács je kopií
seklersko-maďarské varianty. Koláč z proužků kynutého těsta spirálovitě
navinutých na tyč a pečených otáčením nad uhlíky je na Slovensku v
podstatě neměnný, zatímco v maďarsky mluvící oblasti se tento koláč
vyskytuje široce a jeho přítomnost lze průběžně dokumentovat písemnými
prameny již od konce 17. století.

Dochované písemné a tištěné prameny naznačují, že se tento jemnější
koláč z kynutého těsta v Sedmihradsku usadil až v 17. století, kde byl
zpočátku slavnostním koláčem pouze elity, šlechtické vrstvy, a od konce
18. století se postupně široce rozšířil a usadil i v jiných
společenských vrstvách, začal se připravovat i v kuchyních městských
rodin a venkovských zemědělců. Protože se otevřená topeniště v
seklerských vesnických domech zachovala až do konce 19. století, bylo
možné kürtőskalács nadále připravovat za tradičních podmínek nad žhavými
uhlíky krbu při různých rodinných a kalendářních svátcích. Tento
prestižní koláč tak mohl zůstat významným druhem koláče na seklerských
maďarských rodinných a společenských svátcích i na konci 19. století.
Protože v seklerských vesnických zemědělských komunitách se cukr začal
ve větší míře používat právě v té době, cukrová forma kürtőskaláče se
mohla vyvinout až po širším rozšíření cukru vyráběného v továrnách.

Významný symbol maďarské vesnické zemědělské kultury se připravoval
především na významné události vesnických rodin (křtiny, svatby,
přijímání významných hostů). Je zajímavé, že v ostatních částech
maďarsky mluvící oblasti, vlivem rychlejší a dřívější měšťanizace a
následné modernizace, se již nepekl. V maďarských osadách mimo Seklersko
jeho místo zaujaly především dorty městského měšťanského původu.

V životě seklerského kürtőskaláče nastala nová situace s rumunským
ideologickým uvolněním v roce 1968. Bukurešťská moc, bezprostředně po
invazi do Československa, dočasně zmírnila svou dřívější protimenšinovou
politiku a z taktických důvodů krátce tolerovala prodej maďarských
lidových uměleckých předmětů v různých turistických centrech. Mnoho
agilních, podnikavých seklerských rodin brzy využilo této situace a
začalo péct a prodávat kürtőskalács nejen v sedmihradských, ale i v
valašských a moldavských rumunských letoviscích. Ve druhé polovině 20.
století se v důsledku ideologického uvolnění v roce 1968 poměrně rychle
rozšířil i do rumunských regionů mimo Seklersko, především v pobřežních
a horských turistických centrech.

Rumunská změna režimu v roce 1989 přinesla ještě zajímavější fenomén
obrození. Zpočátku se jím nabízeli pouze maďarští turisté navštěvující
seklerské vesnice a postupně se stal nejvýznamnějším, symbolickým
koláčem stále populárnějších místních slavností. Koláč, který se dříve
připravoval pouze v Seklersku a rumunských turistických centrech, se
tedy stále více šířil jako typická seklerská sladkost. Populární
kürtőskalács se po roce 1990 nejprve postupně šířil na západ, tedy stal
se oblíbenou sladkostí v severozápadním pohraničí země, v Partiu, a poté
se postupně usadil i v různých maďarských městech a letoviscích.

Kürtőskalács se v důsledku prezentovaného procesu, podobně jako guláš,
dnes stal nejen seklerským, ale celomaďarským symbolem, organickou
součástí seklerské a maďarské identity, přičemž i nadále zůstává
významným prvkem turistiky směřující do Seklerska. Kürtőskalács se na
přelomu 20. a 21. století jednoznačně stal seklerským maďarským
symbolem. Zatímco se stal nepostradatelným prvkem obrazu Seklerska a
Sedmihradska propagovaného

Díky pracovní migraci a v neposlední řadě turismu se dnes připravuje v
mnoha zemích, kde se stal nejen seklerským nebo maďarským, ale i
typickým evropským koláčem, specifickým gastronomickým symbolem.

Životní dráha kürtőskaláče dobře ilustruje, jak se gastronomický prvek
rozšířil z elitních kruhů, postupně se začlenil do populární kultury,
jak se během sta let stal organickou tradicí ve venkovských komunitách a
jak se nakonec v rámci folklorismu a turismu vrátil k vyšším vrstvám
společnosti.

Když Rumunsko vstoupilo do Evropské unie, začalo se postupně právně
chránit zdejší typické, často zdůrazňované tradice s etnickým významem.
Když si Slovensko nechalo ochránit skalický trdelník, nastala nová
situace i v postmoderním životě kürtőskaláče, který byl dříve považován
pouze za seklerskou tradici.

Především v Seklersku se rozvinula zajímavá rivalita a
hyperreprezentace: v rámci zde pořádaných místních slavností se místo
tradičního 40-50 centimetrového koláče pekly obrovské, 10-20 metrů
dlouhé, nadrozměrné seklerské koláče. Vzhledem k tomu, že
východoevropské komunity s různými jazyky žily po staletí mozaikovitě
vedle sebe, rozvinula se prudká debata a soutěž o ochranu mnoha dalších
tradičních kulinářských prvků (např. maďarská pálenka a rumunská cujka,
rumunský brânză a slovenská bryndza, maďarské a slovenské tokajské víno
atd.). Seklerský kürtőskalács se v posledních desetiletích stal vybraným
prvkem a součástí této symbolické mezi-etnické soutěže.

