\chapter*{Úvod}

\begin{tcolorbox}[title=Upozornění]
   Stránky jsou ve výstavbě, většina textu je zatím neuspořádaná a neúplná. Zveřejňuji je především kvůli kapitole \nameref{Anotovanuxe1ux20chronologickuxe1ux20bibliografie} na straně \pageref{Anotovanuxe1ux20chronologickuxe1ux20bibliografie}.
\end{tcolorbox}

Trdelník je české a slovenské označení pro pečivo původně pečené na rožni. Takové pečivo existuje
v různých variantách v několika zemích střední Evropy, ale i jinde, nejméně od konce středověku. 

Existují tři základní typy: 

\begin{enumerate}
  \item pruh těsta namotaný na formu
  \item těsto rozválené do širokého plátu, poté opečené na formě
  \item řídké těsto, které se na formu lije v několika vrstvách, poté co se předešlé vrstvy zapečou
\end{enumerate}

Historicky nejstarší je první typ (z něj se vyvinulo, to, co dnes nazýváme jako trdelník
nebo \foreignlanguage{hungarian}{Kürtőskalács} v Rumunsku a Maďarsku). 

Podobné pečivo bylo známé už antice, ale reálně z
kuchařek je od 15. století známý jako Spiesskuchen, česky vaječník. Z našeho
území ho zmiňuje roku 1554 ve svém kázání Johannes Mathesius z Jáchymova. Pak se o něm
zmiňuje Komenský a nachází se v několika slovnících.

Druhý typ je o dost vzácnější, našel jsem o něm jen několik zmínek. Ale v
kuchařce J. C. Thiema z roku 1694 je na něj recept, označenej jako Bohmische
Küchlein.

Třetí typ je nejnovější, je to v podstatě dort a je známej z Německa jako
baumkuchen, z Rakouska prügelkrapfen, je rozšířený i v Polsku, Švédsku a
Litvě. Na Moravě byl rozšířený na Znojemsku. V Čechách byl populární koncem
19. století a v první půlce století 20. jako dort trdlovec. Nedělal se ale
doma, nabízely ho  cukrárny, také se prodával na poutích. Zajímavé je, už
tenkrát se prodával jako staročeský trdlovec a lidi z toho měli legraci. Dneska
je u nás už prakticky zapomenutý.

Termín trdelník se objevuje začátkem 19. století. Je tak pojmenovaný recept na
trdelník z litého těsta v překladu kuchařky od Sibilly Dorizio. Není úplně
jisté datum vydání, ale nejspíš je to rok 1816. Každopádně to není recept na
typ trdelníku, jak ho známe dneska.

V roce 1813 vyšla sbírka básní Muza Morawská od J. H. A. Gallaše. Trdelník se
vyskytuje ve třech básních, třeba v Nestřídmost v jídle jedné hanácké osoby.
Gallaš žil v Hranicích a popisoval prostředí Valašska a Haný.

Pak je celkem velká mezera, ale od 80. let 19. století je o trdelnících velký
množství zmínek, protože byly populární různý vlastenecký spolky, který
zkoumaly lidovou kulturu a trdelníky a náčiní pro jejich pečení zkoumali jako
historickou kuriozitu. V roce 1895 byla v Praze velká národopisná výstava a
před ní byly menší výstavy i v regionech. Trdelníky jako tradiční pečivo
představovali na Slovácku, Brněnsku, Haný nebo Třebíčsku.

Už v týhle době byl tradiční trdelník pečenej na ohni na ústupu, protože v
domácnostech se přecházelo z černý kuchyně (to byl v podstatě otevřenej oheň v
kuchyni) na sporáky s pecí a klasický trdelníky se v ní dělat nemohly. Vznikly
proto jiný variaty, který se daly píct v troubě, nebo se smažily v sádle.
Protože byly mnohem menší, daly se plnit krémem a vzniklo něco podobnýho
kremrolím. Respektive kremrole samotný možná vznikly původně z trdelníku.

Během první půlky 20. století je o trdelníku ještě spousta zmínek a vyskytuje
se ve spoustě kuchařek (jak litá, tak navíjená varianta), ale po 2. světový
válce už celkem mizí a vyskytuje se spíš v etnografických publikacích, ale i
kuchařkách. Třeba Lidové pečivo v Čechách a na Moravě z roku 1988 se trdelníkům
věnuje dost podrobně a popisuje recepty na všechny možný varianty. V našem
století se pak někde začaly obnovovat tradice a trdelníky se zase začaly
objevovat v některých lokalitách, kde bejval oblíbenej.

Na Slovensku pak v 80. letech začala skalická pobočka Západoslovenských pekární
vyrábět trdelníky a ty se staly celkem popuární, dokonce získal cenu Zlatý
kosák na výstave Agrokomplex. Pak se prodával po různých jarmarcích a podobně.

Pokud jde o historii trdelníku v Praze, tak podle tohohle článku se objevil
poprvý v roce 2000. Jeden chlapík si ho všiml ve Štúrově na jarmarku a zaujal
ho natolik, že ho začal vyrábět. Nejdřív taky na různejch jarmarcích a pak na
vánočních trzích na Staromáku. Stal se z toho hit, začali ho kopírovat další
prodejci, vznikaly různý varianty a tím jsme se dostali do dnešního stavu.
Takže pražskej trdelník původně vycházel ze Skalickýho trdelníku, ale používají
jinou recepturu na těsto, aby se líp připravovalo (a taky je levnější) a různý
náplně a posypky jsou veskrze místní inovace.

V Maďarsku je ten průběh šíření dost podobnej. Původně byl kürtőskalács
rozšířenej docela hojně, ale v průběhu 20. století mizel, až se zachoval jen v
Sedmihradsku v Rumunsku. Někdy v 70. nebo 80. letech, když se začalo povoloval
drobný podnikání, začal se šířit v různejch stáncích i jinam, postupně se
dostal i do Maďarska, kde se stal taky populárním.

Pokud jde o legendu o původu trdelníku ve Skalici, nejstarší zmínka o ní,
kterou jsem našel, je z roku 1998. I v dokumentu pro registraci chráněnýho
geografickýho označení se píše, že se to místně traduje. Vzhledem k tomu, že
byl u nás spiesskuchen známej stovky let před tím, než se měl údajně dostat do
Skalice spolu s kuchařem ze Sedmihradska, a že jen pár let potom ho popisuje
Galaš ze střední Moravy, není moc pravděpodobný, že by se u nás skutečně
rozšířil z Rumunska.

