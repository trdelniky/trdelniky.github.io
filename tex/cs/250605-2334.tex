\subsection{Einen Eyerkuchen oder Spißkuchen zu
machen}\label{250605-2334}

Zdroj: Coler, Johannes: Buch III: Vom Kochen. Aus: Oeconomia. Oder
Haußbuch. Bd. 1. Wittenberg, 1593. S. 102-208. Strana: 85

\url{https://www.deutschestextarchiv.de/book/view/coler_kochbuch_1593/?p=85}

Překlad:

Jak připravit vaječný koláč nebo špízový koláč (dort)

Chceš-li upéct vaječný koláč, vezmi dobrou sladkou smetanu a jednu nebo
dvě lžíce droždí. Pokud je špíz velký, musíš vzít více droždí. Nasyp do
toho hodně šafránu, aby to bylo pěkně žluté. Přilij také trochu másla a
všechno dobře prohněť, aby to bylo hezky hladké. Vytvoř z toho těsto,
které je pěkně volné a nelepí se ti na ruce. Přidej do něj malé rozinky.

Špíz namaž máslem, jako na jiný vaječný koláč, ale ne příliš mastně, aby
z něj koláč nespadl. Vezmi těsto a ulom z něj kousky. Vyválej je na
prkénku do dlouhých proužků a namáčej si ruce do mouky, aby se ti těsto
nelepilo. Z těsta udělej pět nebo šest kusů, podle množství těsta.

Když to budeš chtít namotat na špíz, poklepej to trochu rukou, aby se to
roztáhlo. Polož to na začátek špízu a nech někoho, aby ti špíz otáčel,
zatímco ty budeš těsto namotávat kolem dokola, jako bys namotával
provázek. Když jsi jeden kus těsta omotal, trochu ho stlač, aby se
roztáhl a pokryl špíz.

Poté vezmi další kousek těsta, přilož ho na špíz a znovu ho omotej, jako
ten předchozí. Dělej to tak dlouho, dokud se špíz nezaplní. Pak to celé
rovnoměrně uhlaď, aby to bylo všude stejné a nelepilo se to na špíz ani
se netrhalo.

Vezmi nit a volně to s ní svaž. Pak to dej k ohni a nech to péct. Když
je to z poloviny upečené, pokapej to horkým máslem. Těsto předtím osol
správným množstvím a také ho pokrop máslem na špízu.

Když odřezáváš roury (pravděpodobně části koláče), můžeš přidat máslo a
dát ho do malých misek, aby se do nich vaječný koláč namáčel a podával
vedle něj.
